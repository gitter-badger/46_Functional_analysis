\documentclass[12pt]{article}
\usepackage{pmmeta}
\pmcanonicalname{NormedVectorSpace}
\pmcreated{2013-03-22 12:13:45}
\pmmodified{2013-03-22 12:13:45}
\pmowner{rspuzio}{6075}
\pmmodifier{rspuzio}{6075}
\pmtitle{normed vector space}
\pmrecord{14}{31604}
\pmprivacy{1}
\pmauthor{rspuzio}{6075}
\pmtype{Definition}
\pmcomment{trigger rebuild}
\pmclassification{msc}{46B99}
\pmsynonym{normed space}{NormedVectorSpace}
\pmsynonym{normed linear space}{NormedVectorSpace}
\pmrelated{CauchySchwarzInequality}
\pmrelated{VectorNorm}
\pmrelated{PseudometricSpace}
\pmrelated{MetricSpace}
\pmrelated{UnitVector}
\pmrelated{ProofOfGramSchmidtOrthogonalizationProcedure}
\pmrelated{EveryNormedSpaceWithSchauderBasisIsSeparable}
\pmrelated{EveryNormedSpaceWithSchauderBasisIsSeparable2}
\pmrelated{FrobeniusProduct}
\pmdefines{norm}
\pmdefines{metric induced by a norm}
\pmdefines{metric induced by the norm}
\pmdefines{induced norm}

\endmetadata

% this is the default PlanetMath preamble.  as your knowledge
% of TeX increases, you will probably want to edit this, but
% it should be fine as is for beginners.

% almost certainly you want these
\usepackage{amssymb}
\usepackage{amsmath}
\usepackage{amsfonts}

% used for TeXing text within eps files
%\usepackage{psfrag}
% need this for including graphics (\includegraphics)
%\usepackage{graphicx}
% for neatly defining theorems and propositions
%\usepackage{amsthm}
% making logically defined graphics
%%%%\usepackage{xypic} 

% there are many more packages, add them here as you need them

% define commands here

\usepackage{amsthm}
\theoremstyle{remark}
\newtheorem*{remark}{remark}

\newcommand{\norm}[1]{\lVert #1 \rVert}
\newcommand{\abs}[1]{\lvert #1 \rvert}
\newcommand{\ip}[2]{\langle #1 , #2 \rangle}
\begin{document}
Let $\mathbb{F}$ be a field which is either $\mathbb{R}$ or $\mathbb{C}$.  A \emph{\PMlinkescapetext{normed vector space}} over $\mathbb{F}$ is a pair $(V,\norm{\cdot})$ where $V$ is a vector space over $\mathbb{F}$ and $\norm{\cdot}\colon V\to\mathbb{R}$ is a function such that
\begin{enumerate}
\item $\norm{v}\geq 0$ for all $v\in V$ and $\norm{v}=0$ if and only if $v=0$ in $V$ (\emph{positive definiteness})
\item $\norm{\lambda v} = \abs{\lambda}\norm{v}$ for all $v\in V$ and all $\lambda\in\mathbb{F}$
\item $\norm{v+w}\leq\norm{v}+\norm{w}$ for all $v,w\in V$ (the \emph{triangle inequality})
\end{enumerate}

The function $\norm{\cdot}$ is called a \emph{norm} on $V$.

Some properties of norms:

\begin{enumerate}
\item
If $W$ is a subspace of $V$ then $W$ can be made into a normed space by simply restricting the norm on $V$ to $W$.  This is called the induced norm on $W$.

\item
Any normed vector space $(V,\norm{\cdot})$ is a metric space under the metric $d\colon V \times V \to \mathbb{R}$ given by $d(u,v)=\norm{u-v}$.  This is called the \emph{metric induced by the norm $\norm{\cdot}$}.

\item
It follows that any normed space is a locally convex topological vector space, in the topology induced by the metric defined above.

\item
In this metric, the norm defines a continuous map from $V$ to $\mathbb{R}$ - this is an easy consequence of the triangle inequality.

\item
If $(V, \ip{}{})$ is an inner product space, then there is a natural induced norm given by $\norm{v} = \sqrt{\ip{v}{v}}$ for all $v \in V$.

\item
The norm is a convex function of its argument.
\end{enumerate}
%%%%%
%%%%%
%%%%%
\end{document}
