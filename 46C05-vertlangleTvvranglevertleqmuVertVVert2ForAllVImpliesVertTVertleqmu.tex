\documentclass[12pt]{article}
\usepackage{pmmeta}
\pmcanonicalname{vertlangleTvvranglevertleqmuVertVVert2ForAllVImpliesVertTVertleqmu}
\pmcreated{2013-03-22 15:25:33}
\pmmodified{2013-03-22 15:25:33}
\pmowner{Gorkem}{3644}
\pmmodifier{Gorkem}{3644}
\pmtitle{$\vert\langle Tv,v \rangle\vert \leq {\mu} \Vert v \Vert ^2$ for all $v$ implies $\Vert T \Vert \leq {\mu}$}
\pmrecord{16}{37271}
\pmprivacy{1}
\pmauthor{Gorkem}{3644}
\pmtype{Theorem}
\pmcomment{trigger rebuild}
\pmclassification{msc}{46C05}

\endmetadata

% this is the default PlanetMath preamble.  as your knowledge
% of TeX increases, you will probably want to edit this, but
% it should be fine as is for beginners.

% almost certainly you want these
\usepackage{amssymb}
\usepackage{amsmath}
\usepackage{amsfonts}
\usepackage{amsthm}

% used for TeXing text within eps files
%\usepackage{psfrag}
% need this for including graphics (\includegraphics)
%\usepackage{graphicx}
% for neatly defining theorems and propositions
%\usepackage{amsthm}
% making logically defined graphics
%%%\usepackage{xypic}

% there are many more packages, add them here as you need them

% define commands here
\newcommand{\ip}[2]{\left\langle #1,#2\right\rangle}
\newcommand{\norm}[1]{\left\Vert #1 \right\Vert}
\newtheorem*{theorem}{Theorem}
\begin{document}
\begin{theorem}
Let $H$ be a unitary space, $T$ be a self-adjoint
linear operator and $\mu\geq 0$.  If $\vert\ip{Tv}{v}\vert\leq \mu
\|v\|^2$ for all $v\in H$ then $T$ is a bounded operator and
$\|T\|\leq \mu$.
\end{theorem}



\begin{proof}
 We will show that $\norm{Tv}\leq \mu\norm{v}$ for all $v\in H$.
This is trivial if $\norm{Tv}$ or $\norm{v}$ is zero, so assume
they are not. Let $\lambda$ be any positive number.
\begin{align*}
\norm{Tv}^2 & = \ip{Tv}{Tv}\\
&=\frac{1}{4}\left[ \ip{T\left(\lambda v +
\frac{1}{\lambda}Tv\right)}{\left(\lambda v +
\frac{1}{\lambda}Tv\right)} - \ip{T\left(\lambda v -
\frac{1}{\lambda}Tv\right)}{\left(\lambda v
-\frac{1}{\lambda}Tv\right)}\right]\\
&\leq \frac{\mu}{4}\left[\norm{\lambda v + \frac{1}{\lambda}Tv}^2
+\norm{\lambda v - \frac{1}{\lambda}Tv}^2 \right]\\
&\leq\frac{\mu}{2}\left[\lambda^2\norm{v}^2 +
\frac{1}{\lambda^2}\norm{Tv}^2 \right]
\end{align*}
Now if we put $\displaystyle \lambda^2 =
\frac{\norm{Tv}}{\norm{v}}$ we get $\norm{Tv}^2 \leq
\mu\norm{Tv}\norm{v}$ hence $\norm{Tv}\leq \mu \norm{v}$.
\end{proof}

\textbf{ Reference:}

F. Riesz  and B. Sz-Nagy, \emph{Functional Analysis}, F. Ungar
Publishing, 1955, chap VI.
%%%%%
%%%%%
\end{document}
