\documentclass[12pt]{article}
\usepackage{pmmeta}
\pmcanonicalname{L1GHasAnApproximateIdentity}
\pmcreated{2013-03-22 17:42:40}
\pmmodified{2013-03-22 17:42:40}
\pmowner{asteroid}{17536}
\pmmodifier{asteroid}{17536}
\pmtitle{$L^1(G)$ has an approximate identity}
\pmrecord{9}{40155}
\pmprivacy{1}
\pmauthor{asteroid}{17536}
\pmtype{Theorem}
\pmcomment{trigger rebuild}
\pmclassification{msc}{46K05}
\pmclassification{msc}{43A20}
\pmclassification{msc}{22D05}
\pmclassification{msc}{22A10}
\pmdefines{$L^1(G)$ has an identity element iff $G$ is discrete}

% this is the default PlanetMath preamble.  as your knowledge
% of TeX increases, you will probably want to edit this, but
% it should be fine as is for beginners.

% almost certainly you want these
\usepackage{amssymb}
\usepackage{amsmath}
\usepackage{amsfonts}

% used for TeXing text within eps files
%\usepackage{psfrag}
% need this for including graphics (\includegraphics)
%\usepackage{graphicx}
% for neatly defining theorems and propositions
%\usepackage{amsthm}
% making logically defined graphics
%%%\usepackage{xypic}

% there are many more packages, add them here as you need them

% define commands here

\begin{document}
Let $G$ be a locally compact topological group. In general, the Banach *-algebra $L^1(G)$ (\PMlinkname{parent entry}{L1GIsABanachAlgebra}) does not have an identity element. In fact:

{\bf \PMlinkescapetext{Proposition} -} $L^1(G)$ has an identity element if and only if $G$ is discrete.

When $G$ is discrete the identity element of $L^1(G)$ is just the Dirac delta, i.e. the function that takes the value $1$ on the identity element of $G$ and vanishes everywhere else.

Nevertheless, $L^1(G)$ has always an approximate identity.

{\bf Theorem -} $L^1(G)$ has an approximate identity $(e_{\lambda})_{\lambda \in \Lambda}$. Moreover the approximate identity $(e_{\lambda})_{\lambda \in \Lambda}$ can be chosen to \PMlinkescapetext{satisfy} the following \PMlinkescapetext{properties}:
\begin{itemize}
\item $e_{\lambda}$ is \PMlinkname{self-adjoint}{InvolutaryRing},
\item $\|e_\lambda\|_1 = 1$,
\item $e_{\lambda} \in C_c(G)$
\end{itemize}
where $C_c(G)$ stands for the space of continuous functions $G \longrightarrow \mathbb{C}$ with compact support.
%%%%%
%%%%%
\end{document}
