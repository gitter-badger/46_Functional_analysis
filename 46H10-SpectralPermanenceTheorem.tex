\documentclass[12pt]{article}
\usepackage{pmmeta}
\pmcanonicalname{SpectralPermanenceTheorem}
\pmcreated{2013-03-22 17:29:50}
\pmmodified{2013-03-22 17:29:50}
\pmowner{asteroid}{17536}
\pmmodifier{asteroid}{17536}
\pmtitle{spectral permanence theorem}
\pmrecord{5}{39885}
\pmprivacy{1}
\pmauthor{asteroid}{17536}
\pmtype{Theorem}
\pmcomment{trigger rebuild}
\pmclassification{msc}{46H10}
\pmclassification{msc}{46H05}

% this is the default PlanetMath preamble.  as your knowledge
% of TeX increases, you will probably want to edit this, but
% it should be fine as is for beginners.

% almost certainly you want these
\usepackage{amssymb}
\usepackage{amsmath}
\usepackage{amsfonts}

% used for TeXing text within eps files
%\usepackage{psfrag}
% need this for including graphics (\includegraphics)
%\usepackage{graphicx}
% for neatly defining theorems and propositions
%\usepackage{amsthm}
% making logically defined graphics
%%%\usepackage{xypic}

% there are many more packages, add them here as you need them

% define commands here

\begin{document}
Let $\mathcal{A}$ be a unital complex Banach algebra and $\mathcal{B} \subseteq \mathcal{A}$ a Banach subalgebra that contains the identity of $\mathcal{A}$.

For every element $x \in \mathcal{B}$ it makes sense to speak of the spectrum $\sigma_{\mathcal{B}}(x)$ of $x$ relative to  $\mathcal{B}$ as well as the spectrum $\sigma_{\mathcal{A}}(x)$ of $x$ relative to $\mathcal{A}$.

We provide here three results of increasing sophistication which relate both these spectrums, $\sigma_{\mathcal{B}}(x)$ and $\sigma_{\mathcal{A}}(x)$. Any of the last two is usually refered to as the {\bf spectral permanence theorem}.

{\bf \PMlinkescapetext{Proposition} -} Let $\mathcal{B} \subseteq \mathcal{A}$ be as above. For every element $x \in \mathcal{B}$ we have
\begin{displaymath}
\sigma_{\mathcal{A}}(x) \subseteq \sigma_{\mathcal{B}}(x).
\end{displaymath}

This first result is purely \PMlinkescapetext{algebraic}. It is a straightforward consequence of the fact that invertible elements in $\mathcal{B}$ are also invertible in $\mathcal{A}$.

The other inclusion, $\sigma_{\mathcal{B}}(x) \subseteq \sigma_{\mathcal{A}}(x)$, is not necessarily true. It is true, however, if one considers the boundary $\partial \sigma_{\mathcal{B}}(x)$ instead.

{\bf Theorem -} Let $\mathcal{B} \subseteq \mathcal{A}$ be as above. For every element $x \in \mathcal{B}$ we have
\begin{displaymath}
\partial \sigma_{\mathcal{B}}(x) \subseteq \sigma_{\mathcal{A}}(x).
\end{displaymath}

Since the spectrum is a non-empty compact set in $\mathbb{C}$, one can decompose $\mathbb{C} - \sigma_{\mathcal{A}}(x)$ into its connected components, obtaining an unbounded component $\Omega_{\infty}$ together with a sequence of bounded components $\Omega_1, \Omega_2, \dots$,
\begin{displaymath}
\mathbb{C}-\sigma_{\mathcal{A}}(x) = \Omega_{\infty} \cup \Omega_{1} \cup \Omega_{2} \cup \cdots
\end{displaymath}
Of course there may be only a finite number of bounded components or \PMlinkescapetext{even} none.

{\bf Theorem -} Let $ x \in \mathcal{B} \subseteq \mathcal{A}$ be as above. Then $\sigma_{\mathcal{B}}(x)$ is obtained from $\sigma_{\mathcal{A}}(x)$ by adjoining to it some (possibly none) bounded components of $\mathbb{C}-\sigma_{\mathcal{A}}(x)$.

As an example, if $\sigma_{\mathcal{A}}(x)$ is the unit circle, then $\sigma_{\mathcal{B}}(x)$ can only possibly be the unit circle or the closed unit disk.
%%%%%
%%%%%
\end{document}
