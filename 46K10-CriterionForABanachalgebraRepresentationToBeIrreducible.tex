\documentclass[12pt]{article}
\usepackage{pmmeta}
\pmcanonicalname{CriterionForABanachalgebraRepresentationToBeIrreducible}
\pmcreated{2013-03-22 17:27:43}
\pmmodified{2013-03-22 17:27:43}
\pmowner{asteroid}{17536}
\pmmodifier{asteroid}{17536}
\pmtitle{criterion for a Banach *-algebra representation to be irreducible}
\pmrecord{9}{39845}
\pmprivacy{1}
\pmauthor{asteroid}{17536}
\pmtype{Theorem}
\pmcomment{trigger rebuild}
\pmclassification{msc}{46K10}

\endmetadata

% this is the default PlanetMath preamble.  as your knowledge
% of TeX increases, you will probably want to edit this, but
% it should be fine as is for beginners.

% almost certainly you want these
\usepackage{amssymb}
\usepackage{amsmath}
\usepackage{amsfonts}

% used for TeXing text within eps files
%\usepackage{psfrag}
% need this for including graphics (\includegraphics)
%\usepackage{graphicx}
% for neatly defining theorems and propositions
%\usepackage{amsthm}
% making logically defined graphics
%%%\usepackage{xypic}

% there are many more packages, add them here as you need them

% define commands here

\begin{document}
\PMlinkescapeword{irreducible}
\PMlinkescapeword{representation}

{\bf Theorem -} Let $\mathcal{A}$ be a Banach *-algebra, $H$ an Hilbert space and $I$ the identity operator in $H$. A \PMlinkname{representation}{BanachAlgebraRepresentation} $\pi : \mathcal{A} \longrightarrow H$ is topologically irreducible if and only if $\pi(\mathcal{A})' = \mathbb{C}I$, i.e. if and only if the commutant of $\pi(\mathcal{A})$ consists of scalar multiples of the identity operator.

{\bf Proof :} $(\Longrightarrow)$

As $\pi(\mathcal{A})$ is selfadjoint, $\pi(\mathcal{A})'$ is a von Neumann algebra.

Suppose $\pi(\mathcal{A})' \neq \mathbb{C}I$. Then the dimension of $\pi(\mathcal{A})'$ is greater than one.

It is known that von Neumann algebras of dimension greater than one contain non-trivial projections, so there is a projection $P \in \pi(\mathcal{A})'$ such that $P \neq 0$ and $P \neq I$.

As $P \in \pi(\mathcal{A})'$, $P$ commutes with every operator $T \in \pi(\mathcal{A})$, that is $PT=TP$.

Thus $Ran \; P$ is an invariant subspace of every $T \in \pi(\mathcal{A})$. Therefore $\pi$ is not an irreducible representation.

$(\Longleftarrow)$

Conversely, suppose that $\pi$ is not an irreducible representation. There exists a closed $\pi(\mathcal{A})$-invariant subspace different from $\{0\}$ and $H$.

Let $P$ be the projection onto that closed invariant subspace.

Invariance can be expressed as: $\pi(a)P = P \pi(a) P$ for every $a \in \mathcal{A}$. It follows that
\begin{displaymath}
P\pi(a) = (\pi(a)^*P)^*=(\pi(a^*)P)^*=(P\pi(a^*)P)^*=P\pi(a^*)^*P=P\pi(a)P=\pi(a)P
\end{displaymath}
for every $a \in \mathcal{A}$.

We conclude that $P$ commutes with every element of $\pi(\mathcal{A})$, i.e. $P \in \pi(\mathcal{A})'$.

Thus $\pi(\mathcal{A})' \neq \mathbb{C}I \;\; \square$ 
%%%%%
%%%%%
\end{document}
