\documentclass[12pt]{article}
\usepackage{pmmeta}
\pmcanonicalname{GelfandMazurTheorem}
\pmcreated{2013-03-22 17:29:03}
\pmmodified{2013-03-22 17:29:03}
\pmowner{asteroid}{17536}
\pmmodifier{asteroid}{17536}
\pmtitle{Gelfand-Mazur theorem}
\pmrecord{7}{39871}
\pmprivacy{1}
\pmauthor{asteroid}{17536}
\pmtype{Theorem}
\pmcomment{trigger rebuild}
\pmclassification{msc}{46H05}

\endmetadata

% this is the default PlanetMath preamble.  as your knowledge
% of TeX increases, you will probably want to edit this, but
% it should be fine as is for beginners.

% almost certainly you want these
\usepackage{amssymb}
\usepackage{amsmath}
\usepackage{amsfonts}

% used for TeXing text within eps files
%\usepackage{psfrag}
% need this for including graphics (\includegraphics)
%\usepackage{graphicx}
% for neatly defining theorems and propositions
%\usepackage{amsthm}
% making logically defined graphics
%%%\usepackage{xypic}

% there are many more packages, add them here as you need them

% define commands here

\begin{document}
{\bf Theorem -} Let $\mathcal{A}$ be a unital Banach algebra over $\mathbb{C}$ that is also a division algebra (i.e. every non-zero element is invertible). Then $\mathcal{A}$ is isometrically isomorphic to $\mathbb{C}$.

{\bf Proof :} Let $e$ denote the unit of $\mathcal{A}$.

Let $x \in \mathcal{A}$ and $\sigma(x)$ be its spectrum. It is known that the \PMlinkname{spectrum is a non-empty set}{SpectrumIsANonEmptyCompactSet} in $\mathbb{C}$.

Let $\lambda \in \sigma(x)$. Since $x-\lambda e$ is not invertible and $\mathcal{A}$ is a division algebra, we must have $x-\lambda e = 0$ and so $x=\lambda e$

Let $\phi : \mathbb{C} \longrightarrow \mathcal{A}$ be defined by $\phi(\lambda)=\lambda e$.

It is clear that $\phi$ is an injective algebra homomorphism.

By the above discussion, $\phi$ is also surjective.

It is isometric because $\|\lambda e\| = |\lambda| \|e\| = |\lambda|$

Therefore, $\mathcal{A}$ is isometrically isomorphic to $\mathbb{C}$. $\square$
%%%%%
%%%%%
\end{document}
