\documentclass[12pt]{article}
\usepackage{pmmeta}
\pmcanonicalname{VectorPnorm}
\pmcreated{2013-03-22 11:43:03}
\pmmodified{2013-03-22 11:43:03}
\pmowner{Andrea Ambrosio}{7332}
\pmmodifier{Andrea Ambrosio}{7332}
\pmtitle{vector p-norm}
\pmrecord{21}{30092}
\pmprivacy{1}
\pmauthor{Andrea Ambrosio}{7332}
\pmtype{Definition}
\pmcomment{trigger rebuild}
\pmclassification{msc}{46B20}
\pmclassification{msc}{05Cxx}
\pmclassification{msc}{05-01}
\pmclassification{msc}{20H15}
\pmclassification{msc}{20B30}
\pmsynonym{Minkowski norm}{VectorPnorm}
\pmsynonym{Euclidean vector norm}{VectorPnorm}
\pmsynonym{vector Euclidean norm}{VectorPnorm}
\pmsynonym{vector 1-norm}{VectorPnorm}
\pmsynonym{vector 2-norm}{VectorPnorm}
\pmsynonym{vector infinity-norm}{VectorPnorm}
\pmsynonym{L^p metric}{VectorPnorm}
\pmsynonym{L^p}{VectorPnorm}
\pmrelated{VectorNorm}
\pmrelated{CauchySchwartzInequality}
\pmrelated{HolderInequality}
\pmrelated{FrobeniusMatrixNorm}
\pmrelated{LpSpace}
\pmrelated{CauchySchwarzInequality}
\pmdefines{Manhattan metric}
\pmdefines{Taxicab}
\pmdefines{L^1 norm}
\pmdefines{L^1 metric}
\pmdefines{L^2 metric}
\pmdefines{L^2 norm}
\pmdefines{L^\infty norm}

\usepackage{graphicx}
%%%%%%%%%%%%\usepackage{xypic} 
\usepackage{bbm}
\newcommand{\Z}{\mathbbmss{Z}}
\newcommand{\C}{\mathbbmss{C}}
\newcommand{\R}{\mathbbmss{R}}
\newcommand{\Q}{\mathbbmss{Q}}
\newcommand{\mathbb}[1]{\mathbbmss{#1}}
\newcommand{\figura}[1]{\begin{center}\includegraphics{#1}\end{center}}
\newcommand{\figuraex}[2]{\begin{center}\includegraphics[#2]{#1}\end{center}}
\begin{document}
A class of vector norms, called a $p$-norm and denoted $||\cdot||_p$, is defined as

\begin{displaymath}
    ||\,x\,||_p = (|x_1|^p + \cdots + |x_n|^p)^\frac{1}{p}\qquad p\geq1, x\in\R^n
\end{displaymath}

The most widely used are the 1-norm, 2-norm, and $\infty$-norm:

\begin{eqnarray*}
    ||\,x\,||_1 & =& |x_1| + \cdots + |x_n| \\
    ||\,x\,||_2 & =& \sqrt{|x_1|^2 + \cdots + |x_n|^2} = \sqrt{x^Tx} \\
    ||\,x\,||_\infty & =& \displaystyle\max_{1\leq i\leq n}|x_i|
\end{eqnarray*}

The 2-norm is sometimes called the Euclidean vector norm, because
$||\,x-y\,||_2$ yields the Euclidean distance between any two vectors $x,y\in \R^n$. The 1-norm is also called the taxicab metric (sometimes Manhattan metric) since the distance of two points can be viewed as the distance a taxi would travel on a city (horizontal and vertical movements).

A useful fact is that for finite dimensional spaces (like $\R^n$) the three mentioned norms are \PMlinkid{equivalent}{4312}. Moreover, all $p$-norms are equivalent. This can be proved using that any norm  has to be continuous in the $2$-norm and working in the unit circle.

The \PMlinkname{$L^p$-norm}{LpSpace} in function spaces is a generalization of these norms by using counting measure.
%%%%%
%%%%%
%%%%%
%%%%%
%%%%%
%%%%%
%%%%%
%%%%%
%%%%%
%%%%%
%%%%%
\end{document}
