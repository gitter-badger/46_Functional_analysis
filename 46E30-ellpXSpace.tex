\documentclass[12pt]{article}
\usepackage{pmmeta}
\pmcanonicalname{ellpXSpace}
\pmcreated{2013-03-22 17:55:59}
\pmmodified{2013-03-22 17:55:59}
\pmowner{asteroid}{17536}
\pmmodifier{asteroid}{17536}
\pmtitle{$\ell^p(X)$ space}
\pmrecord{10}{40428}
\pmprivacy{1}
\pmauthor{asteroid}{17536}
\pmtype{Definition}
\pmcomment{trigger rebuild}
\pmclassification{msc}{46E30}
\pmclassification{msc}{46B26}
\pmclassification{msc}{28B15}
\pmsynonym{$\ell^p(X)$}{ellpXSpace}
\pmsynonym{$\ell^p(X)$-space}{ellpXSpace}
\pmrelated{Lp}
\pmrelated{ClassificationOfHilbertSpaces}
\pmrelated{RieszFischerTheorem}
\pmdefines{$\ell^2(X)$}
\pmdefines{$\ell^2(X)$ space}
\pmdefines{$\ell^p(X)$ is nonseparable iff $X$ is uncountable}
\pmdefines{orthonormal basis of $\ell^2(X)$ have the cardinality of $X$}

% this is the default PlanetMath preamble.  as your knowledge
% of TeX increases, you will probably want to edit this, but
% it should be fine as is for beginners.

% almost certainly you want these
\usepackage{amssymb}
\usepackage{amsmath}
\usepackage{amsfonts}

% used for TeXing text within eps files
%\usepackage{psfrag}
% need this for including graphics (\includegraphics)
%\usepackage{graphicx}
% for neatly defining theorems and propositions
%\usepackage{amsthm}
% making logically defined graphics
%%%\usepackage{xypic}

% there are many more packages, add them here as you need them

% define commands here

\begin{document}
\PMlinkescapephrase{property}

\subsubsection{Definition of $\ell^p(X)$}

Let $p$ be a real number such that $1 \leq p < \infty$.

Let $X$ be a set and let $\mu$ be the counting measure on $X$, defined on the \PMlinkname{$\sigma$-algebra}{SigmaAlgebra} $\mathfrak{B}$ of all subsets of $X$. The $\ell^p(X)$ space is a particular \PMlinkescapetext{type} of a \PMlinkname{$L^p$-space}{LpSpace}, defined as

\begin{displaymath}
\ell^p(X) := L^p(X, \mathfrak{B}, \mu)
\end{displaymath}

Thus, the $\ell^p(X)$ space consists of all functions $f:X \longrightarrow \mathbb{C}$ such that

\begin{displaymath}
\sum_{x \in X} |f(x)|^p < \infty
\end{displaymath}

Of course, for the above sum to be finite one must necessarily have $f(x) \neq 0$ only for a countable number of $x \in X$ (see \PMlinkname{this entry}{SupportOfIntegrableFunctionWithRespectToCountingMeasureIsCountable}).

\subsubsection{Properties}
\begin{itemize}
\item By the corresponding property on $L^p$-spaces, the space $\ell^p(X)$ is a Banach space and its norm amounts to
\begin{displaymath}
\|f\|_p = \left ( \sum_{x \in X} |f(x)|^p \right )^{\frac{1}{p}}
\end{displaymath}
\end{itemize}
\begin{itemize}
\item By the corresponding property on \PMlinkname{$L^2$-spaces}{L2SpacesAreHilbertSpaces}, the space $\ell^2(X)$ is a Hilbert space and its inner product amounts to
\begin{displaymath}
\langle f, g\rangle = \sum_{x \in X} f(x)\overline{g(x)}
\end{displaymath}
\end{itemize}

\subsubsection{Nonseparability of $\ell^p(X)$ for uncountable $X$}

\emph{\PMlinkescapetext{Proposition}} - The space $\ell^p(X)$ is separable if and only if $X$ is a countable set. Moreover, $\ell^p(X)$ admits a Schauder basis if and only if $X$ is countable.

$\,$

A Schauder basis for $\ell^p(X)$, when it exists, can be just the set of functions $\{\delta_{x_0}: x_0 \in X\}$ defined by
\begin{displaymath}
\delta_{x_0}(x) :=
\begin{cases}
1, & $if$\;\; x=x_0\\
0 & $if$\;\; x \neq x_0
\end{cases}
\end{displaymath}

\subsubsection{Orthonormal basis of $\ell^2(X)$}

The set of functions $\{\delta_{x_0}: x_0 \in X\}$ 
is an orthonormal basis of $\ell^2(X)$. Hence, the \PMlinkname{dimension}{OrthonormalBasis} of $\ell^2(X)$ is given by the cardinality of $X$ (as all orthonormal bases have the same cardinality).

It can be shown that all Hilbert spaces are isometrically isomorphic (hence, preserving the inner product) to a $\ell^2(X)$ space, for a suitable set $X$.
%%%%%
%%%%%
\end{document}
