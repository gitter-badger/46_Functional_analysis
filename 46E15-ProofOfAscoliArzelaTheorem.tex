\documentclass[12pt]{article}
\usepackage{pmmeta}
\pmcanonicalname{ProofOfAscoliArzelaTheorem}
\pmcreated{2013-03-22 13:16:19}
\pmmodified{2013-03-22 13:16:19}
\pmowner{paolini}{1187}
\pmmodifier{paolini}{1187}
\pmtitle{proof of Ascoli-Arzel\`a theorem}
\pmrecord{12}{33753}
\pmprivacy{1}
\pmauthor{paolini}{1187}
\pmtype{Proof}
\pmcomment{trigger rebuild}
\pmclassification{msc}{46E15}

\endmetadata

% this is the default PlanetMath preamble.as yourknowledge
% of TeX increases, you will probably want to edit this, but
% it should be fine as is for beginners.

% almost certainly you want these
\usepackage{amssymb}
\usepackage{amsmath}
\usepackage{amsfonts}

% used for TeXing text within eps files
%\usepackage{psfrag}
% need this for including graphics (\includegraphics)
%\usepackage{graphicx}
% for neatly defining theorems and propositions
%\usepackage{amsthm}
% making logically defined graphics
%%%\usepackage{xypic}

% there are many more packages, add them here as you need 
% define commands here

%\newtheorem{theorem}
\begin{document}
Given $\epsilon>0$ we aim at finding a $4\epsilon$-net in $F$ i.e. a finite set of points $F_\epsilon$ such that 
\[
  \bigcup_{f\in F_\epsilon} B_{4\epsilon}(f)\supset F
\]
(see the definition of totally bounded).
Let $\delta>0$ be given with respect to $\epsilon$ in the definition
of equi-continuity (see uniformly equicontinuous) of $F$.
Let $X_\delta$ be a $\delta$-lattice in
$X$ and $Y_\epsilon$ be a  $\epsilon$-lattice in $Y$. 
Let now $Y_\epsilon^{X_\delta}$ be the set of functions from
$X_\delta$ 
to $Y_\epsilon$ and define
$G_\epsilon\subset Y_\epsilon^{X_\delta}$ by
\[
  G_\epsilon = \{ g\in Y_\epsilon^{X_\delta}\colon 
	\exists f\in F\ \forall x\in X_\delta\quad d(f(x),g(x))<\epsilon\}.
\]
Since $Y_\epsilon^{X_\delta}$ is a finite set, 
$G_\epsilon$ is finite too: say $G_\epsilon=\{g_1,\ldots, g_N\}$.
Then define $F_\epsilon\subset F$, $F_\epsilon=\{f_1,\ldots, f_N\}$ 
where $f_k\colon X\to Y$ is a function in $F$ such that 
$d(f_k(x),g_k(x))<\epsilon$ for all $x\in X_\delta$ (the existence of
such a function is guaranteed by the definition of $G_\epsilon$).

We now will prove that $F_\epsilon$ is a $4\epsilon$-lattice in $F$.
Given $f\in F$ choose $g\in
{Y_\epsilon}^{X_\delta}$ such that for all $x\in X_\delta$ it holds
$d(f(x),g(x)) < \epsilon$ (this is possible as for all
$x\in X_\delta$ there exists $y\in Y_\epsilon$ with $d(f(x),y)<\epsilon$). 
We conclude that $g\in G_\epsilon$ and hence $g=g_k$ for some
$k\in\{1,\ldots, N\}$. Notice also that for all $x\in X_\delta$ we
have
 $d(f(x),f_k(x))\le d(f(x),g_k(x)) + d(g_k(x),f_k(x)) < 2\epsilon$.

Given any $x\in X$ we know that there exists
$x_\delta\in X_\delta$ such that $d(x,x_\delta)<\delta$. 
So, by equicontinuity of $F$, 
\[
  d(f(x),f_k(x)) \le
	d(f(x),f(x_\delta)) 
	+ d(f_k(x),f_k(x_\delta)) + d(f(x_\delta),f_k(x_\delta))
	< 4\epsilon.
\]
%%%%%
%%%%%
\end{document}
