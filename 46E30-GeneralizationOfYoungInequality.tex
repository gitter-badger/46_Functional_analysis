\documentclass[12pt]{article}
\usepackage{pmmeta}
\pmcanonicalname{GeneralizationOfYoungInequality}
\pmcreated{2013-03-22 15:43:08}
\pmmodified{2013-03-22 15:43:08}
\pmowner{Andrea Ambrosio}{7332}
\pmmodifier{Andrea Ambrosio}{7332}
\pmtitle{generalization of Young inequality}
\pmrecord{25}{37667}
\pmprivacy{1}
\pmauthor{Andrea Ambrosio}{7332}
\pmtype{Result}
\pmcomment{trigger rebuild}
\pmclassification{msc}{46E30}

% this is the default PlanetMath preamble.  as your knowledge
% of TeX increases, you will probably want to edit this, but
% it should be fine as is for beginners.

% almost certainly you want these
\usepackage{amssymb}
\usepackage{amsmath}
\usepackage{amsfonts}

% used for TeXing text within eps files
%\usepackage{psfrag}
% need this for including graphics (\includegraphics)
%\usepackage{graphicx}
% for neatly defining theorems and propositions
%\usepackage{amsthm}
% making logically defined graphics
%%%\usepackage{xypic}

% there are many more packages, add them here as you need them

% define commands here
\begin{document}
It's straightforward to extend \PMlinkname{Young inequality}{YoungInequality}
to an arbitrary finite number of \PMlinkescapetext{terms}: provided that $%
a_{i}>0$, $c_{i}>0$ and $\sum_{i=1}^{n}\frac{1}{c_{i}}=\frac{1}{r}$, 
\[
\left( \prod_{i=1}^{n}a_{i}\right) ^{r}\leq r\sum_{i=1}^{n}\frac{%
a_{i}^{c_{i}}}{c_{i}}
\]%
In fact, 
\begin{eqnarray*}
\left( \prod_{i=1}^{n}a_{i}\right) ^{r} &=&\exp \left[ \log \left(
\prod_{i=1}^{n}a_{i}\right) ^{r}\right]  \\
&=&\exp \left[ r\sum_{i=1}^{n}\log a_{i}\right]  \\
&=&\exp \left[ r\sum_{i=1}^{n}\frac{1}{c_{i}}\log \left( a_{i}^{c_{i}}\right)
\right]  \\
&=&\exp \left[ \frac{\sum_{i=1}^{n}\frac{1}{c_{i}}\log \left( a_{i}^{c_{i}}\right)
}{\frac{1}{r}}\right]  \\
\text{(by Jensen's inequality and monotonicity of exp)} &\leq &\exp \left[ \log \left( \frac{%
\sum_{i=1}^{n}\frac{1}{c_{i}}a_{i}^{c_{i}}}{\frac{1}{r}}\right) \right]  \\
&&=r\sum_{i=1}^{n}\frac{a_{i}^{c_{i}}}{c_{i}}
\end{eqnarray*}

\bigskip 

Remark: in the case 
\[
\frac{1}{c_{i}}=1\text{ \ \ \ \ }\forall i
\]
one obtains:%
\[
\left( \prod_{i=1}^{n}a_{i}\right) ^{\frac{1}{n}}\leq \frac{1}{n}%
\sum_{i=1}^{n}a_{i}
\]
that is, the usual arithmetic-geometric mean inequality, which suggests
Young inequality could be regarded as a generalization of this classical result.
Actually, let's consider the following restatement of
Young inequality. Having defined:
$w_{i}=\frac{1}{c_{i}}$, \ $\ \ \sum_{i=1}^{n}w_{i}=W=\frac{1}{r}$, 
$\ \ \ x_{i}=a_{i}^{\frac{1}{w_{i}}}$
we have:
\[
\left( \prod_{i=1}^{n}x_{i}^{w_{i}}\right) ^{\frac{1}{W}}\leq \frac{1}{W}%
\sum_{i=1}^{n}w_{i}x_{i}
\]
This expression shows that Young inequality is nothing else than
geometric-arithmentic \textit{weighted} \PMlinkname{mean}{ArithmeticMean} inequality.
%%%%%
%%%%%
\end{document}
