\documentclass[12pt]{article}
\usepackage{pmmeta}
\pmcanonicalname{BanachSpacesOfInfiniteDimensionDoNotHaveACountableHamelBasis}
\pmcreated{2013-03-22 14:59:12}
\pmmodified{2013-03-22 14:59:12}
\pmowner{yark}{2760}
\pmmodifier{yark}{2760}
\pmtitle{Banach spaces of infinite dimension do not have a countable Hamel basis}
\pmrecord{21}{36691}
\pmprivacy{1}
\pmauthor{yark}{2760}
\pmtype{Result}
\pmcomment{trigger rebuild}
\pmclassification{msc}{46B15}

\endmetadata

\usepackage{amssymb}
\usepackage{amsmath}
\usepackage{amsfonts}

% The below lines should work as the command
% \renewcommand{\bibname}{References}
% without creating havoc when rendering an entry in 
% the page-image mode.
\makeatletter
\@ifundefined{bibname}{}{\renewcommand{\bibname}{References}}
\makeatother
\begin{document}
\PMlinkescapeword{algebraic}
\PMlinkescapeword{basis}
\PMlinkescapeword{even}
\PMlinkescapeword{examples}
\PMlinkescapeword{mean}

A Banach space of infinite dimension does not have a countable Hamel basis.

\textbf{Proof}

Let $E$ be such space, and suppose it does have a countable Hamel basis, say $B = (v_{k})_{k \in \mathbb{N}}$. 

Then, by definition of Hamel basis and linear combination, we have that $x \in E$ if and only if $x = \lambda_1 \cdot v_1 + \dots + \lambda_n \cdot v_n$ for some $n \in \mathbb{N}$. Consequently,
$$
E = \bigcup \limits_{i=1}^\infty {(\operatorname{span}(v_j)_{j=1}^i)}.
$$
This would mean that $E$ is a countable union of proper subspaces of finite dimension (they are proper because $E$ has infinite dimension), but every finite dimensional proper subspace of a normed space is nowhere dense, and then $E$ would be first category. This is absurd, by the Baire Category Theorem.

\textbf{Note}

In fact, the Hamel dimension of an infinite-dimensional Banach space
is always at least the cardinality of the continuum
(even if the Continuum Hypothesis fails).
A one-page proof of this has been given by H.\ Elton Lacey\cite{hel}.

\textbf{Examples}

Consider the set of all real-valued infinite sequences $(x_n)$
such that $x_n=0$ for all but finitely many $n$.

This is a vector space, with the known operations. Morover, it has infinite dimension: a possible basis is $(e_k)_{k \in \mathbb{N}}$, where 
$$
e_i(n)=\begin{cases}
1, & \text{if }n=i\\
0, & \text{otherwise}.
\end{cases}
$$
So, it has infinite dimension and a countable Hamel basis.
Using our result, it follows directly that there is no way to define a norm in this vector space such that it is a complete metric space under the induced metric.

\begin{thebibliography}{9}
\bibitem{hel}
 H.\ Elton Lacey,
 {\it The Hamel Dimension of any Infinite Dimensional Separable Banach Space is c},
 Amer.\ Math.\ Mon.\ 80 (1973), 298.
\end{thebibliography}
%%%%%
%%%%%
\end{document}
