\documentclass[12pt]{article}
\usepackage{pmmeta}
\pmcanonicalname{SchursTest}
\pmcreated{2013-03-22 19:01:19}
\pmmodified{2013-03-22 19:01:19}
\pmowner{karstenb}{16623}
\pmmodifier{karstenb}{16623}
\pmtitle{Schur's Test}
\pmrecord{6}{41893}
\pmprivacy{1}
\pmauthor{karstenb}{16623}
\pmtype{Theorem}
\pmcomment{trigger rebuild}
\pmclassification{msc}{46G99}
%\pmkeywords{Lp Inequality}
%\pmkeywords{Bounded Operator}
%\pmkeywords{Kernel}

\endmetadata

\usepackage{amssymb}
\usepackage{amsmath}
\usepackage{amsfonts}
\usepackage{amsthm}
\usepackage{mathrsfs}
\usepackage[sort&compress]{natbib}

%\usepackage{psfrag}
%\usepackage{graphicx}
%%%\usepackage{xypic}

%theorems
\theoremstyle{definition}
\newtheorem{Def}{Definition}

\theoremstyle{plain}
\newtheorem{Lem}{Theorem}
\newtheorem{Lem2}{Lemma}
\newtheorem{Cor}{Corollary}
\newtheorem{Rem}{Remark}

\begin{document}
\begin{Lem}
\textbf{(Schur's Test)} Let $(X, \mu)$ be a measure space ($\mu$ a positive measure). Let $K$ be a positive, measurable function on $X \times X$. Define the operator
\begin{align*}
T f(x) &:= \int_X K(x,y) f(y) \,d\mu(y), x \in X
\end{align*}

If for some $1 < p < \infty$ there exists a measurable, strictly positive function $h$ and a constant $M > 0$ such that
\begin{align*}
&\int_X K(x,y) h(y)^q \,d\mu(y) \leq M h(x)^q \\
&\int_X K(x,y) h(x)^p \,d\mu(x) \leq M h(y)^p
\end{align*}

with $p^{-1} + q^{-1} = 1$, then $||T|| \leq M$ in $L^p(X, d\mu)$. 
\end{Lem}


\begin{proof}
Let $f \in L^p(X, d\mu)$. We have
\begin{align*}
|T f(x)| &\leq \int_{X} h(y) h(y)^{-1} |f(y)| K(x,y) \,d\mu(y) 
\end{align*}

hence by Hoelder's inequality
\begin{align*}
|T f(x)| &\leq \left[\int_X K(x,y) h(y)^q \,d\mu(y) \right]^{\frac{1}{q}} \left[\int_X K(x,y) h(y)^{-p} |f(y)|^p \,d\mu(y)\right]^{\frac{1}{p}}
\end{align*}

By the first inequality in the assumption we have
\begin{align*}
|T f(x)| &\leq M^{\frac{1}{q}} h(x) \left[\int_X K(x,y) h(y)^{-p} |f(y)|^p \,d\mu(y) \right]^{\frac{1}{p}}
\end{align*}

Evaluating $||T f||_p^p$ by Fubini and the second inequality in the assumption we obtain
\begin{align*}
\int_X |T f(x)|^p \,d\mu(x) &\leq M^p \int_X |f(y)|^p \,d\mu(y)
\end{align*}

This completes the proof. 
\end{proof}


A noted special case is Young's Inequality

\begin{Cor}
\textbf{(Young)}

Let $K \colon \mathbb{R}^n \times \mathbb{R}^n \to \mathbb{C}$ be Borel-measurable such that there is a constant $C > 0$ with
\begin{align*}
& \sup_{x \in \mathbb{R}^n} \int_{\mathbb{R}^n} |K(x,y)| \,d\lambda^n(y) \leq C \\
& \sup_{y \in \mathbb{R}^n} \int_{\mathbb{R}^n} |K(x,y)| \,d\lambda^n(x) \leq C
\end{align*}

For $f \in L^p(\mathbb{R}^n)$ ($1 \leq p \leq +\infty$) define
\begin{align*}
T(f)(x) &:= \int_{\mathbb{R}^n} K(x,y) f(y)\,d\lambda^n(y)
\end{align*}

Then $||T f||_p \leq C ||f||_p$. 
\end{Cor}

\begin{thebibliography}{99}
\bibitem[Hedenmalm 2000]{hedenmalm} H. Hedenmalm, Boris Korenblum, Kehe Zhu \emph{Theory of Bergman spaces}, Springer Verlag, New York, 2000
\end{thebibliography}

%%%%%
%%%%%
\end{document}
