\documentclass[12pt]{article}
\usepackage{pmmeta}
\pmcanonicalname{CalgebrasHaveApproximateIdentities}
\pmcreated{2013-03-22 17:30:40}
\pmmodified{2013-03-22 17:30:40}
\pmowner{asteroid}{17536}
\pmmodifier{asteroid}{17536}
\pmtitle{$C^*$-algebras have approximate identities}
\pmrecord{4}{39901}
\pmprivacy{1}
\pmauthor{asteroid}{17536}
\pmtype{Theorem}
\pmcomment{trigger rebuild}
\pmclassification{msc}{46L05}

\endmetadata

% this is the default PlanetMath preamble.  as your knowledge
% of TeX increases, you will probably want to edit this, but
% it should be fine as is for beginners.

% almost certainly you want these
\usepackage{amssymb}
\usepackage{amsmath}
\usepackage{amsfonts}

% used for TeXing text within eps files
%\usepackage{psfrag}
% need this for including graphics (\includegraphics)
%\usepackage{graphicx}
% for neatly defining theorems and propositions
%\usepackage{amsthm}
% making logically defined graphics
%%%\usepackage{xypic}

% there are many more packages, add them here as you need them

% define commands here

\begin{document}
In this entry $\leq$ has three different meanings:
\begin{enumerate}
\item - The \PMlinkname{ordering of self-adjoint elements}{OrderingOfSelfAdjoints} of a given \PMlinkname{$C^*$-algebra}{CAlgebra}.
\item - The usual \PMlinkname{order}{PartialOrder} in $\mathbb{R}$.
\item - The \PMlinkescapetext{order} of a directed set taken as the domain of a given net.
\end{enumerate}
It will be clear from the context which one is being used.

{\bf Theorem -} Every $C^*$-algebra has an approximate identity $(e_{\lambda})_{\lambda \in \Lambda}$. Moreover, the approximate identity $(e_{\lambda})_{\lambda \in \Lambda}$ can be chosen to \PMlinkescapetext{satisfy} the following \PMlinkescapetext{properties}:
\begin{itemize}
\item $0\leq e_{\lambda}\;\;\;\;\forall_{\lambda \in \Lambda}$
\item $\|e_{\lambda}\| \leq 1\;\;\;\;\forall_{\lambda \in \Lambda}$
\item $\lambda \leq \mu\; \Rightarrow \;e_{\lambda}\leq e_{\mu}$, i.e. $(e_{\lambda})_{\lambda \in \Lambda}$ is increasing.
\end{itemize}

For \PMlinkname{separable}{Separable} $C^*$-algebras the approximate identity can be chosen as an increasing sequence $0\leq e_1 \leq e_2 \leq \dots$ of norm-one elements.
%%%%%
%%%%%
\end{document}
