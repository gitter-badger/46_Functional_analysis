\documentclass[12pt]{article}
\usepackage{pmmeta}
\pmcanonicalname{NormedAlgebra}
\pmcreated{2013-03-22 16:11:38}
\pmmodified{2013-03-22 16:11:38}
\pmowner{CWoo}{3771}
\pmmodifier{CWoo}{3771}
\pmtitle{normed algebra}
\pmrecord{13}{38286}
\pmprivacy{1}
\pmauthor{CWoo}{3771}
\pmtype{Definition}
\pmcomment{trigger rebuild}
\pmclassification{msc}{46H05}
\pmrelated{GelfandTornheimTheorem}
\pmrelated{SuperfieldsSuperspace}
\pmdefines{normed ring}
\pmdefines{topological algebra}
\pmdefines{real normed algebra}
\pmdefines{complex normed algebra}

\usepackage{amssymb,amscd}
\usepackage{amsmath}
\usepackage{amsfonts}

% used for TeXing text within eps files
%\usepackage{psfrag}
% need this for including graphics (\includegraphics)
%\usepackage{graphicx}
% for neatly defining theorems and propositions
%\usepackage{amsthm}
% making logically defined graphics
%%\usepackage{xypic}
\usepackage{pst-plot}
\usepackage{psfrag}

% define commands here

\begin{document}
A ring $A$ is said to be a \emph{normed ring} if $A$ possesses a norm $\| \cdot \|$, that is, a non-negative real-valued function $\|\cdot \|:A\to \mathbb{R}$ such that for any $a,b\in A$,
\begin{enumerate}
\item $\|a\|=0$ iff $a=0$,
\item $\|a+b\|\le \|a\|+\|b\|$, 
\item $\|-a\|=\|a\|$, and
\item $\|ab\|\le \|a\|\|b\|$.
\end{enumerate}

\textbf{Remarks}.  
\begin{itemize}
\item If $A$ contains the multiplicative identity $1$, then $0<\|1\| \le\|1\|\|1\|$ and so $1\le \|1\|$.  
\item However, it is usually required that in a normed ring, $\|1\|=1$.
\item $\|\cdot\|$ defines a metric $d$ on $A$ given by $d(a,b)=\|a-b\|$, so that $A$ with $d$ is a metric space and one can set up a topology on $A$ by defining its subbasis a collection of $B(a,r):=\lbrace x\in A\mid d(a,x)< r\rbrace$ called \emph{open balls} for any $a\in A$ and $r>0$.  With this definition, it is easy to see that $\|\cdot\|$ is continuous.
\item Given a sequence $\lbrace a_n\rbrace$ of elements in $A$, we say that $a$ is a limit point of $\lbrace a_n\rbrace$, if $$\lim_{n\to\infty}\|a_n-a\|=0.$$  By the triangle inequality, $a$, if it exists, is unique, and so we also write $$a=\lim_{n\to\infty}a_n.$$
\item In addition, the last condition ensures that the ring multiplication is continuous.
\end{itemize}

An algebra $A$ over a field $k$ is said to be a \emph{normed algebra} if 
\begin{enumerate}
\item $A$ is a normed ring with norm $\|\cdot\|$,
\item $k$ is equipped with a valuation $| \cdot |$, and
\item $\|\alpha a\|=|\alpha|\|a\|$ for any $\alpha \in k$ and $a\in A$.
\end{enumerate}

\textbf{Remarks}.  
\begin{itemize}
\item Alternatively, a normed algebra $A$ can be defined as a normed vector space with a multiplication defined on $A$ such that multiplication is continuous with respect to the norm $\|\cdot\|$.
\item Typically, $k$ is either the reals $\mathbb{R}$ or the complex numbers $\mathbb{C}$, and $A$ is called a \emph{real normed algebra} or a \emph{complex normed algebra} correspondingly.
\item A normed algebra that is complete with respect to the norm is called Banach algebra (the underlying field must be complete and algebraically closed), paralleling with the analogy with a Banach space versus a normed vector space.
\item Normed rings and normed algebras are special cases of the more general notions of a topological ring and a \emph{topological algebra}, the latter of which is defined as a topological ring over a field such that the scalar multiplication is continuous.
\end{itemize}

\begin{thebibliography}{7}
\bibitem{nm} M. A. Naimark: {\em Normed Rings}, Noordhoff, (1959).
\bibitem{cr} C. E. Rickart: {\em General Theory of Banach Algebras}, Van Nostrand, 1960.
\end{thebibliography}
%%%%%
%%%%%
\end{document}
