\documentclass[12pt]{article}
\usepackage{pmmeta}
\pmcanonicalname{BalancedSet}
\pmcreated{2013-03-22 15:33:16}
\pmmodified{2013-03-22 15:33:16}
\pmowner{matte}{1858}
\pmmodifier{matte}{1858}
\pmtitle{balanced set}
\pmrecord{5}{37453}
\pmprivacy{1}
\pmauthor{matte}{1858}
\pmtype{Definition}
\pmcomment{trigger rebuild}
\pmclassification{msc}{46-00}
\pmrelated{AbsorbingSet}
\pmdefines{balanced subset}
\pmdefines{balanced hull}
\pmdefines{balanced evelope}
\pmdefines{circled}
\pmdefines{\'equilibr\'e}

\endmetadata

% this is the default PlanetMath preamble.  as your knowledge
% of TeX increases, you will probably want to edit this, but
% it should be fine as is for beginners.

% almost certainly you want these
\usepackage{amssymb}
\usepackage{amsmath}
\usepackage{amsfonts}
\usepackage{amsthm}

\usepackage{mathrsfs}

% used for TeXing text within eps files
%\usepackage{psfrag}
% need this for including graphics (\includegraphics)
%\usepackage{graphicx}
% for neatly defining theorems and propositions
%
% making logically defined graphics
%%%\usepackage{xypic}

% there are many more packages, add them here as you need them

% define commands here

\newcommand{\sR}[0]{\mathbb{R}}
\newcommand{\sC}[0]{\mathbb{C}}
\newcommand{\sN}[0]{\mathbb{N}}
\newcommand{\sZ}[0]{\mathbb{Z}}

 \usepackage{bbm}
 \newcommand{\Z}{\mathbbmss{Z}}
 \newcommand{\C}{\mathbbmss{C}}
 \newcommand{\F}{\mathbbmss{F}}
 \newcommand{\R}{\mathbbmss{R}}
 \newcommand{\Q}{\mathbbmss{Q}}



\newcommand*{\norm}[1]{\lVert #1 \rVert}
\newcommand*{\abs}[1]{| #1 |}



\newtheorem{thm}{Theorem}
\newtheorem{defn}{Definition}
\newtheorem{prop}{Proposition}
\newtheorem{lemma}{Lemma}
\newtheorem{cor}{Corollary}
\begin{document}
\PMlinkescapeword{proposition}
\PMlinkescapeword{terms}
\PMlinkescapeword{term}

{\bf Definition} \cite{rudin_fap,edwards, horvath, cristescu} 
Let $V$ be a vector space over $\sR$ (or $\sC$),
and let $S$ be a subset of $V$. If $\lambda S\subset S$ for all scalars $\lambda$ such
that $|\lambda|\le 1$, then $S$ is a {\bf balanced set} in $V$.
The {\bf balanced hull} of $S$,
denoted by $\operatorname{eq}(S)$, is the smallest 
balanced set containing $S$. 

In the above, 
 $\lambda S = \{ \lambda s\mid s\in S\}$, 
and $|\cdot|$ is the absolute value (in $\sR$),
or the modulus of a complex number (in $\sC$).

\subsubsection{Examples and properties}
\begin{enumerate}
\item Let $V$ be a normed space with norm $||\cdot||$. Then the unit ball
$\{v\in V\mid ||v||\le 1\}$ is a balanced set.
\item Any vector subspace is a balanced set. Thus, in $\sR^3$, lines and planes passing
through the origin are balanced sets.
\end{enumerate}

\subsubsection{Notes}
A balanced set is also sometimes called {\bf circled} \cite{horvath}.
The term {\bf balanced evelope} is also used for the balanced hull \cite{edwards}.
Bourbaki uses the term {\bf \'equilibr\'e} \cite{edwards}, c.f. $\operatorname{eq}(A)$ 
above. In \cite{reed}, a balanced set is defined as above, but with the condition $|\lambda|=1$ instead of $|\lambda|\le 1$. 


 \begin{thebibliography}{9}
 \bibitem{rudin_fap}
 W. Rudin, \emph{Functional Analysis},
McGraw-Hill Book Company, 1973.
\bibitem{edwards} R.E. Edwards, \emph{Functional Analysis: Theory and Applications},
 Dover Publications, 1995.
\bibitem{horvath} J. Horv\'ath, \emph{Topological Vector Spaces and Distributions},
Addison-Wsley Publishing Company, 1966.
 \bibitem{cristescu} R. Cristescu, \emph{Topological vector spaces},
 Noordhoff International Publishing, 1977.
\bibitem{reed} M. Reed, B. Simon,
 \emph{Methods of Modern Mathematical Physics: Functional Analysis I},
 Revised and enlarged edition, Academic Press, 1980.
 
 \end{thebibliography}
%%%%%
%%%%%
\end{document}
