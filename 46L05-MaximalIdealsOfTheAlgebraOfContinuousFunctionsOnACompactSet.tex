\documentclass[12pt]{article}
\usepackage{pmmeta}
\pmcanonicalname{MaximalIdealsOfTheAlgebraOfContinuousFunctionsOnACompactSet}
\pmcreated{2013-03-22 17:44:57}
\pmmodified{2013-03-22 17:44:57}
\pmowner{asteroid}{17536}
\pmmodifier{asteroid}{17536}
\pmtitle{maximal ideals of the algebra of continuous functions on a compact set}
\pmrecord{5}{40200}
\pmprivacy{1}
\pmauthor{asteroid}{17536}
\pmtype{Theorem}
\pmcomment{trigger rebuild}
\pmclassification{msc}{46L05}
\pmclassification{msc}{46J20}
\pmclassification{msc}{46J10}
\pmclassification{msc}{16W80}
\pmsynonym{character space of the algebra of continuous functions on a compact set}{MaximalIdealsOfTheAlgebraOfContinuousFunctionsOnACompactSet}

% this is the default PlanetMath preamble.  as your knowledge
% of TeX increases, you will probably want to edit this, but
% it should be fine as is for beginners.

% almost certainly you want these
\usepackage{amssymb}
\usepackage{amsmath}
\usepackage{amsfonts}

% used for TeXing text within eps files
%\usepackage{psfrag}
% need this for including graphics (\includegraphics)
%\usepackage{graphicx}
% for neatly defining theorems and propositions
%\usepackage{amsthm}
% making logically defined graphics
%%%\usepackage{xypic}

% there are many more packages, add them here as you need them

% define commands here

\begin{document}
Let $X$ be a compact Hausdorff space and $C(X)$ the Banach algebra of continuous functions $X \longrightarrow \mathbb{C}$ (with the sup norm).

In this entry we are interested in identifying the maximal ideals and the character space of $C(X)$. Since $C(X)$ is a Banach algebra with an identity element, there is a bijective correspondence between the character space of $C(X)$ and the set of maximal ideals of this algebra, given by
\begin{displaymath}
\phi \longleftrightarrow Ker\; \phi
\end{displaymath}
Hence, by identifying the character space of $C(X)$ we are able to identify its maximal ideals.\\

$\;$

{\bf Theorem 1 -} Let $\Delta$ be the character space of $C(X)$. For each $x \in X$ let $ev_x \in \Delta$ be the point-evaluation at $x$, i.e.
\begin{displaymath}
ev_x(f)=f(x)\;, \qquad\qquad\;f \in C(X)
\end{displaymath}
Then the mapping $x \longmapsto ev_x$ is an homeomorphism between $\Delta$ and $X$.\\

$\;$

Thus, the character space of $C(X)$ is homeomorphic to $X$ via point-evaluations.

Now, the maximal ideals of $C(X)$ correspond to the kernels of the point-evaluation functions. The kernel of $ev_x$, the point-evaluation at $x$, is just
\begin{align*}
\{f \in C(X) : f(x) = 0 \}
\end{align*}
i.e., the functions that vanish at $x$.

Thus, each maximal ideal of $C(X)$ is just the set of functions that vanish in a given point.

\subsection{Generalization to locally compact Hausdorff spaces}
Now, let $X$ be a locally compact Hausdorff space and $C_0(X)$ the space of continuous functions $X \longrightarrow \mathbb{C}$ that vanish at infinity.

There is a generalization of Theorem 1 above that allows one to identify the character space of $C_0(X)$, but since this algebra is not unital unless $X$ is compact, we cannot identify its maximal ideals by the above method.\\

$\;$

{\bf Theorem 2-} Let $\Delta$ be the character space of $C_0(X)$. For each $x \in X$ let $ev_x \in \Delta$ be the point-evaluation at $x$, i.e.
\begin{displaymath}
ev_x(f)=f(x)\;, \qquad\qquad\;f \in C_0(X)
\end{displaymath}
Then the mapping $x \longmapsto ev_x$ is an homeomorphism between $\Delta$ and $X$.\\

$\;$

Thus, the character space of $C_0(X)$ is also homeomorphic to $X$ via point-evaluations.
%%%%%
%%%%%
\end{document}
