\documentclass[12pt]{article}
\usepackage{pmmeta}
\pmcanonicalname{ProofOfBanachAlaogluTheorem}
\pmcreated{2013-03-22 15:10:03}
\pmmodified{2013-03-22 15:10:03}
\pmowner{Mathprof}{13753}
\pmmodifier{Mathprof}{13753}
\pmtitle{proof  of Banach-Alaoglu theorem}
\pmrecord{12}{36917}
\pmprivacy{1}
\pmauthor{Mathprof}{13753}
\pmtype{Proof}
\pmcomment{trigger rebuild}
\pmclassification{msc}{46B10}

\endmetadata

% this is the default PlanetMath preamble.  as your knowledge
% of TeX increases, you will probably want to edit this, but
% it should be fine as is for beginners.

% almost certainly you want these
\usepackage{amssymb}
\usepackage{amsmath}
\usepackage{amsfonts}

% used for TeXing text within eps files
%\usepackage{psfrag}
% need this for including graphics (\includegraphics)
%\usepackage{graphicx}
% for neatly defining theorems and propositions
%\usepackage{amsthm}
% making logically defined graphics
%%%\usepackage{xypic}

% there are many more packages, add them here as you need them

% define commands here
\begin{document}
For any $x\in X$, let $D_x=\{z\in\mathbb{C}: |z|\leq \|x\|\}$ and $D=\Pi_{x\in X} D_x$. Since $D_x$ is a compact subset of $\mathbb{C}$, $D$ is compact in product topology by Tychonoff theorem.

We prove the theorem by finding a homeomorphism that maps the closed unit ball 
$B_{X^*}$ of $X^*$ onto a closed subset of $D$. Define $\Phi_x:B_{X^*}\to D_x$ by 
$\Phi_x(f)=f(x)$ and $\Phi:B_{X^*}\to D$ by $\Phi=\Pi_{x\in X}\Phi_x$, so that
$\Phi(f)=(f(x))_{x\in X}$. Obviously, $\Phi$ is one-to-one, and a net $(f_\alpha)$ in $B_{X^*}$ converges to $f$ in weak-* topology of $X^*$ iff $\Phi(f_\alpha)$ converges to $\Phi(f)$ in product topology, therefore $\Phi$ is continuous and so is its inverse $\Phi^{-1}:\Phi(B_{X^*})\to B_{X^*}$.

It remains to show that $\Phi(B_{X^*})$ is closed. If $(\Phi(f_\alpha))$ is a net
in $\Phi(B_{X^*})$, converging to a point $d=(d_x)_{x\in X}\in D$, we can define a function
$f:X\to \mathbb{C}$ by $f(x)=d_x$. As $\lim_\alpha \Phi(f_\alpha(x))=d_x$ for all $x\in X$ by definition of weak-* convergence, one can easily see that $f$ is a linear functional in $B_{X^*}$ and that $\Phi(f)=d$. This shows that $d$ is actually in $\Phi(B_{X^*})$ and finishes the proof.
%%%%%
%%%%%
\end{document}
