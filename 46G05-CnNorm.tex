\documentclass[12pt]{article}
\usepackage{pmmeta}
\pmcanonicalname{CnNorm}
\pmcreated{2013-03-22 14:59:46}
\pmmodified{2013-03-22 14:59:46}
\pmowner{rspuzio}{6075}
\pmmodifier{rspuzio}{6075}
\pmtitle{$C^n$ norm}
\pmrecord{9}{36702}
\pmprivacy{1}
\pmauthor{rspuzio}{6075}
\pmtype{Definition}
\pmcomment{trigger rebuild}
\pmclassification{msc}{46G05}
\pmclassification{msc}{26B05}
\pmclassification{msc}{26Axx}
\pmclassification{msc}{26A24}
\pmclassification{msc}{26A15}

\endmetadata

% this is the default PlanetMath preamble.  as your knowledge
% of TeX increases, you will probably want to edit this, but
% it should be fine as is for beginners.

% almost certainly you want these
\usepackage{amssymb}
\usepackage{amsmath}
\usepackage{amsfonts}

% used for TeXing text within eps files
%\usepackage{psfrag}
% need this for including graphics (\includegraphics)
%\usepackage{graphicx}
% for neatly defining theorems and propositions
%\usepackage{amsthm}
% making logically defined graphics
%%%\usepackage{xypic}

% there are many more packages, add them here as you need them

% define commands here
\begin{document}
One can define an extended norm on the space $C^n (I)$ where $I$ is a subset of $\mathbb{R}$ as follows:
 $$\| f \|_{C^n} = \sup_{x \in I} \sup_{k \le n} \left| \frac{d^k f}{dx^k} \right|$$
If $f$ is a function of more than one variable (i.e. lies in $C^n (D)$ for a subset $D \in \mathbb{R}^m$), then one needs to take the supremum over all partial derivatives of order up to $n$.  

That $$\| \cdot \|_{C^n}$$ satisfies the defining conditions for an extended norm follows trivially from the properties of the absolute value (positivity, homogeneity, and the triangle inequality) and the inequality
 $$\sup (|f| + |g|) < \sup |f| + \sup |g|.$$

If we are considering functions defined on the whole of $\mathbb{R}^m$ or an unbounded subset of $\mathbb{R}^m$, the $C^n$ norm may be infinite.  For example,
 $$\| e^x \|_{C^n} = \infty$$
for all $n$ because the $n$-th derivative of $e^x$ is again $e^x$, which blows up as $x$ approaches infinity.  If we are considering functions on a compact (closed and bounded) subset of $\mathbb{R}^m$ however, the $C^n$ norm is always finite as a consequence of the fact that every continuous function on a compact set attains a maximum.  This also means that we may replace the ``$\sup$'' with a ``$\max$'' in our definition in this case.

Having a sequence of functions converge under this norm is the same as having their $n$-th derivatives converge uniformly.  Therefore, it follows from the fact that the uniform limit of continuous functions is continuous that $C^n$ is complete under this norm. (In other words, it is a Banach space.)

In the case of $C^{\infty}$, there is no natural way to impose a norm, so instead one uses all the $C^n$ norms to define the topology in $C^\infty$.  One does this by declaring that a subset of $C^\infty$ is closed if it is closed in all the $C^n$ norms.  A space like this whose topology is defined by an infinite collection of norms is known as a multi-normed space.
%%%%%
%%%%%
\end{document}
