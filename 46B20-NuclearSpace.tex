\documentclass[12pt]{article}
\usepackage{pmmeta}
\pmcanonicalname{NuclearSpace}
\pmcreated{2013-03-22 16:37:28}
\pmmodified{2013-03-22 16:37:28}
\pmowner{Simone}{5904}
\pmmodifier{Simone}{5904}
\pmtitle{nuclear space}
\pmrecord{6}{38823}
\pmprivacy{1}
\pmauthor{Simone}{5904}
\pmtype{Definition}
\pmcomment{trigger rebuild}
\pmclassification{msc}{46B20}
%\pmkeywords{Fr\'echet space}
%\pmkeywords{Banach space}
%\pmkeywords{semi-norm}

\endmetadata

% this is the default PlanetMath preamble.  as your knowledge
% of TeX increases, you will probably want to edit this, but
% it should be fine as is for beginners.

% almost certainly you want these
\usepackage{amssymb}
\usepackage{amsmath}
\usepackage{amsfonts}

% used for TeXing text within eps files
%\usepackage{psfrag}
% need this for including graphics (\includegraphics)
%\usepackage{graphicx}
% for neatly defining theorems and propositions
%\usepackage{amsthm}
% making logically defined graphics
%%%\usepackage{xypic}

% there are many more packages, add them here as you need them

% define commands here

\begin{document}
If $E$ is a Fréchet space and $(p_j)$ an increasing sequence of semi-norms on $E$ defining the topology of $E$, we have
$$
E=\underset{\longleftarrow}{\lim}\,\widehat E_{p_j},
$$
where $\widehat E_{p_j}$ is the Hausdorff completion of $(E,p_j)$ and $\widehat E_{p_{j+1}}\to\widehat E_{p_j}$ the canonical morphism. Here $\widehat E_{p_j}$ is a Banach space for the induced norm $\widehat p_j$.

A Fréchet space $E$ is said to be \emph{nuclear} if the topology of $E$ can be defined by an increasing sequence of semi-norms $p_j$ such that each canonical morphism $\widehat E_{p_{j+1}}\to\widehat E_{p_j}$ of Banach spaces is nuclear.

Recall that a morphism $f\colon E\to F$ of complete locally convex spaces is said to be nuclear if $f$ can be written as
$$
f(x)=\sum\lambda_j\xi_j(x)y_j
$$
where $(\lambda_j)$ is a sequence of scalars with $\sum|\lambda_j|<+\infty$,$\xi_j\in E'$ an equicontinuous sequence of linear forms and $y_j\in F$ a bounded sequence.
%%%%%
%%%%%
\end{document}
