\documentclass[12pt]{article}
\usepackage{pmmeta}
\pmcanonicalname{ProjectionsAndClosedSubspaces}
\pmcreated{2013-03-22 17:52:57}
\pmmodified{2013-03-22 17:52:57}
\pmowner{asteroid}{17536}
\pmmodifier{asteroid}{17536}
\pmtitle{projections and closed subspaces}
\pmrecord{5}{40364}
\pmprivacy{1}
\pmauthor{asteroid}{17536}
\pmtype{Theorem}
\pmcomment{trigger rebuild}
\pmclassification{msc}{46C07}
\pmclassification{msc}{46B20}
\pmsynonym{projection along a closed subspace}{ProjectionsAndClosedSubspaces}
\pmsynonym{orthogonal projections onto Hilbert subspaces}{ProjectionsAndClosedSubspaces}

\endmetadata

% this is the default PlanetMath preamble.  as your knowledge
% of TeX increases, you will probably want to edit this, but
% it should be fine as is for beginners.

% almost certainly you want these
\usepackage{amssymb}
\usepackage{amsmath}
\usepackage{amsfonts}

% used for TeXing text within eps files
%\usepackage{psfrag}
% need this for including graphics (\includegraphics)
%\usepackage{graphicx}
% for neatly defining theorems and propositions
%\usepackage{amsthm}
% making logically defined graphics
%%%\usepackage{xypic}

% there are many more packages, add them here as you need them

% define commands here

\begin{document}
{\bf Theorem  1 -} Let $X$ be a Banach space and $M$ a closed subspace. Then,
\begin{itemize}
\item $M$ is topologically complemented in $X$ if and only if there exists a continuous projection onto $M$.
\end{itemize}
\begin{itemize}
\item Given a topological complement $N$ of $M$, there exists a unique continuous projection $P$ onto $M$ such that $P(x+y)=x$ for all $x \in M$ and $y \in N$.
\end{itemize}


$\;$

The projection $P$ in the second part of the above theorem is sometimes called the \emph{projection onto $M$ along $N$}.

The above result can be further improved for Hilbert spaces. 


$\,$


{\bf Theorem 2 - } Let $X$ be a Hilbert space and $M$ a closed subspace. Then, $M$ is topologically complemented in $X$ if and only if there exists an orthogonal projection onto $M$ (which is unique).

$\,$

Since, by the orthogonal decomposition theorem, a closed subspace of a Hilbert space is always topologically complemented by its orthogonal complement ($X=M \oplus M^{\perp}$), it follows that

$\,$

{\bf Corollary -} Let $X$ be a Hilbert space and $M$ a closed subspace. Then, there exists a unique orthogonal projection onto $M$. This establishes a bijective correspondence between orthogonal projections and closed subspaces.
%%%%%
%%%%%
\end{document}
