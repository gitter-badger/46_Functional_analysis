\documentclass[12pt]{article}
\usepackage{pmmeta}
\pmcanonicalname{OrthonormalBasis}
\pmcreated{2013-03-22 14:02:29}
\pmmodified{2013-03-22 14:02:29}
\pmowner{yark}{2760}
\pmmodifier{yark}{2760}
\pmtitle{orthonormal basis}
\pmrecord{19}{35346}
\pmprivacy{1}
\pmauthor{yark}{2760}
\pmtype{Definition}
\pmcomment{trigger rebuild}
\pmclassification{msc}{46C05}
\pmsynonym{Hilbert basis}{OrthonormalBasis}
\pmrelated{RieszSequence}
\pmrelated{Orthonormal}
\pmrelated{ClassificationOfHilbertSpaces}
\pmdefines{dimension of a Hilbert space}
\pmdefines{dimension}

\def\ip#1{{\langle #1\rangle}}

\begin{document}
\PMlinkescapeword{even}
\PMlinkescapeword{finite}
\PMlinkescapeword{terms}
\PMlinkescapeword{properties}

\section*{Definition}

An \emph{orthonormal basis} (or \emph{Hilbert basis})
of an inner product space $V$
is a subset $B$ of $V$ satisfying the following two properties:
\begin{itemize}
\item $B$ is an orthonormal set.
\item The linear span of $B$ is dense in $V$.
\end{itemize}

The first condition means that all elements of $B$ have norm $1$
and every element of $B$ is \PMlinkname{orthogonal}{OrthogonalVectors} to every other element of $B$.
The second condition says that every element of $V$ can be approximated arbitrarily closely by (finite) linear combinations of elements of $B$.

\section*{Orthonormal bases of Hilbert spaces}

Every Hilbert space has an orthonormal basis.
The cardinality of this orthonormal basis
is called the \emph{dimension} of the Hilbert space.
(This is well-defined,
as the cardinality does not depend on the choice of orthonormal basis.
This dimension is not in general the same as
\PMlinkname{the usual concept of dimension for vector spaces}{Dimension2}.)

If $B$ is an orthonormal basis of a Hilbert space $H$,
then for every $x\in H$ we have
\[
  x=\sum_{b\in B}\ip{x,b}b.
\]
Thus $x$ is expressed as a (possibly infinite)
``linear combination'' of elements of $B$.
The expression is well-defined,
because only countably many of the terms $\ip{x,b}b$ are non-zero
(even if $B$ itself is uncountable),
and if there are infinitely many non-zero terms
the series is unconditionally convergent.
For any $x,y\in H$ we also have
\[
  \ip{x,y}=\sum_{b\in B}\ip{x,b}\ip{b,y}.
\]

%%%%%
%%%%%
\end{document}
