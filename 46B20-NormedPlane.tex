\documentclass[12pt]{article}
\usepackage{pmmeta}
\pmcanonicalname{NormedPlane}
\pmcreated{2013-03-22 16:50:30}
\pmmodified{2013-03-22 16:50:30}
\pmowner{Mathprof}{13753}
\pmmodifier{Mathprof}{13753}
\pmtitle{normed plane}
\pmrecord{6}{39086}
\pmprivacy{1}
\pmauthor{Mathprof}{13753}
\pmtype{Definition}
\pmcomment{trigger rebuild}
\pmclassification{msc}{46B20}
\pmdefines{Minkowski plane}
\pmdefines{Minkowski geometry}

\endmetadata

% this is the default PlanetMath preamble.  as your knowledge
% of TeX increases, you will probably want to edit this, but
% it should be fine as is for beginners.

% almost certainly you want these
\usepackage{amssymb}
\usepackage{amsmath}
\usepackage{amsfonts}

% used for TeXing text within eps files
%\usepackage{psfrag}
% need this for including graphics (\includegraphics)
%\usepackage{graphicx}
% for neatly defining theorems and propositions
%\usepackage{amsthm}
% making logically defined graphics
%%%\usepackage{xypic}

% there are many more packages, add them here as you need them

% define commands here

\begin{document}
A \emph{normed plane} is a pair $(\mathbb{R}^2, ||\cdot||)$, where the function 
$x \to ||x||$ is a norm.

If we define a distance function $d(x,y) = ||x-y||$ then 
the metric space  $(\mathbb{R}^2, d)$ is called a \emph{Minkowski plane} or
a \emph{Minkowski geometry}.


The classical examples of Minkowski and normed planes are
the $p$-norm $||x||_p = (|x_1|^p + |x_2|^p)^{1/p}$ where
$1 \leq p < \infty$ and the maximum or supremum norm
$||x||_{\infty} = \max\{|x_1| , |x_2|\}$.

%%%%%
%%%%%
\end{document}
