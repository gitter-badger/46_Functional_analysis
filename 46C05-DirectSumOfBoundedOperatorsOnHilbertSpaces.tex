\documentclass[12pt]{article}
\usepackage{pmmeta}
\pmcanonicalname{DirectSumOfBoundedOperatorsOnHilbertSpaces}
\pmcreated{2013-03-22 18:00:32}
\pmmodified{2013-03-22 18:00:32}
\pmowner{asteroid}{17536}
\pmmodifier{asteroid}{17536}
\pmtitle{direct sum of bounded operators on Hilbert spaces}
\pmrecord{7}{40523}
\pmprivacy{1}
\pmauthor{asteroid}{17536}
\pmtype{Definition}
\pmcomment{trigger rebuild}
\pmclassification{msc}{46C05}
\pmclassification{msc}{47A05}
%\pmkeywords{uniformly bounded family of operators}
%\pmkeywords{direct sum of operators}

% this is the default PlanetMath preamble.  as your knowledge
% of TeX increases, you will probably want to edit this, but
% it should be fine as is for beginners.

% almost certainly you want these
\usepackage{amssymb}
\usepackage{amsmath}
\usepackage{amsfonts}

% used for TeXing text within eps files
%\usepackage{psfrag}
% need this for including graphics (\includegraphics)
%\usepackage{graphicx}
% for neatly defining theorems and propositions
%\usepackage{amsthm}
% making logically defined graphics
%%%\usepackage{xypic}

% there are many more packages, add them here as you need them

% define commands here

\begin{document}
\PMlinkescapephrase{direct sum}

\subsection{Definition}
Let $\{ H_i\}_{i \in I}$ be a family of Hilbert spaces indexed by a set $I$. For each $i \in I$ let $T_i:H_i \longrightarrow H_i$ be a bounded linear operator on $H_i$ such that the family $\{T_i\}_{i \in I}$ of bounded linear operators is uniformly bounded, i.e. $\sup\,\{\|T_i\|: i \in I\} < \infty$.

{\bf Definition -} The \emph{direct sum} of the uniformly bounded family $\{T_i\}_{i \in I}$ is the operator
\begin{align*}
\bigoplus_{i \in I} T_i : \bigoplus_{i \in I} H_i \longrightarrow \bigoplus_{i \in I} H_i
\end{align*}
on the direct sum of Hilbert spaces $\bigoplus_{i \in I} H_i$ defined by
\begin{align*}
\left( \bigoplus_{i \in I} T_i \;(x)\right)_i := T_i x_i
\end{align*}

It can be seen that $\bigoplus_{i \in I} T_i$ is well-defined and is in fact a bounded linear operator, whose norm is
\begin{align*}
\left\|\bigoplus_{i \in I} T_i \right\| = \sup\,\{\|T_i\| : i \in I\}
\end{align*}

\subsection{Properties}
\begin{itemize}
\item $\displaystyle \bigoplus_{i \in I}  (aT_i + b S_i) = a \bigoplus_{i \in I} T_i + b\bigoplus_{i \in I} S_i$, where $a, b \in \mathbb{C}$.
\end{itemize}
\begin{itemize}
\item $\displaystyle \left(\bigoplus_{i \in I} T_i\right)^* = \bigoplus_{i \in I} T_i^*$.
\end{itemize}
\begin{itemize}
\item $\displaystyle \left(\bigoplus_{i \in I} T_i\right) \left(\bigoplus_{i \in I} S_i\right)  = \bigoplus_{i \in I} T_iS_i$.
\end{itemize}
%%%%%
%%%%%
\end{document}
