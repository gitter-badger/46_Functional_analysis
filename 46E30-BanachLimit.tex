\documentclass[12pt]{article}
\usepackage{pmmeta}
\pmcanonicalname{BanachLimit}
\pmcreated{2013-03-22 15:23:00}
\pmmodified{2013-03-22 15:23:00}
\pmowner{stevecheng}{10074}
\pmmodifier{stevecheng}{10074}
\pmtitle{Banach limit}
\pmrecord{7}{37213}
\pmprivacy{1}
\pmauthor{stevecheng}{10074}
\pmtype{Definition}
\pmcomment{trigger rebuild}
\pmclassification{msc}{46E30}
\pmclassification{msc}{40A05}
\pmrelated{AlmostConvergent}
\pmrelated{ConstructionOfBanachLimitUsingLimitAlongAnUltrafilter}
\pmrelated{ConstructionOfBanachLimitUsingLimitAlongAnUltrafilter2}

\endmetadata

% this is the default PlanetMath preamble.  as your knowledge
% of TeX increases, you will probably want to edit this, but
% it should be fine as is for beginners.

% almost certainly you want these
\usepackage{amssymb}
\usepackage{amsmath}
\usepackage{amsfonts}
\usepackage{enumerate}

% used for TeXing text within eps files
%\usepackage{psfrag}
% need this for including graphics (\includegraphics)
%\usepackage{graphicx}
% for neatly defining theorems and propositions
%\usepackage{amsthm}
% making logically defined graphics
%%%\usepackage{xypic}

% there are many more packages, add them here as you need them

% define commands here

\newcommand{\nat}{\mathbb{N}}
\newcommand{\complex}{\mathbb{C}}
\providecommand{\defnterm}[1]{\emph{#1}}
\newcommand{\elli}{\ell^\infty}
\providecommand{\norm}[1]{\lVert#1\rVert}
\providecommand{\normW}[1]{\left\lVert#1\right\rVert}
\providecommand{\normB}[1]{\Bigl\lVert#1\Bigr\rVert}
\DeclareMathOperator{\zRe}{Re}
\DeclareMathOperator{\zIm}{Im}
\begin{document}
Consider the set $c_0$ of all convergent complex-valued sequences $\{ x(n) \}_{n \in \nat}$.
The limit operation $x \mapsto \lim_{n \to \infty} x(n)$
is a linear functional on $c_0$, by the usual limit laws.
A \defnterm{Banach limit} is, loosely speaking, any linear functional that generalizes $\lim$ to apply to non-convergent sequences as well. The formal definition follows:

Let $\elli$ be the set of bounded complex-valued sequences $\{ x(n) \}_{n \in \nat}$, equipped
with the sup norm.
Then $c_0 \subset \elli$, and $\lim\colon c_0 \to \complex$ is a linear functional.
A \defnterm{Banach limit} is any continuous linear functional $\phi \in (\elli)^*$ satisfying:
\begin{enumerate}[i]
\item
$\phi(x) = \lim_n x(n)$ if $x \in c_0$ (That is, $\phi$ extends $\lim$.)
\item
$\norm{\phi} = 1$.
\item
$\phi (Sx) = \phi(x)$, where $S\colon \elli \to \elli$ is the shift operator defined by $Sx(n) = x(n+1)$. (Shift invariance)
\item
If $x(n) \geq 0$ for all $n$, then $\phi(x) \geq 0$. (Positivity)
\end{enumerate}

There is not necessarily a unique Banach limit. Indeed, Banach limits are often constructed
by extending $\lim$ with the Hahn-Banach theorem (which in turn invokes the Axiom of Choice).

Like the limit superior and limit inferior, the Banach limit can be applied for situations
where one wants to algebraically manipulate limit equations or inequalities, 
even when it is not assured beforehand
that the limits in question exist (in the classical sense).

\section{Some consequences of the definition}
The positivity condition ensures that the Banach limit of a real-valued sequence is real-valued,
and that limits can be compared: 
if $x \leq y$, then $\phi(x) \leq \phi(y)$.
In particular, given a real-valued sequence $x$, by comparison with the sequences
$y(n) = \inf_{k \geq n} x(k)$ and $z(n) = \sup_{k \geq n} x(k)$,
it follows that $\liminf_n x(n) \leq \phi(x) \leq \limsup_n x(n)$.

The shift invariance allows any finite number of terms of the sequence to be neglected when taking the Banach limit, as is possible with the classical limit.

On the other hand, $\phi$ can never be multiplicative, meaning that
$\phi(xy) = \phi(x) \phi(y)$ fails.
For a counter-example, set $x = (0, 1, 0, 1, \dotsc)$; then we would have
$\phi(0) = \phi(x \cdot Sx) = \phi(x) \phi(Sx) = \phi(x)^2$, so $\phi(x) = 0$,
but $1 = \phi(1) = \phi(x + Sx) = \phi(x) + \phi(Sx) = 2\phi(x) = 0$.

That $\phi$ is continuous means it is compatible with limits in $\elli$.
For example, suppose that $\{ x_k \}_{k \in \nat} \subset \ell^\infty$,
and that $\sum_{k=0}^\infty x_k$ is absolutely convergent in $\elli$.
(In other words, $\sum_{k=0}^\infty \norm{x_k}_\infty < \infty$.)
Then $\phi(\sum_{k=0}^\infty x_k) = \sum_{k=0}^\infty \phi(x_k)$
by continuity.
Observe that this is just the dominated convergence theorem, specialized
to the case of the counting measure on $\nat$, in disguise.

\section{Other definitions}
In some definitions of the Banach limit, condition (i) is replaced
by the seemingly weaker condition that $\phi(1) = 1$ --- 
the Banach limit of a constant sequence is that constant.
In fact, the latter condition together with shift invarance implies condition (i).

If we restrict to real-valued sequences,
condition (ii) is clearly redundant, in view of the other conditions.
%%%%%
%%%%%
\end{document}
