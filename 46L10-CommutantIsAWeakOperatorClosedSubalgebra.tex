\documentclass[12pt]{article}
\usepackage{pmmeta}
\pmcanonicalname{CommutantIsAWeakOperatorClosedSubalgebra}
\pmcreated{2013-03-22 18:39:32}
\pmmodified{2013-03-22 18:39:32}
\pmowner{asteroid}{17536}
\pmmodifier{asteroid}{17536}
\pmtitle{commutant is a weak operator closed subalgebra}
\pmrecord{7}{41402}
\pmprivacy{1}
\pmauthor{asteroid}{17536}
\pmtype{Theorem}
\pmcomment{trigger rebuild}
\pmclassification{msc}{46L10}

% this is the default PlanetMath preamble.  as your knowledge
% of TeX increases, you will probably want to edit this, but
% it should be fine as is for beginners.

% almost certainly you want these
\usepackage{amssymb}
\usepackage{amsmath}
\usepackage{amsfonts}

% used for TeXing text within eps files
%\usepackage{psfrag}
% need this for including graphics (\includegraphics)
%\usepackage{graphicx}
% for neatly defining theorems and propositions
%\usepackage{amsthm}
% making logically defined graphics
%%%\usepackage{xypic}

% there are many more packages, add them here as you need them

% define commands here

\begin{document}
Let $H$ be a Hilbert space and $B(H)$ the algebra of bounded operators in $H$. Recall that the commutant of a subset $\mathcal{F} \subset B(H)$ is the set of all bounded operators that commute with those of $\mathcal{F}$, i.e.

\begin{align*}
\mathcal{F}':=\{T \in B(H):\; TS=ST \,,\;\;\; \forall S \in \mathcal{F}\}.
\end{align*}

{\bf \PMlinkescapetext{Proposition} -} If $\mathcal{F} \subset B(H)$, then $\mathcal{F}'$ is a subalgebra of $B(H)$ that contains the identity operator and is closed in the weak operator topology.

{\bf \emph{\PMlinkescapetext{Proof}:}} It is clear that $\mathcal{F}'$ contains the identity operator, since it commutes with all operators in $B(H)$ and in particular with those of $\mathcal{F}$.

Let us now see that $\mathcal{F}'$ is a subalgebra of $B(H)$. Let $T_1, T_2 \in \mathcal{F}'$ and $\lambda \in \mathbb{C}$. We have that, for all $S \in \mathcal{F}$,

\begin{align*}
S(T_1+T_2)= ST_1 + ST_2 = T_1S + T_2S = (T_1+T_2)S\\
S(\lambda T_1) = \lambda ST_1 = \lambda T_1 S\\
S(T_1T_2) = T_1ST_2 = T_1T_2S
\end{align*}

thus, $T_1 +T_2$, $\lambda T_1$ and $T_1T_2$ all belong to $\mathcal{F}'$, and therefore $\mathcal{F}'$ is a subalgebra of $B(H)$.

It remains to see that $\mathcal{F}'$ is weak operator closed. Suppose $(T_i)$ is a net in $\mathcal{F}'$ that converges to $T$ in the weak operator topology. Then, for all $x, y \in H$ we have that $\langle T_ix, y \rangle \to \langle Tx,y\rangle$. Thus, for all $S \in \mathcal{F}$, we have

\begin{eqnarray*}
\langle (TS - ST)x, y \rangle & = & \langle TSx, y \rangle - \langle Tx, S^*y \rangle \\
& = & \lim \big( \langle T_iSx, y \rangle - \langle T_ix, S^*y \rangle \big)\\
& = & \lim \, \langle (T_iS - ST_i)x, y \rangle\\
& = & \lim\, \langle (T_iS - T_iS)x, y \rangle\\
& = & 0
\end{eqnarray*}

Hence, $TS-ST=0$, so that $T \in \mathcal{F}'$. We conclude that $\mathcal{F}'$ is closed in the weak operator topology. $\square$
%%%%%
%%%%%
\end{document}
