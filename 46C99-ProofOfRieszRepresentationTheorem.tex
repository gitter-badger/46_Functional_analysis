\documentclass[12pt]{article}
\usepackage{pmmeta}
\pmcanonicalname{ProofOfRieszRepresentationTheorem}
\pmcreated{2013-03-22 17:32:37}
\pmmodified{2013-03-22 17:32:37}
\pmowner{asteroid}{17536}
\pmmodifier{asteroid}{17536}
\pmtitle{proof of Riesz representation theorem}
\pmrecord{5}{39943}
\pmprivacy{1}
\pmauthor{asteroid}{17536}
\pmtype{Proof}
\pmcomment{trigger rebuild}
\pmclassification{msc}{46C99}

\endmetadata

% this is the default PlanetMath preamble.  as your knowledge
% of TeX increases, you will probably want to edit this, but
% it should be fine as is for beginners.

% almost certainly you want these
\usepackage{amssymb}
\usepackage{amsmath}
\usepackage{amsfonts}

% used for TeXing text within eps files
%\usepackage{psfrag}
% need this for including graphics (\includegraphics)
%\usepackage{graphicx}
% for neatly defining theorems and propositions
%\usepackage{amsthm}
% making logically defined graphics
%%%\usepackage{xypic}

% there are many more packages, add them here as you need them

% define commands here

\begin{document}
{\bf Existence -} If $f=0$ we can just take $u=0$ and thereby have $f(x) =0= \langle x, 0\rangle$ for all $x \in \mathcal{H}$.

Suppose now $f \neq 0$, i.e. $Ker f \neq \mathcal{H}$.

Recall that, since $f$ is \PMlinkname{continuous}{ContinuousMap}, $Ker f$ is a closed subspace of $\mathcal{H}$ (continuity of $f$ implies that $f^{-1}(0)$ is closed in $\mathcal{H}$). It then follows from the orthogonal decomposition theorem that
\begin{displaymath}
\mathcal{H} = Ker f \oplus (Ker f)^{\perp}
\end{displaymath}
and as $Ker f \neq \mathcal{H}$ we can find $z \in (Ker f)^{\perp}$ such that $\|z\| = 1$.

It follows easily from the linearity of $f$ that for every $x \in \mathcal{H}$ we have
\begin{displaymath}
f(x)z-f(z)x \in Ker f
\end{displaymath}
and since $z \in (Ker f)^{\perp}$
\begin{eqnarray*}
\quad\quad\quad\quad\quad\quad 0 & = & \langle f(x)z -f(z)x, z \rangle \\
& = & f(x)\langle z, z\rangle - f(z)\langle x, z\rangle \\
& = & f(x)\|z\|^2 -  \langle x, \overline{f(z)} z\rangle \\
& = & f(x) -  \langle x, \overline{f(z)} z\rangle
\end{eqnarray*}

which implies
\begin{displaymath}
f(x) = \langle x, \overline{f(z)} z\rangle \; .
\end{displaymath}

The theorem then follows by taking $u = \overline{f(z)} z$.

{\bf Uniqueness -} Suppose there were $u_1, u_2 \in \mathcal{H}$ such that for every $x \in \mathcal{H}$
\begin{displaymath}
f(x)= \langle x, u_1 \rangle = \langle x, u_2 \rangle .
\end{displaymath}

Then $\langle x, u_1 -u_2 \rangle = 0$ for every $x \in \mathcal{H}$. Taking $x = u_1 - u_2$ we obtain $\|u_1-u_2\|^2 = 0$, which implies $u_1 = u_2$. $\square$
%%%%%
%%%%%
\end{document}
