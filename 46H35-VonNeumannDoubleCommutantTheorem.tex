\documentclass[12pt]{article}
\usepackage{pmmeta}
\pmcanonicalname{VonNeumannDoubleCommutantTheorem}
\pmcreated{2013-03-22 18:40:27}
\pmmodified{2013-03-22 18:40:27}
\pmowner{asteroid}{17536}
\pmmodifier{asteroid}{17536}
\pmtitle{von Neumann double commutant theorem}
\pmrecord{4}{41421}
\pmprivacy{1}
\pmauthor{asteroid}{17536}
\pmtype{Theorem}
\pmcomment{trigger rebuild}
\pmclassification{msc}{46H35}
\pmclassification{msc}{46K05}
\pmclassification{msc}{46L10}
\pmsynonym{double commutant theorem}{VonNeumannDoubleCommutantTheorem}
\pmsynonym{bicommutant theorem}{VonNeumannDoubleCommutantTheorem}
\pmsynonym{von Neumann bicommutant theorem}{VonNeumannDoubleCommutantTheorem}
\pmsynonym{von Neumann density theorem}{VonNeumannDoubleCommutantTheorem}

% this is the default PlanetMath preamble.  as your knowledge
% of TeX increases, you will probably want to edit this, but
% it should be fine as is for beginners.

% almost certainly you want these
\usepackage{amssymb}
\usepackage{amsmath}
\usepackage{amsfonts}

% used for TeXing text within eps files
%\usepackage{psfrag}
% need this for including graphics (\includegraphics)
%\usepackage{graphicx}
% for neatly defining theorems and propositions
%\usepackage{amsthm}
% making logically defined graphics
%%%\usepackage{xypic}

% there are many more packages, add them here as you need them

% define commands here

\begin{document}
The von Neumann double commutant theorem is a remarkable result in the theory of self-adjoint algebras of operators on Hilbert spaces, as it expresses purely topological aspects of these algebras in terms of purely algebraic properties.

$\,$

{\bf Theorem - von Neumann \PMlinkescapetext{double commutant} -} Let $H$ be a \PMlinkname{Hilbert space}{HilbertSpace} and $B(H)$ its algebra of bounded operators. Let $\mathcal{M}$ be a *-subalgebra of $B(H)$ that contains the identity operator. The following statements are equivalent:

\begin{enumerate}
\item $\mathcal M = \mathcal M''$, i.e. $\mathcal M$ equals its double commutant.
\item $\mathcal M$ is closed in the weak operator topology.
\item $\mathcal M$ is closed in the strong operator topology.
\end{enumerate}

$\,$

Thus, a purely topological property of a $\mathcal{M}$, as being closed for some operator topology, is equivalent to a purely algebraic property, such as being equal to its double commutant. 

This result is also known as the \emph{bicommutant theorem} or the \emph{von Neumann density theorem}.
%%%%%
%%%%%
\end{document}
