\documentclass[12pt]{article}
\usepackage{pmmeta}
\pmcanonicalname{EquivalentNorms}
\pmcreated{2013-03-22 13:39:28}
\pmmodified{2013-03-22 13:39:28}
\pmowner{matte}{1858}
\pmmodifier{matte}{1858}
\pmtitle{equivalent norms}
\pmrecord{10}{34312}
\pmprivacy{1}
\pmauthor{matte}{1858}
\pmtype{Definition}
\pmcomment{trigger rebuild}
\pmclassification{msc}{46B99}

% this is the default PlanetMath preamble.  as your knowledge
% of TeX increases, you will probably want to edit this, but
% it should be fine as is for beginners.

% almost certainly you want these
\usepackage{amssymb}
\usepackage{amsmath}
\usepackage{amsfonts}

% used for TeXing text within eps files
%\usepackage{psfrag}
% need this for including graphics (\includegraphics)
%\usepackage{graphicx}
% for neatly defining theorems and propositions
%\usepackage{amsthm}
% making logically defined graphics
%%%\usepackage{xypic}

% there are many more packages, add them here as you need them

% define commands here

\newcommand{\sR}[0]{\mathbb{R}}
\newcommand{\sC}[0]{\mathbb{C}}
\newcommand{\sN}[0]{\mathbb{N}}
\newcommand{\sZ}[0]{\mathbb{Z}}
\begin{document}
Let $\Vert x\Vert$ and $\Vert x\Vert' $ be two norms on
a vector space $V$. These norms are \emph{equivalent norms} if
there exists a number $C>1$ such that
\begin{eqnarray}
\label{condC}
 \frac{1}{C} \Vert x\Vert \le \Vert x\Vert' \le C \Vert x\Vert
\end{eqnarray}
for all $x\in V$.

Since equation \eqref{condC} is equivalent to
\begin{eqnarray}
\label{condC1}
 \frac{1}{C} \Vert x\Vert ' \le \Vert x\Vert \le C \Vert x\Vert'
\end{eqnarray}
it follows that the definition is well defined. In other words,
$\Vert\cdot \Vert$ and $\Vert\cdot \Vert'$ are equivalent if and only if
$\Vert\cdot \Vert'$ and $\Vert\cdot \Vert$ are equivalent.
An alternative condition is that there exist positive real
numbers $c,d$ such that
$$
 c\Vert x\Vert \le \Vert x \Vert' \le d \Vert x\Vert.
$$
However, this condition is equivalent to the above
by setting $C=\max\{1/c,d\}$.

Some key results are as follows:

\begin{enumerate}
\item If $\gamma>0$ and $\Vert x \Vert' = \gamma \Vert x \Vert$, then
$\Vert\cdot \Vert$ and $\Vert\cdot \Vert'$ are equivalent. For example,
if $\gamma>1$, then condition \eqref{condC} holds with $C=\gamma$, and
for $\gamma<1$, condition \eqref{condC1} holds with $C=1/\gamma$.

\item Suppose norms $\Vert \cdot \Vert$ and $\Vert \cdot \Vert'$ are equivalent norms
 as in equation \eqref{condC}, and let $B_r(x)$ and $B_r'(x)$ be the
 open balls with respect to $\Vert \cdot \Vert$ and $\Vert \cdot \Vert'$, respectively.
 By \PMlinkname{this result}{ScalingOfTheOpenBallInANormedVectorSpace}
 it follows that
 $$
 C B_{\varepsilon}(x) \subseteq B'_\varepsilon(x)\subseteq \frac{1}{C} B_{\varepsilon}(x).
 $$ 

It follows that the identity map from $(V,\Vert \cdot \Vert)$ to $(V,\Vert \cdot \Vert')$
is a homeomorphism. Or, alternatively, equivalent norms on $V$ induce the same
topology on $V$.

\item The converse of the last paragraph is also true, i.e. if two norms induce the same topology on $V$ then they are equivalent. This follows from the fact that every continuous linear function between two normed vector spaces is \PMlinkname{bounded}{BoundedOperator} (see \PMlinkname{this entry}{BoundedOperator}).

\item Suppose $\langle\cdot,\cdot\rangle$ and $\langle\cdot,\cdot\rangle'$ are inner product. Suppose further that the induced norms $\Vert\cdot\Vert$ and $\Vert\cdot\Vert'$ are equivalent as in equation \ref{condC}. Then, by the polarization identity, the inner products satisfy 
$$
  \frac{1}{C^2}\langle v,w \rangle'  \le \langle v,w \rangle  \le C^2\langle v,w \rangle.
$$

\item On a finite dimensional vector space all norms are equivalent
(see \PMlinkname{this page}{ProofThatAllNormsOnFiniteVectorSpaceAreEquivalent}).
This is easy to understand as the unit sphere is compact if and only if 
a space is finite dimensional.
On infinite dimensional spaces this result does not hold (see
\PMlinkname{this page}{AllNormsAreNotEquivalent}).

It follows that on a finite dimensional vector space,
one can check continuity and convergence with respect with any norm.
If a sequence converges in one norm, it converges in all norms.
In matrix analysis this is particularly useful as one can choose the norm that
is most easily calculated. 

\item The concept of equivalent norms also generalize to possibly non-symmetric norms. In this setting, all norms are also equivalent on a finite dimensional vector space. In particular, $\Vert\cdot \Vert$ and $\Vert-\cdot\Vert$ are
equivalent, and there exists $C>0$ such that 
$$
  \Vert - v\Vert \le C \Vert v\Vert,\quad v\in V.
$$
\end{enumerate}
%%%%%
%%%%%
\end{document}
