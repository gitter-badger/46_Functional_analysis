\documentclass[12pt]{article}
\usepackage{pmmeta}
\pmcanonicalname{SpaceOfRapidlyDecreasingFunctions}
\pmcreated{2013-03-22 13:44:50}
\pmmodified{2013-03-22 13:44:50}
\pmowner{matte}{1858}
\pmmodifier{matte}{1858}
\pmtitle{space of rapidly decreasing functions}
\pmrecord{8}{34444}
\pmprivacy{1}
\pmauthor{matte}{1858}
\pmtype{Definition}
\pmcomment{trigger rebuild}
\pmclassification{msc}{46F05}
\pmsynonym{Schwartz space}{SpaceOfRapidlyDecreasingFunctions}
\pmrelated{DiscreteTimeFourierTransformInRelationWithItsContinousTimeFourierTransfrom}

\endmetadata

% this is the default PlanetMath preamble.  as your knowledge
% of TeX increases, you will probably want to edit this, but
% it should be fine as is for beginners.

% almost certainly you want these
\usepackage{amssymb}
\usepackage{amsmath}
\usepackage{amsfonts}

% used for TeXing text within eps files
%\usepackage{psfrag}
% need this for including graphics (\includegraphics)
%\usepackage{graphicx}
% for neatly defining theorems and propositions
%\usepackage{amsthm}
% making logically defined graphics
%%%\usepackage{xypic}

% there are many more packages, add them here as you need them

% define commands here

\newcommand{\sR}[0]{\mathbb{R}}
\newcommand{\sC}[0]{\mathbb{C}}
\newcommand{\sN}[0]{\mathbb{N}}
\newcommand{\sZ}[0]{\mathbb{Z}}
\begin{document}
\newcommand{\cD}[0]{\mathcal{D}}
 \newcommand{\scomp}[0]{C^\infty_0}
 \newcommand{\cS}[0]{\mathcal{S}}

The function space of rapidly decreasing functions $\cS$ has the
important property that the Fourier transform is an endomorphism
on this space. This property enables one, by duality, to 
define the Fourier transform
for elements in the dual space of $\cS$, that is, for tempered 
distributions. 

{\bf Definition}  
The \emph{space of rapidly decreasing functions} on $\sR^n$
is the function space
\begin{eqnarray*}
\cS(\sR^n)=\{ f \in C^\infty(\sR^n) \mid \sup_{x\in \sR^n} \mid \, ||f||_{\alpha,\beta} <
 \infty\, \mbox{for all multi-indices}  \, \alpha, \beta \},
\end{eqnarray*}
where $C^\infty(\sR^n)$ is the set of smooth functions 
from $\sR^n$ to $\sC$, and 
$$||f||_{\alpha,\beta}=||x^\alpha D^\beta f||_\infty.$$
Here, $||\cdot||_\infty$ is the supremum norm, and we use 
multi-index notation.
When the dimension $n$ is clear, it is convenient to write 
$\cS=\cS(\sR^n)$. The space $\cS$ is also called the 
\emph{Schwartz space}, after Laurent Schwartz
(1915-2002) \cite{schwartz_bib}.


\subsubsection{Examples of functions in $\cS$} 
\begin{enumerate}
\item If $i$ is a multi-index, and $a$ is a positive
real number, then
$$ x^i \exp\{-a x^2\} \in \cS.$$
\item Any smooth function with compact support $f$ is in $\cS$.
This is clear since any derivative of $f$ is continuous, so
$x^\alpha D^\beta f$ has a maximum in $\sR^n$. 
\end{enumerate}

\subsubsection{Properties}
\begin{enumerate}
\item 
$\cS$ is a complex vector space. In other words, 
$\cS$ is closed under point-wise addition and under 
multiplication by a complex scalar. 
\item Using Leibniz' rule, it follows that $\cS$ is also closed
under point-wise multiplication; if $f,g\in \cS$, then
$fg: x\mapsto f(x)g(x)$ is also in $\cS$.
\item For any $1\le p\le \infty$, we have \cite{reed}
$$ \cS\subset L^p,$$
and if $p<\infty$, then $\cS$ is also dense in $L^p$. 
\item The Fourier transform is a linear isomorphism $\cS\to\cS$. 
\end{enumerate}


\begin{thebibliography}{9}
 \bibitem{hormander}
 L. H\"ormander, \emph{The Analysis of Linear Partial Differential Operators I,
 (Distribution theory and Fourier Analysis)}, 2nd ed, Springer-Verlag, 1990.
 \bibitem{schwartz_bib} 
 The MacTutor History of Mathematics archive,
 \PMlinkexternal{Laurent Schwartz}{http://www-gap.dcs.st-and.ac.uk/~history/Mathematicians/Schwartz.html}
\bibitem{reed} M. Reed, B. Simon,
  \emph{Methods of Modern Mathematical Physics: Functional Analysis I},
Revised and enlarged edition,  Academic Press, 1980.
\bibitem{wikiTemp} Wikipedia, 
 \PMlinkexternal{Tempered distributions}{http://en.wikipedia.org/wiki/Tempered_distribution}
 \end{thebibliography}
%%%%%
%%%%%
\end{document}
