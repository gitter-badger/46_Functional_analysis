\documentclass[12pt]{article}
\usepackage{pmmeta}
\pmcanonicalname{ProofOfHilbertSpaceIsUniformlyConvexSpace}
\pmcreated{2013-03-22 15:20:48}
\pmmodified{2013-03-22 15:20:48}
\pmowner{Mathprof}{13753}
\pmmodifier{Mathprof}{13753}
\pmtitle{proof of  Hilbert space is uniformly convex space}
\pmrecord{16}{37167}
\pmprivacy{1}
\pmauthor{Mathprof}{13753}
\pmtype{Proof}
\pmcomment{trigger rebuild}
\pmclassification{msc}{46C15}
\pmclassification{msc}{46H05}

% this is the default PlanetMath preamble.  as your knowledge
% of TeX increases, you will probably want to edit this, but
% it should be fine as is for beginners.

% almost certainly you want these
\usepackage{amssymb}
\usepackage{amsmath}
\usepackage{amsfonts}

% used for TeXing text within eps files
%\usepackage{psfrag}
% need this for including graphics (\includegraphics)
%\usepackage{graphicx}
% for neatly defining theorems and propositions
%\usepackage{amsthm}
% making logically defined graphics
%%%\usepackage{xypic}

% there are many more packages, add them here as you need them

% define commands here
\begin{document}
We prove that in fact an inner product space is uniformly convex.
Let $\epsilon >0$, $u,v \in H$ such that $\|u\|\leq 1$, $\|v\|\leq 1$, $\|u-v\|\geq \epsilon$. Put $\delta = 1-\frac{1}{2}\sqrt{4-\epsilon^2}$.
Then $\delta >0$ and by the parallelogram law
\begin{eqnarray*}
\Vert u+v\Vert^2 & = & \Vert u + v \Vert^2 + \Vert u-v\Vert^2 - \Vert u-v\Vert^2\\
&=& 2 \Vert u \Vert^2 + 2\Vert v \Vert^2 - \Vert u-v\Vert^2 \\
& \leq &  4  - \epsilon^2 \\
& = & 4(1-\delta)^2.\\
\end{eqnarray*}
Hence,  $\Vert\frac{u+v}{2}\Vert\leq 1-\delta$. 

Since a Hilbert space is an inner product space, 
 a Hilbert space \PMlinkescapetext{satisfies} the conditions of a uniformly convex space.
%%%%%
%%%%%
\end{document}
