\documentclass[12pt]{article}
\usepackage{pmmeta}
\pmcanonicalname{PolarizationIdentity}
\pmcreated{2013-03-22 17:37:20}
\pmmodified{2013-03-22 17:37:20}
\pmowner{asteroid}{17536}
\pmmodifier{asteroid}{17536}
\pmtitle{polarization identity}
\pmrecord{4}{40042}
\pmprivacy{1}
\pmauthor{asteroid}{17536}
\pmtype{Theorem}
\pmcomment{trigger rebuild}
\pmclassification{msc}{46C05}

% this is the default PlanetMath preamble.  as your knowledge
% of TeX increases, you will probably want to edit this, but
% it should be fine as is for beginners.

% almost certainly you want these
\usepackage{amssymb}
\usepackage{amsmath}
\usepackage{amsfonts}

% used for TeXing text within eps files
%\usepackage{psfrag}
% need this for including graphics (\includegraphics)
%\usepackage{graphicx}
% for neatly defining theorems and propositions
%\usepackage{amsthm}
% making logically defined graphics
%%%\usepackage{xypic}

% there are many more packages, add them here as you need them

% define commands here

\begin{document}
{\bf Theorem} [polarization identity] {\bf -} Let $X$ be an inner product space over $\mathbb{R}$. The following identity holds for every $x, y \in X$:
\begin{displaymath}
\langle x, y \rangle = \frac{1}{4}(\|x +y \|^2 - \|x-y\|^2)
\end{displaymath}

If $X$ is an inner product space over $\mathbb{C}$ instead, the identity becomes
\begin{displaymath}
\langle x, y \rangle = \frac{1}{4}(\|x +y \|^2 - \|x-y\|^2) + \frac{1}{4}i(\|x+iy\|^2-\|x-iy\|^2)
\end{displaymath}

{\bf Remark -} This result shows that the inner product of $X$ is determined by the norm. Moreover, it can be shown that if a normed space $V$ \PMlinkescapetext{satisfies} the parallelogram law, the above formulas define an inner product compatible with the norm of $V$.
%%%%%
%%%%%
\end{document}
