\documentclass[12pt]{article}
\usepackage{pmmeta}
\pmcanonicalname{BesselInequality}
\pmcreated{2013-03-22 12:46:38}
\pmmodified{2013-03-22 12:46:38}
\pmowner{ariels}{338}
\pmmodifier{ariels}{338}
\pmtitle{Bessel inequality}
\pmrecord{5}{33089}
\pmprivacy{1}
\pmauthor{ariels}{338}
\pmtype{Theorem}
\pmcomment{trigger rebuild}
\pmclassification{msc}{46C05}

\endmetadata

% this is the default PlanetMath preamble.  as your knowledge
% of TeX increases, you will probably want to edit this, but
% it should be fine as is for beginners.

% almost certainly you want these
\usepackage{amssymb}
\usepackage{amsmath}
\usepackage{amsfonts}

% used for TeXing text within eps files
%\usepackage{psfrag}
% need this for including graphics (\includegraphics)
%\usepackage{graphicx}
% for neatly defining theorems and propositions
%\usepackage{amsthm}
% making logically defined graphics
%%%\usepackage{xypic}

% there are many more packages, add them here as you need them

% define commands here

\newcommand{\Prob}[2]{\mathbb{P}_{#1}\left\{#2\right\}}
\newcommand{\Expect}{\mathbb{E}}
\newcommand{\norm}[1]{\left\|#1\right\|}

% Some sets
\newcommand{\Nats}{\mathbb{N}}
\newcommand{\Ints}{\mathbb{Z}}
\newcommand{\Reals}{\mathbb{R}}
\newcommand{\Complex}{\mathbb{C}}



%%%%%% END OF SAVED PREAMBLE %%%%%%
\begin{document}
\newcommand{\Hilb}{\mathcal{H}}
\newcommand{\size}[1]{\left|#1\right|}
\newcommand{\scalar}[2]{\left\langle#1,#2\right\rangle}
Let $\Hilb$ be a Hilbert space, and suppose $e_1, e_2, \ldots \in \Hilb$ is an orthonormal sequence.  Then for any $x\in\Hilb$,
$$
\sum_{k=1}^{\infty}\size{\scalar{x}{e_k}}^2 \le \norm{x}^2.
$$

Bessel's inequality immediately lets us define the sum
$$
x' = \sum_{k=1}^{\infty}\scalar{x}{e_k}e_k.
$$
The inequality means that the series converges.

For a complete orthonormal series, we have Parseval's theorem, which replaces inequality with equality (and consequently $x'$ with $x$).
%%%%%
%%%%%
\end{document}
