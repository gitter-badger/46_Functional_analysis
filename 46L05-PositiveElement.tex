\documentclass[12pt]{article}
\usepackage{pmmeta}
\pmcanonicalname{PositiveElement}
\pmcreated{2013-03-22 17:30:31}
\pmmodified{2013-03-22 17:30:31}
\pmowner{asteroid}{17536}
\pmmodifier{asteroid}{17536}
\pmtitle{positive element}
\pmrecord{8}{39898}
\pmprivacy{1}
\pmauthor{asteroid}{17536}
\pmtype{Definition}
\pmcomment{trigger rebuild}
\pmclassification{msc}{46L05}
\pmclassification{msc}{47L07}
\pmclassification{msc}{47A05}
\pmsynonym{positive}{PositiveElement}
\pmdefines{positive operator}
\pmdefines{positive cone}
\pmdefines{square root of positive element}

% this is the default PlanetMath preamble.  as your knowledge
% of TeX increases, you will probably want to edit this, but
% it should be fine as is for beginners.

% almost certainly you want these
\usepackage{amssymb}
\usepackage{amsmath}
\usepackage{amsfonts}

% used for TeXing text within eps files
%\usepackage{psfrag}
% need this for including graphics (\includegraphics)
%\usepackage{graphicx}
% for neatly defining theorems and propositions
%\usepackage{amsthm}
% making logically defined graphics
%%%\usepackage{xypic}

% there are many more packages, add them here as you need them

% define commands here

\begin{document}
Let $H$ be a complex Hilbert space. Let $T:H \longrightarrow H$ be a bounded operator in $H$.

{\bf Definition -} $T$ is said to be a {\bf positive operator} if there exists a bounded operator $A: H \longrightarrow H$ such that
\begin{displaymath}
T=A^*A
\end{displaymath}
where $A^*$ denotes the adjoint of $A$.

Every positive operator $T$ satisfies the very strong condition $\langle T v , v \rangle \geq 0$ for every $v \in H$ since
\begin{displaymath}
\langle T v , v \rangle = \langle A^*A v , v \rangle = \langle A v , Av \rangle = \|Av\|^2 \geq 0
\end{displaymath}

The converse is also true, although it is not so \PMlinkescapetext{simple} to prove. Indeed,

{\bf Theorem -} $T$ is positive if and only if $\langle Tv, v \rangle \geq 0 \;\;\;\;\forall_{v \in H}$

\subsection{Generalization to $C^*$-algebras}

The above notion can be generalized to elements in an arbitrary \PMlinkname{$C^*$-algebra}{CAlgebra}.

In what follows $\mathcal{A}$ denotes a $C^*$-algebra.

{\bf Definition -} An element $x \in \mathcal{A}$ is said to be {\bf positive} (and denoted $0 \leq x$) if
\begin{displaymath}
x=a^*a
\end{displaymath}
for some element $a \in \mathcal{A}$.

$Remark -$ Positive elements are clearly \PMlinkname{self-adjoint}{InvolutaryRing}.

\subsection{Positive spectrum}

It can be shown that the positive elements of $\mathcal{A}$ are precisely the normal elements of $\mathcal{A}$ with a positive spectrum. We \PMlinkescapetext{state} it here as a theorem:

{\bf Theorem -} Let $x \in \mathcal{A}$ and $\sigma(x)$ denote its spectrum. Then $x$ is positive if and only if $x$ is \PMlinkescapetext{normal} and $\sigma(x)\subset \mathbb{R}_{0}^+$.

\subsection{Square roots}

Positive elements admit a unique positive square root.

{\bf Theorem -} Let $x$ be a positive element in  $\mathcal{A}$. There is a unique $b \in \mathcal{A}$ such that
\begin{itemize}
\item $b$ is positive
\item $x=b^2$.
\end{itemize}

The converse is also true (with \PMlinkescapetext{even weaker} assumptions): If $x$ admits a \PMlinkescapetext{self-adjoint} square root then $x$ is positive, since
\begin{displaymath}
x=b^2=bb=b^*b
\end{displaymath}

\subsection{The positive cone}

Another interesting fact about positive elements is that they form a \PMlinkname{proper convex cone}{Cone5} in $\mathcal{A}$, usually called the {\bf positive cone} of $\mathcal{A}$. That is stated in following theorem:

{\bf Theorem -} Let $a, b$ be positive elements in $\mathcal{A}$. Then
\begin{itemize}
\item $a+b$ is also positive
\item $\lambda a$ is also positive for every $\lambda \geq 0$
\item If both $a$ and $-a$ are positive then $a=0$.
\end{itemize}

\subsection{Norm closure}
{\bf Theorem -} The set of positive elements in $\mathcal{A}$ is norm closed.
%%%%%
%%%%%
\end{document}
