\documentclass[12pt]{article}
\usepackage{pmmeta}
\pmcanonicalname{MultiresolutionAnalysis}
\pmcreated{2013-03-22 14:26:48}
\pmmodified{2013-03-22 14:26:48}
\pmowner{swiftset}{1337}
\pmmodifier{swiftset}{1337}
\pmtitle{multiresolution analysis}
\pmrecord{5}{35961}
\pmprivacy{1}
\pmauthor{swiftset}{1337}
\pmtype{Definition}
\pmcomment{trigger rebuild}
\pmclassification{msc}{46C99}
\pmsynonym{level of detail}{MultiresolutionAnalysis}
\pmrelated{Wavelet}
\pmdefines{scaling function}

\endmetadata

% this is the default PlanetMath preamble.  as your knowledge
% of TeX increases, you will probably want to edit this, but
% it should be fine as is for beginners.

% almost certainly you want these
\usepackage{amssymb}
\usepackage{amsmath}
\usepackage{amsfonts}

% used for TeXing text within eps files
%\usepackage{psfrag}
% need this for including graphics (\includegraphics)
%\usepackage{graphicx}
% for neatly defining theorems and propositions
%\usepackage{amsthm}
% making logically defined graphics
%%%\usepackage{xypic}

% there are many more packages, add them here as you need them

% define commands here
\begin{document}
\PMlinkescapeword{function}
\PMlinkescapeword{density}
\PMlinkescapeword{separation}

\paragraph{Definition} 
A \emph{multiresolution analysis} is a sequence $(V_j)_{j\in \mathbb Z}$ of subspaces of $L_2({\mathbb R})$ such that
\begin{enumerate}
\item (nesting) $ \ldots \subset V_{-1} \subset V_0 \subset V_1 \subset \ldots $
\item (density) $\overline{\mathop{\rm span} \bigcup_{j \in \mathbb Z} V_j } = L_2({\mathbb R}) $ 
\item (separation) $ \bigcap_{j \in \mathbb Z} V_j = \{0\}$
\item (scaling) $f(x) \in V_j$ if and only if $f(2^{-j} x) \in V_0$
\item (orthonormal basis) there exists a function $\Phi \in V_0$, called a \emph{scaling function}, such that the system $\{ \Phi(t -m) \}_{m \in \mathbb Z} \}$ is an orthonormal basis in $V_0.$
\end{enumerate}

\paragraph{Notes}
Multiresolution analysis, particularly scaling functions, are used to derive wavelets. The $V_j$ are called approximation spaces. Several choices of scaling functions may exist for a given set of approximation spaces--- each determines a unique multiresolution analysis.
%%%%%
%%%%%
\end{document}
