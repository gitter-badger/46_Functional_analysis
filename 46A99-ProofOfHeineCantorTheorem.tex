\documentclass[12pt]{article}
\usepackage{pmmeta}
\pmcanonicalname{ProofOfHeineCantorTheorem}
\pmcreated{2013-03-22 15:09:43}
\pmmodified{2013-03-22 15:09:43}
\pmowner{drini}{3}
\pmmodifier{drini}{3}
\pmtitle{proof of Heine-Cantor theorem}
\pmrecord{10}{36911}
\pmprivacy{1}
\pmauthor{drini}{3}
\pmtype{Proof}
\pmcomment{trigger rebuild}
\pmclassification{msc}{46A99}

\endmetadata

% this is the default PlanetMath preamble.  as your knowledge
% of TeX increases, you will probably want to edit this, but
% it should be fine as is for beginners.

% almost certainly you want these
\usepackage{amssymb}
\usepackage{amsmath}
\usepackage{amsfonts}

% used for TeXing text within eps files
%\usepackage{psfrag}
% need this for including graphics (\includegraphics)
%\usepackage{graphicx}
% for neatly defining theorems and propositions
%\usepackage{amsthm}
% making logically defined graphics
%%%\usepackage{xypic}

% there are many more packages, add them here as you need them

% define commands here
\begin{document}
We seek to show that $f:K \to X$ is continuous with $K$ a compact metric space, then $f$ is uniformly continuous. Recall that for $f:K\to X$, uniform continuity 
is the condition that for any $\varepsilon>0$, there exists $\delta$ such that
\[
d_K(x,y) < \delta 
\implies d_X (f(x),f(y)) < \epsilon
\]
for all $x,y\in K$

Suppose $K$ is a compact metric space, $f$ continuous on $K$. Let $\epsilon > 0$. For each $k \in K$ choose 
$\delta_k$ such that $d(k,x) \leq \delta_k$ implies $d(f(k),f(x)) \leq \frac{\epsilon}{2}$. Note that the collection of balls 
$B(k, \frac{\delta_k}{2} )$ covers $K$, so by compactness there is a finite subcover,
say involving $k_1, \ldots, k_n$. Take 
\begin{equation*}
\delta = \min_{i=1,\ldots,n} \frac{\delta_{k_i}}{2}
\end{equation*}
Then, suppose $d(x,y) \leq \delta$. By the choice of $k_1,\ldots,k_n$ and the triangle inequality, there exists an $i$ such that 
$d(x,k_i),d(y,k_i) \leq \delta_{k_i}$. Hence,
\begin{eqnarray}
d(f(x),f(y)) &\leq& d(f(x),f(k_i)) + d(f(y),f(k_i)) \\
             &\leq& \frac{\epsilon}{2} + \frac{\epsilon}{2}
\end{eqnarray}

As $x,y$ were arbitrary, we have that $f$ is uniformly continuous.\\
This proof is similar to one found in Mathematical Principles of Analysis, Rudin.
%%%%%
%%%%%
\end{document}
