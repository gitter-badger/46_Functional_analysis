\documentclass[12pt]{article}
\usepackage{pmmeta}
\pmcanonicalname{ProofOfClosedGraphTheorem}
\pmcreated{2013-03-22 14:48:47}
\pmmodified{2013-03-22 14:48:47}
\pmowner{Koro}{127}
\pmmodifier{Koro}{127}
\pmtitle{proof of closed graph theorem}
\pmrecord{5}{36472}
\pmprivacy{1}
\pmauthor{Koro}{127}
\pmtype{Proof}
\pmcomment{trigger rebuild}
\pmclassification{msc}{46A30}

\endmetadata

% this is the default PlanetMath preamble.  as your knowledge
% of TeX increases, you will probably want to edit this, but
% it should be fine as is for beginners.

% almost certainly you want these
\usepackage{amssymb}
\usepackage{amsmath}
\usepackage{amsfonts}
\usepackage{mathrsfs}

% used for TeXing text within eps files
%\usepackage{psfrag}
% need this for including graphics (\includegraphics)
%\usepackage{graphicx}
% for neatly defining theorems and propositions
%\usepackage{amsthm}
% making logically defined graphics
%%%\usepackage{xypic}

% there are many more packages, add them here as you need them

% define commands here
\newcommand{\C}{\mathbb{C}}
\newcommand{\R}{\mathbb{R}}
\newcommand{\N}{\mathbb{N}}
\newcommand{\Z}{\mathbb{Z}}
\newcommand{\Per}{\operatorname{Per}}
\begin{document}
Let $T\colon X\to Y$ be a linear mapping. Denote its graph by $G(T)$, and let $p_1\colon X\times Y\to X$ and $p_2\colon X\times Y\to Y$ be the projections onto $X$ and $Y$, respectively. We remark that these projections are continuous, by definition of the product of Banach spaces.

If $T$ is bounded, then given a sequence 
$\{(x_i, Tx_i)\}$ in $G(T)$ which converges to $(x,y)\in X\times Y$, we have that $$x_i = p_1(x_i,Tx_i) \xrightarrow[i\to\infty]{} p_1(x,y) = x$$ 
and  $$Tx_i = p_2(x_i,Tx_i) \xrightarrow[i\to\infty]{} p_2(x,y) = y,$$
by continuity of the projections. 
But then, since $T$ is continuous,
$$Tx = \lim_{i\to\infty} Tx_i = y.$$
Thus $(x,y) = (x,Tx)\in G(T)$, proving that $G(T)$ is closed.

Now suppose $G(T)$ is closed. We remark that $G(T)$ is a vector subspace of $X\times Y$, and being closed, it is a Banach space. Consider the operator 
$\tilde T:X\to G(T)$ defined by $\tilde Tx = (x,Tx)$. It is clear that $\tilde T$ is a bijection, its inverse being $p_1|_{G(T)}$, the restriction of $p_1$ to $G(T)$. Since $p_1$ is continuous on $X\times Y$, the restriction is continuous as well; and since it is also surjective, the open mapping theorem implies that $p_1|_{G(T)}$ is an open mapping, so its inverse must be continuous. That is, $\tilde T$ is continuous, and consequently $T = p_2\circ\tilde T$ is continuous.
%%%%%
%%%%%
\end{document}
