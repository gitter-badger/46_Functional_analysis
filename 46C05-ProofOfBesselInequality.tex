\documentclass[12pt]{article}
\usepackage{pmmeta}
\pmcanonicalname{ProofOfBesselInequality}
\pmcreated{2013-03-22 12:46:41}
\pmmodified{2013-03-22 12:46:41}
\pmowner{ariels}{338}
\pmmodifier{ariels}{338}
\pmtitle{proof of Bessel inequality}
\pmrecord{4}{33090}
\pmprivacy{1}
\pmauthor{ariels}{338}
\pmtype{Proof}
\pmcomment{trigger rebuild}
\pmclassification{msc}{46C05}

\endmetadata

% this is the default PlanetMath preamble.  as your knowledge
% of TeX increases, you will probably want to edit this, but
% it should be fine as is for beginners.

% almost certainly you want these
\usepackage{amssymb}
\usepackage{amsmath}
\usepackage{amsfonts}

% used for TeXing text within eps files
%\usepackage{psfrag}
% need this for including graphics (\includegraphics)
%\usepackage{graphicx}
% for neatly defining theorems and propositions
%\usepackage{amsthm}
% making logically defined graphics
%%%\usepackage{xypic}

% there are many more packages, add them here as you need them

% define commands here

\newcommand{\Prob}[2]{\mathbb{P}_{#1}\left\{#2\right\}}
\newcommand{\Expect}{\mathbb{E}}
\newcommand{\norm}[1]{\left\|#1\right\|}

% Some sets
\newcommand{\Nats}{\mathbb{N}}
\newcommand{\Ints}{\mathbb{Z}}
\newcommand{\Reals}{\mathbb{R}}
\newcommand{\Complex}{\mathbb{C}}



%%%%%% END OF SAVED PREAMBLE %%%%%%
\begin{document}
\newcommand{\Hilb}{\mathcal{H}}
\newcommand{\size}[1]{\left|#1\right|}
\newcommand{\scalar}[2]{\left\langle#1,#2\right\rangle}
Let
$$
r_n = x - \sum_{k=1}^{n} \scalar{x}{e_k}\cdot e_k.
$$
Then for $j=1,\ldots,n$,
\begin{eqnarray}
\scalar{r_n}{e_j} &=
\scalar{x}{e_j} -
\sum_{k=1}^{n} \scalar{\scalar{x}{e_k}\cdot e_k}{e_j} \\
&= \scalar{x}{e_j} -
\scalar{x}{e_j}\scalar{e_j}{e_j} = 0
\end{eqnarray}
so $e_1,\ldots,e_n,r_n$ is an orthogonal series.

Computing norms, we see that
$$
\norm{x}^2 = \norm{r_n + \sum_{k=1}^{n} \scalar{x}{e_k}\cdot e_k}^2 =
\norm{r_n}^2 + \sum_{k=1}^{n} \size{\scalar{x}{e_k}}^2 \ge
\sum_{k=1}^{n} \size{\scalar{x}{e_k}}^2.
$$
So the series
$$
\sum_{k=1}^{\infty} \size{\scalar{x}{e_k}}^2
$$
converges and is bounded by $\norm{x}^2$, as required.
%%%%%
%%%%%
\end{document}
