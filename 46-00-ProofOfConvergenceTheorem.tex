\documentclass[12pt]{article}
\usepackage{pmmeta}
\pmcanonicalname{ProofOfConvergenceTheorem}
\pmcreated{2013-03-22 13:44:36}
\pmmodified{2013-03-22 13:44:36}
\pmowner{matte}{1858}
\pmmodifier{matte}{1858}
\pmtitle{proof of convergence theorem}
\pmrecord{10}{34437}
\pmprivacy{1}
\pmauthor{matte}{1858}
\pmtype{Proof}
\pmcomment{trigger rebuild}
\pmclassification{msc}{46-00}
\pmclassification{msc}{46F05}

% this is the default PlanetMath preamble.  as your knowledge
% of TeX increases, you will probably want to edit this, but
% it should be fine as is for beginners.

% almost certainly you want these
\usepackage{amssymb}
\usepackage{amsmath}
\usepackage{amsfonts}

% used for TeXing text within eps files
%\usepackage{psfrag}
% need this for including graphics (\includegraphics)
%\usepackage{graphicx}
% for neatly defining theorems and propositions
%\usepackage{amsthm}
% making logically defined graphics
%%%\usepackage{xypic}

% there are many more packages, add them here as you need them

% define commands here

\newcommand{\sR}[0]{\mathbb{R}}
\newcommand{\sC}[0]{\mathbb{C}}
\newcommand{\sN}[0]{\mathbb{N}}
\newcommand{\sZ}[0]{\mathbb{Z}}
\begin{document}
\newcommand{\cD}[0]{\mathcal{D}}
\newcommand{\scomp}[0]{C^\infty_0}
Let us show the equivalence of (2) and (3).
First, the proof  that (3) implies (2) is a direct calculation. 
Next, let us show that (2) implies (3): 
Suppose $Tu_i \to 0$ in $\sC$, and if $K$ is a  compact set in $U$, and  
$\{u_i\}_{i=1}^\infty$ is a sequence in $\cD_K$ such that
for any multi-index $\alpha$, we have 
$$ D^\alpha u_i \to 0$$
in the supremum norm $\lVert\cdot\rVert_\infty$ as $i\to \infty$.
For a contradiction, suppose there is a compact set $K$ in $U$
such that for all constants $C>0$ and $k\in\{0, 1,2,\ldots\}$ there exists 
a function $u\in \cD_K$ such that 
$$|T(u)|> C\sum_{|\alpha|\le k} ||D^\alpha u||_\infty.$$
Then, for $C=k=1,2,\ldots$ we obtain functions $u_1,u_2,\ldots$ in $\cD(K)$
such that 
$ |T(u_i)| > i\sum_{|\alpha|\le i} ||D^\alpha u_i||_\infty.$
Thus $|T(u_i)|>0$ for all $i$, so for $v_i=u_i/|T(u_i)|$, we have
$$ 1> i\sum_{|\alpha|\le i} ||D^\alpha v_i||_\infty.$$
It follows that $||D^\alpha u_i||_\infty< 1/i$ 
for any multi-index $\alpha$ with $|\alpha|\le i$.
Thus $\{v_i\}_{i=1}^\infty$ satisfies our assumption, whence $T(v_i)$ should tend to $0$. 
However, for all $i$, we have $T(v_i)= 1$. This contradiction 
completes the proof.
%%%%%
%%%%%
\end{document}
