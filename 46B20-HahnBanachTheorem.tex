\documentclass[12pt]{article}
\usepackage{pmmeta}
\pmcanonicalname{HahnBanachTheorem}
\pmcreated{2013-03-22 12:54:09}
\pmmodified{2013-03-22 12:54:09}
\pmowner{rmilson}{146}
\pmmodifier{rmilson}{146}
\pmtitle{Hahn-Banach theorem}
\pmrecord{10}{33252}
\pmprivacy{1}
\pmauthor{rmilson}{146}
\pmtype{Theorem}
\pmcomment{trigger rebuild}
\pmclassification{msc}{46B20}
\pmdefines{bound}
\pmdefines{bounded}

\usepackage{amsmath}
\usepackage{amsfonts}
\usepackage{amssymb}
\newcommand{\reals}{\mathbb{R}}
\newcommand{\natnums}{\mathbb{N}}
\newcommand{\cnums}{\mathbb{C}}
\newcommand{\znums}{\mathbb{Z}}
\newcommand{\lp}{\left(}
\newcommand{\rp}{\right)}
\newcommand{\lb}{\left[}
\newcommand{\rb}{\right]}
\newcommand{\supth}{^{\text{th}}}
\newtheorem{proposition}{Proposition}
\newtheorem{definition}[proposition]{Definition}

\newtheorem{theorem}[proposition]{Theorem}

\newcommand{\bu}{\mathbf{u}}
\newcommand{\bv}{\mathbf{v}}
\newcommand{\pnorm}{\operatorname{p}}
\newcommand{\snorm}[1]{\pnorm(#1)}
\begin{document}
The Hahn-Banach theorem is a foundational result in functional
analysis.  Roughly speaking, it asserts the existence of a great
variety of bounded (and hence continuous) linear functionals on an
normed vector space, even if that space happens to be
infinite-dimensional. We first consider an
abstract version of this theorem, and then give the more classical
result as a corollary.

Let $V$ be a real, or a complex vector space, with $K$
denoting the corresponding field of scalars, and let
$$\pnorm:V\rightarrow\reals^+$$ 
be a seminorm on $V$.

\begin{theorem}
  Let $f:U\to K$ be a linear functional defined on a subspace
  $U\subset V$.  If the restricted functional satisfies
  $$\vert f(\bu)\vert\leq \snorm{\bu},\quad \bu\in U,$$
  then it can be extended to all of $V$ without violating the above
  property.  To be more precise, there exists a linear functional
  $F:V\to K$ such that
  \begin{align*}
    F(\bu) &= f(\bu),\quad \bu\in U\\
    \vert F(\bu) \vert &\leq \snorm{\bu},\quad \bu\in V.
  \end{align*}  
\end{theorem}

\begin{definition}
  We say that a linear functional $f:V\to K$ is \emph{bounded} if 
  there exists a bound $B\in\reals^+$ such that
  \begin{equation}
    \label{eq:bdef}
    \vert f(\bu)\vert \leq B \snorm{\bu},\quad \bu\in V.    
  \end{equation}
  If $f$ is a bounded linear functional, we define $\Vert f\Vert$, the
  norm of $f$, according to
  $$\Vert f \Vert = \sup \{ \vert f(\bu)\vert : \snorm{\bu} = 1 \}.$$
  One can show that $\Vert f\Vert$ is the infimum of all the possible
  $B$ that satisfy \eqref{eq:bdef}
\end{definition}

\begin{theorem}[Hahn-Banach]
  Let $f:U\to K$ be a bounded linear functional defined on a subspace
  $U\subset V$. Let $\Vert f \Vert_U$ denote the norm of $f$ relative
  to the restricted seminorm on $U$.  Then there exists a bounded
  extension $F:V\to K$ with the same norm, i.e.
  $$\Vert F\Vert_V = \Vert f\Vert_U.$$
\end{theorem}
%%%%%
%%%%%
\end{document}
