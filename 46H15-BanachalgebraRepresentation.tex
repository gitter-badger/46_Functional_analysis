\documentclass[12pt]{article}
\usepackage{pmmeta}
\pmcanonicalname{BanachalgebraRepresentation}
\pmcreated{2013-03-22 17:27:37}
\pmmodified{2013-03-22 17:27:37}
\pmowner{asteroid}{17536}
\pmmodifier{asteroid}{17536}
\pmtitle{Banach *-algebra representation}
\pmrecord{23}{39843}
\pmprivacy{1}
\pmauthor{asteroid}{17536}
\pmtype{Definition}
\pmcomment{trigger rebuild}
\pmclassification{msc}{46H15}
\pmclassification{msc}{46K10}
\pmdefines{subrepresentation}
\pmdefines{cyclic representation}
\pmdefines{cyclic vector}
\pmdefines{nondegenerate representation}
\pmdefines{topologically irreducible}
\pmdefines{algebrically irreducible}
\pmdefines{direct sum of representations}
\pmdefines{unitarily equivalent}

% this is the default PlanetMath preamble.  as your 
% almost certainly you want these
\usepackage{amssymb}
\usepackage{amsmath}
\usepackage{amsfonts}

% used for TeXing text within eps files
%\usepackage{psfrag}
% need this for including graphics (\includegraphics)
%\usepackage{graphicx}
% for neatly defining theorems and propositions
%\usepackage{amsthm}
% making logically defined graphics
%%%\usepackage{xypic}

% there are many more packages, add them here as you need them

% define commands here

\begin{document}
\PMlinkescapeword{representation}
\PMlinkescapeword{representations}
\PMlinkescapeword{nondegenerate}

\subsection*{Definition:}
A {\bf representation} of a Banach *-algebra $\mathcal{A}$ is a *-homomorphism $\pi : \mathcal{A} \longrightarrow \mathcal{B}(H)$ of $\mathcal{A}$ into the *-algebra of bounded operators on some Hilbert space $H$.

The set of all representations of $\mathcal{A}$ on a Hilbert space $H$ is denoted $rep(\mathcal{A},H)$.


\subsection*{Special kinds of representations:}

\begin{itemize}
\item A {\bf subrepresentation} of a representation $\pi \in rep(\mathcal{A},H)$ is a representation $\pi_0 \in rep(\mathcal{A},H_0)$ obtained from $\pi$ by restricting to a closed $\pi(\mathcal{A})$-\PMlinkname{invariant subspace}{InvariantSubspace} \footnote{by a $\pi(\mathcal{A})$-\PMlinkescapetext{invariant subspace} we \PMlinkescapetext{mean} a subspace which is invariant under every operator $\pi(a)$ with $a \in \mathcal{A}$} $H_0 \subseteq H$.
\end{itemize}

\begin{itemize}
\item A representation $\pi \in rep(\mathcal{A},H)$ is said to be {\bf nondegenerate} if one of the following equivalent conditions hold:
\begin{enumerate}
\item $\pi(x)\xi = 0 \;\;\;\;\; \forall x\in \mathcal{A}\; \Longrightarrow \; \xi = 0$, where $\xi \in H$.
\item $H$ is the closed linear span of the set of vectors $\pi(\mathcal{A})H := \{\pi(x)\xi : x \in \mathcal{A}, \xi \in H\}$
\end{enumerate}
\end{itemize}

\begin{itemize}
\item A representation $\pi \in rep(\mathcal{A},H)$ is said to be {\bf topologically irreducible} (or just \PMlinkescapetext{{\bf irreducible}}) if the only closed $\pi(\mathcal{A})$-invariant \PMlinkescapetext{subspaces} of $H$ are the trivial ones, $\{0\}$ and $H$.
\end{itemize}

\begin{itemize}
\item A representation $\pi \in rep(\mathcal{A},H)$ is said to be {\bf algebrically irreducible} if the only $\pi(\mathcal{A})$-invariant \PMlinkescapetext{subspaces} of $H$ (not necessarily closed) are the trivial ones, $\{0\}$ and $H$.
\end{itemize}

\begin{itemize}
\item Given two representations $\pi_1 \in rep(\mathcal{A},H_1)$ and $\pi_2 \in rep(\mathcal{A},H_2)$, the \PMlinkescapetext{{\bf direct sum}} of $\pi_1$ and $\pi_2$ is the representation $\pi_1 \oplus \pi_2 \in rep(\mathcal{A}, H_1 \oplus H_2)$ given by $\pi_1 \oplus \pi_2 (x) := \pi_1 (x) \oplus \pi_2 (x), \;\;\; x \in \mathcal{A}$.

More generally, given a family $\{\pi_i\}_{i \in I}$ of representations, with $\pi_i \in rep(\mathcal{A}, H_i)$, their \PMlinkescapetext{{\bf direct sum}} is the representation $\bigoplus_{i \in I} \pi_i \in rep(\mathcal{A}, \bigoplus_{i \in I}H_i)$, in the direct sum of Hilbert spaces $\bigoplus_{i \in I}H_i$, such that $\left(\bigoplus_{i \in I} \pi_i\right) (x):= \bigoplus_{i \in I} \pi_i(x)$ is the \PMlinkname{direct sum of the family of bounded operators}{DirectSumOfBoundedOperatorsOnHilbertSpaces} $\{\pi_i(x)\}_{i \in I}$.
\end{itemize}

\begin{itemize}
\item Two representations $\pi_1 \in rep(\mathcal{A},H_1)$ and $\pi_2 \in rep(\mathcal{A},H_2)$ of a  Banach *-algebra $\mathcal{A}$ are said to be {\bf unitarily equivalent} if there is a unitary $U : H_1 \longrightarrow H_2$ such that
\begin{displaymath}
\pi_2(a) = U \pi_1(a) U^* \;\;\;\;\; \forall a \in \mathcal{A}
\end{displaymath}
\end{itemize}

\begin{itemize}
\item A representation $\pi \in rep(\mathcal{A},H)$ is said to be {\bf \PMlinkescapetext{cyclic}} if there exists a vector $\xi \in H$ such that the set
\begin{displaymath}
\pi(A)\,\xi := \{\pi(a)\,\xi : a \in \mathcal{A}\}
\end{displaymath}
is \PMlinkname{dense}{Dense} in $H$. Such a vector is called a {\bf cyclic vector} for the representation $\pi$.
\end{itemize}



Linked file: http://aux.planetmath.org/files/objects/9843/BanachAlgebraRepresentation.pdf

%%%%%
%%%%%
\end{document}
