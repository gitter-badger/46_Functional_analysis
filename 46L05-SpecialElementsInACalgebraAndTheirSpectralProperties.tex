\documentclass[12pt]{article}
\usepackage{pmmeta}
\pmcanonicalname{SpecialElementsInACalgebraAndTheirSpectralProperties}
\pmcreated{2013-03-22 17:28:36}
\pmmodified{2013-03-22 17:28:36}
\pmowner{asteroid}{17536}
\pmmodifier{asteroid}{17536}
\pmtitle{special elements in a $C^*$-algebra and their spectral properties}
\pmrecord{11}{39862}
\pmprivacy{1}
\pmauthor{asteroid}{17536}
\pmtype{Definition}
\pmcomment{trigger rebuild}
\pmclassification{msc}{46L05}
\pmdefines{normal elements and spectral radius}
\pmdefines{spectrum of self-adjoint elements}
\pmdefines{spectrum of unitary elements}
\pmdefines{spectrum of projections}
\pmdefines{spectrum of positive elements}

\endmetadata

% this is the default PlanetMath preamble.  as your knowledge
% of TeX increases, you will probably want to edit this, but
% it should be fine as is for beginners.

% almost certainly you want these
\usepackage{amssymb}
\usepackage{amsmath}
\usepackage{amsfonts}

% used for TeXing text within eps files
%\usepackage{psfrag}
% need this for including graphics (\includegraphics)
%\usepackage{graphicx}
% for neatly defining theorems and propositions
%\usepackage{amsthm}
% making logically defined graphics
%%%\usepackage{xypic}

% there are many more packages, add them here as you need them

% define commands here

\begin{document}
\PMlinkescapeword{self-adjoint}
\PMlinkescapeword{normal}
\PMlinkescapeword{unitary}
\PMlinkescapeword{positive}
\PMlinkescapeword{projection}
\PMlinkescapeword{projections}
\PMlinkescapeword{partial isometry}

{\bf Definition -} Suppose $\mathcal{A}$ is a \PMlinkname{$C^*$-algebra}{CAlgebra}. An element $x \in \mathcal{A}$ is said to be:
\begin{itemize}
\item {\bf normal} if $x^*x = xx^*$
\item {\bf self-adjoint} if $x^* = x$
\item {\bf unitary} if $\mathcal{A}$ has an identity element $e$ and $x^*x = xx^* = e$
\item {\bf positive} if $x = y^*y$ for some element $y \in \mathcal{A}$
\item {\bf projection} if $x^* = x$ and $x^2=x$
\item {\bf partial isometry} if $x^*x$ and $xx^*$ are both projections
\end{itemize}

\subsubsection{Properties of the special elements in terms of their spectrum}

In the following $\sigma(x)$ denotes the spectrum of an element $x$ and $R_{\sigma}(x)$ its spectral radius.

{\bf Theorem 1 -} Suppose $\mathcal{A}$ is a $C^*$-algebra and $x \in \mathcal{A}$. If $x$ is normal then $\|x\| = R_{\sigma}(x)$

{\bf Theorem 2 -} Suppose $\mathcal{A}$ is a $C^*$-algebra and $x \in \mathcal{A}$.

\begin{itemize}
\item If $x$ is self-adjoint, then $\sigma(x) \subset \mathbb{R}$.
\item If $x$ is unitary, then $\sigma(x) \subset \partial D$, where $D \subset \mathbb{C}$ is the unit disk.
\item If $x$ is positive, then $\sigma(x) \subset \mathbb{R}^{+}$
\item If $x$ is a projection, then $\sigma(x) \subset \{0,1\}$
\end{itemize}

{\bf Theorem 3 -} Suppose $\mathcal{A}$ is a commutative $C^*$-algebra and $x \in \mathcal{A}$. Then

\begin{itemize}
\item $x$ is self-adjoint if and only if $\sigma(x) \subset \mathbb{R}$.
\item $x$ is unitary if and only if $\sigma(x) \subset \partial D$, where $D \subset \mathbb{C}$ is the unit disk.
\item $x$ is positive if and only if $\sigma(x) \subset \mathbb{R}^{+}$
\item $x$ is a projection if and only if $\sigma(x) \subset \{0,1\}$
\end{itemize}

{\bf Theorem 4 -} Suppose $\mathcal{A}$ is a $C^*$-algebra and $x$ is normal in $\mathcal{A}$. Then

\begin{itemize}
\item $x$ is self-adjoint if and only if $\sigma(x) \subset \mathbb{R}$.
\item $x$ is unitary if and only if $\sigma(x) \subset \partial D$, where $D \subset \mathbb{C}$ is the unit disk.
\item $x$ is positive if and only if $\sigma(x) \subset \mathbb{R}^{+}$
\item $x$ is a projection if and only if $\sigma(x) \subset \{0,1\}$
\end{itemize}

%%%%%
%%%%%
\end{document}
