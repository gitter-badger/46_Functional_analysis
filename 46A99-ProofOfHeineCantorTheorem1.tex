\documentclass[12pt]{article}
\usepackage{pmmeta}
\pmcanonicalname{ProofOfHeineCantorTheorem}
\pmcreated{2013-03-22 13:31:26}
\pmmodified{2013-03-22 13:31:26}
\pmowner{paolini}{1187}
\pmmodifier{paolini}{1187}
\pmtitle{proof of Heine-Cantor theorem}
\pmrecord{5}{34114}
\pmprivacy{1}
\pmauthor{paolini}{1187}
\pmtype{Proof}
\pmcomment{trigger rebuild}
\pmclassification{msc}{46A99}

\endmetadata

% this is the default PlanetMath preamble.  as your knowledge
% of TeX increases, you will probably want to edit this, but
% it should be fine as is for beginners.

% almost certainly you want these
\usepackage{amssymb}
\usepackage{amsmath}
\usepackage{amsfonts}

% used for TeXing text within eps files
%\usepackage{psfrag}
% need this for including graphics (\includegraphics)
%\usepackage{graphicx}
% for neatly defining theorems and propositions
%\usepackage{amsthm}
% making logically defined graphics
%%%\usepackage{xypic}

% there are many more packages, add them here as you need them

% define commands here
\begin{document}
We prove this theorem in the case when $X$ and $Y$ are metric spaces.

Suppose $f$ is not uniformly continuous. Then
\[
  \exists \epsilon>0\ \forall \delta>0\ \exists x,y\in X \quad
  d(x,y)< \delta \ \mathrm{but}\ d(f(x),f(y))\ge \epsilon.
\]
In particular by letting $\delta=1/k$ we can construct two sequences $x_k$ and $y_k$ such that
\[
  d(x_k,y_k) < 1/k\ \mathrm{and}\ d(f(x_k),f(y_k)\ge \epsilon.
\]

Since $X$ is compact the two sequence have convergent subsequences i.e. 
\[
  x_{k_j} \to \bar x \in X, \quad y_{k_j} \to \bar y \in X.
\]
Since $d(x_k,y_k)\to 0$ we have $\bar x = \bar y$. Being $f$ continuous we hence conclude $d(f(x_{k_j}),f(y_{k_j})) \to 0$ which is a contradiction being $d(f(x_k),f(y_k))\ge \epsilon$.
%%%%%
%%%%%
\end{document}
