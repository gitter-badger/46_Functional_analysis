\documentclass[12pt]{article}
\usepackage{pmmeta}
\pmcanonicalname{TopologyOfLocallyConvexSpacesIsGeneratedBySeminorms}
\pmcreated{2013-03-22 17:43:29}
\pmmodified{2013-03-22 17:43:29}
\pmowner{asteroid}{17536}
\pmmodifier{asteroid}{17536}
\pmtitle{topology of locally convex spaces is generated by seminorms}
\pmrecord{4}{40171}
\pmprivacy{1}
\pmauthor{asteroid}{17536}
\pmtype{Theorem}
\pmcomment{trigger rebuild}
\pmclassification{msc}{46-00}
\pmclassification{msc}{46A03}

% this is the default PlanetMath preamble.  as your knowledge
% of TeX increases, you will probably want to edit this, but
% it should be fine as is for beginners.

% almost certainly you want these
\usepackage{amssymb}
\usepackage{amsmath}
\usepackage{amsfonts}

% used for TeXing text within eps files
%\usepackage{psfrag}
% need this for including graphics (\includegraphics)
%\usepackage{graphicx}
% for neatly defining theorems and propositions
%\usepackage{amsthm}
% making logically defined graphics
%%%\usepackage{xypic}

% there are many more packages, add them here as you need them

% define commands here

\begin{document}
{\bf Theorem -} Let $V$ be a topological vector space over $\mathbb{R}$ or $\mathbb{C}$. Then $V$ is \PMlinkname{locally convex}{LocallyConvexTopologicalVectorSpace} if and only if the topology of $V$ is \PMlinkescapetext{generated by} a family of seminorms.

Moreover, $V$ is Hausdorff and locally \PMlinkescapetext{convex} if and only if the topology of $V$ is \PMlinkescapetext{generated by} a \PMlinkescapetext{separating} family of seminorms.
%%%%%
%%%%%
\end{document}
