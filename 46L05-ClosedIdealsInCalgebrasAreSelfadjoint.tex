\documentclass[12pt]{article}
\usepackage{pmmeta}
\pmcanonicalname{ClosedIdealsInCalgebrasAreSelfadjoint}
\pmcreated{2013-03-22 17:30:42}
\pmmodified{2013-03-22 17:30:42}
\pmowner{asteroid}{17536}
\pmmodifier{asteroid}{17536}
\pmtitle{closed ideals in $C^*$-algebras are self-adjoint}
\pmrecord{9}{39902}
\pmprivacy{1}
\pmauthor{asteroid}{17536}
\pmtype{Theorem}
\pmcomment{trigger rebuild}
\pmclassification{msc}{46L05}
\pmclassification{msc}{46H10}

\endmetadata

% this is the default PlanetMath preamble.  as your knowledge
% of TeX increases, you will probably want to edit this, but
% it should be fine as is for beginners.

% almost certainly you want these
\usepackage{amssymb}
\usepackage{amsmath}
\usepackage{amsfonts}

% used for TeXing text within eps files
%\usepackage{psfrag}
% need this for including graphics (\includegraphics)
%\usepackage{graphicx}
% for neatly defining theorems and propositions
%\usepackage{amsthm}
% making logically defined graphics
%%%\usepackage{xypic}

% there are many more packages, add them here as you need them

% define commands here

\begin{document}
\PMlinkescapeword{self-adjoint}
\PMlinkescapeword{closed}

{\bf Theorem -} Every \PMlinkname{closed}{ClosedSet} \PMlinkname{two-sided ideal}{IdealOfAnAlgebra} $\mathcal{I}$ of a \PMlinkname{$C^*$-algebra}{CAlgebra} $\mathcal{A}$ is \PMlinkname{self-adjoint}{InvolutaryRing}, i.e.
\begin{center}
if $x \in \mathcal{I}$ then $x^* \in \mathcal{I}$.

\end{center}

{\bf Proof :} Let $\mathcal{I}^* := \{a^* : a \in \mathcal{I}\}$.

Since $\mathcal{I}$ is closed and the involution mapping is continuous, it follows that $\mathcal{I}^*$ is also closed.

We claim that $\mathcal{I}^*$ is also a \PMlinkescapetext{two-sided ideal} of $\mathcal{A}$. To see this let $a,b \in \mathcal{I}$, $x \in \mathcal{A}$ and $\lambda \in \mathbb{C}$. Then
\begin{itemize}
\item $a^* + \lambda b^* = (a+\overline{\lambda}b)^* \in \mathcal{I}^*$ since $a + \overline{\lambda} b \in \mathcal{I}$ 
\item $xa^* =(ax^*)^* \in \mathcal{I}^*$ since $ax^* \in \mathcal{I}$.
\item $a^*x=(x^*a)^* \in \mathcal{I}^*$ since $x^*a \in \mathcal{I}$
\end{itemize}

Let $\mathcal{B} := \mathcal{I} \cap \mathcal{I}^*$.

$\mathcal{B}$ is a $C^*$-subalgebra of $\mathcal{A}$ (it is a norm-closed, involution-closed, subalgebra of $\mathcal{A}$).

It is known that every $C^*$-algebra has an approximate identity consisting of positive elements with norm less than $1$ (see this \PMlinkname{entry}{CAlgebrasHaveApproximateIdentities}).

Let $(e_{\lambda})_{\lambda \in \Lambda}$ be an approximate identity for $\mathcal{B}$ with the above \PMlinkescapetext{properties}:
\begin{enumerate}
\item each $e_{\lambda}$ is positive (hence self-adjoint) and
\item $\|e_{\lambda}\| \leq 1\;\;\;\forall_{\lambda \in \Lambda}$
\end{enumerate}  

We now prove $\mathcal{I}$ is self-adjoint:

Let $a \in \mathcal{I}$. We have that
\begin{eqnarray*}
\|a^*-a^*e_{\lambda}\|^2 & = & \|(a^*-a^*e_{\lambda})^*\cdot(a^*-a^*e_{\lambda})\| \\
& = & \|(a- e_{\lambda}a)\cdot(a^*- a^*e_{\lambda})\| \\
& = & \|(aa^*-aa^*e_{\lambda}) -e_{\lambda}(aa^*-aa^*e_{\lambda})\| \\
& \leq & \|aa^*-aa^*e_{\lambda}\| + \|e_{\lambda}\|\cdot \|aa^*-aa^*e_{\lambda}\| \\
& \leq & \|aa^*-aa^*e_{\lambda}\| + \|aa^*-aa^*e_{\lambda}\| \\
& = & 2 \|aa^*-aa^*e_{\lambda}\|
\end{eqnarray*}

Taking limits in both \PMlinkescapetext{sides} we obtain
\begin{displaymath}
\lim_{\lambda}\|a^*-a^*e_{\lambda}\|^2 \leq \lim_{\lambda}\;2 \|aa^*-aa^*e_{\lambda}\| = 0
\end{displaymath}
since $aa^* \in \mathcal{I}\cap\mathcal{I}^*=\mathcal{B}$ and $(e_{\lambda})_{\lambda \in\Lambda}$ is an approximate identity for $\mathcal{B}$.

As $e_{\lambda} \in \mathcal{I}$ we see that $a^*e_{\lambda} \in \mathcal{I}$.

We conclude from the limit above that $a^*$ is in the closure of $\mathcal{I}$. Therefore $a^*\in \mathcal{I}$.

Hence, $\mathcal{I}$ is self-adjoint. $\square$
%%%%%
%%%%%
\end{document}
