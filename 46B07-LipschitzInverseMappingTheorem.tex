\documentclass[12pt]{article}
\usepackage{pmmeta}
\pmcanonicalname{LipschitzInverseMappingTheorem}
\pmcreated{2013-03-22 14:25:13}
\pmmodified{2013-03-22 14:25:13}
\pmowner{Koro}{127}
\pmmodifier{Koro}{127}
\pmtitle{Lipschitz inverse mapping theorem}
\pmrecord{7}{35927}
\pmprivacy{1}
\pmauthor{Koro}{127}
\pmtype{Theorem}
\pmcomment{trigger rebuild}
\pmclassification{msc}{46B07}
\pmclassification{msc}{47J07}

% this is the default PlanetMath preamble.  as your knowledge
% of TeX increases, you will probably want to edit this, but
% it should be fine as is for beginners.

% almost certainly you want these
\usepackage{amssymb}
\usepackage{amsmath}
\usepackage{amsfonts}
\usepackage{mathrsfs}

% used for TeXing text within eps files
%\usepackage{psfrag}
% need this for including graphics (\includegraphics)
%\usepackage{graphicx}
% for neatly defining theorems and propositions
%\usepackage{amsthm}
% making logically defined graphics
%%%\usepackage{xypic}

% there are many more packages, add them here as you need them

% define commands here
\newcommand{\lip}{\operatorname{Lip}}
\newcommand{\C}{\mathbb{C}}
\newcommand{\R}{\mathbb{R}}
\newcommand{\N}{\mathbb{N}}
\newcommand{\Z}{\mathbb{Z}}
\newcommand{\Per}{\operatorname{Per}}
\begin{document}
Let $(E,\|\cdot\|)$ be a Banach space and let $A\colon E\to E$ be a
bounded linear isomorphism with
bounded inverse (i.e. a topological linear automorphism);
let $B(r)$ be the ball with center 0
and radius $r$ (we allow $r=\infty$). Then for any Lipschitz map
$\phi\colon B(r)\to E$
such that $\lip \phi < \|A^{-1}\|^{-1}$ and $\phi(0)=0$, there are open sets
$U\subset E$ and $V\subset B(r)$ and a map $T \colon U\to V$ such that $T(A+\phi) = I|_V$ and $(A+\phi)T = I|_U$.
In other words, there is a local inverse of $A+\phi$ near zero. Furthermore, the inverse $T$ is Lipschitz with $\lip T \leq (\|A\|+\lip \phi)^{-1}$ and
$$B\left(r(\|A^{-1}\|^{-1} - \lip \phi)\right)\subset U.$$

\emph{Remark.} The inclusion above implies that $A+\phi\colon E\to E$ is invertible if $r=\infty$.

\emph{Remark.} $\lip \phi$ denotes the smallest Lipschitz constant of $\phi$.
%%%%%
%%%%%
\end{document}
