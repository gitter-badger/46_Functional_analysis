\documentclass[12pt]{article}
\usepackage{pmmeta}
\pmcanonicalname{WeakTopology}
\pmcreated{2013-03-22 15:07:05}
\pmmodified{2013-03-22 15:07:05}
\pmowner{jirka}{4157}
\pmmodifier{jirka}{4157}
\pmtitle{weak-* topology}
\pmrecord{8}{36855}
\pmprivacy{1}
\pmauthor{jirka}{4157}
\pmtype{Definition}
\pmcomment{trigger rebuild}
\pmclassification{msc}{46A03}
\pmsynonym{weak-* topology}{WeakTopology}
\pmsynonym{weak-$*$ topology}{WeakTopology}
\pmsynonym{weak-star topology}{WeakTopology}
\pmrelated{WeakHomotopyAdditionLemma}
\pmdefines{weak topology}

\endmetadata

% this is the default PlanetMath preamble.  as your knowledge
% of TeX increases, you will probably want to edit this, but
% it should be fine as is for beginners.

% almost certainly you want these
\usepackage{amssymb}
\usepackage{amsmath}
\usepackage{amsfonts}

% used for TeXing text within eps files
%\usepackage{psfrag}
% need this for including graphics (\includegraphics)
%\usepackage{graphicx}
% for neatly defining theorems and propositions
\usepackage{amsthm}
% making logically defined graphics
%%%\usepackage{xypic}

% there are many more packages, add them here as you need them

% define commands here
\theoremstyle{theorem}
\newtheorem*{thm}{Theorem}
\newtheorem*{lemma}{Lemma}
\newtheorem*{conj}{Conjecture}
\newtheorem*{cor}{Corollary}
\newtheorem*{example}{Example}
\newtheorem*{prop}{Proposition}
\theoremstyle{definition}
\newtheorem*{defn}{Definition}
\theoremstyle{remark}
\newtheorem*{rmk}{Remark}
\begin{document}
Let $X$ be a locally convex topological vector space (over $\mathbb{C}$ or $\mathbb{R}$), and let $X^*$ be the set of continuous linear functionals on $X$ (the continuous dual of $X$). 
If $f \in X^*$ then let $p_{f}$ denote the seminorm $p_f(x) = \lvert f(x) \rvert$, and let $p_x(f)$ denote the seminorm $p_x(f) = \lvert f(x) \rvert$.
Obviously any normed space is a locally convex topological vector space so $X$ could be a normed space.

\begin{defn}
The topology on $X$ defined by the seminorms $\{ p_f \mid f \in X^* \}$ is called the {\em weak topology} and the topology on $X^*$ defined by the seminorms $\{ p_x \mid x \in X \}$ is called the {\em weak-$*$ topology}.
\end{defn}

The weak topology on $X$ is usually denoted by $\sigma(X,X^*)$ and the weak-$*$
topology on $X^*$ is usually denoted by $\sigma(X^*,X)$.  Another common notation is $(X,wk)$ and $(X^*,wk-*)$

Topology defined on a space $Y$ by seminorms $p_\iota$, $\iota \in I$ means that we take the sets $\{ y \in Y \mid p_\iota(y) < \epsilon \}$ for all $\iota \in I$ and $\epsilon > 0$ as a subbase for the topology (that is finite intersections of such sets form the basis).

The most striking result about weak-$*$ topology is the Alaoglu's theorem which asserts that for $X$ being a normed space, a closed ball (in the operator norm) of $X^*$ is weak-$*$ compact.  There is no similar result for the weak topology on $X$, unless $X$ is a reflexive space.

Note that $X^*$ is sometimes used for the algebraic dual of a space and $X'$ is used for the continuous dual.  In functional analysis $X^*$ always means the continuous dual and hence the term {\em weak-$*$ topology}.

\begin{thebibliography}{9}
\bibitem{Conway:funcanal}
John B.\@ Conway.
{\em \PMlinkescapetext{A Course in Functional Analysis}},
Springer-Verlag, New York, New York, 1990.
\end{thebibliography}
%%%%%
%%%%%
\end{document}
