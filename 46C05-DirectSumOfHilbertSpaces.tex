\documentclass[12pt]{article}
\usepackage{pmmeta}
\pmcanonicalname{DirectSumOfHilbertSpaces}
\pmcreated{2013-03-22 14:43:55}
\pmmodified{2013-03-22 14:43:55}
\pmowner{asteroid}{17536}
\pmmodifier{asteroid}{17536}
\pmtitle{direct sum of Hilbert spaces}
\pmrecord{10}{36363}
\pmprivacy{1}
\pmauthor{asteroid}{17536}
\pmtype{Definition}
\pmcomment{trigger rebuild}
\pmclassification{msc}{46C05}
\pmrelated{CategoryOfHilbertSpaces}

% this is the default PlanetMath preamble.  as your 

% almost certainly you want these
\usepackage{amssymb}
\usepackage{amsmath}
\usepackage{amsfonts}

% used for TeXing text within eps files
%\usepackage{psfrag}
% need this for including graphics (\includegraphics)
%\usepackage{graphicx}
% for neatly defining theorems and propositions
%\usepackage{amsthm}
% making logically defined graphics
%%%\usepackage{xypic}

% there are many more packages, add them here as you need 

% define commands here

\begin{document}
Let $\{H_i\}_{i \in I}$ be a family of Hilbert spaces indexed by a set $I$.  The direct sum of this family of Hilbert spaces, denoted as
 $$\bigoplus_{i \in I} H_i$$
consists of all elements $v$ of the \PMlinkname{Cartesian product}{GeneralizedCartesianProduct} of $\{H_i\}_{i \in I}$ such that $\sum \| v_i\|^2 < \infty$. Of course, for the previous sum to be finite only at most a countable number of $v_i$ can be non-zero.

Vector addition and scalar multiplication are defined termwise:  If $u, v \in \bigoplus_{i \in I} H_i$, then $(u+v)_i = u_i + v_i$ and $(sv)_i = s v_i$.

 The inner product of two vectors is defined as
 $$\langle u, v \rangle = \sum_{i \in I} \langle u_i, v_i \rangle$$


Linked PDF file:

http://images.planetmath.org/cache/objects/6363/pdf/DirectSumOfHilbertSpaces.pdf

%%%%%
%%%%%
\end{document}
