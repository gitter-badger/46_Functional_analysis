\documentclass[12pt]{article}
\usepackage{pmmeta}
\pmcanonicalname{ProofThatLpSpacesAreComplete}
\pmcreated{2013-03-22 14:40:09}
\pmmodified{2013-03-22 14:40:09}
\pmowner{Simone}{5904}
\pmmodifier{Simone}{5904}
\pmtitle{proof that $L^p$ spaces are complete}
\pmrecord{8}{36270}
\pmprivacy{1}
\pmauthor{Simone}{5904}
\pmtype{Proof}
\pmcomment{trigger rebuild}
\pmclassification{msc}{46B25}

\endmetadata

% this is the default PlanetMath preamble.  as your knowledge
% of TeX increases, you will probably want to edit this, but
% it should be fine as is for beginners.

% almost certainly you want these
\usepackage{amssymb}
\usepackage{amsmath}
\usepackage{amsfonts}

% used for TeXing text within eps files
%\usepackage{psfrag}
% need this for including graphics (\includegraphics)
%\usepackage{graphicx}
% for neatly defining theorems and propositions
%\usepackage{amsthm}
% making logically defined graphics
%%%\usepackage{xypic}

% there are many more packages, add them here as you need them

% define commands here
\begin{document}
Let's prove completeness for the classical Banach spaces, say $L^p[0,1]$
where $p \geq 1$. 

Since the case $p=\infty$ is elementary, we may assume $1 \le p < \infty$. 
Let $[f_{\cdot}] \in (L^p)^{\mathbf{N}}$ be a Cauchy sequence. Define $[g_0] := [f_0]$ and for $n > 0$ define $[g_n] := [f_n - f_{n-1}]$. Then $[\sum_{n=0}^N g_n] = [f_N]$ and we see that
$$\sum_{n=0}^\infty \|g_n\| = \sum_{n=0}^\infty \|f_n - f_{n-1}\| \leq ??? < \infty.$$
Thus it suffices to prove that etc.
%$\|f_n - f_m\| < \varepsilon$
%Define $$F(x) := \left\{\begin{array}{cc}\lim f_{\cdot}(x) & \mbox{if it %exists}\\0 & \mbox{otherwise}\end{array}\right.$$ and calculate its $L^p$-norm. %We have $$\|F\|^p = \int |F|^p \mathrm{d}\mu = $$

It suffices to prove that each absolutely summable series in $L^p$ is 
summable in $L^p$ to some element in $L^p$.

Let $\{f_n\}$ be a sequence in $L^p$ with 
$\sum_{n=1}^\infty \|f_n\|=M<\infty$, and define functions $g_n$ by 
setting $g_n(x)=\sum_{k=1}^n|f_k(x)|$. From the Minkowski inequality we 
have
$$
\|g_n\|\le\sum_{k=1}^n\|f_k\|\le M.
$$
Hence
$$
\int g_n^p\le M^p.
$$
For each $x$, $\{g_n(x)\}$ is an increasing sequence of (extended) real 
numbers and so must converge to an extended real number $g(x)$. The 
function $g$ so defined is measurable, and, since $g_n\ge 0$, we have
$$
\int g^p\le M^p
$$
by Fatou's Lemma. Hence $g^p$ is integrable, and $g(x)$ is finite for 
almost all $x$.

For each $x$ such that $g(x)$ is finite the series $\sum_{k=1}^\infty f_k(x)$ 
is an absolutely summable series of real numbers and so must be summable 
to a real number $s(x)$. If we set $s(x)=0$ for those $x$ where 
$g(x)=\infty$, we have defined a function $s$ which is the limit almost 
everywhere of the partial sums $s_n=\sum_{k=1}^n f_k$. Hence $s$ is 
measurable. Since $|s_n(x)|\le g(x)$, we have $|s(x)|\le g(x)$. 
Consequently, $s$ is in $L^p$ and we have
$$
|s_n(x)-s(x)|^p\le 2^p\,[g(x)]^p.
$$
Since $2^pg^p$ is integrable and $|s_n(x)-s(x)|^p$ converges to $0$ for 
almost all $x$, we have
$$
\int|s_n-s|^p\to 0
$$
by the Lebesgue Convergence Theorem. Thus $\|s_n-s\|^p\to 0$, whence 
$\|s_n-s\|\to 0$. Consequently, the series $\{f_n\}$ has in $L^p$ the sum 
$s$.
\begin{thebibliography}
{}Royden, H. L. \emph{Real analysis. Third edition}. Macmillan Publishing Company, New York, 1988.
\end{thebibliography}
%%%%%
%%%%%
\end{document}
