\documentclass[12pt]{article}
\usepackage{pmmeta}
\pmcanonicalname{GaborFrame}
\pmcreated{2013-03-22 17:08:28}
\pmmodified{2013-03-22 17:08:28}
\pmowner{ErlendA}{6587}
\pmmodifier{ErlendA}{6587}
\pmtitle{Gabor frame}
\pmrecord{5}{39448}
\pmprivacy{1}
\pmauthor{ErlendA}{6587}
\pmtype{Definition}
\pmcomment{trigger rebuild}
\pmclassification{msc}{46C99}
\pmdefines{Gabor frame}
\pmdefines{Gabor super-frame}
\pmdefines{Vector-valued Gabor frame}

% this is the default PlanetMath preamble.  as your knowledge
% of TeX increases, you will probably want to edit this, but
% it should be fine as is for beginners.

% almost certainly you want these
\usepackage{amssymb}
\usepackage{amsmath}
\usepackage{amsfonts}

% used for TeXing text within eps files
%\usepackage{psfrag}
% need this for including graphics (\includegraphics)
%\usepackage{graphicx}
% for neatly defining theorems and propositions
\usepackage{amsthm}
% making logically defined graphics
%%%\usepackage{xypic}

% there are many more packages, add them here as you need them

% define commands here

\newtheorem*{defn}{Definition}
\newtheorem*{thm}{Theorem}
\newtheorem*{cor}{Corollary}
\newtheorem*{lemma}{Lemma}
\newtheorem*{conj}{Conjecture}
\newtheorem*{prop}{Proposition}
\newtheorem*{open}{Open Question}
\newtheorem*{cond}{Condition}
\newenvironment{skproof}{\noindent\emph{Sketch Proof.}}{\qed\\}
\newtheorem*{rem}{Remark}

\newcommand{\Z}{\ensuremath{\mathbb{Z}}}
\newcommand{\N}{\ensuremath{\mathbb{N}}}
\newcommand{\Q}{\ensuremath{\mathbb{Q}}}
\newcommand{\R}{\ensuremath{\mathbb{R}}}

\newcommand{\eq}{\ensuremath{\thicksim}}
\newcommand{\iso}{\cong}
\newcommand{\normal}{\lhd}
\newcommand{\Aut}[1]{{\ensuremath{\mathrm{Aut}({#1})}}}
\newcommand{\GL}[2]{{\ensuremath{\mathrm{GL}({#1},{#2})}}}
\newcommand{\pad}{\ensuremath{\Box}}
\begin{document}
\noindent One may be interested in Gabor frames and its related theory if one looks further into the frame framework. First, denote a lattice by $\Lambda=A \mathbb{Z}^{2d}$, where $A$ is an invertible matrix, and let $\pi(\xi,\phi)f=e^{2 \pi i \xi x}f(x-\phi)$

\begin{defn}
Let $g \in L^2(\mathbb{R}^d)$ be a nonzero window, and let $\lambda \in \Lambda$, then
\begin{equation*}
G(g,\lambda)=\left\{ \pi(\lambda)g : \lambda \in \Lambda \right\}
\end{equation*}
\noindent is a Gabor system. If $G(g,\lambda)$ is a frame, it's called a Gabor frame for $L^2(\mathbb{R}^d)$


\end{defn}

Supose now that one wants to look at a more general framework, and work with functions in $L^2(\mathbb{R}^d,\mathbb{C}^n)$. Then the definition above generalises to

\begin{defn}
Let $\boldsymbol{g} \in L^2(\mathbb{R}^d,\mathbb{C}^n)$ be a nonzero window and let $\lambda \in \Lambda$, then

\begin{equation*}
\boldsymbol{G}(\boldsymbol{g},\lambda)=\left\{ \pi(\lambda) \boldsymbol{g} : \lambda \in \Lambda \right\}
\end{equation*}

\noindent is a Gabor super-frame if the frame inequalities hold, where
\[ 
\pi( \xi, \phi ) \boldsymbol{g}=e^{2 \pi \i x \cdot \xi}\left( g_1(x-\phi),g_2(x-\phi),...,g_n(x-\phi) \right)
\]

\noindent and for $ \boldsymbol{f},\boldsymbol{h} \in L^2(\mathbb{R}^d,\mathbb{C}^n) $
\[
\left\langle \boldsymbol{f},\boldsymbol{h} \right\rangle_{L^2(\mathbb{R}^d,\mathbb{C}^n)} 
= \sum_{i=1}^n \left\langle f_i , h_i \right\rangle_{L^2(\mathbb{R}^d)}
\]


\end{defn}



\begin{thebibliography}{1}
\bibitem{gg} Karlheinz Gröchenig, "Foundations of Time-Frequency Analysis," {\it Birkhhäuser} (2000)
\end{thebibliography}
%%%%%
%%%%%
\end{document}
