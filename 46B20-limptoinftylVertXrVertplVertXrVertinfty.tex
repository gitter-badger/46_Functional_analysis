\documentclass[12pt]{article}
\usepackage{pmmeta}
\pmcanonicalname{limptoinftylVertXrVertplVertXrVertinfty}
\pmcreated{2013-03-22 14:02:53}
\pmmodified{2013-03-22 14:02:53}
\pmowner{Koro}{127}
\pmmodifier{Koro}{127}
\pmtitle{$\lim_{p \to \infty} \lVert x \rVert_p = \lVert x \rVert_{\infty}$}
\pmrecord{12}{35399}
\pmprivacy{1}
\pmauthor{Koro}{127}
\pmtype{Result}
\pmcomment{trigger rebuild}
\pmclassification{msc}{46B20}
\pmrelated{PowerMean}

\endmetadata

% this is the default PlanetMath preamble.  as your knowledge
% of TeX increases, you will probably want to edit this, but
% it should be fine as is for beginners.

% almost certainly you want these
\usepackage{amssymb}
\usepackage{amsmath}
\usepackage{amsfonts}

% used for TeXing text within eps files
%\usepackage{psfrag}
% need this for including graphics (\includegraphics)
%\usepackage{graphicx}
% for neatly defining theorems and propositions
%\usepackage{amsthm}
% making logically defined graphics
%%%\usepackage{xypic}

% there are many more packages, add them here as you need them

% define commands here

\newcommand{\sR}[0]{\mathbb{R}}
\newcommand{\sC}[0]{\mathbb{C}}
\newcommand{\sN}[0]{\mathbb{N}}
\newcommand{\sZ}[0]{\mathbb{Z}}

% The below lines should work as the command
% \renewcommand{\bibname}{References}
% without creating havoc when rendering an entry in 
% the page-image mode.
\makeatletter
\@ifundefined{bibname}{}{\renewcommand{\bibname}{References}}
\makeatother

\newcommand*{\norm}[1]{\lVert #1 \rVert}
\newcommand*{\abs}[1]{| #1 |}
\begin{document}
Suppose $x=(x_1,\ldots, x_n)$ is a point in $\sR^n$, and let $\norm{x}_p$
and $\norm{x}_\infty$ be the usual $p$-norm and $\infty$-norm;
\begin{eqnarray*}
\norm{x}_p &=& \big(|x_1|^p + \cdots + |x_n|^p\big)^{1/p},\\
\norm{x}_\infty &=& \max \{|x_1|, \ldots, |x_n|\}.
\end{eqnarray*}

Our claim is that
\begin{eqnarray}
\label{claim}
\lim_{p \to \infty} \norm{x}_p &=& \norm{x}_{\infty}.
\end{eqnarray}
In other words, for any fixed $x\in \sR^n$, the above limit holds.
This, or course,  justifies the notation for the $\infty$-norm.

{\bf Proof.}
Since both norms
stay invariant if we exchange two components in $x$, we can arrange things
such that $\norm{x}_\infty = |x_1|$. Then for any real $p>0$, we have
\begin{eqnarray*}
\norm{x}_\infty  &=&\abs{x_1} =(\abs{x_1}^p)^{1/p} \leq \norm{x}_p
\end{eqnarray*}
and
\begin{eqnarray*}
\norm{x}_p  &\le & n^{1/p} |x_1| =n^{1/p} \norm{x}_\infty.
\end{eqnarray*}
Taking the limit of the above inequalities (see 
\PMlinkname{this page}{InequalityForRealNumbers})
we obtain
\begin{eqnarray*}
\norm{x}_\infty  &\le& \lim_{p\to \infty} \norm{x}_p,\\
\lim_{p\to \infty} \norm{x}_p  &\le & \norm{x}_\infty,
\end{eqnarray*}
which combined yield the result. $\Box$
%%%%%
%%%%%
\end{document}
