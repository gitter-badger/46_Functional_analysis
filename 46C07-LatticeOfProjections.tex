\documentclass[12pt]{article}
\usepackage{pmmeta}
\pmcanonicalname{LatticeOfProjections}
\pmcreated{2013-03-22 17:53:29}
\pmmodified{2013-03-22 17:53:29}
\pmowner{asteroid}{17536}
\pmmodifier{asteroid}{17536}
\pmtitle{lattice of projections}
\pmrecord{14}{40377}
\pmprivacy{1}
\pmauthor{asteroid}{17536}
\pmtype{Feature}
\pmcomment{trigger rebuild}
\pmclassification{msc}{46C07}
\pmclassification{msc}{46C05}
\pmclassification{msc}{06C15}
\pmclassification{msc}{46L10}
\pmsynonym{projections in Hilbert spaces}{LatticeOfProjections}
\pmrelated{OrthomodularLattice}
\pmrelated{QuantumLogic}
\pmrelated{ContinuousGeometry}
\pmdefines{minimal projection}

\endmetadata

% this is the default PlanetMath preamble.  as your knowledge
% of TeX increases, you will probably want to edit this, but
% it should be fine as is for beginners.

% almost certainly you want these
\usepackage{amssymb}
\usepackage{amsmath}
\usepackage{amsfonts}

% used for TeXing text within eps files
%\usepackage{psfrag}
% need this for including graphics (\includegraphics)
%\usepackage{graphicx}
% for neatly defining theorems and propositions
%\usepackage{amsthm}
% making logically defined graphics
%%%\usepackage{xypic}

% there are many more packages, add them here as you need them

% define commands here

\begin{document}
\PMlinkescapeword{properties}
\PMlinkescapeword{satisfy}
\PMlinkescapeword{satisfies}
\PMlinkescapeword{structure}
\PMlinkescapeword{type}
\PMlinkescapeword{orthogonal}
\PMlinkescapeword{perpendicular}
\PMlinkescapeword{terms}
\PMlinkescapeword{minimal}
\PMlinkescapeword{section}
\PMlinkescapeword{corolary}

Let $H$ be a Hilbert space and $B(H)$ the algebra of bounded operators in $H$. By a projection in $B(H)$ we always \PMlinkescapetext{mean} an orthogonal projection.

Recall that a projection $P$ in $B(H)$ is a \PMlinkname{bounded}{BoundedOperator} self-adjoint operator satisfying $P^2=P$.

The set of projections in $B(H)$, although not forming a vector space, has a very rich structure. In this entry we are going to endow this set with a partial ordering in a \PMlinkescapetext{way} that it becomes a complete lattice. The lattice structure of the set of projections has profound consequences on the structure of von Neumann algebras.

\section{ The Lattice of Projections}

In Hilbert spaces there is a bijective correspondence between closed subspaces and projections (see \PMlinkname{this entry}{ProjectionsAndClosedSubspaces}). This correspondence is given by
\begin{displaymath}
P \;\longleftrightarrow\; \mathrm{Ran}(P)
\end{displaymath}
where $P$ is a projection and $\mathrm{Ran}(P)$ denotes the range of $P$.

Since the set of closed subspaces can be partially ordered by inclusion, we can define a partial order $\leq$ in the set of projections using the above correspondence:
\begin{displaymath}
P \leq Q\;\; \Longleftrightarrow \;\;\mathrm{Ran}(P) \subseteq \mathrm{Ran}(Q)
\end{displaymath}


But since projections are self-adjoint operators (in fact they are positive operators, as $P=P^*P$), they inherit the natural \PMlinkname{partial ordering of self-adjoint operators}{OrderingOfSelfAdjoints}, which we denote by $\leq_{sa}$, and whose definition we recall now
\begin{displaymath}
P \leq_{sa} Q \;\; \Longleftrightarrow \;\; Q - P \text{is a positive operator}
\end{displaymath}
As the following theorem shows, these two orderings coincide. Thus, we shall not make any more distinctions of notation between them.

$\,$

{\bf Theorem 1 -} Let $P, Q$ be projections in $B(H)$. The following conditions are equivalent:
\begin{itemize}
\item $\mathrm{Ran}(P) \subseteq \mathrm{Ran}(Q)\;\;$ (i.e. $P \leq Q$) 
\item $QP = P$
\item $PQ = P$
\item $\|Px\| \leq \|Qx\|$ for all $x \in H$
\item $P \leq_{sa} Q$
\end{itemize}

$\;$

Two closed subspaces $Y, Z$ in $H$ have a greatest lower bound $Y \land Z$ and a least upper bound $Y \lor Z$. Specifically, $Y \land Z$ is precisely the intersection $Y \cap Z$ and $Y \lor Z$ is precisely the closure of the subspace generated by $Y$ and $Z$. Hence, if $P, Q$ are projections in $B(H)$ then $P \land Q$ is the projection onto $\mathrm{Ran}(P) \cap \mathrm{Ran}(Q)$ and $P \lor Q$ is the projection onto the closure of $\mathrm{Ran}(P) + \mathrm{Ran}(Q)$.

The above discussion clarifies that the set of projections in $B(H)$ has a lattice structure. In fact, the set of projections forms a complete lattice, by somewhat \PMlinkescapetext{similar arguments} as above:

Every family $\{Y_{\alpha}\}$ of closed subspaces in $H$ possesses an infimum $\bigwedge Y_{\alpha}$ and a supremum $\bigvee Y_{\alpha}$, which are, respectively, the intersection of all $Y_{\alpha}$ and the closure of the subspace generated by all $Y_{\alpha}$. There is, of course, a correspondent in terms of projections: every family $\{P_{\alpha}\}$ of projections has an infimum $\bigwedge P_{\alpha}$ and a supremum $\bigvee P_{\alpha}$, which are, respectively, the projection onto the intersection of all $\mathrm{Ran}(P_{\alpha})$ and the projection onto the closure of the subspace generated by all $\mathrm{Ran}(P_{\alpha})$.

\section{Additional Lattice Features}
\begin{itemize}
\item The lattice of projections in $B(H)$ is never \PMlinkname{distributive}{DistributiveLattice} (unless $H$ is one-dimensional).
\end{itemize}
\begin{itemize}
\item Also, it is modular if and only if $H$ is finite dimensional. Nevertheless, there are important \PMlinkescapetext{classes} of von Neumann algebras (a particular type of subalgebras of $B(H)$ that are "rich" in projections) over an infinite-dimensional $H$, whose lattices of projections are in fact modular.
\end{itemize}
\begin{itemize}
\item Projections on one-dimensional subspaces are usually called {\bf minimal projections} and they are in fact minimal in the sense that: there are no closed subspaces strictly between $\{0\}$ and a one-dimensional subspace, and every closed subspace other than $\{0\}$ contains a one-dimensional subspace. This means that the lattice of projections in $B(H)$ is an atomic lattice and its atoms are precisely the projections on one-dimensional subspaces.

Moreover, every closed subspace of $H$ is the closure of the span of its one-dimensional subspaces. Thus, the lattice of projections in $B(H)$ is an atomistic lattice.
\end{itemize}
\begin{itemize}
\item In Hilbert spaces every closed subspace $Z$ is topologically complemented by its orthogonal complement ($H=Z \oplus Z^{\perp}$), and this fact is reflected in the structure of projections. The lattice of projections is then an orthocomplemented lattice, where the orthocomplement of each projection $P$ is the projection $I-P$ (onto $\mathrm{Ran}(P)^{\perp}$).
\end{itemize}
\begin{itemize}
\item We shall see further ahead in this entry, when we discuss orthogonal projections, that the lattice of projections in $B(H)$ is an orthomodular lattice.
\end{itemize}

\section{Commuting and Orthogonal Projections}

When two projections $P, Q$ commute, the projections $P \land Q$ and $P \lor Q$ can be described algebraically in a very \PMlinkescapetext{simple way}. We shall see at the end of this section that $P$ and $Q$ commute precisely when its corresponding subspaces $\mathrm{Ran}(P)$ and $\mathrm{Ran}(Q)$ are "perpendicular".

$\;$

{\bf Theorem 2 -} Let $P, Q$ be commuting projections (i.e. $PQ = QP$), then
\begin{itemize}
\item $P \land Q = PQ$
\item $P \lor Q = P + Q - PQ$
\item $\mathrm{Ran}(P) \lor \mathrm{Ran}(Q) = \mathrm{Ran}(P) + \mathrm{Ran}(Q)$. In particular, $\mathrm{Ran}(P) + \mathrm{Ran}(Q)$ is closed.
\end{itemize}

$\,$

Two projections $P, Q$ are said to be {\bf orthogonal} if $P \leq Q^{\perp}$. This is equivalent to say that its corresponding subspaces are orthogonal ($\mathrm{Ran}(P)$ lies in the orthogonal complement of $\mathrm{Ran}(Q)$).

$\,$

{\bf Corollary 1 -} Two projections $P, Q$ are orthogonal if and only if $PQ= 0$. When this is so, then $P \lor Q = P + Q$.

$\,$

{\bf Corollary 2 -} Let $P, Q$ be projections in $B(H)$ such that $P \leq Q$. Then $Q-P$ is the projection onto $\mathrm{Ran}(Q) \cap \mathrm{Ran}(P)^{\perp}$.

$\,$

We can now see that $P, Q$ commute if and only if $\mathrm{Ran}(P)$ and $\mathrm{Ran}(Q)$ are "perpendicular". A somewhat informal and intuitive definition of "perpendicular" is that of requiring the two subspaces to be orthogonal outside their intersection (this is different of \PMlinkescapetext{orthogonality}, since orthogonal subspaces do not \PMlinkescapetext{even} intersect each other). More rigorously, $P$ and $Q$ commute if and only if the subspaces $\mathrm{Ran}(P) \cap (\mathrm{Ran}(P) \cap \mathrm{Ran}(Q))^{\perp}$ and $\mathrm{Ran}(Q) \cap (\mathrm{Ran}(P) \cap \mathrm{Ran}(Q))^{\perp}$ are orthogonal. 

This can be proved using all the above results: The two subspaces are orthogonal iff
\begin{displaymath}
0 = (P - P \land Q)(Q - P \land Q) = PQ - P \land Q
\end{displaymath}
and $PQ = P \land Q$ iff 
\begin{displaymath}
PQ = P \land Q = (P \land Q)^* = (PQ)^*= QP
\end{displaymath}

We can now also see that the lattice of projections is orthomodular: Suppose $P \leq Q$. Then, using the above results,
\begin{displaymath}
P \lor (Q \land P^{\perp}) = P \lor (Q-P) = P + (Q-P) - P(Q-P) = Q
\end{displaymath}



\section{Nets of Projections}

In the following we discuss some useful and interesting results about convergence and limits of projections.

Let $\Lambda$ be a poset. A net of projections $\{P_{\alpha}\}_{\alpha \in \Lambda}$ is said to be increasing
if $\alpha \leq \beta \Longrightarrow P_{\alpha} \leq P_{\beta}$. Decreasing nets are
defined similarly.

$\,$

{\bf Theorem 3 -} Let $\{P_{\alpha}\}$ be an increasing net of projections. Then
 $\lim_{\alpha} P_{\alpha} x = \bigvee_{\alpha} P_{\alpha}\;x$ for every $x \in H$.

In other words, $P_{\alpha}$ converges to $\bigvee_{\alpha} P_{\alpha}$ in the strong operator
 topology.

$\,$

Similarly for decreasing nets of projections,

$\;$

{\bf Theorem 4 -} Let $\{P_{\alpha}\}$ be a decreasing net of projections. Then
 $\lim_{\alpha} P_{\alpha} x = \bigwedge_{\alpha} P_{\alpha}\;x$ for every $x \in H$.

In other words, $P_{\alpha}$ converges to $\bigwedge_{\alpha} P_{\alpha}$ in the strong operator
 topology.

$\,$

{\bf Theorem 5 -} Let $\Lambda$ be a set and $\{P_{\alpha}\}_{\alpha \in \Lambda}$ be a family of pairwise orthogonal projections. Then $\sum P_{\alpha}$ is summable and $\sum P_{\alpha}\,x = \bigvee_{\alpha} P_{\alpha}\;x$ for all $x \in H$.


%%%%%
%%%%%
\end{document}
