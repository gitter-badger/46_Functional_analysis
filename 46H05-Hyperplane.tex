\documentclass[12pt]{article}
\usepackage{pmmeta}
\pmcanonicalname{Hyperplane}
\pmcreated{2013-03-22 15:15:12}
\pmmodified{2013-03-22 15:15:12}
\pmowner{georgiosl}{7242}
\pmmodifier{georgiosl}{7242}
\pmtitle{hyperplane}
\pmrecord{9}{37035}
\pmprivacy{1}
\pmauthor{georgiosl}{7242}
\pmtype{Definition}
\pmcomment{trigger rebuild}
\pmclassification{msc}{46H05}
\pmdefines{real hyperplane}
\pmdefines{complex hyperplane}

% this is the default PlanetMath preamble.  as your knowledge
% of TeX increases, you will probably want to edit this, but
% it should be fine as is for beginners.

% almost certainly you want these
\usepackage{amssymb}
\usepackage{amsmath}
\usepackage{amsfonts}

% used for TeXing text within eps files
%\usepackage{psfrag}
% need this for including graphics (\includegraphics)
%\usepackage{graphicx}
% for neatly defining theorems and propositions
%\usepackage{amsthm}
% making logically defined graphics
%%%\usepackage{xypic}

% there are many more packages, add them here as you need them

% define commands here
\begin{document}
Let $E$ be a linear space over a field $k$.  A hyperplane $H$ in $E$ is defined as the set of the form $$H=\{x\in E:f(x)=a\}$$ where  $a \in k$ and $f$ is a nonzero linear functional, $f \colon E \to k$.  If $k=\mathbb{R}$ or $\mathbb{C}$, then $H$ is called a \emph{real hyperplane} or \emph{complex hyperplane} respectively.

\textbf{Remark}.  When $k=\mathbb{C}$, the word ``hyperplane'' also has a more restrictive meaning: it is the zero set of a complex linear functional (by setting $a=0$ above).
%%%%%
%%%%%
\end{document}
