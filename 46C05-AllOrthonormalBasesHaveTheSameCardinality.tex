\documentclass[12pt]{article}
\usepackage{pmmeta}
\pmcanonicalname{AllOrthonormalBasesHaveTheSameCardinality}
\pmcreated{2013-03-22 17:56:10}
\pmmodified{2013-03-22 17:56:10}
\pmowner{asteroid}{17536}
\pmmodifier{asteroid}{17536}
\pmtitle{all orthonormal bases have the same cardinality}
\pmrecord{7}{40433}
\pmprivacy{1}
\pmauthor{asteroid}{17536}
\pmtype{Theorem}
\pmcomment{trigger rebuild}
\pmclassification{msc}{46C05}
\pmsynonym{dimension of an Hilbert space is well-defined}{AllOrthonormalBasesHaveTheSameCardinality}

% this is the default PlanetMath preamble.  as your knowledge
% of TeX increases, you will probably want to edit this, but
% it should be fine as is for beginners.

% almost certainly you want these
\usepackage{amssymb}
\usepackage{amsmath}
\usepackage{amsfonts}

% used for TeXing text within eps files
%\usepackage{psfrag}
% need this for including graphics (\includegraphics)
%\usepackage{graphicx}
% for neatly defining theorems and propositions
%\usepackage{amsthm}
% making logically defined graphics
%%%\usepackage{xypic}

% there are many more packages, add them here as you need them

% define commands here

\begin{document}
{\bf Theorem. --} All orthonormal bases of an Hilbert space $H$ have the same cardinality. It follows that the concept of dimension of a Hilbert space is well-defined.

$\,$

{\bf \emph{Proof:}} When $H$ is finite-dimensional (as a vector space), every orthonormal basis is a Hamel basis of $H$. Thus, the result follows from the fact that all Hamel bases of a vector space have the same cardinality (see \PMlinkname{this entry}{AllBasesForAVectorSpaceHaveTheSameCardinality}).

We now consider the case where $H$ is infinite-dimensional (as a vector space). Let $\{e_i\}_{i \in I}$ and $\{f_j\}_{j \in J}$ be two orthonormal basis of $H$, indexed by the sets $I$ and $J$, respectively. Since $H$ is infinite dimensional the sets $I$ and $J$ must be infinite.

We know, from Parseval's equality, that for every $x \in H$
\begin{displaymath}
\|x\|^2 = \sum_{i \in I} |\langle x, e_i \rangle|^2
\end{displaymath}
We know that, in the above sum, $\langle x, e_i \rangle \neq 0$ for only a countable number of $i \in I$.  Thus, considering $x$ as $f_j$, the set $I_j := \{ i \in I: \langle f_j, e_i \rangle \neq 0 \}$ is countable.   Since for each $i \in I$ we also have
\begin{displaymath}
\|e_i\|^2 = \sum_{j \in J}| \langle e_i, f_j \rangle|^2
\end{displaymath}
there must be $j \in J$ such that $\langle f_j, e_i \rangle \neq 0$. We conclude that $\displaystyle I = \bigcup_{j \in J} I_j$.

Hence, since each $I_j$ is countable, $I \leq J\!\times\!\mathbb{N} \cong J$ (because $J$ is infinite).

An analogous \PMlinkescapetext{argument} proves that $J \leq I$.  Hence, by the Schroeder-Bernstein theorem $J$ and $I$ have the same cardinality. $\square$
%%%%%
%%%%%
\end{document}
