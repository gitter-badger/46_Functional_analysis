\documentclass[12pt]{article}
\usepackage{pmmeta}
\pmcanonicalname{ProofOfParallelogramLaw1}
\pmcreated{2013-03-22 16:08:15}
\pmmodified{2013-03-22 16:08:15}
\pmowner{Wkbj79}{1863}
\pmmodifier{Wkbj79}{1863}
\pmtitle{proof of parallelogram law}
\pmrecord{6}{38211}
\pmprivacy{1}
\pmauthor{Wkbj79}{1863}
\pmtype{Proof}
\pmcomment{trigger rebuild}
\pmclassification{msc}{46C05}
\pmrelated{ProofOfParallelogramLaw}
\pmrelated{AlternateProofOfParallelogramLaw}

\usepackage{amssymb}
\usepackage{amsmath}
\usepackage{amsfonts}

\usepackage{psfrag}
\usepackage{graphicx}
\usepackage{amsthm}
%%\usepackage{xypic}
\begin{document}
The proof supplied here for the parallelogram law uses the properties of norms and inner products.  See the entries about these \PMlinkescapetext{objects} for more details regarding the following calculations.

\begin{proof}

\begin{center}
\begin{tabular}{ll}
$\Vert x+y \Vert^2+\Vert x-y \Vert^2$ & $=\langle x+y,x+y \rangle + \langle x-y,x-y \rangle$ \\
& $=\langle x,x+y \rangle + \langle y,x+y \rangle + \langle x,x-y \rangle - \langle y,x-y \rangle$ \\
& $=\overline{\langle x+y,x \rangle}+\overline{\langle x+y,y \rangle}+\overline{\langle x-y,x \rangle}-\overline{\langle x-y,y \rangle}$ \\
& $\displaystyle =\overline{\langle x,x \rangle + \langle y,x \rangle}+\overline{\langle x,y \rangle + \langle y,y \rangle}+\overline{\langle x,x \rangle - \langle y,x \rangle}-\left( \overline{\langle x,y \rangle - \langle y,y \rangle} \right)$ \\
& $=\overline{\langle x,x \rangle}+\overline{\langle y,x \rangle}+\overline{\langle x,y \rangle}+\overline{\langle y,y \rangle}+\overline{\langle x,x \rangle}-\overline{\langle y,x \rangle}-\overline{\langle x,y \rangle}+\overline{\langle y,y \rangle}$ \\
& $=\langle x,x \rangle + \langle y,y \rangle + \langle x,x \rangle + \langle y,y \rangle$ \\
& $=2\langle x,x \rangle +2 \langle y,y \rangle$ \\
& $=2 \Vert x \Vert^2+2 \Vert y \Vert^2$. \end{tabular}
\end{center}
\end{proof}
%%%%%
%%%%%
\end{document}
