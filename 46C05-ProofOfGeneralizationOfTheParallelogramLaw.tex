\documentclass[12pt]{article}
\usepackage{pmmeta}
\pmcanonicalname{ProofOfGeneralizationOfTheParallelogramLaw}
\pmcreated{2013-03-22 16:08:58}
\pmmodified{2013-03-22 16:08:58}
\pmowner{Mathprof}{13753}
\pmmodifier{Mathprof}{13753}
\pmtitle{proof of generalization of the parallelogram law}
\pmrecord{7}{38229}
\pmprivacy{1}
\pmauthor{Mathprof}{13753}
\pmtype{Proof}
\pmcomment{trigger rebuild}
\pmclassification{msc}{46C05}

% this is the default PlanetMath preamble.  as your knowledge
% of TeX increases, you will probably want to edit this, but
% it should be fine as is for beginners.

% almost certainly you want these
\usepackage{amssymb}
\usepackage{amsmath}
\usepackage{amsfonts}

% used for TeXing text within eps files
%\usepackage{psfrag}
% need this for including graphics (\includegraphics)
%\usepackage{graphicx}
% for neatly defining theorems and propositions
%\usepackage{amsthm}
% making logically defined graphics
%%%\usepackage{xypic}

% there are many more packages, add them here as you need them

% define commands here

\begin{document}
Let $g(x,y) = \Vert x + y \Vert^2 - \Vert x \Vert^2$ and $m(x,y) = \langle x, y \rangle + \langle y , x \rangle $.
Then
$$
g(x,y) = \Vert y \Vert^2 + m(x,y).
$$
Hence, taking $x_1 = x_4 = x, x_2 = y, x_3 = z$ we have:
\begin{eqnarray*}
\sum_{i=1}^3 \Vert x_i + x_{i+1} \Vert^2 - \sum_{i=1}^3 \Vert x_i \Vert^2 & = & \sum_{i=1}^3 g(x_i , x_{i+1} ) \\
& =& \sum_{i=1}^3\Vert x_i \Vert^2 + \sum_{i=1}^3 m(x_i, x_{i+1} ) \\ 
& = & \Big\Vert \sum_{i=1}^3 x_i \Big\Vert^2.
\end{eqnarray*}
%%%%%
%%%%%
\end{document}
