\documentclass[12pt]{article}
\usepackage{pmmeta}
\pmcanonicalname{AbsoluteValueInAVectorLattice}
\pmcreated{2013-03-22 17:03:16}
\pmmodified{2013-03-22 17:03:16}
\pmowner{CWoo}{3771}
\pmmodifier{CWoo}{3771}
\pmtitle{absolute value in a vector lattice}
\pmrecord{10}{39345}
\pmprivacy{1}
\pmauthor{CWoo}{3771}
\pmtype{Definition}
\pmcomment{trigger rebuild}
\pmclassification{msc}{46A40}
\pmclassification{msc}{06F20}
\pmdefines{absolute value}

\usepackage{amssymb,amscd}
\usepackage{amsmath}
\usepackage{amsfonts}
\usepackage{mathrsfs}

% used for TeXing text within eps files
%\usepackage{psfrag}
% need this for including graphics (\includegraphics)
%\usepackage{graphicx}
% for neatly defining theorems and propositions
\usepackage{amsthm}
% making logically defined graphics
%%\usepackage{xypic}
\usepackage{pst-plot}
\usepackage{psfrag}

% define commands here
\newtheorem{prop}{Proposition}
\newtheorem{thm}{Theorem}
\newtheorem{ex}{Example}
\newcommand{\real}{\mathbb{R}}
\newcommand{\pdiff}[2]{\frac{\partial #1}{\partial #2}}
\newcommand{\mpdiff}[3]{\frac{\partial^#1 #2}{\partial #3^#1}}
\begin{document}
Let $V$ be a vector lattice over $\mathbb{R}$, and $V^+$ be its positive cone.  We define three functions from $V$ to $V^+$ as follows.  For any $x\in V$,
\begin{itemize}
\item $x^+:=x\vee 0$,
\item $x^-:=(-x)\vee 0$, 
\item $|x|:=(-x)\vee x$.
\end{itemize}

It is easy to see that these functions are well-defined.  Below are some properties of the three functions:
\begin{enumerate}
\item $x^+=(-x)^-$ and $x^-=(-x)^+$.
\item $x=x^+-x^-$, since $x^+-x^-=(x\vee 0)-(-x)\vee 0=(x\vee 0)+(x\wedge 0)=x+0=x$.
\item $|x|=x^++x^-$, since $x^++x^-=x+2x^-=x+(-2x)\vee 0=(x-2x)\vee (x+0)=|x|$.
\item If $0\le x$, then $x^+=x$, $x^-=0$ and $|x|=x$.  Also, $x\le 0$ implies $x^+=0$, $x^-=-x$ and $|x|=-x$.
\item $|x|=0$ iff $x=0$.  The ``only if'' part is obvious.  For the ``if'' part, if $|x|=0$, then $(-x)\vee x=0$, so $x\le 0$ and $-x\le 0$.  But then $0\le x$, so $x=0$.
\item $|rx|=|r||x|$ for any $r\in \mathbb{R}$.  If $0\le r$, then $|rx|=(-rx)\vee (rx)=r\big((-x)\vee x\big)=r|x|=|r||x|$.  On the other hand, if $r\le 0$, then $|rx|=(-rx)\vee (rx)=(-r)\big(x\vee (-x)\big)=-r|x|=|r||x|$.
\item $|x|+|y|=|x+y|\vee |x-y|$, since $$LHS=(-x)\vee x+(-y)\vee y=(-x-y)\vee (-x+y)\vee (x-y)\vee (x+y)=RHS.$$
\item (triangle inequality). $|x+y|\le |x|+|y|$, since $|x+y|\le |x+y|\vee |x-y|=|x|+|y|$.
\end{enumerate}

Properties 5, 6, and 8 satisfy the axioms of an absolute value, and therefore $|x|$ is called the \emph{absolute value} of $x$.  However, it is not the ``norm'' of a vector in the traditional sense, since it is not a real-valued function.
%%%%%
%%%%%
\end{document}
