\documentclass[12pt]{article}
\usepackage{pmmeta}
\pmcanonicalname{PointwiseLimitOfBoundedOperatorsIsBounded}
\pmcreated{2013-03-22 17:32:10}
\pmmodified{2013-03-22 17:32:10}
\pmowner{asteroid}{17536}
\pmmodifier{asteroid}{17536}
\pmtitle{pointwise limit of bounded operators is bounded}
\pmrecord{4}{39933}
\pmprivacy{1}
\pmauthor{asteroid}{17536}
\pmtype{Corollary}
\pmcomment{trigger rebuild}
\pmclassification{msc}{46B99}
\pmclassification{msc}{47A05}

% this is the default PlanetMath preamble.  as your knowledge
% of TeX increases, you will probably want to edit this, but
% it should be fine as is for beginners.

% almost certainly you want these
\usepackage{amssymb}
\usepackage{amsmath}
\usepackage{amsfonts}

% used for TeXing text within eps files
%\usepackage{psfrag}
% need this for including graphics (\includegraphics)
%\usepackage{graphicx}
% for neatly defining theorems and propositions
%\usepackage{amsthm}
% making logically defined graphics
%%%\usepackage{xypic}

% there are many more packages, add them here as you need them

% define commands here

\begin{document}
The following result is a corollary of the principle of uniform boundedness.

{\bf Theorem -} Let $X$ be a Banach space and $Y$ a normed vector space. Let $(T_n) \in B(X,Y)$ be a sequence of bounded operators from $X$ to $Y$. If $(T_nx)$ converges for every $x \in X$, then the operator
\begin{center}
$T:X \longrightarrow Y$
\end{center}

\begin{displaymath}
Tx = \lim_{n \rightarrow \infty} T_n x
\end{displaymath}
is linear and \PMlinkescapetext{bounded}. Moreover, the sequence $(\|T_n\|)$ is \PMlinkname{bounded}{Bounded}.

{\bf Proof :} It is clear that the operator $T$ is linear. 

For each $x \in X$ we have $\displaystyle \;\sup_n \|T_nx\| < \infty\;$ since $(T_nx)$ is \PMlinkescapetext{convergent}. By the \PMlinkname{principle of uniform boundedness}{BanachSteinhausTheorem} we must also have $\displaystyle M := \sup_n \|T_n\| < \infty$.

Then for each $x \in X$ we have
\begin{displaymath}
\|Tx\| = \lim_{n \rightarrow \infty} \|T_nx\| \leq M\|x\|
\end{displaymath}

which means that $T$ is \PMlinkescapetext{bounded}. $\square$
%%%%%
%%%%%
\end{document}
