\documentclass[12pt]{article}
\usepackage{pmmeta}
\pmcanonicalname{WeakDerivative}
\pmcreated{2013-03-22 14:54:52}
\pmmodified{2013-03-22 14:54:52}
\pmowner{paolini}{1187}
\pmmodifier{paolini}{1187}
\pmtitle{weak derivative}
\pmrecord{16}{36600}
\pmprivacy{1}
\pmauthor{paolini}{1187}
\pmtype{Definition}
\pmcomment{trigger rebuild}
\pmclassification{msc}{46E35}
\pmrelated{SobolevSpaces}

% this is the default PlanetMath preamble.  as your knowledge
% of TeX increases, you will probably want to edit this, but
% it should be fine as is for beginners.

% almost certainly you want these
\usepackage{amssymb}
\usepackage{amsmath}
\usepackage{amsfonts}

% used for TeXing text within eps files
%\usepackage{psfrag}
% need this for including graphics (\includegraphics)
%\usepackage{graphicx}
% for neatly defining theorems and propositions
%\usepackage{amsthm}
% making logically defined graphics
%%%\usepackage{xypic}

% there are many more packages, add them here as you need them

% define commands here
\newcommand{\R}{\mathbf R}
\begin{document}
Let $f\colon \Omega\to \R$ and $g=(g_1,\ldots,g_n)\colon \Omega\to \R^n$ be locally integrable 
functions defined on an open set $\Omega\subset \mathbf R^n$.
We say that $g$ is the \emph{weak derivative} of $f$ if the equality
\[
  \int_\Omega f \frac{\partial \phi}{\partial x_i} = - \int_\Omega g_i \phi
\]
holds true for all functions $\phi\in\mathcal C^\infty_c(\Omega)$ (smooth functions with compact support in $\Omega$) and for all $i=1,\ldots, n$. Notice that the integrals involved are well defined since $\phi$ is bounded with compact support and because $f$ is assumed to be integrable on compact subsets of $\Omega$.

\subsection*{Comments}
\begin{enumerate}
\item
If $f$ is of class $\mathcal C^1$ then the gradient of $f$ is the weak derivative of $f$ in view of Gauss Green Theorem. So the weak derivative is an extension of the classical derivative. 
\item
The weak derivative is unique (as an element of the Lebesgue space $L^1_{\mathrm loc}$) in view of a result about locally integrable functions.
\item
The same definition can be given for functions with complex values.
\end{enumerate}
%%%%%
%%%%%
\end{document}
