\documentclass[12pt]{article}
\usepackage{pmmeta}
\pmcanonicalname{MazurUlamTheorem}
\pmcreated{2013-03-22 16:22:50}
\pmmodified{2013-03-22 16:22:50}
\pmowner{yark}{2760}
\pmmodifier{yark}{2760}
\pmtitle{Mazur-Ulam theorem}
\pmrecord{12}{38524}
\pmprivacy{1}
\pmauthor{yark}{2760}
\pmtype{Theorem}
\pmcomment{trigger rebuild}
\pmclassification{msc}{46B04}

\usepackage{amssymb}
\usepackage{amsthm}

\newtheorem*{thm*}{Theorem}

\def\C{\mathbb{C}}
\def\R{\mathbb{R}}

% The below lines should work as the command
% \renewcommand{\bibname}{References}
% without creating havoc when rendering an entry in
% the page-image mode.
\makeatletter
\@ifundefined{bibname}{}{\renewcommand{\bibname}{References}}
\makeatother

\begin{document}
\PMlinkescapeword{isometric}
\PMlinkescapeword{isometries}
\PMlinkescapeword{isometry}
\PMlinkescapeword{theorem}

\begin{thm*}
Every \PMlinkname{isometry}{Isometry} between normed vector spaces over $\R$
is an affine transformation.
\end{thm*}

Note that we consider isometries to be surjective by definition.
The result is not in general true for non-surjective isometric mappings.

The result does not extend to normed vector spaces over $\C$,
as can be seen from the fact that complex conjugation is an isometry $\C\to\C$
but is not affine over $\C$.
(But complex conjugation is clearly affine over $\R$,
and in general any normed vector space over $\C$
can be considered as a normed vector space over $\R$,
to which the theorem can be applied.)

This theorem was first proved by Mazur and Ulam.\cite{mazurulam}
A simpler proof has been given by Jussi V\"{a}is\"{a}l\"{a}.\cite{vaisala}

\begin{thebibliography}{9}
\bibitem{mazurulam}
 S.\ Mazur and S.\ Ulam,
 {\it Sur les transformations isom\'etriques d'espaces vectoriels norm\'es},
 C.\ R.\ Acad.\ Sci., Paris 194 (1932), 946--948.
\bibitem{vaisala}
 Jussi V\"{a}is\"{a}l\"{a},
 {\it A proof of the Mazur--Ulam theorem},
 Amer.\ Math.\ Mon.\ 110, \#7 (2003), 633--635.
 (A preprint is \PMlinkexternal{available on V\"{a}is\"{a}l\"{a}'s website}{http://www.helsinki.fi/\%7Ejvaisala/mazurulam.pdf}.)
\end{thebibliography}

%%%%%
%%%%%
\end{document}
