\documentclass[12pt]{article}
\usepackage{pmmeta}
\pmcanonicalname{PropertiesOfMinkowskisFunctional}
\pmcreated{2013-03-22 15:45:04}
\pmmodified{2013-03-22 15:45:04}
\pmowner{georgiosl}{7242}
\pmmodifier{georgiosl}{7242}
\pmtitle{properties of Minkowski's functional}
\pmrecord{10}{37704}
\pmprivacy{1}
\pmauthor{georgiosl}{7242}
\pmtype{Theorem}
\pmcomment{trigger rebuild}
\pmclassification{msc}{46B20}

% this is the default PlanetMath preamble.  as your knowledge
% of TeX increases, you will probably want to edit this, but
% it should be fine as is for beginners.

% almost certainly you want these
\usepackage{amssymb}
\usepackage{amsmath}
\usepackage{amsfonts}

% used for TeXing text within eps files
%\usepackage{psfrag}
% need this for including graphics (\includegraphics)
%\usepackage{graphicx}
% for neatly defining theorems and propositions
%\usepackage{amsthm}
% making logically defined graphics
%%%\usepackage{xypic}

% there are many more packages, add them here as you need them

% define commands here
\begin{document}
Let $X$ be a normed space, $K$ convex subset of $X$ and $0$ belongs to the interior of $K$.Then
\begin{enumerate} 
\item $\rho_{K}(x)\geq 0$ for all $x\in X$
\item $\rho_{K}(0)= 0$
\item $\rho_{K}(\lambda x)= \lambda \rho_{K}(x)$, for all $\lambda\geq 0$ and $x\in X$
\item $\rho_{K}(x+y)\leq \rho_{K}(x)+\rho_{K}(y)$\,for all $x,y \in K$ 
\item $\{x\in X\colon \rho_{K}(x)<1\}\subset K \subset \{x\in X\colon \rho_{K}(x)\leq 1\}$
\item $K^{0}=\{x\in X\colon \rho_{K}(x)<1\}$ where $K^{0}$ denotes the interior of $K$
\item $\bar K=\{x\in X\colon \rho_{K}(x)\leq 1\}$  where $\bar K$ denotes the closure of $K$
\item $Bd(K)= \{x\in X\colon \rho_{K}(x)= 1\}$ where the $Bd(K)$ denotes the boundary of $K$.
\end{enumerate}
Minkowski's functional is a useful tool to prove propositions and  solve  exercises. Let us see an example\\
\textbf{Example} Let $K$ be a convex subset of $X$. Show that $Ex(K)\subset Bd(K)$, where $Ex(K)$ denotes the 
set of extreme points of $K$.
\\If $x\in Ex(K)$ then from this follows that $x\in 1K$ and $\rho_{K}(x)= 1$.
Now we hypothesize that $\rho_{K}(x)<1$ then there is a real number $s$ such that $\rho_{K}(x)<s<1$ and 
so $\rho_{K}(\frac{x}{s})<1$. Therefore we have that $x=s\frac{x}{s}+(1-s)0 \in K$, that contradicts to the 
fact that $x\in Ex(K).$
%%%%%
%%%%%
\end{document}
