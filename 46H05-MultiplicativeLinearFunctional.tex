\documentclass[12pt]{article}
\usepackage{pmmeta}
\pmcanonicalname{MultiplicativeLinearFunctional}
\pmcreated{2013-03-22 17:22:25}
\pmmodified{2013-03-22 17:22:25}
\pmowner{asteroid}{17536}
\pmmodifier{asteroid}{17536}
\pmtitle{multiplicative linear functional}
\pmrecord{29}{39737}
\pmprivacy{1}
\pmauthor{asteroid}{17536}
\pmtype{Definition}
\pmcomment{trigger rebuild}
\pmclassification{msc}{46H05}
\pmsynonym{character (of an algebra)}{MultiplicativeLinearFunctional}
\pmrelated{LinearFunctional}
\pmrelated{GelfandTransform}
\pmrelated{BanachAlgebra}
\pmdefines{character}
\pmdefines{maximal ideal space}
\pmdefines{character space}

\endmetadata

% this is the default PlanetMath preamble.  as your knowledge
% of TeX increases, you will probably want to edit this, but
% it should be fine as is for beginners.

% almost certainly you want these
\usepackage{amssymb}
\usepackage{amsmath}
\usepackage{amsfonts}

% used for TeXing text within eps files
%\usepackage{psfrag}
% need this for including graphics (\includegraphics)
%\usepackage{graphicx}
% for neatly defining theorems and propositions
%\usepackage{amsthm}
% making logically defined graphics
%%%\usepackage{xypic}

% there are many more packages, add them here as you need them

% define commands here

\begin{document}
\section{Definition}
Let $\mathcal{A}$ be an algebra over $\mathbb{C}$.

A {\bf multiplicative linear functional} is an nontrivial algebra homomorphism $\phi :\mathcal{A} \longrightarrow
 \mathbb{C}$, i.e. $\phi$ is a non-zero linear functional such that $\;\phi(x\cdot y) = \phi(x)\cdot\phi(y), \;\;\;\forall x,y \in \mathcal{A}$.

Multiplicative linear functionals are also called {\bf characters} of $\mathcal{A}$.

\section{Properties}

\begin{itemize}
\item If $\phi$ is a multiplicative linear functional in a Banach algebra $\mathcal{A}$ over $\mathbb{C}$ then $\phi$ is continuous. Moreover, if $\mathcal{A}$ has an identity element then $\|\phi\| = 1$.
\end{itemize}

\begin{itemize}
\item Suppose $\mathcal{A}$ is a Banach algebra over $\mathbb{C}$. The set of multiplicative linear functionals in $\mathcal{A}$ is a locally compact Hausdorff space in the weak-* topology. Moreover, this set is compact if $\mathcal{A}$
 has an identity element.
\end{itemize}

\begin{itemize}
\item Suppose $\mathcal{A}$ is a commutative Banach algebra over $\mathbb{C}$ with an identity element. There is a bijective correspondence
 between the set of maximal ideals in $\mathcal{A}$ and the set of multiplicative linear functionals
 in $\mathcal{A}$. This correspondence is given by

\begin{displaymath}
\phi \longmapsto Ker\; \phi
\end{displaymath}
\end{itemize}

\begin{itemize}
\item Suppose $\mathcal{A}$ is a commutative \PMlinkname{$C^*$-algebra}{CAlgebra}. Multiplicative linear functionals in $\mathcal{A}$ are exactly the \PMlinkname{irreducible representations}{BanachAlgebraRepresentation} of $\mathcal{A}$.
\end{itemize}

\section{Character space of a Banach algebra}
As stated above, the set of all multiplicative linear functionals in a Banach algebra $\mathcal{A}$ is a locally compact Hausdorff space with the weak-* topology. It becomes a compact set if $\mathcal{A}$ has an identity element.

There are several designations for this space, such as:
 the {\bf \PMlinkescapetext{spectrum}} of $\mathcal{A}$, the {\bf maximal ideal space}, the {\bf character space}.

\section{Examples}
\begin{itemize}
\item Let $X$ be a topological space and $C(X)$ the algebra of continuous functions $X \longrightarrow \mathbb{C}$. Every point evaluation is a multiplicative linear functional of $C(X)$. In other words, for every point $x \in X$, the function
\begin{align*}
ev_x : C(X) \longrightarrow \mathbb{C}\\
ev_x ( f) = f(x)
\end{align*}
that gives the evaluation in $x$, is a multiplicative linear functional of $C(X)$.
\end{itemize}
%%%%%
%%%%%
\end{document}
