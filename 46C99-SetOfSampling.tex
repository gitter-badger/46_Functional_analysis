\documentclass[12pt]{article}
\usepackage{pmmeta}
\pmcanonicalname{SetOfSampling}
\pmcreated{2013-03-22 14:27:50}
\pmmodified{2013-03-22 14:27:50}
\pmowner{swiftset}{1337}
\pmmodifier{swiftset}{1337}
\pmtitle{set of sampling}
\pmrecord{4}{35984}
\pmprivacy{1}
\pmauthor{swiftset}{1337}
\pmtype{Definition}
\pmcomment{trigger rebuild}
\pmclassification{msc}{46C99}
\pmsynonym{sampling set}{SetOfSampling}
\pmrelated{Frame2}
\pmdefines{set of sampling}
\pmdefines{sampling operator}

\endmetadata

% this is the default PlanetMath preamble.  as your knowledge
% of TeX increases, you will probably want to edit this, but
% it should be fine as is for beginners.

% almost certainly you want these
\usepackage{amssymb}
\usepackage{amsmath}
\usepackage{amsfonts}

% used for TeXing text within eps files
%\usepackage{psfrag}
% need this for including graphics (\includegraphics)
%\usepackage{graphicx}
% for neatly defining theorems and propositions
%\usepackage{amsthm}
% making logically defined graphics
%%%\usepackage{xypic}

% there are many more packages, add them here as you need them

% define commands here
\begin{document}
\paragraph{Definition}
Let $F$ be a Hilbert space of functions defined on a domain $D$. Let $T = \{t_i\}_{i\in I}$ be a finite or infinite sequence of points in $D$. $T$ is said to be a \emph{set of sampling} for $F$ if the sampling operator $S: F \rightarrow l^2_{|T|}$ defined by 
$$ 
S: f \mapsto 
\begin{pmatrix}
f(t_1) \\
f(t_2) \\
\vdots 
\end{pmatrix}
$$
is bounded (i.e. continuous) and bounded below; i.e. if 
$$\exists A,B>0 \hbox{ such that } \forall f \in F, A\|f\|^2 \leq \sum_{i=1}^{|T|} |f(t_i)|^2 \leq B \|f\|^2.$$

\paragraph{Relation to Frames}
Using the Riesz Representation Theorem, it is easy to show that every set of sampling determines a unique frame in such a way that the analysis operator of that frame is the sampling operator associated with the set of sampling. In fact, let $t=\{t_i\}$ be a set of sampling with sampling operator $S_t$. Use the Riesz representation theorem to rewrite $S_t$ in terms of vectors $\{g_i\}$ in $F$:
$$ 
S: f \mapsto 
\begin{pmatrix}
f(t_1) \\
f(t_2) \\
\vdots
\end{pmatrix}
=
\begin{pmatrix}
\langle f, g_1 \rangle \\
\langle f, g_2 \rangle \\
\vdots
\end{pmatrix}
$$
then note that 
$$ \forall f\in F, A\|f\|^2 \leq \sum_{i} \left| \langle f, g_i \rangle \right|^2 \leq B\|f\|^2, $$
so the $\{g_i\}$ form a frame with bounds $A, B$, and $S_t = \theta_g.$

\paragraph{Reconstruction}
Particularly nice sets of sampling are those that correspond to tight frames, because then $\theta_g^\ast\theta_g=\theta_g^\ast S_t=AI$, and it is possible to reconstruct the function $f$, given its values over the set of sampling:
$$ f = \frac{1}{A}\sum_i f(t_i) g_i.$$
Sets of sampling which correspond to tight frames are referred to as tight sets of sampling.
%%%%%
%%%%%
\end{document}
