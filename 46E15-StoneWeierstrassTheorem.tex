\documentclass[12pt]{article}
\usepackage{pmmeta}
\pmcanonicalname{StoneWeierstrassTheorem}
\pmcreated{2013-03-22 12:42:06}
\pmmodified{2013-03-22 12:42:06}
\pmowner{rspuzio}{6075}
\pmmodifier{rspuzio}{6075}
\pmtitle{Stone-Weierstrass theorem}
\pmrecord{9}{32984}
\pmprivacy{1}
\pmauthor{rspuzio}{6075}
\pmtype{Theorem}
\pmcomment{trigger rebuild}
\pmclassification{msc}{46E15}

\endmetadata

% this is the default PlanetMath preamble.  as your knowledge
% of TeX increases, you will probably want to edit this, but
% it should be fine as is for beginners.

% almost certainly you want these
\usepackage{amssymb}
\usepackage{amsmath}
\usepackage{amsfonts}

% used for TeXing text within eps files
%\usepackage{psfrag}
% need this for including graphics (\includegraphics)
%\usepackage{graphicx}
% for neatly defining theorems and propositions
%\usepackage{amsthm}
% making logically defined graphics
%%%\usepackage{xypic}

% there are many more packages, add them here as you need them

% define commands here
\begin{document}
Let $X$ be a compact space and let $C^0(X,\mathbb{R})$ be the algebra of continuous real functions 
defined over $X$. Let $\mathcal{A}$ be a subalgebra of $C^0(X,\mathbb{R})$ for which the following 
conditions hold:
\begin{enumerate}
\item $\forall x, y \in X, x \ne y, \exists f \in \mathcal{A} : f(x) \neq f(y)$
\item $1 \in \mathcal{A}$
\end{enumerate}
Then $\mathcal{A}$ is dense in $C^0(X,\mathbb{R})$.

This theorem is a generalization of the classical Weierstrass approximation theorem to general spaces.
%%%%%
%%%%%
\end{document}
