\documentclass[12pt]{article}
\usepackage{pmmeta}
\pmcanonicalname{SpectrumIsANonemptyCompactSet}
\pmcreated{2013-03-22 17:25:05}
\pmmodified{2013-03-22 17:25:05}
\pmowner{asteroid}{17536}
\pmmodifier{asteroid}{17536}
\pmtitle{spectrum is a non-empty compact set}
\pmrecord{10}{39791}
\pmprivacy{1}
\pmauthor{asteroid}{17536}
\pmtype{Theorem}
\pmcomment{trigger rebuild}
\pmclassification{msc}{46H05}

% this is the default PlanetMath preamble.  as your knowledge
% of TeX increases, you will probably want to edit this, but
% it should be fine as is for beginners.

% almost certainly you want these
\usepackage{amssymb}
\usepackage{amsmath}
\usepackage{amsfonts}

% used for TeXing text within eps files
%\usepackage{psfrag}
% need this for including graphics (\includegraphics)
%\usepackage{graphicx}
% for neatly defining theorems and propositions
%\usepackage{amsthm}
% making logically defined graphics
%%%\usepackage{xypic}

% there are many more packages, add them here as you need them

% define commands here

\begin{document}
{\bf Theorem -} Let $\mathcal{A}$ be a complex Banach algebra with identity element. The spectrum of each $a \in \mathcal{A}$ is a non-empty compact set in $\mathbb{C}$.

$\emph{Remark}$ : For Banach algebras over $\mathbb{R}$ the spectrum of an element is also a compact set, although it can be empty. To assure that it is not the empty set, proofs usually involve \PMlinkname{Liouville's theorem}{LiouvillesTheorem2} for \PMlinkescapetext{functions} of a complex \PMlinkescapetext{variable} with values in a Banach algebra.

{\bf \emph{Proof :}} Let $e$ be the identity element of $\mathcal{A}$. Let $\sigma(a)$ denote the spectrum of the element $a \in \mathcal{A}$.
\begin{itemize}
\item {\bf \PMlinkescapetext{Compactness} -} For each $\lambda \in \mathbb{C}$ such that $|\lambda| > \|a\|$ one has $\|\lambda^{-1}a\|<1$, and so, by the \PMlinkname{Neumann series}{NeumannSeriesInBanachAlgebras}, $e-\lambda^{-1}a$ is invertible. Since
\begin{displaymath}
a-\lambda e=-\lambda(e-\lambda^{-1}a)
\end{displaymath}
we see that $a-\lambda e$ is also invertible.

We conclude that $\sigma(a)$ is contained in a disk of radius $\|a\|$, and therefore it is bounded.

Let $\phi : \mathbb{C} \longrightarrow \mathcal{A}$ be the function defined by
\begin{displaymath}
\phi(\lambda) = a - \lambda e
\end{displaymath}

It is known that the set $\mathcal{G}$ of the invertible elements of $\mathcal{A}$ is open (see \PMlinkname{this entry}{InvertibleElementsInABanachAlgebraFormAnOpenSet}).

Since $\phi^{-1}(\mathcal{G})=\mathbb{C}-\sigma(a)$ and $\phi$ is a continuous function we see that that $\sigma(a)$ is a closed set in $\mathbb{C}$.

As $\sigma(a)$ is a bounded closed subset of $\mathbb{C}$, it is compact.

\item {\bf Non-emptiness -} Suppose that $\sigma(a)$ was empty. Then the resolvent \PMlinkescapetext{function} $R_a$ is defined in $\mathbb{C}$.

We can see that $R_a$ is bounded since it is continuous in the closed disk $|\lambda|<\|a\|$ and, for $\lambda >\|a\|$, we have (again, by the \PMlinkname{Neumann series}{NeumannSeriesInBanachAlgebras})
\begin{eqnarray*}
\|R_a(\lambda)\| & = & \|(a-\lambda e)^{-1}\| \\ 
& = & \|\lambda^{-1}(e-\lambda^{-1}a)^{-1}\| \\ 
& \leq & \frac{|\lambda|^{-1}}{1-|\lambda|^{-1}\|a\|} \\
& = & \frac{1}{|\lambda|-\|a\|}
\end{eqnarray*}
and therefore $\displaystyle \lim_{|\lambda| \rightarrow \infty} R_a(\lambda) = 0$, which shows that $R_a$ is bounded.

The resolvent function, $R_a$, is \PMlinkname{analytic}{BanachSpaceValuedAnalyticFunctions} (see \PMlinkname{this entry}{ResolventFunctionIsAnalytic}). As it is defined in $\mathbb{C}$, it is a bounded entire function. Applying \PMlinkname{Liouville's theorem}{LiouvillesTheorem2} we conclude that it must be constant (see this \PMlinkname{this entry}{BanachSpaceValuedAnalyticFunctions} for an idea of how 
\PMlinkescapetext{Liouville's theorem} holds for Banach space valued functions).

Since $R_a(\lambda)$ converges to $0$ as $|\lambda| \rightarrow \infty$ we see that $R_a$ must be identically zero.

Thus, we have arrived to a contradiction since $0$ is not invertible.

Therefore $\sigma(a)$ is non-empty.$\square$
\end{itemize}
%%%%%
%%%%%
\end{document}
