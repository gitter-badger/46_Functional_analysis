\documentclass[12pt]{article}
\usepackage{pmmeta}
\pmcanonicalname{RieszSequence}
\pmcreated{2013-03-22 14:26:53}
\pmmodified{2013-03-22 14:26:53}
\pmowner{swiftset}{1337}
\pmmodifier{swiftset}{1337}
\pmtitle{Riesz sequence}
\pmrecord{7}{35963}
\pmprivacy{1}
\pmauthor{swiftset}{1337}
\pmtype{Definition}
\pmcomment{trigger rebuild}
\pmclassification{msc}{46C05}
\pmrelated{OrthonormalBasis}
\pmrelated{HilbertSpace}
\pmrelated{Frame2}
\pmrelated{AnEquivalentConditionForTheTranslatesOfAnL_2FunctionToFormARieszBasis}
\pmdefines{Riesz basis}

\endmetadata

% this is the default PlanetMath preamble.  as your knowledge
% of TeX increases, you will probably want to edit this, but
% it should be fine as is for beginners.

% almost certainly you want these
\usepackage{amssymb}
\usepackage{amsmath}
\usepackage{amsfonts}

% used for TeXing text within eps files
%\usepackage{psfrag}
% need this for including graphics (\includegraphics)
%\usepackage{graphicx}
% for neatly defining theorems and propositions
%\usepackage{amsthm}
% making logically defined graphics
%%%\usepackage{xypic}

% there are many more packages, add them here as you need them

% define commands here
\begin{document}
\paragraph{Definition}
A sequence of vectors $(x_n)$ in a Hilbert space $H$ is called a Riesz sequence if there exist constants $0 < c \leq C$ such that 
$$ c\left( \sum_n |a_n|^2 \right) \leq \left\| \sum_n a_n x_n \right\|^2 \leq C \left( \sum_n |a_n|^2 \right) $$
for all sequences of scalars $(a_n) \in l^2$. A Riesz sequence is called a Riesz basis if $\overline{\mathop{\rm span} (x_n)} = H$.

If $H$ is a finite-dimensional space, then every basis of $H$ is a Riesz basis.
%%%%%
%%%%%
\end{document}
