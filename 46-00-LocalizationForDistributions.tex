\documentclass[12pt]{article}
\usepackage{pmmeta}
\pmcanonicalname{LocalizationForDistributions}
\pmcreated{2013-03-22 13:46:17}
\pmmodified{2013-03-22 13:46:17}
\pmowner{drini}{3}
\pmmodifier{drini}{3}
\pmtitle{localization for distributions}
\pmrecord{9}{34477}
\pmprivacy{1}
\pmauthor{drini}{3}
\pmtype{Definition}
\pmcomment{trigger rebuild}
\pmclassification{msc}{46-00}
\pmclassification{msc}{46F05}

% this is the default PlanetMath preamble.  as your knowledge
% of TeX increases, you will probably want to edit this, but
% it should be fine as is for beginners.

% almost certainly you want these
\usepackage{amssymb}
\usepackage{amsmath}
\usepackage{amsfonts}

% used for TeXing text within eps files
%\usepackage{psfrag}
% need this for including graphics (\includegraphics)
%\usepackage{graphicx}
% for neatly defining theorems and propositions
%\usepackage{amsthm}
% making logically defined graphics
%%%\usepackage{xypic}

% there are many more packages, add them here as you need them

% define commands here

\newcommand{\sR}[0]{\mathbb{R}}
\newcommand{\sC}[0]{\mathbb{C}}
\newcommand{\sN}[0]{\mathbb{N}}
\newcommand{\sZ}[0]{\mathbb{Z}}
\begin{document}
\PMlinkescapeword{index}

\newcommand{\supp}[0]{\operatorname{supp}}
\newcommand{\cD}[0]{\mathcal{D}}

{\bf Definition}
Suppose $U$ is an open set in $\sR^n$ and $T$ is a 
distribution $T\in \cD'(U)$. Then we say that $T$ \emph{vanishes} on
an open set $V\subset U$, if the restriction of $T$ to $V$ is 
the zero distribution on $V$. In other words, $T$ vanishes on $V$, if
$T(v)=0$ for all $v\in C_0^\infty(V)$. (Here $C_0^\infty(V)$ is the set
of smooth function with compact support in $V$.) Similarly, we say that
two distributions $S,T\in \cD'(U)$ are \emph{equal}, or 
\emph{coincide} on $V$, if $S-T$ vanishes on $V$. We then 
write: $S=T$ on $V$. 

{\bf Theorem}\cite{folland, hormander} 
Suppose $U$ is an open set in $\sR^n$ and 
$\{U_i\}_{i\in I}$ is an open cover of $U$, i.e., 
$$U=\bigcup_{i\in I} U_i.$$
Here, $I$ is an arbitrary index set. If $S,T$ are distributions on $U$, 
such that $S=T$ on each $U_i$, then $S=T$ (on U). 

{\bf Proof.} Suppose $u\in \cD(U)$. Our aim is to show that $S(u)=T(u)$. 
First, we have $\supp u \subset K$ for some compact $K\subset U$. 
\PMlinkname{It follows}{YIsCompactIfAndOnlyIfEveryOpenCoverOfYHasAFiniteSubcover} that there exist a finite collection of $U_i$:s from the open cover, 
say $U_1, \ldots, U_N$, such that $K\subset \cup_{i=1}^N U_i$.
By a smooth partition of unity, there
are smooth functions $\phi_1, \ldots, \phi_N: U\to \sR$ such that
\begin{enumerate}
\item$ \supp \phi_i \subset U_i$ for all $i$.
\item  $\phi_i(x) \in [0,1]$ for all $x\in U$ and all $i$,
\item $\sum_{i=1}^N \phi_i(x) = 1$ for all $x\in K$.
\end{enumerate}
From the first property, and from a property for the \PMlinkname{support of a function}{SupportOfFunction},
it follows that 
$\supp \phi_i u \subset \supp \phi_i \cap \supp u \subset U_i$. 
Therefore,  for each $i$,
$S(\phi_i u)=T(\phi_i u)$ since $S$ and $T$ conicide 
on $U_i$. 
Then
\begin{eqnarray*}
S(u) = \sum_{i=1}^N S(\phi_i u) = \sum_{i=1}^N T(\phi_i u) = T(u),
\end{eqnarray*}
and the theorem follows. $\Box$

\begin{thebibliography}{9}
\bibitem{folland}
 G.B. Folland, \emph{Real Analysis: Modern Techniques and Their Applications}, 2nd ed, John Wiley \& Sons, Inc., 1999.
\bibitem{rudin_fap}
 W. Rudin, \emph{Functional Analysis},
 McGraw-Hill Book Company, 1973.
 \bibitem{hormander}
 L. H\"ormander, \emph{The Analysis of Linear Partial Differential Operators I,
 (Distribution theory and Fourier Analysis)}, 2nd ed, Springer-Verlag, 1990.
 \end{thebibliography}
%%%%%
%%%%%
\end{document}
