\documentclass[12pt]{article}
\usepackage{pmmeta}
\pmcanonicalname{ProofOfBoundedLinearFunctionalsOnLpmu}
\pmcreated{2013-03-22 18:38:19}
\pmmodified{2013-03-22 18:38:19}
\pmowner{gel}{22282}
\pmmodifier{gel}{22282}
\pmtitle{proof of bounded linear functionals on $L^p(\mu)$}
\pmrecord{4}{41379}
\pmprivacy{1}
\pmauthor{gel}{22282}
\pmtype{Proof}
\pmcomment{trigger rebuild}
\pmclassification{msc}{46E30}
\pmclassification{msc}{28A25}
%\pmkeywords{$L^p$ space}

\endmetadata

% almost certainly you want these
\usepackage{amssymb}
\usepackage{amsmath}
\usepackage{amsfonts}

% used for TeXing text within eps files
%\usepackage{psfrag}
% need this for including graphics (\includegraphics)
%\usepackage{graphicx}
% for neatly defining theorems and propositions
\usepackage{amsthm}
% making logically defined graphics
%%%\usepackage{xypic}

% there are many more packages, add them here as you need them

% define commands here
\newtheorem*{theorem*}{Theorem}
\newtheorem*{lemma*}{Lemma}
\newtheorem*{corollary*}{Corollary}
\newtheorem*{definition*}{Definition}
\newtheorem{theorem}{Theorem}
\newtheorem{lemma}{Lemma}
\newtheorem{corollary}{Corollary}
\newtheorem{definition}{Definition}

\begin{document}
If $(X,\mathfrak{M},\mu)$ is a $\sigma$-finite measure-space and $p,q$ are \PMlinkname{H\"older conjugates}{ConjugateIndex} with $p<\infty$, then we show that $L^q$ is isometrically isomorphic to the dual space of $L^p$.

For any $g\in L^q$, define the linear map
\begin{equation*}
\Phi_g\colon L^p\rightarrow\mathbb{C},\ f\mapsto \Phi_g(f)=\int fg\,d\mu.
\end{equation*}
This is a bounded linear map with operator norm $\Vert\Phi_g\Vert=\Vert g\Vert_q$ (see \PMlinkname{$L^p$-norm is dual to $L^q$}{LpNormIsDualToLq}), so the map $g\mapsto\Phi_g$ gives an isometric embedding from $L^q$ to the dual space of $L^p$. It only remains to show that it is onto.

So, suppose that $\Phi\colon L^p\rightarrow\mathbb{C}$ is a bounded linear map. It needs to be shown that there is a $g\in L^q$ with $\Phi=\Phi_g$.
As \PMlinkname{any $\sigma$-finite measure is equivalent to a probability measure}{AnySigmaFiniteMeasureIsEquivalentToAProbabilityMeasure}, there is a bounded $h>0$ such that $\int h\,d\mu=1$.
Let $\tilde\Phi\colon L^\infty\rightarrow\mathbb{C}$ be the bounded linear map given by $\tilde\Phi(f)=\Phi(hf)$. Then, there is a $g_0\in L^1$ such that
\begin{equation*}
\Phi(hf)=\tilde\Phi(f)=\int fg_0\,d\mu
\end{equation*}
for every $f\in L^\infty$ (see \PMlinkname{bounded linear functionals on $L^\infty$}{BoundedLinearFunctionalsOnLinftymu}). Set $g=h^{-1}g_0$ and, for any $f\in L^p$, let $f_n$ be the sequence
\begin{equation*}
f_n = f 1_{\{|h^{-1}f|<n\}}.
\end{equation*}
As $h^{-1}f_n\in L^\infty$,
\begin{equation*}
\Vert f_ng\Vert_1 =\Vert h^{-1}f_ng_0\Vert_1=\Phi(\operatorname{sign}(fg_0)f_n)\le\Vert\Phi\Vert \Vert f_n\Vert_p.
\end{equation*}
Letting $n$ tend to infinity, dominated convergence says that $f_n\rightarrow f$ in the $L^p$-norm, so Fatou's lemma gives
\begin{equation*}
\Vert f g\Vert_1\le\liminf_{n\rightarrow\infty}\Vert f_n g\Vert_1\le\Vert\Phi\Vert\Vert f\Vert_p.
\end{equation*}
In particular, $\Vert g\Vert_q\le\Vert\Phi\Vert$ (see \PMlinkname{$L^p$-norm is dual to $L^q$}{LpNormIsDualToLq}), so $g\in L^q$. As $|f_ng|\le|fg|$ are in $L^1$, dominated convergence finally gives
\begin{equation*}
\int fg\,d\mu =\lim_{n\rightarrow\infty}\int f_ng\,d\mu=\lim_{n\rightarrow\infty}\Phi(f_n)=\Phi(f)
\end{equation*}
so $\Phi_g=\Phi$ as required.

%%%%%
%%%%%
\end{document}
