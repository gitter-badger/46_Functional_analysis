\documentclass[12pt]{article}
\usepackage{pmmeta}
\pmcanonicalname{OrderingOfSelfadjoints}
\pmcreated{2013-03-22 17:30:37}
\pmmodified{2013-03-22 17:30:37}
\pmowner{asteroid}{17536}
\pmmodifier{asteroid}{17536}
\pmtitle{ordering of self-adjoints}
\pmrecord{7}{39900}
\pmprivacy{1}
\pmauthor{asteroid}{17536}
\pmtype{Theorem}
\pmcomment{trigger rebuild}
\pmclassification{msc}{46L05}

% this is the default PlanetMath preamble.  as your knowledge
% of TeX increases, you will probably want to edit this, but
% it should be fine as is for beginners.

% almost certainly you want these
\usepackage{amssymb}
\usepackage{amsmath}
\usepackage{amsfonts}

% used for TeXing text within eps files
%\usepackage{psfrag}
% need this for including graphics (\includegraphics)
%\usepackage{graphicx}
% for neatly defining theorems and propositions
%\usepackage{amsthm}
% making logically defined graphics
%%%\usepackage{xypic}

% there are many more packages, add them here as you need them

% define commands here

\begin{document}
Let $\mathcal{A}$ be a \PMlinkname{$C^*$-algebra}{CAlgebra}. Let $\mathcal{A}^+$ denote the set of positive elements of $\mathcal{A}$ and $\mathcal{A}_{sa}$ denote the set of self-adjoint elements of $\mathcal{A}$.

Since $\mathcal{A}^+$ is a \PMlinkname{proper convex cone}{Cone5} (see this \PMlinkname{entry}{PositiveElement3}), we can define a partial order $\leq$ on the set $\mathcal{A}_{sa}$, by setting
\begin{center}
$a\leq b$ if and only if $b-a \in \mathcal{A}^+$, i.e. $b-a$ is positive.
\end{center} 

{\bf Theorem -} The relation $\leq$ is a partial order relation on $\mathcal{A}_{sa}$. Moreover, $\leq$ turns $\mathcal{A}_{sa}$ into an ordered topological vector space.

\subsubsection{Properties:}

\begin{itemize}
\item $a\leq b \;\Rightarrow\; c^*a\,c\leq c^*b\,c\;\;$ for every $c \in \mathcal{A}$.
\item If $a$ and $b$ are invertible and $a \leq b$, then $b^{-1} \leq a^{-1}$.
\item If $\mathcal{A}$ has an identity element $e$, then $-\|a\|e \leq a \leq \|a\|e\;$ for every $a \in \mathcal{A}_{sa}$.
\item $-b \leq a \leq b \;\Rightarrow \;\|a\| \leq \|b\|$.
\end{itemize}

\subsubsection{Remark:}
The proof that $\leq$ is partial order makes no use of the self-adjointness \PMlinkescapetext{property}. In fact, $\mathcal{A}$ itself is an ordered topological vector space under the relation $\leq$.

However, it turns out that this ordering relation provides its most usefulness when restricted to self-adjoint elements. For example, some of the above \PMlinkescapetext{properties} would not hold if we did not restrict to $\mathcal{A}_{sa}$.
%%%%%
%%%%%
\end{document}
