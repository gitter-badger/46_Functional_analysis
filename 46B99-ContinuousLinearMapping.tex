\documentclass[12pt]{article}
\usepackage{pmmeta}
\pmcanonicalname{ContinuousLinearMapping}
\pmcreated{2013-03-22 13:15:41}
\pmmodified{2013-03-22 13:15:41}
\pmowner{Koro}{127}
\pmmodifier{Koro}{127}
\pmtitle{continuous linear mapping}
\pmrecord{7}{33741}
\pmprivacy{1}
\pmauthor{Koro}{127}
\pmtype{Definition}
\pmcomment{trigger rebuild}
\pmclassification{msc}{46B99}
\pmsynonym{bounded linear mapping}{ContinuousLinearMapping}
\pmrelated{HomomorphismsOfCAlgebrasAreContinuous}
\pmrelated{CAlgebra}
\pmrelated{BoundedLinearFunctionalsOnLpmu}
\pmdefines{bounded linear transform}
\pmdefines{bounded linear operator}

% this is the default PlanetMath preamble.  as your knowledge
% of TeX increases, you will probably want to edit this, but
% it should be fine as is for beginners.

% almost certainly you want these
\usepackage{amssymb}
\usepackage{amsmath}
\usepackage{amsfonts}

% used for TeXing text within eps files
%\usepackage{psfrag}
% need this for including graphics (\includegraphics)
%\usepackage{graphicx}
% for neatly defining theorems and propositions
%\usepackage{amsthm}
% making logically defined graphics
%%%\usepackage{xypic}

% there are many more packages, add them here as you need them

% define commands here
\begin{document}
If $(V_1,\|\cdot\|_1)$ and $(V_2,\|\cdot\|_2)$ are normed vector spaces, a linear mapping $T:V_1\rightarrow V_2$ is continuous if it is continuous in the metric induced by the norms. 

If there is a nonnegative constant $c$ such that 
$\|T(x)\|_2\leq c\|x\|_1$ for each $x\in V_1$, we say that $T$ is \emph{\PMlinkescapetext{bounded}}. This should not be confused with the usual terminology referring to a bounded function as one that has bounded range. In fact, bounded linear mappings usually have unbounded ranges. 

The expression \emph{bounded linear mapping} is often  used in functional analysis to refer to continuous linear mappings as well. This is because the two definitions are equivalent:

If $T$ is bounded, then $\|T(x)-T(y)\|_2 = \|T(x-y)\|_2 \leq c\|x-y\|_1$, so $T$ is a Lipschitz function. Now suppose $T$ is continuous. Then there exists $r>0$ such that $\|T(x)\|_2 \leq 1$ when $\|x\|_1\leq r$. For any $x\in V_1$,  we then have \[\frac{r}{\|x\|_1}\|T(x)\|_2 = \|T\left(\frac{r}{\|x\|_1}x\right)\|_2 \leq 1,\]
hence $\|T(x)\|_2\leq r\|x\|_1$; so $T$ is bounded.

It can be shown that a linear mapping between two topological vector spaces is continuous if and only if it is \PMlinkname{continuous at}{Continuous} $0$ \cite{rudin_fap}.

\begin{thebibliography}{9}
 \bibitem{rudin_fap}
 W. Rudin, \emph{Functional Analysis},
 McGraw-Hill Book Company, 1973.
\end{thebibliography}
%%%%%
%%%%%
\end{document}
