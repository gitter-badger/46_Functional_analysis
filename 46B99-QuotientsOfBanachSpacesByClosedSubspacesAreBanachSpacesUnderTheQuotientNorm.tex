\documentclass[12pt]{article}
\usepackage{pmmeta}
\pmcanonicalname{QuotientsOfBanachSpacesByClosedSubspacesAreBanachSpacesUnderTheQuotientNorm}
\pmcreated{2013-03-22 17:23:01}
\pmmodified{2013-03-22 17:23:01}
\pmowner{asteroid}{17536}
\pmmodifier{asteroid}{17536}
\pmtitle{quotients of Banach spaces by closed subspaces are Banach spaces under the quotient norm}
\pmrecord{7}{39750}
\pmprivacy{1}
\pmauthor{asteroid}{17536}
\pmtype{Theorem}
\pmcomment{trigger rebuild}
\pmclassification{msc}{46B99}

\endmetadata

% this is the default PlanetMath preamble.  as your knowledge
% of TeX increases, you will probably want to edit this, but
% it should be fine as is for beginners.

% almost certainly you want these
\usepackage{amssymb}
\usepackage{amsmath}
\usepackage{amsfonts}

% used for TeXing text within eps files
%\usepackage{psfrag}
% need this for including graphics (\includegraphics)
%\usepackage{graphicx}
% for neatly defining theorems and propositions
%\usepackage{amsthm}
% making logically defined graphics
%%%\usepackage{xypic}

% there are many more packages, add them here as you need them

% define commands here

\begin{document}
{\bf Theorem -} Let $X$ be a Banach space and $M$ a closed subspace. Then $X/M$ with the quotient norm is a
 Banach space.

{\bf Proof :} In {\PMlinkescapetext{order}} to prove that $X/M$ is a Banach space it is enough to prove that every series in $X/M$
 that converges absolutely also converges in $X/M$.

Let $\sum_{n} X_n$ be an absolutely convergent series in $X/M$, i.e., $\sum_{n} \|X_n\|_{X/M} < \infty$.
By definition of the quotient norm, there exists $x_n \in X_n$ such that

\begin{displaymath}
\|x_n \| \le \|X_n \|_{X/M} + 2^{-n}
\end{displaymath}

It is clear that $\sum_{n} \|x_n\| < \infty$ and so, as $X$ is a Banach space, $\sum_{n} x_n$ is
 convergent.

Let $x = \sum_{n} x_n$ and $s_k = \sum_{n=1}^k x_n$. We have that

\begin{displaymath}
x - s_k + M = (x+M) - (s_k +M) = (x+M) - \sum_{n=1}^k (x_n +M) = (x+M) - \sum_{n=1}^k X_n
\end{displaymath}

Since $\|x-s_k + M \|_{X/M} \leq \|x-s_k\| \longrightarrow 0$ we see that $\sum_n X_n$ converges in $X/M$
 to $x+M$. $\square$
%%%%%
%%%%%
\end{document}
