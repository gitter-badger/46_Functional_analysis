\documentclass[12pt]{article}
\usepackage{pmmeta}
\pmcanonicalname{RellichSelectionTheorem}
\pmcreated{2013-03-22 14:38:55}
\pmmodified{2013-03-22 14:38:55}
\pmowner{rspuzio}{6075}
\pmmodifier{rspuzio}{6075}
\pmtitle{Rellich selection theorem}
\pmrecord{7}{36239}
\pmprivacy{1}
\pmauthor{rspuzio}{6075}
\pmtype{Theorem}
\pmcomment{trigger rebuild}
\pmclassification{msc}{46C05}

\endmetadata

% this is the default PlanetMath preamble.  as your knowledge
% of TeX increases, you will probably want to edit this, but
% it should be fine as is for beginners.

% almost certainly you want these
\usepackage{amssymb}
\usepackage{amsmath}
\usepackage{amsfonts}

% used for TeXing text within eps files
%\usepackage{psfrag}
% need this for including graphics (\includegraphics)
%\usepackage{graphicx}
% for neatly defining theorems and propositions
%\usepackage{amsthm}
% making logically defined graphics
%%%\usepackage{xypic}

% there are many more packages, add them here as you need them

% define commands here
\begin{document}
Let $D$ be an open subset of $\mathbb{R}^n$.  If, for a sequence of functions $f_i \colon D \to \mathbb{R}$, $i = 1,2,\ldots$ there exists a constant $B>0$ such that 
 $$(\forall i) \qquad \| f_i \|_{L^2 (D)} = \int_D f_i^2 \, d^n x < B$$
and
 $$(\forall i) \, (\forall j \in \{1, \ldots n \}) \qquad \int_D \left( {\partial f_i \over \partial x_j} \right)^2 \, d^n x < B$$
then there exists a subsequence which is convergent in the $L^2 (D)$ norm.
%%%%%
%%%%%
\end{document}
