\documentclass[12pt]{article}
\usepackage{pmmeta}
\pmcanonicalname{ScalingOfTheOpenBallInANormedVectorSpace}
\pmcreated{2013-03-22 15:33:25}
\pmmodified{2013-03-22 15:33:25}
\pmowner{matte}{1858}
\pmmodifier{matte}{1858}
\pmtitle{scaling of the open ball in a normed vector space}
\pmrecord{7}{37458}
\pmprivacy{1}
\pmauthor{matte}{1858}
\pmtype{Theorem}
\pmcomment{trigger rebuild}
\pmclassification{msc}{46B99}

% this is the default PlanetMath preamble.  as your knowledge
% of TeX increases, you will probably want to edit this, but
% it should be fine as is for beginners.

% almost certainly you want these
\usepackage{amssymb}
\usepackage{amsmath}
\usepackage{amsfonts}
\usepackage{amsthm}

\usepackage{mathrsfs}

% used for TeXing text within eps files
%\usepackage{psfrag}
% need this for including graphics (\includegraphics)
%\usepackage{graphicx}
% for neatly defining theorems and propositions
%
% making logically defined graphics
%%%\usepackage{xypic}

% there are many more packages, add them here as you need them

% define commands here

\newcommand{\sR}[0]{\mathbb{R}}
\newcommand{\sC}[0]{\mathbb{C}}
\newcommand{\sN}[0]{\mathbb{N}}
\newcommand{\sZ}[0]{\mathbb{Z}}

 \usepackage{bbm}
 \newcommand{\Z}{\mathbbmss{Z}}
 \newcommand{\C}{\mathbbmss{C}}
 \newcommand{\F}{\mathbbmss{F}}
 \newcommand{\R}{\mathbbmss{R}}
 \newcommand{\Q}{\mathbbmss{Q}}



\newcommand*{\norm}[1]{\lVert #1 \rVert}
\newcommand*{\abs}[1]{| #1 |}



\newtheorem{thm}{Theorem}
\newtheorem{defn}{Definition}
\newtheorem{prop}{Proposition}
\newtheorem{lemma}{Lemma}
\newtheorem{cor}{Corollary}
\begin{document}
% This is written from scratch with no reference

Let $V$ be a vector space over a field $F$ (real or complex), and let 
$\Vert\cdot \Vert$ be a norm on $V$. Further, 
for $r>0$, $v\in V$, let
$$
 B_r(v) = \{ w\in V: \Vert w-v\Vert < r \}.
$$
Then for any non-zero $\lambda\in F$, we have 
$$
  \lambda B_r(v) = B_{|\lambda| r}(\lambda v).
$$

The claim is clear for $\lambda =0$, so we can assume that $\lambda \neq 0$. 
Then
\begin{eqnarray*}
\lambda B_r(v) &=&  \{ z\in V: \Vert w-v\Vert < r\ \mbox{and}\ z=\lambda w \} \\
               &=&  \{ z\in V: \Vert \frac{z}{\lambda}-v\Vert < r \} \\
               &=&  \{ z\in V: \Vert z-\lambda v\Vert < |\lambda| r \} \\
               &=&  B_{|\lambda| r}(\lambda v).
\end{eqnarray*}
%%%%%
%%%%%
\end{document}
