\documentclass[12pt]{article}
\usepackage{pmmeta}
\pmcanonicalname{PolynomialFunctionalCalculus}
\pmcreated{2013-03-22 18:48:23}
\pmmodified{2013-03-22 18:48:23}
\pmowner{asteroid}{17536}
\pmmodifier{asteroid}{17536}
\pmtitle{polynomial functional calculus}
\pmrecord{8}{41608}
\pmprivacy{1}
\pmauthor{asteroid}{17536}
\pmtype{Feature}
\pmcomment{trigger rebuild}
\pmclassification{msc}{46H30}
\pmclassification{msc}{47A60}
\pmrelated{FunctionalCalculus}
\pmrelated{ContinuousFunctionalCalculus2}
\pmrelated{BorelFunctionalCalculus}
\pmdefines{polynomial spectral mapping theorem}

% this is the default PlanetMath preamble.  as your knowledge
% of TeX increases, you will probably want to edit this, but
% it should be fine as is for beginners.

% almost certainly you want these
\usepackage{amssymb}
\usepackage{amsmath}
\usepackage{amsfonts}

% used for TeXing text within eps files
%\usepackage{psfrag}
% need this for including graphics (\includegraphics)
%\usepackage{graphicx}
% for neatly defining theorems and propositions
%\usepackage{amsthm}
% making logically defined graphics
%%%\usepackage{xypic}

% there are many more packages, add them here as you need them

% define commands here

\begin{document}
\PMlinkescapeword{similar}

Let $\mathcal{A}$ be an unital associative algebra over $\mathbb{C}$ with identity element $e$ and let $a \in \mathcal{A}$.

The {\bf polynomial functional calculus} is the most basic form of a functional calculus. It allows the expression
\begin{align*}
p(a)
\end{align*}
to make sense as an element of $\mathcal{A}$, for any polynomial $p:\mathbb{C} \longrightarrow \mathbb{C}$.

This is achieved in the following natural way: for any polynomial $p(\lambda) := \sum c_n \, \lambda^n$ we \PMlinkescapetext{associate} the element $p(a):= \sum c_n\, a^n \in \mathcal{A}$.

\section{Definition}

Recall that the set of polynomial functions in $\mathbb{C}$, denoted by $\mathbb{C}[\lambda]$, is an associative algebra over $\mathbb{C}$ under pointwise operations and is generated by the constant polynomial $1$ and the variable $\lambda$ (corresponding to the identity function in $\mathbb{C}$).

Moreover, any homomorphism from the algebra $\mathbb{C}[\lambda]$ is perfectly determined by the values of $1$ and $\lambda$.

{\bf Definition -} \emph{Consider the algebra homomorphism $\pi: \mathbb{C}[\lambda] \longrightarrow \mathcal{A}$ such that $\pi(1) = e$ and $\pi(\lambda) = a$. This homomorphism is denoted by}
\begin{align*}
p \longmapsto p(a)
\end{align*}
\emph{and it is called the} {\bf polynomial functional calculus} \emph{for $a$.}


It is clear that for any polynomial $p(\lambda) := \sum c_n \lambda^n$ we have $p(a) = \sum c_n\, a^n$.

\section{Spectral Properties}

We will denote by $\sigma(x)$ the \PMlinkname{spectrum}{Spectrum} of an element $x \in \mathcal{A}$.


{\bf Theorem - (polynomial spectral mapping theorem) -} \emph{Let $\mathcal{A}$ be an unital associative algebra over $\mathbb{C}$ and $a$ an element in $\mathcal{A}$. For any polynomial $p$ we have that}
\begin{align*}
\sigma(p(a)) = p(\sigma(a))
\end{align*}

{\bf \emph{\PMlinkescapetext{Proof}:}} Let us first prove that $\sigma(p(a)) \subseteq p(\sigma(a))$. Suppose $\widetilde{\lambda} \in \sigma(p(a))$, which means that $p(a) - \widetilde{\lambda} e$ is not invertible. Now consider the polynomial in $\mathbb{C}$ given by $q:= p - \widetilde{\lambda}$. It is clear that $q(a) = p(a) - \widetilde{\lambda} e$, and therefore $q(a)$ is not invertible. Since $\mathbb{C}$ is \PMlinkname{algebraically closed}{FundamentalTheoremOfAlgebra}, we have that
\begin{align*}
q(\lambda) = (\lambda - \lambda_1)^{n_1} \cdots (\lambda - \lambda_k)^{n_k}
\end{align*}
for some $\lambda_1, \dots, \lambda_k \in \mathbb{C}$ and $n_1, \dots, n_k \in \mathbb{N}$. Thus, we can also write a similar product for $q(a)$ as
\begin{align*}
q(a) = (a - \lambda_1e)^{n_1} \cdots (a - \lambda_k e)^{n_k}
\end{align*}

Now, since $q(a)$ is not invertible we must have that at least one of the factors $(a-\lambda_i e)$ is not invertible, which means that for that particular $\lambda_i$ we have $\lambda_i \in \sigma(a)$. But we also have that $q(\lambda_i) = 0$, i.e. $p(\lambda_i) = \widetilde{\lambda}$, and hence $\widetilde{\lambda} \in p(\sigma(a))$.

We now prove the \PMlinkescapetext{opposite} inclusion $\sigma(p(a)) \supseteq p(\sigma(a))$. Suppose $\widetilde{\lambda} \in p(\sigma(a))$, which means that $\widetilde{\lambda} = p(\lambda_0)$ for some $\lambda_0 \in \sigma(a)$. The polynomial $p - \widetilde{\lambda}$ has a zero at $\lambda_0$, hence there is a polynomial $d$ such that
\begin{align*}
p(\lambda) - \widetilde{\lambda}= d(\lambda) (\lambda - \lambda_0)\,, \qquad\qquad \lambda \in \mathbb{C}
\end{align*}
Thus, we can also write a similar product for $q(a)$ as
\begin{align*}
p(a) - \widetilde{\lambda} e = d(a) (a - \lambda_0 e)
\end{align*}
If $p(a) - \widetilde{\lambda} e$ was invertible, then we would see that $a - \lambda_0 e$ had a \PMlinkname{left}{InversesInRings} and a \PMlinkname{right inverse}{InversesInRings}, thus being invertible. But we know that $\lambda_0 \in \sigma(a)$, hence we conclude that $p(a) - \widetilde{\lambda} e$ cannot be invertible, i.e. $\widetilde{\lambda} \in \sigma(p(a))$. $\square$
%%%%%
%%%%%
\end{document}
