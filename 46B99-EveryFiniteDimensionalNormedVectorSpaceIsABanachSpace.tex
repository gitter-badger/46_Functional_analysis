\documentclass[12pt]{article}
\usepackage{pmmeta}
\pmcanonicalname{EveryFiniteDimensionalNormedVectorSpaceIsABanachSpace}
\pmcreated{2013-03-22 14:56:31}
\pmmodified{2013-03-22 14:56:31}
\pmowner{matte}{1858}
\pmmodifier{matte}{1858}
\pmtitle{every finite dimensional normed vector space is a Banach space}
\pmrecord{10}{36633}
\pmprivacy{1}
\pmauthor{matte}{1858}
\pmtype{Theorem}
\pmcomment{trigger rebuild}
\pmclassification{msc}{46B99}

\endmetadata

% this is the default PlanetMath preamble.  as your knowledge
% of TeX increases, you will probably want to edit this, but
% it should be fine as is for beginners.

% almost certainly you want these
\usepackage{amssymb}
\usepackage{amsmath}
\usepackage{amsfonts}
\usepackage{amsthm}

\usepackage{mathrsfs}

% used for TeXing text within eps files
%\usepackage{psfrag}
% need this for including graphics (\includegraphics)
%\usepackage{graphicx}
% for neatly defining theorems and propositions
%
% making logically defined graphics
%%%\usepackage{xypic}

% there are many more packages, add them here as you need them

% define commands here

\newcommand{\sR}[0]{\mathbb{R}}
\newcommand{\sC}[0]{\mathbb{C}}
\newcommand{\sN}[0]{\mathbb{N}}
\newcommand{\sZ}[0]{\mathbb{Z}}

 \usepackage{bbm}
 \newcommand{\Z}{\mathbbmss{Z}}
 \newcommand{\C}{\mathbbmss{C}}
 \newcommand{\R}{\mathbbmss{R}}
 \newcommand{\Q}{\mathbbmss{Q}}



\newcommand*{\norm}[1]{\lVert #1 \rVert}
\newcommand*{\abs}[1]{| #1 |}



\newtheorem{thm}{Theorem}
\newtheorem{defn}{Definition}
\newtheorem{prop}{Proposition}
\newtheorem{lemma}{Lemma}
\newtheorem{cor}{Corollary}
\begin{document}
\begin{thm}
Every finite dimensional normed vector space is a Banach space.
\end{thm}

\emph{Proof.} Suppose $(V,\Vert\cdot\Vert)$ is the normed vector space, 
    and $(e_i)_{i=1}^N$ is a basis for $V$. 
For $x=\sum_{j=1}^N \lambda_j e_j$, we can then define 
$$
  \Vert x \Vert' = \sqrt{\sum_{j=1}^N |\lambda_j|^2}
$$
whence $\Vert\cdot\Vert'\colon V\to \sR$ is a norm for $V$. 
Since 
\PMlinkname{all norms on a finite dimensional vector space are equivalent}{ProofThatAllNormsOnFiniteVectorSpaceAreEquivalent}, 
there is a constant $C>0$ such that
$$
 \frac{1}{C} \Vert x \Vert' \le \Vert x \Vert \le C \Vert x \Vert', \quad x\in V.
$$
To prove that $V$ is a Banach space, let $x_1,x_2,\ldots$ be a Cauchy sequence
in $(V,\Vert\cdot \Vert)$. That is, 
   for all $\varepsilon>0$ there is an $M\ge 1$ such that 
$$
  \Vert x_j-x_k \Vert <\varepsilon, \ \ \mbox{for all} j,k\ge M.
$$
Let us write each $x_k$ in this sequence in the basis $(e_j)$ 
   as $x_k=\sum_{j=1}^N \lambda_{k,j} e_j$ for some constants 
   $\lambda_{k,j}\in \C$. 
For $k,l\ge 1$ we then have
\begin{eqnarray*}
\Vert x_k-x_l\Vert   &\ge& \frac{1}{C} \Vert x_k-x_l \Vert' \\
   &\ge& \frac{1}{C} \sqrt{\sum_{j=1}^N |\lambda_{k,j}-\lambda_{l,j}|^2} \\
   &\ge& \frac{1}{C} |\lambda_{k,j}-\lambda_{l,j}|
\end{eqnarray*}
for all $j=1,\ldots, N$.
It follows that 
   $(\lambda_{k,1})_{k=1}^\infty, \ldots, (\lambda_{k,N})_{k=1}^\infty$
   are Cauchy sequences in $\C$. As $\C$ is complete, these converge to 
   some complex numbers $\lambda_1, \ldots, \lambda_N$. 
   Let $x=\sum_{j=1}^N \lambda_j e_j$. 

For each $k=1,2,\ldots$, we then have
\begin{eqnarray*}
\Vert x-x_k\Vert &\le& C \Vert x-x_k\Vert' \\
                 &\le& C \sqrt{\sum_{j=1}^N |\lambda_{j}-\lambda_{k,j}|^2}.
\end{eqnarray*}
By taking $k\to \infty$ it follows that $(x_j)$ converges to $x\in V$. $\Box$
%%%%%
%%%%%
\end{document}
