\documentclass[12pt]{article}
\usepackage{pmmeta}
\pmcanonicalname{CharacterizationOfMaximalIdealsOfTheAlgebraOfContinuousFunctionsOnACompactSet}
\pmcreated{2013-03-22 17:45:08}
\pmmodified{2013-03-22 17:45:08}
\pmowner{rspuzio}{6075}
\pmmodifier{rspuzio}{6075}
\pmtitle{characterization of maximal ideals of the algebra of continuous functions on a compact set}
\pmrecord{9}{40203}
\pmprivacy{1}
\pmauthor{rspuzio}{6075}
\pmtype{Theorem}
\pmcomment{trigger rebuild}
\pmclassification{msc}{46L05}
\pmclassification{msc}{46J20}
\pmclassification{msc}{46J10}
\pmclassification{msc}{16W80}

\endmetadata

% this is the default PlanetMath preamble.  as your knowledge
% of TeX increases, you will probably want to edit this, but
% it should be fine as is for beginners.

% almost certainly you want these
\usepackage{amssymb}
\usepackage{amsmath}
\usepackage{amsfonts}

% used for TeXing text within eps files
%\usepackage{psfrag}
% need this for including graphics (\includegraphics)
%\usepackage{graphicx}
% for neatly defining theorems and propositions
\usepackage{amsthm}
% making logically defined graphics
%%%\usepackage{xypic}

% there are many more packages, add them here as you need them

% define commands here

\newtheorem{thm}{Theorem}
\begin{document}
Let $X$ be a compact topological space and let $C(X)$ be the algebra of continuous
real-valued functions on this space.  In this entry, we shall examine the maximal
ideals of this algebra.

\begin{thm}  Let $X$ be a compact topological space and $I$ be an ideal of $C(X)$.
Then either $I = C(x)$ or there exists a point $p \in X$ such that $f(p) = 0$ for
all $f \in I$.
\end{thm}

\begin{proof}
Assume that, for every point $p \in X$, there exists a continuous function $f \in I$ 
such that $f(p) \not= 0$.  Then, by continuity, there must exist an
open set $U$ containing $p$ so that $f(q) \not= 0$ for all $q \in U$.  Thus, we may
assign to each point $p \in X$ a continuous function $f \in I$ and
an open set $U$ of $X$ such that $f(q) \neq 0$ for all $q \in U$.  Since this 
collection of open sets covers $X$, which is compact, there must exists a finite
subcover which also covers $X$.  Call this subcover $U_1, \ldots, U_n$ and the
corresponding functions $f_1, \ldots f_n$.  Consider the function $g$ defined as
$g(x) = (f_1(x))^2 + \cdots + (f_n(x))^2$.  Since $I$ is an ideal, $g \in I$.  For every
point $p \in X$, there exists an integer $i$ between $1$ and $n$ such that $f_i (p) 
\not= 0$.  This implies that $g(p) \neq 0$.  Since $g$ is a continuous function on
a compact set, it must attain a minimum.  By construction of $g$, the value of $g$
at its minimum cannot be negative; by what we just showed, it cannot equal zero either.
Hence being bounded from below by a positive number, $g$ has a continuous inverse.  
But, if an ideal contains an invertible element, it must be the whole algebra.  Hence,
we conclude that either there exists a point $p \in x$ such that $f(p) = 0$ for all 
$f \in I$ or $I = C(x)$.
\end{proof}

\begin{thm}
Let $X$ be a compact Hausdorff topological space.  Then an ideal is maximal if
and only if it is the ideal of all points which go zero at a given point.
\end{thm}

\begin{proof}
By the previous theorem, every non-trivial ideal must be a subset of an ideal of
functions which vanish at a given point.  Hence, it only remains to prove that
ideals of functions vanishing at a point is maxiamal.

Let $p$ be a point of $X$.  Assume that the ideal of functions vanishing at $p$
is properly contained in ideal $I$.  Then there must exist a function $f \in I$
such that $f (p) \neq 0$ (otherwise, the inclusion would not be proper).  Since
$f$ is continuous, there will exist an open neighborhood $U$ of $p$ such that
$f(x) \neq 0$ when $x \in U$.  By Urysohn's theorem, there exists a continuous 
function $h \colon X \to \mathbb{R}$ such that $f(p) = 0$ and $f(x) = 0$ for
all $x \in X \setminus U$.  Since $I$ was assumed to contain all functions vanishing 
at $p$, we must have $f \in I$.  Hence, the function $g$ defined by $g(x) = 
(f(x))^2 + (h(x))^2$ must also lie in $I$.  By construction, $g(g) > 0$
when $x \in U$ and when $g(x) \in X \setminus U$.  Because $X$ is compact,
$g$ must attain a minimum somewhere, hence is bounded from below by a
positive number.  Thus $g$ has a continuous inverse, so $I = C(X)$, hence the
ideal of functions vanishing at $p$ is maximal.
\end{proof}

%%%%%
%%%%%
\end{document}
