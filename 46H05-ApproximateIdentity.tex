\documentclass[12pt]{article}
\usepackage{pmmeta}
\pmcanonicalname{ApproximateIdentity}
\pmcreated{2013-03-22 17:30:22}
\pmmodified{2013-03-22 17:30:22}
\pmowner{asteroid}{17536}
\pmmodifier{asteroid}{17536}
\pmtitle{approximate identity}
\pmrecord{6}{39895}
\pmprivacy{1}
\pmauthor{asteroid}{17536}
\pmtype{Definition}
\pmcomment{trigger rebuild}
\pmclassification{msc}{46H05}
\pmsynonym{approximate unit}{ApproximateIdentity}
\pmdefines{left approximate identity}
\pmdefines{right approximate identity}

% this is the default PlanetMath preamble.  as your knowledge
% of TeX increases, you will probably want to edit this, but
% it should be fine as is for beginners.

% almost certainly you want these
\usepackage{amssymb}
\usepackage{amsmath}
\usepackage{amsfonts}

% used for TeXing text within eps files
%\usepackage{psfrag}
% need this for including graphics (\includegraphics)
%\usepackage{graphicx}
% for neatly defining theorems and propositions
%\usepackage{amsthm}
% making logically defined graphics
%%%\usepackage{xypic}

% there are many more packages, add them here as you need them

% define commands here

\begin{document}
Let $\mathcal{A}$ be a Banach algebra.

A {\bf left approximate identity} for $\mathcal{A}$ is a net $(e_{\lambda})_{\lambda \in \Lambda}$ in $\mathcal{A}$ which \PMlinkescapetext{satisfies}:
\begin{enumerate}
\item $\|e_{\lambda}\| < C \;\;\;\; \forall_{\lambda \in \Lambda} \;$, for some constant $C$.
\item $e_{\lambda}a \longrightarrow a\;$, for every $a \in \mathcal{A}$.
\end{enumerate}

Similarly, a {\bf right approximate identity} for $\mathcal{A}$ is a net $(e_{\lambda})_{\lambda \in \Lambda}$ in $\mathcal{A}$ which \PMlinkescapetext{satisfies}:
\begin{enumerate}
\item $\|e_{\lambda}\| < C \;\;\;\; \forall_{\lambda \in \Lambda} \;$, for some constant $C$.
\item $ae_{\lambda} \longrightarrow a\;$, for every $a \in \mathcal{A}$.
\end{enumerate}

An {\bf approximate identity} for a $\mathcal{A}$ is a net $(e_{\lambda})_{\lambda \in \Lambda}$ in $\mathcal{A}$ which is both a left and right approximate identity.

\subsubsection{Remarks:}
\begin{itemize}
\item There are examples of Banach algebras that do not have approximate \PMlinkescapetext{identities}.
\item If $\mathcal{A}$ has an identity element $e$, then clearly $e$ itself is an approximate identity for $\mathcal{A}$.
\end{itemize}
%%%%%
%%%%%
\end{document}
