\documentclass[12pt]{article}
\usepackage{pmmeta}
\pmcanonicalname{SpectralMappingTheorem}
\pmcreated{2013-03-22 17:30:08}
\pmmodified{2013-03-22 17:30:08}
\pmowner{asteroid}{17536}
\pmmodifier{asteroid}{17536}
\pmtitle{spectral mapping theorem}
\pmrecord{4}{39891}
\pmprivacy{1}
\pmauthor{asteroid}{17536}
\pmtype{Theorem}
\pmcomment{trigger rebuild}
\pmclassification{msc}{46L05}
\pmclassification{msc}{47A60}
\pmclassification{msc}{46H30}

\endmetadata

% this is the default PlanetMath preamble.  as your knowledge
% of TeX increases, you will probably want to edit this, but
% it should be fine as is for beginners.

% almost certainly you want these
\usepackage{amssymb}
\usepackage{amsmath}
\usepackage{amsfonts}

% used for TeXing text within eps files
%\usepackage{psfrag}
% need this for including graphics (\includegraphics)
%\usepackage{graphicx}
% for neatly defining theorems and propositions
%\usepackage{amsthm}
% making logically defined graphics
%%%\usepackage{xypic}

% there are many more packages, add them here as you need them

% define commands here

\begin{document}
Let $\mathcal{A}$ be a unital \PMlinkname{$C^*$-algebra}{CAlgebra}. Let $x$ be a normal element in $\mathcal{A}$ and $\sigma(x)$ be its spectrum.

The continuous functional calculus provides a $C^*$-isomorphism
\begin{center}
$C(\sigma(x)) \longrightarrow \mathcal{A}[x]$

$\;\;f \mapsto f(x)$
\end{center}
between the $C^*$-algebra $C(\sigma(x))$ of complex valued continuous functions on $\sigma(x)$ and the $C^*$-subalgebra $\mathcal{A}[x] \subseteq \mathcal{A}$ generated by $x$ and the identity of $\mathcal{A}$.

{\bf Spectral Mapping Theorem -} Let $x \in \mathcal{A}$ be as above. Let $f \in C(\sigma(x))$. Then
\begin{displaymath}
\sigma(f(x))=f(\sigma(x)).
\end{displaymath}

{\bf Proof :} Since $C(\sigma(x))$ and $\mathcal{A}[x]$ are isomorphic we must have
\begin{displaymath}
\sigma(f) = \sigma_{\mathcal{A}[x]}(f(x))
\end{displaymath}
where $\sigma_{\mathcal{A}[x]}(f(x))$ denotes the spectrum of $f(x)$ relative to the subalgebra $\mathcal{A}[x]$.

By the spectral invariance theorem we have $\sigma_{\mathcal{A}[x]}(f(x))=\sigma(f(x))$. Hence
\begin{displaymath}
\sigma(f) = \sigma(f(x))
\end{displaymath}

Thus, we only have to prove that $f(\sigma(x)) = \sigma(f)$.

$f$ is defined on $\sigma(x)$ so $f(\sigma(x))$ is precisely the image of $f$.

Let $\lambda \in \mathbb{C}$. The function $f - \lambda$ is invertible if and only if $f - \lambda$ has no zeros.

Equivalently, $f - \lambda$ is not invertible if and only if $f - \lambda$ has a zero, i.e. $f(\lambda_0) = \lambda$ for some $\lambda_0$.

The previous statement can be reformulated as: $\lambda \in \sigma(f)$ if and only if $\lambda$ is in the image of $f$.

We conclude that $\sigma(f)=f(\sigma(x))$, and this proves the theorem. $\square$
%%%%%
%%%%%
\end{document}
