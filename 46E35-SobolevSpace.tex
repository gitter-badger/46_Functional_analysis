\documentclass[12pt]{article}
\usepackage{pmmeta}
\pmcanonicalname{SobolevSpace}
\pmcreated{2013-03-22 14:54:55}
\pmmodified{2013-03-22 14:54:55}
\pmowner{paolini}{1187}
\pmmodifier{paolini}{1187}
\pmtitle{Sobolev space}
\pmrecord{10}{36601}
\pmprivacy{1}
\pmauthor{paolini}{1187}
\pmtype{Definition}
\pmcomment{trigger rebuild}
\pmclassification{msc}{46E35}
\pmsynonym{Sobolev function}{SobolevSpace}
\pmrelated{WeakDerivative}

% this is the default PlanetMath preamble.  as your knowledge
% of TeX increases, you will probably want to edit this, but
% it should be fine as is for beginners.

% almost certainly you want these
\usepackage{amssymb}
\usepackage{amsmath}
\usepackage{amsfonts}

% used for TeXing text within eps files
%\usepackage{psfrag}
% need this for including graphics (\includegraphics)
%\usepackage{graphicx}
% for neatly defining theorems and propositions
%\usepackage{amsthm}
% making logically defined graphics
%%%\usepackage{xypic}

% there are many more packages, add them here as you need them

% define commands here
\newcommand{\R}{\mathbf R}
\begin{document}
We define the Sobolev spaces of functions $W^{m,p}(\Omega)$ where $\Omega$ is an open subset of $\R^n$, $m\ge 0$ is an integer and $p\in[1,+\infty]$.

The spaces $W^{0,p}(\Omega)$ are simply defined to be the spaces $L^p(\Omega)$ of Lebesgue $p$-summable functions. 
We then define the space $W^{m,p}(\Omega)$ to be the space of functions $u\in L^p(\Omega)$ which have weak derivatives $g=(g_1,\ldots,g_n)$ such that $g_i\in W^{m-1,p}(\Omega)$.

The space $W^{m,p}$ turns out to be a Banach space when endowed with the norm
\[
  \Vert u \Vert_{W^{m,p}}=
   \sum_{k=0}^m 
   \sum_{i_1=1}^n \cdots \sum_{i_k=1}^n
  \left[\int_\Omega 
\left|\frac{\partial^k u(x)}{\partial x_{i_1}\cdots\partial x_{i_k}}\right|^p
\, dx\right]^{\frac 1 p}
\]
i.e.\ the sum of the $L^p$ norms of $u$ and of all weak derivatives of $u$ up to the $m$-th order. 

Of particular interest are the spaces $H^m(\Omega):=W^{m,2}(\Omega)$ which turn out to be Hilbert spaces with the scalar product given by
\[
 (u,v)_{H^m(\Omega)}=\sum_{k=0}^m \sum_{i_1=1}^n \cdots \sum_{i_k=1}^n
  \int_\Omega 
\frac{\partial^k u(x)}{\partial x_{i_1}\cdots\partial x_{i_k}}
\frac{\partial^k v(x)}{\partial x_{i_1}\cdots\partial x_{i_k}}
\, dx.
\]
%%%%%
%%%%%
\end{document}
