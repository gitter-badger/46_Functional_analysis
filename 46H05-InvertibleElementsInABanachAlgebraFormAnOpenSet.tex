\documentclass[12pt]{article}
\usepackage{pmmeta}
\pmcanonicalname{InvertibleElementsInABanachAlgebraFormAnOpenSet}
\pmcreated{2013-03-22 17:23:22}
\pmmodified{2013-03-22 17:23:22}
\pmowner{asteroid}{17536}
\pmmodifier{asteroid}{17536}
\pmtitle{invertible elements in a Banach algebra form an open set}
\pmrecord{6}{39757}
\pmprivacy{1}
\pmauthor{asteroid}{17536}
\pmtype{Theorem}
\pmcomment{trigger rebuild}
\pmclassification{msc}{46H05}

% this is the default PlanetMath preamble.  as your knowledge
% of TeX increases, you will probably want to edit this, but
% it should be fine as is for beginners.

% almost certainly you want these
\usepackage{amssymb}
\usepackage{amsmath}
\usepackage{amsfonts}

% used for TeXing text within eps files
%\usepackage{psfrag}
% need this for including graphics (\includegraphics)
%\usepackage{graphicx}
% for neatly defining theorems and propositions
%\usepackage{amsthm}
% making logically defined graphics
%%%\usepackage{xypic}

% there are many more packages, add them here as you need them

% define commands here

\begin{document}
{\bf Theorem -} Let $\mathcal{A}$ be a Banach algebra with identity element $e$ and $G(\mathcal{A})$ be the set of invertible elements in $\mathcal{A}$. Let $B_r(x)$ denote the open ball of radius $r$ centered in $x$.

Then, for all $x \in G(\mathcal{A})$ we have that

\begin{displaymath}
B_{\|x^{-1}\|^{-1}}(x) \subseteq G(\mathcal{A})
\end{displaymath}

and therefore $G(\mathcal{A})$ is open in $\mathcal{A}$.

{\bf Proof :} Let $x \in G(\mathcal{A})$ and $y \in B_{\|x^{-1}\|^{-1}}(x)$. We have that

\begin{displaymath}
\|e-x^{-1}y\| = \|x^{-1}x-x^{-1}y\| = \|x^{-1}(x-y)\| \le \|x^{-1}\|\|x-y\| < \|x^{-1}\|\|x^{-1}\|^{-1} = 1
\end{displaymath}

So, by the \PMlinkname{Neumann series}{NeumannSeriesInBanachAlgebras} we conclude that $e-(e-x^{-1}y)$ is invertible,
 i.e. $x^{-1}y \in G(\mathcal{A})$.

As $G(\mathcal{A})$ is a group we must have $y \in G(\mathcal{A})$.

So $B_{\|x^{-1}\|^{-1}}(x) \subseteq G(\mathcal{A})$ and the theorem follows. $\square$
%%%%%
%%%%%
\end{document}
