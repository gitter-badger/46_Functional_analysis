\documentclass[12pt]{article}
\usepackage{pmmeta}
\pmcanonicalname{Ellp}
\pmcreated{2013-03-22 12:19:03}
\pmmodified{2013-03-22 12:19:03}
\pmowner{rspuzio}{6075}
\pmmodifier{rspuzio}{6075}
\pmtitle{ell^p}
\pmrecord{25}{31929}
\pmprivacy{1}
\pmauthor{rspuzio}{6075}
\pmtype{Definition}
\pmcomment{trigger rebuild}
\pmclassification{msc}{46B99}
\pmclassification{msc}{54E50}
\pmrelated{EllpXSpace}
\pmdefines{$\ell^\infty$}
\pmdefines{$\ell^2$}

% this is the default PlanetMath preamble.  as your knowledge
% of TeX increases, you will probably want to edit this, but
% it should be fine as is for beginners.

% almost certainly you want these
\usepackage{amssymb}
\usepackage{amsmath}
\usepackage{amsfonts}

% used for TeXing text within eps files
%\usepackage{psfrag}
% need this for including graphics (\includegraphics)
%\usepackage{graphicx}
% for neatly defining theorems and propositions
%\usepackage{amsthm}
% making logically defined graphics
%%%\usepackage{xypic} 

% there are many more packages, add them here as you need them

% define commands here

\begin{document}
Let $\mathbb{F}$ be either $\mathbb{R}$ or $\mathbb{C}$, and let $p\in\mathbb{R}$ with $p\geq 1$.  We define $\ell^p$ to be the set of all sequences $(a_i)_{i\geq 0}$ in $\mathbb{F}$ such that $$\sum_{i=0}^{\infty}|a_i|^p$$ converges. 

We also define $\ell^{\infty}$ to be the set of all \PMlinkname{bounded}{BoundedInterval} sequences $(a_i)_{i\geq 0}$ with norm given by $$\Vert (a_i)\Vert_{\infty} = \operatorname{sup}\{ |a_i|:i\geq 0\}.$$

By defining addition and scalar multiplication pointwise, $\ell^p(\mathbb{F})$ and
$\ell^\infty(\mathbb{F})$ have a natural vector space stucture.
That the sum of two elements on $\ell^p(\mathbb{F})$ is again an element
in $\ell^p(\mathbb{F})$ follows from Minkowski inequality
(see below).
We can make $\ell^p$ into a normed vector space, by defining the norm as $$\Vert (a_i)\Vert_p = (\sum_{i=0}^{\infty}|a_i|^p)^{1/p}.$$

The normed vector spaces $\ell^{\infty}$ and $\ell^p$ for $p\geq 1$ are complete under these norms, making them into Banach spaces.  Moreover, $\ell^2$ is a Hilbert space under the inner product $$\langle (a_i),(b_i)\rangle = \sum_{i=0}^{\infty}a_i \overline{b_i}$$ where $\overline{x}$ denotes the complex conjugate of $x$.

For $p>1$ the (continuous) dual space of $\ell^p$ is $\ell^q$ where $\frac{1}{p} + \frac{1}{q}=1$, and the dual space of $\ell^1$ is $\ell^{\infty}$.

\subsubsection*{Properties}
\begin{enumerate}
\item If $a=(a_0,a_1, \ldots ) \in \ell^p(\mathbb{F})$ for $1\le p< \infty$, then
$\lim_{k\to \infty} a_k =0$.
(\PMlinkname{proof.}{ThenA_kto0IfSum_k1inftyA_kConverges})
\item For $1\le p<\infty$, $\ell^p(\mathbb{F})$ is separable, and $\ell^\infty(\mathbb{F})$
is not separable.
\item Minkowski inequality. If $a,b\in \ell^p(\mathbb{F})$ where $p\ge 1$, then
$$
\Vert a+b \Vert_p \le \Vert a\Vert_p + \Vert b \Vert_p.
$$
\end{enumerate}

%%%%%
%%%%%
\end{document}
