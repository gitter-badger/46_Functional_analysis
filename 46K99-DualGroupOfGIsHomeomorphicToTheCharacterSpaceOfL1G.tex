\documentclass[12pt]{article}
\usepackage{pmmeta}
\pmcanonicalname{DualGroupOfGIsHomeomorphicToTheCharacterSpaceOfL1G}
\pmcreated{2013-03-22 17:42:49}
\pmmodified{2013-03-22 17:42:49}
\pmowner{asteroid}{17536}
\pmmodifier{asteroid}{17536}
\pmtitle{dual group of $G$ is homeomorphic to the character space of $L^1(G)$}
\pmrecord{5}{40158}
\pmprivacy{1}
\pmauthor{asteroid}{17536}
\pmtype{Theorem}
\pmcomment{trigger rebuild}
\pmclassification{msc}{46K99}
\pmclassification{msc}{43A40}
\pmclassification{msc}{43A20}
\pmclassification{msc}{22D20}
\pmclassification{msc}{22D15}
\pmclassification{msc}{22D35}
\pmclassification{msc}{22B10}
\pmclassification{msc}{22B05}
\pmrelated{L1GIsABanachAlgebra}

\endmetadata

% this is the default PlanetMath preamble.  as your knowledge
% of TeX increases, you will probably want to edit this, but
% it should be fine as is for beginners.

% almost certainly you want these
\usepackage{amssymb}
\usepackage{amsmath}
\usepackage{amsfonts}

% used for TeXing text within eps files
%\usepackage{psfrag}
% need this for including graphics (\includegraphics)
%\usepackage{graphicx}
% for neatly defining theorems and propositions
%\usepackage{amsthm}
% making logically defined graphics
%%%\usepackage{xypic}

% there are many more packages, add them here as you need them

% define commands here

\begin{document}
Let $G$ be a locally compact \PMlinkname{abelian}{AbelianGroup2} \PMlinkname{group}{TopologicalGroup} and $L^1(G)$ its group algebra.

Let $\hat{G}$ denote the Pontryagin dual of $G$ and $\Delta$ the character space of $L^1(G)$, i.e. the set of multiplicative linear functionals of $L^1(G)$ endowed with the weak-* topology.

{\bf Theorem -} The spaces $\hat{G}$ and $\Delta$ are homeomorphic. The homeomorphism is given by
\begin{displaymath}
\omega \longmapsto \phi_{\omega}\;, \qquad\qquad \omega \in \hat{G}
\end{displaymath}
where $\phi_{\omega} \in \Delta$ is defined by
\begin{displaymath}
\phi_{\omega}(f):=\int_G f(s)\omega(s)\;d\mu(s)\;, \qquad\qquad f \in L^1(G)
\end{displaymath}
%%%%%
%%%%%
\end{document}
