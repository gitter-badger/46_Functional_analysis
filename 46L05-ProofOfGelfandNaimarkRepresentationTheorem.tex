\documentclass[12pt]{article}
\usepackage{pmmeta}
\pmcanonicalname{ProofOfGelfandNaimarkRepresentationTheorem}
\pmcreated{2013-03-22 18:01:17}
\pmmodified{2013-03-22 18:01:17}
\pmowner{asteroid}{17536}
\pmmodifier{asteroid}{17536}
\pmtitle{proof of Gelfand-Naimark representation theorem}
\pmrecord{7}{40538}
\pmprivacy{1}
\pmauthor{asteroid}{17536}
\pmtype{Proof}
\pmcomment{trigger rebuild}
\pmclassification{msc}{46L05}
%\pmkeywords{Gelfand-Naimark representation theorem}
%\pmkeywords{algebra of bounded operators of a Hilbert space $H$}
\pmrelated{GelfandNaimarkRepresentationTheorem}

% this is the default PlanetMath preamble.  as your knowledge
% of TeX increases, you will probably want to edit this, but
% it should be fine as is for beginners.

% almost certainly you want these
\usepackage{amssymb}
\usepackage{amsmath}
\usepackage{amsfonts}

% used for TeXing text within eps files
%\usepackage{psfrag}
% need this for including graphics (\includegraphics)
%\usepackage{graphicx}
% for neatly defining theorems and propositions
%\usepackage{amsthm}
% making logically defined graphics
%%%\usepackage{xypic}

% there are many more packages, add them here as you need them

% define commands here

\begin{document}
\PMlinkescapephrase{representation}
\PMlinkescapephrase{representations}

{\bf Proof:} Let $\mathcal{A}$ be a \PMlinkname{$C^*$-algebra}{CAlgebra}. We intend to prove that $\mathcal{A}$ is isometrically isomorphic to a norm closed *-subalgebra of $B(H)$, the algebra of bounded operators of a suitable Hilbert space $H$.

Let $S(\mathcal{A})$ denote the state space of $\mathcal{A}$. For every state $\phi \in S(\mathcal{A})$ the Gelfand-Naimark-Segal construction allows one to construct a \PMlinkname{representation}{BanachAlgebraRepresentation} $\pi_{\phi}: \mathcal{A} \longrightarrow B(H_{\phi})$ of $\mathcal{A}$ in a Hilbert space $H_{\phi}$.

Now consider the \PMlinkname{direct sum}{BanachAlgebraRepresentation} of these representations $\displaystyle \pi := \bigoplus_{\phi \in S(\mathcal{A})} \pi_{\phi}$. Recall that $\pi$ is a representation
\begin{align*}
\pi:\mathcal{A} \longrightarrow B\Big(\bigoplus_{\phi \in S(\mathcal{A})} H_{\phi}\Big)
\end{align*}
of $\mathcal{A}$ in the \PMlinkname{direct sum of the family of Hilbert spaces}{DirectSumOfHilbertSpaces} $\{H_{\phi}\}_{\phi \in S(\mathcal{A})}$.

We now prove that this representation is injective.

Suppose there exists $a \in \mathcal{A}$ such that $\pi(a)=0$. Then, for all $\phi \in S(\mathcal{A})$, $\pi_{\phi}(a)=0$.  Thus, by definition of $\pi_{\phi}$,
\begin{align*}
\phi(a)=\langle \pi_{\phi}(a) \xi_{\phi}, \xi_{\phi} \rangle = 0
\end{align*}
where $\xi_{\phi} \in H_{\phi}$ is the cyclic vector associated with $\pi_{\phi}$. Since for every $\phi \in S(\mathcal{A})$ we have $\phi(a)=0$, we can conclude that $a=0$ (see \PMlinkname{this entry}{PropertiesOfStates}), i.e. $\pi$ is injective.

Since an \PMlinkname{injective *-homomorphism between $C^*$-algebras is isometric}{InjectiveCAlgebraHomomorphismIsIsometric}, we conclude that $\pi$ is also isometric. Hence $\pi(\mathcal{A})$ is a closed *-subalgebra of $\displaystyle B\Big(\bigoplus_{\phi \in S(\mathcal{A})} H_{\phi}\Big)$. Thus, we have proven that $\pi$ is an isometric isomorphism between $\mathcal{A}$ and a closed *-subalgebra of $B(H)$, for a suitable Hilbert space $H$. $\square$
%%%%%
%%%%%
\end{document}
