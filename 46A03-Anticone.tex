\documentclass[12pt]{article}
\usepackage{pmmeta}
\pmcanonicalname{Anticone}
\pmcreated{2013-03-22 17:20:48}
\pmmodified{2013-03-22 17:20:48}
\pmowner{stevecheng}{10074}
\pmmodifier{stevecheng}{10074}
\pmtitle{anti-cone}
\pmrecord{8}{39703}
\pmprivacy{1}
\pmauthor{stevecheng}{10074}
\pmtype{Definition}
\pmcomment{trigger rebuild}
\pmclassification{msc}{46A03}
\pmclassification{msc}{46A20}
\pmsynonym{anticone}{Anticone}
\pmsynonym{dual cone}{Anticone}
\pmrelated{GeneralizedFarkasLemma}

% The standard font packages
\usepackage{amssymb}
\usepackage{amsmath}
\usepackage{amsfonts}

% For neatly defining theorems and definitions
\usepackage{amsthm}

% Including EPS/PDF graphics (\includegraphics)
%\usepackage{graphicx}

% Making matrix-based graphics
%%%\usepackage{xypic}

% Enumeration lists with different styles
%\usepackage{enumerate}

% Set up the theorem environments
%\newtheorem{thm}{Theorem}
%\newtheorem*{thm*}{Theorem}
\newtheorem*{defn*}{Definition}
\newtheorem{prop}{Property}

\newcommand{\defnterm}[1]{\emph{#1}}

% The standard number systems
\newcommand{\complex}{\mathbb{C}}
\newcommand{\real}{\mathbb{R}}
\newcommand{\rat}{\mathbb{Q}}
\newcommand{\nat}{\mathbb{N}}
\newcommand{\intset}{\mathbb{Z}}

% Absolute values and norms
% Normal, wide, and big versions of the delimeters
\newcommand{\abs}[1]{\lvert#1\rvert}
\newcommand{\absW}[1]{\left\lvert#1\right\rvert}
\newcommand{\absB}[1]{\Bigl\lvert#1\Bigr\rvert}
\newcommand{\norm}[1]{\lVert#1\rVert}
\newcommand{\normW}[1]{\left\lVert#1\right\rVert}
\newcommand{\normB}[1]{\Bigl\lVert#1\Bigr\rVert}

% Inverse functions
\newcommand{\inv}[1]{{#1}^{-1}}

% Differentiation operators
\newcommand{\od}[2]{\frac{d #1}{d #2}}
\newcommand{\pd}[2]{\frac{\partial #1}{\partial #2}}
\newcommand{\pdd}[2]{\frac{\partial^2 #1}{\partial #2}}
\newcommand{\ipd}[2]{\partial #1 / \partial #2}

% Differentials on integrals
\newcommand{\dx}{\, dx}
\newcommand{\dt}{\, dt}
\newcommand{\dmu}{\, d\mu}

% Inner products
\newcommand{\ip}[2]{\langle {#1}, {#2} \rangle}

% Complex numbers
\DeclareMathOperator{\zRe}{Re}
\DeclareMathOperator{\zIm}{Im}
\newcommand{\conjug}[1]{\overline{#1}}

% Calligraphic letters
\newcommand{\sF}{\mathcal{F}}
\newcommand{\sD}{\mathcal{D}}

% Standard spaces
\newcommand{\Hilb}{\mathcal{H}}
\newcommand{\Le}{\mathbf{L}}

% Operators and functions occassionally used in my articles
\DeclareMathOperator{\D}{D}
\DeclareMathOperator{\linspan}{span}
\DeclareMathOperator{\rank}{rank}
\DeclareMathOperator{\lindim}{dim}
\DeclareMathOperator{\sinc}{sinc}

% Probability stuff
\newcommand{\PP}{\mathbb{P}}
\newcommand{\E}{\mathbb{E}}

\begin{document}
Let $X$ be a real vector space, and $\Phi$ be a subspace of linear functionals
on $X$.

For any set $S \subseteq X$, 
its \defnterm{anti-cone} $S^+$,
with respect to $\Phi$, is the set
\[
S^+ = \{ \phi \in \Phi \colon \phi(x) \geq 0 \,, \text{ for all } x \in S \}\,.
\]

The anti-cone is also called the \emph{dual cone}.

\subsection*{Usage}

The anti-cone operation is generally applied to subsets of $X$
that are themselves
cones.
Recall that a cone in a real vector space generalize the notion of
linear inequalities in a finite number of real variables.
The dual cone provides a natural way to transfer such
inequalities in the original vector space
to its dual space.
The concept is useful in the theory of 
duality.

The set $\Phi$ in the definition may be taken to be any subspace
of the algebraic dual space $X^*$.
The set $\Phi$ often needs to be restricted
to a subspace smaller than $X^*$, or even 
the continuous dual space $X'$,
in order to obtain 
the nice closure and reflexivity properties below.

\subsection*{Basic properties}

\begin{prop}\label{prop:is-cone}
The anti-cone is a convex cone in $\Phi$. 
\begin{proof}
If $\phi(x)$ is non-negative, then so is $t\phi(x)$ for $t > 0$.
And if $\phi_1(x), \phi_2(x) \geq 0$,
then clearly $(1-t)\phi_1(x) + t\phi_2(x) \geq 0$ for $0 \leq t \leq 1$.
\end{proof}
\end{prop}

\begin{prop}\label{prop:lower-bound}
If $K \subseteq X$ is a cone, then its anti-cone $K^+$ may be equivalently 
characterized as:
\[
K^+ = \{ \phi \in \Phi \colon 
\text{$\phi(x)$ over $x \in K$ is bounded below} \}\,.
\]
\begin{proof}
It suffices to show that if $\inf_{x \in K} \phi(x)$ is bounded below, 
then it is non-negative.
If it were negative, take some $x \in K$ such that
$\phi(x) < 0$.  For any $t > 0$, the vector $tx$ is in the cone $K$,
and the function value $\phi(tx) = t\phi(x)$ would be arbitrarily
large negative, and hence unbounded below.
\end{proof}
\end{prop}

\subsection*{Topological properties}

\textbf{Assumptions.}
Assume that $\Phi$ separates points of $X$.
Let $X$ have the weak topology generated by $\Phi$,
and let $\Phi$ have the weak-* topology generated by $X$;
this makes $X$ and $\Phi$ into Hausdorff topological vector spaces.

Vectors $x \in X$ will be identified
with their images $\hat{x}$ under the natural embedding of $X$
 in its double dual space.

The pairing $(X, \Phi)$ is sometimes called a dual pair; 
and $(\Phi, X)$, where $X$ is identified with its image in the double dual,
is also a dual pair.

\begin{prop}\label{prop:closed}
$S^+$ is weak-* closed.
\begin{proof}
Let $\{ \phi_\alpha \} \subseteq \Phi$ be a net converging to $\phi$
in the weak-* topology.
By definition, $\hat{x}(\phi_\alpha) = \phi_\alpha(x) \geq 0$.  
As the functional $\hat{x}$ is continuous in the weak-* topology,
we have $\hat{x}(\phi_\alpha) \to \hat{x}(\phi) \geq 0$.
Hence $\phi \in S^+$.
\end{proof}
\end{prop}

\begin{prop}\label{prop:closure}
$\overline{S}^+ = S^+$.
\begin{proof}
The inclusion $\overline{S}^+ \subseteq S^+$ is obvious.
And if $\phi(x) \geq 0$ for all $x \in S$,
then by continuity, this holds true for $x \in \overline{S}$ too ---
so $\overline{S}^+ \supseteq S^+$.
\end{proof}
\end{prop}

\subsection*{Properties involving cone inclusion}

\begin{prop}[Farkas' lemma]\label{prop:test}
Let $K \subseteq X$ be a weakly-closed convex cone.
Then $x \in K$ if and only if $\phi(x) \geq 0$ for all $\phi \in K^+$.
\begin{proof}
That $\phi(x) \geq 0$ for $\phi \in K^+$ and $x \in K$ is just the definition.
For the converse, we show that if $x \in X \setminus K$,
then there exists $\phi \in K^+$ such that $\phi(x) < 0$.

If $K=\emptyset$, then the desired $\phi \in \Phi = K^+$
exists because $\Phi$ can separate the points $x$ and $0$.
If $K \neq \emptyset$, by the hyperplane separation theorem,
there is a $\phi \in \Phi$ such that $\phi(x) < \inf_{y \in K} \phi(y)$.
This $\phi$ will automatically be in $K^+$ by Property \ref{prop:lower-bound}.
The zero vector is the weak limit of $ty$, as $t \searrow 0$,
for any vector $y$.
Thus $0 \in K$, and 
we conclude with $\inf_{y \in K} \phi(y) \leq 0$.
\end{proof}
\end{prop}

\begin{prop}\label{prop:double}
$K^{++} = \overline{K}$
for any convex cone $K$.  
(The anti-cone operation on $K^+$ is to be taken with respect to
$X$.)
\begin{proof}
We work with $\overline{K}$, which is a weakly-closed convex cone.
By Property \ref{prop:test},
$x \in \overline{K}$ if and only if $\phi(x) \geq 0$ for all $\phi \in 
\overline{K}^+ = K^+$.
But by definition of the second anti-cone,
$\hat{x} \in (K^+)^+$ if and only if 
$\phi(x) = \hat{x}(\phi) \geq 0$ for all $\phi \in K^+$.
\end{proof}
\end{prop}

\begin{prop}\label{prop:equiv}
Let $K$ and $L$ be convex cones in $X$, with $K$ weakly closed.
Then $K^+ \subseteq L^+$ if and only if $K \supseteq L$.
\begin{proof}
\[
K^+ \subseteq L^+ \implies
K = \overline{K} = K^{++} \supseteq L^{++} = \overline{L} \supseteq L \implies
K^+ \subseteq L^+\,. \qedhere
\]
\end{proof}
\end{prop}




\begin{thebibliography}{6}
\bibitem{CK}
B. D. Craven and J. J. Kohila.
``Generalizations of Farkas' Theorem.''
\emph{SIAM Journal on Mathematical Analysis}. 
Vol. 8, No. 6, November 1977.
\bibitem{Luenberger}
David G. Luenberger. \emph{Optimization by Vector Space Methods}.
John Wiley \& Sons, 1969.
\end{thebibliography}

%%%%%
%%%%%
\end{document}
