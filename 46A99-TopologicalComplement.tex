\documentclass[12pt]{article}
\usepackage{pmmeta}
\pmcanonicalname{TopologicalComplement}
\pmcreated{2013-03-22 17:32:31}
\pmmodified{2013-03-22 17:32:31}
\pmowner{asteroid}{17536}
\pmmodifier{asteroid}{17536}
\pmtitle{topological complement}
\pmrecord{5}{39941}
\pmprivacy{1}
\pmauthor{asteroid}{17536}
\pmtype{Definition}
\pmcomment{trigger rebuild}
\pmclassification{msc}{46A99}
\pmclassification{msc}{15A03}
\pmdefines{topologically complementary}
\pmdefines{topologically complemented}

% this is the default PlanetMath preamble.  as your knowledge
% of TeX increases, you will probably want to edit this, but
% it should be fine as is for beginners.

% almost certainly you want these
\usepackage{amssymb}
\usepackage{amsmath}
\usepackage{amsfonts}

% used for TeXing text within eps files
%\usepackage{psfrag}
% need this for including graphics (\includegraphics)
%\usepackage{graphicx}
% for neatly defining theorems and propositions
%\usepackage{amsthm}
% making logically defined graphics
%%%\usepackage{xypic}

% there are many more packages, add them here as you need them

% define commands here

\begin{document}
\subsubsection{Definition}
Let $X$ be a topological vector space and $M \subseteq X$ a \PMlinkname{closed}{ClosedSet} subspace.

If there exists a closed subspace $N \subseteq X$ such that
\begin{displaymath}
M \oplus N = X
\end{displaymath}
we say that $M$ is {\bf topologically complemented}.

In this case $N$ is said to be a {\bf topological complement} of $M$, and also $M$ and $N$ are said to be {\bf topologically complementary} subspaces.

\subsubsection{Remarks}
\begin{itemize}
\item It is known that every subspace $M \subseteq X$ has an algebraic complement, i.e. there exists a subspace $N \subseteq X$ such that $M \oplus N = X$. The existence of topological complements, however, is not always assured.
\item If $X$ is an Hilbert space, then each closed subspace $M \subseteq X$ is topologically complemented by its orthogonal complement $M^{\perp}$, i.e.
\begin{displaymath}
M \oplus M^{\perp} = X .
\end{displaymath} 
\item Moreover, for Banach spaces the converse of the last paragraph also holds, i.e. if each closed subspace is topologically complemented then $X$ is isomorphic a Hilbert space. This is the \PMlinkname{Lindenstrauss-Tzafriri theorem}{CharacterizationOfAHilbertSpace}.
\end{itemize}
%%%%%
%%%%%
\end{document}
