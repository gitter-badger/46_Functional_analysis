\documentclass[12pt]{article}
\usepackage{pmmeta}
\pmcanonicalname{DeltaDistribution}
\pmcreated{2013-03-22 13:45:52}
\pmmodified{2013-03-22 13:45:52}
\pmowner{matte}{1858}
\pmmodifier{matte}{1858}
\pmtitle{delta distribution}
\pmrecord{6}{34468}
\pmprivacy{1}
\pmauthor{matte}{1858}
\pmtype{Definition}
\pmcomment{trigger rebuild}
\pmclassification{msc}{46-00}
\pmclassification{msc}{46F05}
\pmrelated{ExampleOfDiracSequence}

\endmetadata

% this is the default PlanetMath preamble.  as your knowledge
% of TeX increases, you will probably want to edit this, but
% it should be fine as is for beginners.

% almost certainly you want these
\usepackage{amssymb}
\usepackage{amsmath}
\usepackage{amsfonts}

% used for TeXing text within eps files
%\usepackage{psfrag}
% need this for including graphics (\includegraphics)
%\usepackage{graphicx}
% for neatly defining theorems and propositions
%\usepackage{amsthm}
% making logically defined graphics
%%%\usepackage{xypic}

% there are many more packages, add them here as you need them

% define commands here

\newcommand{\sR}[0]{\mathbb{R}}
\newcommand{\sC}[0]{\mathbb{C}}
\newcommand{\sN}[0]{\mathbb{N}}
\newcommand{\sZ}[0]{\mathbb{Z}}
\begin{document}
\newcommand{\cD}[0]{\mathcal{D}}

Let $U$ be an open subset of $\sR^n$ such that $0\in U$.
Then the \emph{delta distribution} is the mapping 
\begin{eqnarray*}
\delta : \cD(U) &\to & \sC \\
         u   &\mapsto & u(0).
\end{eqnarray*}

{\bf Claim} The delta distribution is a distribution of zeroth order, i.e.,
$\delta\in \cD'^0(U)$.

\emph{Proof.} With obvious notation, we have
\begin{eqnarray*}
\delta(u+v)&=&(u+v)(0)=u(0)+v(0) = \delta(u) + \delta(v),\\
\delta(\alpha u) &=& (\alpha u)(0)=\alpha u(0)=\alpha \delta(u),
\end{eqnarray*}
so $\delta$ is linear. To see that $\delta$ is continuous, we
use condition (3) on this \PMlinkname{this page}{Distribution4}.
Indeed, if $K$ is a compact set in $U$, and $u\in \cD_K$, then
$$ |\delta(u)| = |u(0)| \le ||u||_\infty,$$
where $||\cdot||_\infty$ is the supremum norm. $\Box$
%%%%%
%%%%%
\end{document}
