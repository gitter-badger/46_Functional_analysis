\documentclass[12pt]{article}
\usepackage{pmmeta}
\pmcanonicalname{KreinMilmanTheorem}
\pmcreated{2013-03-22 14:24:58}
\pmmodified{2013-03-22 14:24:58}
\pmowner{jirka}{4157}
\pmmodifier{jirka}{4157}
\pmtitle{Krein-Milman theorem}
\pmrecord{6}{35921}
\pmprivacy{1}
\pmauthor{jirka}{4157}
\pmtype{Theorem}
\pmcomment{trigger rebuild}
\pmclassification{msc}{46A03}
\pmclassification{msc}{52A07}
\pmclassification{msc}{52A99}

\endmetadata

% this is the default PlanetMath preamble.  as your knowledge
% of TeX increases, you will probably want to edit this, but
% it should be fine as is for beginners.

% almost certainly you want these
\usepackage{amssymb}
\usepackage{amsmath}
\usepackage{amsfonts}

% used for TeXing text within eps files
%\usepackage{psfrag}
% need this for including graphics (\includegraphics)
%\usepackage{graphicx}
% for neatly defining theorems and propositions
\usepackage{amsthm}
% making logically defined graphics
%%%\usepackage{xypic}

% there are many more packages, add them here as you need them

% define commands here
\theoremstyle{theorem}
\newtheorem*{thm}{Theorem}
\newtheorem*{lemma}{Lemma}
\newtheorem*{conj}{Conjecture}
\newtheorem*{cor}{Corollary}
\newtheorem*{example}{Example}
\theoremstyle{definition}
\newtheorem*{defn}{Definition}
\begin{document}
\begin{thm}
Let $X$ be a locally convex topological vector space, and let $K \subset X$
be a compact \PMlinkname{convex subset}{ConvexSet}.  Then $K$ is the closed convex hull of its extreme points.
\end{thm}

% FIXME: this needs to go into the convex hull entry or some such
The closed convex hull above is defined as the intersection of all closed convex subsets of $X$ that contain $K$.  This turns out to be the same as the closure of the convex hull in a topological vector space.
                                                                                
\begin{thebibliography}{9}
\bibitem{royden}
H.\@ L.\@ Royden. \emph{\PMlinkescapetext{Real Analysis}}. Prentice-Hall, Englewood Cliffs, New Jersey, 1988
\end{thebibliography}
%%%%%
%%%%%
\end{document}
