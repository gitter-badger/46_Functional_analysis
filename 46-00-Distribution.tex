\documentclass[12pt]{article}
\usepackage{pmmeta}
\pmcanonicalname{Distribution}
\pmcreated{2013-03-22 13:44:08}
\pmmodified{2013-03-22 13:44:08}
\pmowner{matte}{1858}
\pmmodifier{matte}{1858}
\pmtitle{distribution}
\pmrecord{23}{34427}
\pmprivacy{1}
\pmauthor{matte}{1858}
\pmtype{Definition}
\pmcomment{trigger rebuild}
\pmclassification{msc}{46-00}
\pmclassification{msc}{46F05}
\pmsynonym{`generalized function'}{Distribution}
\pmrelated{ExampleOfDiracSequence}
\pmrelated{DiracDeltaFunction}
\pmrelated{DiscreteTimeFourierTransformInRelationWithItsContinousTimeFourierTransfrom}
\pmrelated{QuantumGroups}
\pmrelated{FourierStieltjesAlgebraOfAGroupoid}
\pmrelated{QuantumOperatorAlgebrasInQuantumFieldTheories}
\pmrelated{QFTOrQuantumFieldTheories}
\pmrelated{QuantumGroup}
\pmdefines{distribution of finite order}

% this is the default PlanetMath preamble.  as your knowledge
% of TeX increases, you will probably want to edit this, but
% it should be fine as is for beginners.

% almost certainly you want these
\usepackage{amssymb}
\usepackage{amsmath}
\usepackage{amsfonts}
\usepackage{amsthm}

\usepackage{mathrsfs}

% used for TeXing text within eps files
%\usepackage{psfrag}
% need this for including graphics (\includegraphics)
%\usepackage{graphicx}
% for neatly defining theorems and propositions
%
% making logically defined graphics
%%%\usepackage{xypic}

% there are many more packages, add them here as you need them

% define commands here

\newcommand{\sR}[0]{\mathbb{R}}
\newcommand{\sC}[0]{\mathbb{C}}
\newcommand{\sN}[0]{\mathbb{N}}
\newcommand{\sZ}[0]{\mathbb{Z}}

 \usepackage{bbm}
 \newcommand{\Z}{\mathbbmss{Z}}
 \newcommand{\C}{\mathbbmss{C}}
 \newcommand{\R}{\mathbbmss{R}}
 \newcommand{\Q}{\mathbbmss{Q}}



\newcommand*{\norm}[1]{\lVert #1 \rVert}
\newcommand*{\abs}[1]{| #1 |}



\newtheorem{thm}{Theorem}
\newtheorem{defn}{Definition}
\newtheorem{prop}{Proposition}
\newtheorem{lemma}{Lemma}
\newtheorem{cor}{Corollary}

\begin{document}
\newcommand{\cD}[0]{\mathcal{D}}
\newcommand{\scomp}[0]{C^\infty_0}
\PMlinkescapeword{standard}
\PMlinkescapeword{onto}
\PMlinkescapeword{order}

\subsection*{Motivation}

The main motivation behind distribution theory is to
extend the common linear operators on functions,
such as the derivative, convolution, and the Fourier transform,
so that they also apply to the singular, non-smooth, or non-integrable
functions that regularly appear in both theoretical and applied 
analysis.

Distribution theory also seeks to define suitable structures
on the spaces of functions involved
to ensure the convergence of suitable approximating functions,
and the continuity of certain operators.
For example, the limit of derivatives should be equal
to the derivative of the limit, with some definition of the limiting
operation.

When this program is carried out,
inevitably we find that we have to enlarge the space of objects that we
would consider as ``functions''.  For example, the derivative of a step
function is the Dirac delta function with a spike at the discontinuous step;
the Fourier transform of a constant function is also a Dirac delta
function, with the spike representing infinite spectral magnitude
at one single frequency.  (These facts, of course, had long been 
used in engineering mathematics.)

\textbf{Remark:}
Dirac's notion of delta distributions was introduced to facilitate computations in Quantum Mechanics,
however without having at the beginning a proper mathematical definition. In part as 
a (negative) reaction to such a state of affairs, von Neumann produced a mathematically
well-defined \PMlinkname{foundation of Quantum Mechanics}{QuantumGroupsAndVonNeumannAlgebras} based on actions of 
self-adjoint operators on Hilbert spaces which is still currently in use, with several significant
additions such as Frech\'et nuclear spaces and quantum groups. 

There are several theories of such \emph{`generalized functions'}.
In this entry, we describe Schwartz' theory of \emph{distributions}, 
which is probably the most widely used.

Essentially, a distribution on $\sR$ is a linear mapping that takes a 
smooth function (with compact support) on $\sR$ into a real number. 
For example, the delta distribution is the map,
$$
    f\mapsto f(0)
$$
while any smooth function $g$ on $\sR$ induces a distribution
$$
   f\mapsto \int_{\sR} fg.
$$

Distributions are also well behaved under coordinate changes, and 
can be defined onto a manifold. Differential forms with 
distribution valued coefficients are called currents. 
However, it is not possible to define a product of two 
distributions generalizing the product of usual functions. 


\subsection*{Formal definition}

\emph{A note on notation.} In distribution theory, the 
topological vector space of smooth functions with compact support on 
an open set $U\subseteq \R^n$
is traditionally denoted by $\cD(U)$. Let us also denote by 
$\cD_K(U)$ the subset of $\cD(U)$ of functions with support in a 
compact set $K\subset U$.

\begin{defn}[Distribution] 
A \emph{distribution} is a linear continuous functional on $\cD(U)$,
i.e., a linear continuous mapping $\cD(U)\to\sC$. 
The set of all distributions on $U$ is denoted by $\cD'(U)$. 
\end{defn}

Suppose $T$ is a linear functional on $\cD(U)$. 
Then  $T$ is continuous if and only if $T$ is continuous
in the origin (see \PMlinkname{this page}{ContinuousLinearMapping}).
This condition can be rewritten in various ways, and 
the below theorem gives two convenient conditions that can be used to prove
that a linear mapping is a distribution. 

\begin{thm}
Let $U$ be an open set in $\sR^n$, 
and let $T$ be a linear functional on $\cD(U)$. Then the
following are equivalent:
\begin{enumerate}
\item $T$ is a distribution.
\item If $K$ is a compact set in $U$, and  
$\{u_i\}_{i=1}^\infty$ be a sequence in $\cD_K(U)$, such that
for any multi-index $\alpha$, we have 
$$ 
   D^\alpha u_i \to 0
$$
in the supremum norm as $i\to \infty$, 
then $T(u_i) \to 0$ in $\sC$. 
\item For any compact set $K$ in $U$, there are constants $C>0$ and 
$k\in\{1,2,\ldots\}$ such that for all $u\in \cD_K(U)$, we have
\begin{eqnarray}
\label{ineq1}
|T(u)| &\le& C \sum_{|\alpha|\le k} ||D^\alpha u ||_\infty,
\end{eqnarray}
where $\alpha$ is a multi-index, and 
$||\cdot||_\infty$ is the supremum norm.
\end{enumerate}
\end{thm}

{\bf Proof} The equivalence of (2) and (3) can be 
found on \PMlinkname{this page}{EquivalenceOfConditions2And3},
and the equivalence of (1) and (3) is shown in 
\cite{rudin_fap}. 

\subsubsection*{Distributions of order $k$}
 
If $T$ is a distribution on an open set $U$, 
and the same $k$ can be used for any $K$ 
in the above inequality, then $T$ is a 
\emph{distribution of order $k$}. 
The set of all such distributions is denoted by $D'^k(U)$. 

Both usual functions and the delta distribution are of order $0$. 
One can also show that by differentiating a distribution its order increases
by at most one. Thus, in some sense, the order is a measure of how
''smooth'' a distribution is. 

\subsubsection*{Topology for $\cD'(U)$ }
The standard topology for $\cD'(U)$ is the weak$^\ast$ topology. 
In this topology, a sequence $\{T_i\}_{i=1}^\infty$ of distributions 
(in $\cD'(U)$) converges to a distribution $T\in \cD'(U)$ if and only if
$$ T_i( u) \to T(u)\,\,\,\mbox{(in $\sC$) as $i\to \infty$} $$
for every $u\in \cD(U)$.

\subsection*{Notes}
A common notation for the action of a distribution $T$ onto a test function $u\in \cD(U)$
(i.e., $T(u)$ with above notation)  is $\langle T,u\rangle$. 
The motivation for this
comes from \PMlinkname{this example}{EveryLocallyIntegrableFunctionIsADistribution}.

\begin{thebibliography}{9}
 \bibitem{rudin_fap}
 W. Rudin, {\em Functional Analysis},
 McGraw-Hill Book Company, 1973.
 \bibitem{hormander}
 L. H$\"o$rmander, {\em The Analysis of Linear Partial Differential Operators I,
 (Distribution theory and Fourier Analysis)}, 2nd ed, Springer-Verlag, 1990.
 \end{thebibliography}

%%%%%
%%%%%
\end{document}
