\documentclass[12pt]{article}
\usepackage{pmmeta}
\pmcanonicalname{QuotientsInCalgebras}
\pmcreated{2013-03-22 17:41:39}
\pmmodified{2013-03-22 17:41:39}
\pmowner{asteroid}{17536}
\pmmodifier{asteroid}{17536}
\pmtitle{quotients in $C^*$-algebras}
\pmrecord{5}{40135}
\pmprivacy{1}
\pmauthor{asteroid}{17536}
\pmtype{Theorem}
\pmcomment{trigger rebuild}
\pmclassification{msc}{46L05}
\pmrelated{NuclearCAlgebra}

% this is the default PlanetMath preamble.  as your knowledge
% of TeX increases, you will probably want to edit this, but
% it should be fine as is for beginners.

% almost certainly you want these
\usepackage{amssymb}
\usepackage{amsmath}
\usepackage{amsfonts}

% used for TeXing text within eps files
%\usepackage{psfrag}
% need this for including graphics (\includegraphics)
%\usepackage{graphicx}
% for neatly defining theorems and propositions
%\usepackage{amsthm}
% making logically defined graphics
%%%\usepackage{xypic}

% there are many more packages, add them here as you need them

% define commands here

\begin{document}
\PMlinkescapeword{involution}

{\bf Theorem -} Let $\mathcal{A}$ be a \PMlinkname{$C^*$-algebra}{CAlgebra} and $\mathcal{I}\subseteq \mathcal{A}$ a \PMlinkname{closed}{ClosedSet} ideal. Then the \PMlinkname{involution}{InvolutaryRing} in $\mathcal{A}$ induces a well-defined involution in $\mathcal{A}/\mathcal{I}$ and $\mathcal{A}/\mathcal{I}$ is a $C^*$-algebra with this involution and the quotient norm.
%%%%%
%%%%%
\end{document}
