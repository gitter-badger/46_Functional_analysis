\documentclass[12pt]{article}
\usepackage{pmmeta}
\pmcanonicalname{ExtensionAndRestrictionOfStates}
\pmcreated{2013-03-22 18:09:35}
\pmmodified{2013-03-22 18:09:35}
\pmowner{asteroid}{17536}
\pmmodifier{asteroid}{17536}
\pmtitle{extension and restriction of states}
\pmrecord{10}{40717}
\pmprivacy{1}
\pmauthor{asteroid}{17536}
\pmtype{Theorem}
\pmcomment{trigger rebuild}
\pmclassification{msc}{46L30}
\pmrelated{State}

% this is the default PlanetMath preamble.  as your knowledge
% of TeX increases, you will probably want to edit this, but
% it should be fine as is for beginners.

% almost certainly you want these
\usepackage{amssymb}
\usepackage{amsmath}
\usepackage{amsfonts}

% used for TeXing text within eps files
%\usepackage{psfrag}
% need this for including graphics (\includegraphics)
%\usepackage{graphicx}
% for neatly defining theorems and propositions
%\usepackage{amsthm}
% making logically defined graphics
%%%\usepackage{xypic}

% there are many more packages, add them here as you need them

% define commands here
\newcommand*{\Cset}{\mathbb{C}}

\begin{document}
\PMlinkescapephrase{restriction}

\subsection{Restriction of States}

Let $\mathcal{A}$ be a \PMlinkname{$C^*$-algebra}{CAlgebra} and $\mathcal{B} \subset \mathcal{A}$ a $C^*$-subalgebra, both having the same identity element.

$\,$

{\bf \PMlinkescapetext{Proposition} -} Given a state $\phi$ of $\mathcal{A}$, its \PMlinkname{restriction}{RestrictionOfAFunction} $\phi|_{\mathcal{B}}$ to $\mathcal{B}$ is also a state of $\mathcal{B}$.

$\,$

{\bf Remark -} Note that the requirement that the $C^*$-algebras $\mathcal{A}$ and $\mathcal{B}$ have a (common) identity element is necessary.

For example, let $X$ be a compact space and $C(X)$ the $C^*$-algebra of continuous functions $X \to \mathbb{C}$. Pick a point $x_0 \in X$ and consider the $C^*$-subalgebra of continuous functions $X \to \mathbb{C}$ which vanish at $x_0$. Notice that this subalgebra never has the same identity element of $C(X)$ (the constant function that equals $1$). In fact, this subalgebra may not have an identity at all.

Now the evaluation mapping at $x_0$, i.e. the function $\mathrm{ev}_{x_0}: C(X) \to \mathbb{C}$
\begin{align*}
\mathrm{ev}_{x_0} (f) := f(x_0)
\end{align*}
is a state of $C(X)$. Of course, its restriction to the subalgebra in question is the zero mapping, therefore not being a state.

\subsection{Extension of States}

Let $\mathcal{A}$ be a $C^*$-algebra and $\mathcal{B} \subset \mathcal{A}$ a $C^*$-subalgebra (not necessarily unital).

$\,$

{\bf Theorem 1 -} Every state $\phi$ of $\mathcal{B}$ admits an extension to a state $\widetilde{\phi}$ of $\mathcal{A}$. Moreover, every pure state $\phi$ of $\mathcal{B}$  admits an extension to a pure state $\widetilde{\phi}$ of $\mathcal{A}$.\\

{\bf Theorem 2 -} The set of extensions of a state $\phi$ of $\mathcal{B}$ is a compact and convex subset of $S_{\mathcal{A}}$, the \PMlinkescapetext{state space} of $\mathcal{A}$ endowed with the weak-* topology.

%%%%%
%%%%%
\end{document}
