\documentclass[12pt]{article}
\usepackage{pmmeta}
\pmcanonicalname{VectorNorm}
\pmcreated{2013-03-22 11:43:00}
\pmmodified{2013-03-22 11:43:00}
\pmowner{mike}{2826}
\pmmodifier{mike}{2826}
\pmtitle{vector norm}
\pmrecord{30}{30091}
\pmprivacy{1}
\pmauthor{mike}{2826}
\pmtype{Definition}
\pmcomment{trigger rebuild}
\pmclassification{msc}{46B20}
\pmclassification{msc}{18-01}
\pmclassification{msc}{20H15}
\pmclassification{msc}{20B30}
\pmrelated{Vector}
\pmrelated{Metric}
\pmrelated{Norm}
\pmrelated{VectorPnorm}
\pmrelated{NormedVectorSpace}
\pmrelated{MatrixNorm}
\pmrelated{MatrixPnorm}
\pmrelated{FrobeniusMatrixNorm}
\pmrelated{CauchySchwarzInequality}
\pmrelated{MetricSpace}
\pmrelated{VectorSpace}
\pmrelated{LpSpace}
\pmrelated{OperatorNorm}
\pmrelated{BoundedOperator}
\pmrelated{SemiNorm}
\pmrelated{BanachSpace}
\pmrelated{HilbertSpace}
\pmrelated{UnitVector}
\pmdefines{normed vector space}
\pmdefines{Euclidean norm}

\usepackage{amssymb}
\usepackage{amsmath}
\usepackage{amsfonts}
\begin{document}
A vector norm on the real vector space $V$ is a function $f : V \to \mathbb{R}$ that satisfies the following properties:

\begin{eqnarray*}
    f(x) = 0 \iff x = 0 && \\
    f(x) \ge 0          && x \in V \\
    f(x+y) \leq f(x)+f(y) && x,y \in V \\
    f(\alpha x) = |\alpha|f(x) && \alpha \in \mathbb{R},x\in V
\end{eqnarray*}

Such a function is denoted as $||\,x\,||$.  Particular norms are distinguished by subscripts, such
as $||\,x\,||_V$, when referring to a norm in the space $V$.  A \emph{unit vector} with respect to the norm $||\,\cdot\,||$ is a vector $x$ satisfying
$||\,x\,|| = 1$.\\

A vector norm on a complex vector space is defined similarly.

A common (and useful) example of a real norm is the Euclidean norm given by $||x||=(x_1^2 + x_2^2 +  \cdots + x_n^2)^{1/2}$ defined on $V=\mathbb{R}^n$.
Note, however, that there exists vector spaces which are metric, but upon which it is not possible to define a norm. If it possible, the space is called a {\em normed vector space}. Given a metric on the vector space, a necessary and sufficient condition for this space to be a normed space, is
\begin{eqnarray*}
     d(x+a,y+a)=&d(x,y) & \forall x,y,a \in V\\
     d(\alpha x,\alpha y)=&|\alpha|d(x,y) &\forall x,y \in V, \alpha \in \mathbb{R}\\
\end{eqnarray*}
But given a norm, a metric can always be defined by the equation $ d(x,y)=||x-y||$. Hence every normed space is a metric space.
%%%%%
%%%%%
%%%%%
%%%%%
%%%%%
%%%%%
%%%%%
%%%%%
%%%%%
%%%%%
%%%%%
%%%%%
\end{document}
