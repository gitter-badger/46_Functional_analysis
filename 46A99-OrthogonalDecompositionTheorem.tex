\documentclass[12pt]{article}
\usepackage{pmmeta}
\pmcanonicalname{OrthogonalDecompositionTheorem}
\pmcreated{2013-03-22 17:32:34}
\pmmodified{2013-03-22 17:32:34}
\pmowner{asteroid}{17536}
\pmmodifier{asteroid}{17536}
\pmtitle{orthogonal decomposition theorem}
\pmrecord{4}{39942}
\pmprivacy{1}
\pmauthor{asteroid}{17536}
\pmtype{Theorem}
\pmcomment{trigger rebuild}
\pmclassification{msc}{46A99}
\pmsynonym{closed subspaces of Hilbert spaces are complemented}{OrthogonalDecompositionTheorem}

% this is the default PlanetMath preamble.  as your knowledge
% of TeX increases, you will probably want to edit this, but
% it should be fine as is for beginners.

% almost certainly you want these
\usepackage{amssymb}
\usepackage{amsmath}
\usepackage{amsfonts}

% used for TeXing text within eps files
%\usepackage{psfrag}
% need this for including graphics (\includegraphics)
%\usepackage{graphicx}
% for neatly defining theorems and propositions
%\usepackage{amsthm}
% making logically defined graphics
%%%\usepackage{xypic}

% there are many more packages, add them here as you need them

% define commands here

\begin{document}
{\bf Theorem -} Let $X$ be an Hilbert space and $A \subseteq X$ a closed subspace. Then the \PMlinkname{orthogonal complement}{Complimentary} of $A$, denoted $A^{\perp}$, is a topological complement of $A$. That means $A^{\perp}$ is closed and
\begin{displaymath}
X =A \oplus A^{\perp} \;.
\end{displaymath}

{\bf Proof :}
\begin{itemize}
\item $A^{\perp}$ is closed : 

This follows easily from the continuity of the inner product. If a sequence $(x_n)$ of elements in $A^{\perp}$ converges to an element $x_0 \in X$, then
\begin{displaymath}
\langle x_0, a \rangle = \langle \lim_{n \rightarrow \infty} x_n, a \rangle = \lim_{n \rightarrow \infty}\langle x_n, a \rangle = 0\;\;\; \text{for every} a \in A
\end{displaymath}
which implies that $x_0 \in A^{\perp}$.

\item $X=A \oplus A^{\perp}$ : 

Since $X$ is \PMlinkname{complete}{Complete} and $A$ is closed, $A$ is a \PMlinkescapetext{complete} subspace of $X$. Therefore, for every $x \in X$, there exists a best approximation of $x$ in $A$, which we denote by $a_0 \in A$, that satisfies $x-a_0 \in A^{\perp}$ (see this \PMlinkname{entry}{BestApproximationInInnerProductSpaces}).

This allows one to write $x$ as a sum of elements in $A$ and $A^{\perp}$
\begin{displaymath}
x= a_0 + (x-a_0)
\end{displaymath}
which proves that
\begin{displaymath}
X= A + A^{\perp} \; .
\end{displaymath}

Moreover, it is easy to see that
\begin{displaymath}
A \cap A^{\perp} = \{0\}
\end{displaymath}
since if $y \in A \cap A^{\perp}$ then $\langle y, y \rangle = 0$, which means $y=0$.

We conclude that $X=A \oplus A^{\perp}$. $\square$
\end{itemize}


%%%%%
%%%%%
\end{document}
