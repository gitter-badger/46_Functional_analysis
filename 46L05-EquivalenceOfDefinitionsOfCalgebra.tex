\documentclass[12pt]{article}
\usepackage{pmmeta}
\pmcanonicalname{EquivalenceOfDefinitionsOfCalgebra}
\pmcreated{2013-03-22 17:42:27}
\pmmodified{2013-03-22 17:42:27}
\pmowner{rspuzio}{6075}
\pmmodifier{rspuzio}{6075}
\pmtitle{equivalence of definitions of $C^*$-algebra}
\pmrecord{4}{40151}
\pmprivacy{1}
\pmauthor{rspuzio}{6075}
\pmtype{Theorem}
\pmcomment{trigger rebuild}
\pmclassification{msc}{46L05}
\pmrelated{HomomorphismsOfCAlgebrasAreContinuous}
\pmrelated{CAlgebra}

\endmetadata

% this is the default PlanetMath preamble.  as your knowledge
% of TeX increases, you will probably want to edit this, but
% it should be fine as is for beginners.

% almost certainly you want these
\usepackage{amssymb}
\usepackage{amsmath}
\usepackage{amsfonts}

% used for TeXing text within eps files
%\usepackage{psfrag}
% need this for including graphics (\includegraphics)
%\usepackage{graphicx}
% for neatly defining theorems and propositions
\usepackage{amsthm}
% making logically defined graphics
%%%\usepackage{xypic}

% there are many more packages, add them here as you need them

% define commands here
\newcommand*{\norm}[1]{\Vert #1\Vert}
\newtheorem{thm}{Theorem}
\begin{document}
In this entry, we will prove that the definitions of $C^*$ algebra given in the 
main entry are equivalent.

\begin{thm}
A Banach algebra $A$ with an antilinear involution $*$ such that
$\norm{a}^2 \leq \norm{a^* a}$ for all $a \in A$ is a $C^*$-algebra.
\end{thm}

\begin{proof}
It follows from the product inequality $\norm{ab} \leq \norm{a}\norm{b}$ that
\[ \norm{a}^2 \leq \norm{a^* a} \leq \norm{a^*}\norm{a}. \]
Therefore, $\norm{a} \leq \norm{a^*}$. Putting $a^*$ for $a$, we also have $\norm{a^*} \leq \norm{a^{**}} = \norm{a}$. Thus, the involution is an isometry: $\norm{a} = \norm{a^*}$.
So now,
\[ \norm{a}^2 \leq \norm{a^* a} \leq \norm{a}^2. \]
Hence, $\norm{a^* a} = \norm{a}^2$.
\end{proof}

\begin{thm}
A Banach algebra $A$ with an antilinear involution $*$ such that
$\norm{a^* a} = \norm{a^*}\norm{a}$ is a $C^*$-algebra.
\end{thm}
%%%%%
%%%%%
\end{document}
