\documentclass[12pt]{article}
\usepackage{pmmeta}
\pmcanonicalname{CriteriaForExistenceOfAntidervatives}
\pmcreated{2013-03-22 19:14:00}
\pmmodified{2013-03-22 19:14:00}
\pmowner{scineram}{4030}
\pmmodifier{scineram}{4030}
\pmtitle{criteria for existence of antidervatives}
\pmrecord{9}{42157}
\pmprivacy{1}
\pmauthor{scineram}{4030}
\pmtype{Theorem}
\pmcomment{trigger rebuild}
\pmclassification{msc}{46G05}
%\pmkeywords{derivative antiderivative primitive function differential calculus}

\endmetadata

% this is the default PlanetMath preamble.  as your knowledge
% of TeX increases, you will probably want to edit this, but
% it should be fine as is for beginners.

% almost certainly you want these
\usepackage{amssymb}
\usepackage{amsmath}
\usepackage{amsfonts}
% used for TeXing text within eps files
%\usepackage{psfrag}
% need this for including graphics (\includegraphics)
\usepackage{graphicx}
% for neatly defining theorems and propositions
\usepackage{amsthm}
% making logically defined graphics
%%%\usepackage{xypic}

% there are many more packages, add them here as you need them

% define commands here
\newtheorem{thm}{Theorem}
\newtheorem{cor}{Corollary}

\begin{document}
Let $X$ be a normed space, $Y$ a Banach space, $U\subset X$ a connected open set, $f\colon U\to L(X;Y)$ a continuous function, where $L(X;Y)$ is the space of continuous linear operators. In this article a path is a curve that has bounded variation. The following theorems give necessary and sufficient conditions for $f$ to have an antiderivatives.

\begin{thm}
The following conditions are equivalent:
\begin{enumerate}
\item $f$ has an antiderivative on $U$,
\item for any $\gamma$ closed path in $U$ $\int_\gamma f=0$,
\item for any $\gamma$, $\delta$ paths in $U$ that have the same starting and endpoints $\int_\gamma f=\int_\delta f$.
\end{enumerate}
\end{thm}

The next theorem states criteria for the existence of local antiderivatives.

\begin{thm}
The following conditions are equivalent:
\begin{enumerate}
\item $f$ has an antiderivative locally,
\item for $\gamma$, $\delta$ homotopic closed paths in $U$ $\int_\gamma f=\int_\delta f$,
\item if $\gamma$ is a triangular path such that its convex hull is in $U$, then $\int_\gamma f=0$.
\end{enumerate}
\end{thm}
With the stronger assumption that $f$ is differenciable we can obtain a more easily applicable condition. We introduce the  canonical isometric isomorphism
\[\pi_{1,1}\colon L(X;L(X;Y))\to L_2(X;Y),\quad u\mapsto((x_1,x_2)\mapsto u(x_1)(x_2))\]
where $L_2(X;Y)$ is the space of bilinear operators from $X$ to $Y$. If $F$ is an antiderivative of $f$, then $\pi_{1,1}(Df(x))=D^2F(x)$ and by Clairaut's theorem the second derivative is symmetric. The following theorems assert that the reverse is also true.

\begin{thm}
If $f$ is differentiable, then it has an antiderivative locally if and only if $\pi_{1,1}(Df(x))$ is symmetric for all $x\in U$.
\end{thm}

Combining these three theorems immediately gives the following.

\begin{cor}
If $U$ is simply connected and $f$ is differentiable, then it has an antiderivative on $U$ if and only if $\pi_{1,1}(Df(x))$ is symmetric for all $x\in U$.
\end{cor}

%%%%%
%%%%%
\end{document}
