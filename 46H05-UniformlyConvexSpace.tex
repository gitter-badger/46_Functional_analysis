\documentclass[12pt]{article}
\usepackage{pmmeta}
\pmcanonicalname{UniformlyConvexSpace}
\pmcreated{2013-03-22 15:13:11}
\pmmodified{2013-03-22 15:13:11}
\pmowner{georgiosl}{7242}
\pmmodifier{georgiosl}{7242}
\pmtitle{uniformly convex space}
\pmrecord{32}{36983}
\pmprivacy{1}
\pmauthor{georgiosl}{7242}
\pmtype{Definition}
\pmcomment{trigger rebuild}
\pmclassification{msc}{46H05}
\pmsynonym{uniformly convex}{UniformlyConvexSpace}

% this is the default PlanetMath preamble.  as your knowledge
% of TeX increases, you will probably want to edit this, but
% it should be fine as is for beginners.

% almost certainly you want these
\usepackage{amssymb}
\usepackage{amsmath}
\usepackage{amsfonts}

% used for TeXing text within eps files
%\usepackage{psfrag}
% need this for including graphics (\includegraphics)
%\usepackage{graphicx}
% for neatly defining theorems and propositions
%\usepackage{amsthm}
% making logically defined graphics
%%%\usepackage{xypic}

% there are many more packages, add them here as you need them

% define commands here
\begin{document}
A normed space is \emph{uniformly convex} iff  $\forall \epsilon>0$ there exists $\delta>0$ that satisfies
for $\|x\|\leq1$  $\|y\|\leq1$ and $\|x-y\|> \epsilon$ $\Rightarrow$ $\|\frac{x+y}{2}\|\leq1$-$\delta$.
\\\\For example it is easily seen that the normed space $(\mathbb{R}^2,\|.\|_{2})$ is uniformly convex space.
Also $L^p$ and $l^p$ spaces for $1<p<\infty$ are uniformly convex, see \it{ J.A. Clarkson, "Uniformly convex spaces", Trans. Amer. Math. Society, 40 (1936), 396-414.}

%%%%%
%%%%%
\end{document}
