\documentclass[12pt]{article}
\usepackage{pmmeta}
\pmcanonicalname{CommutantOfBHIsmathbbCI}
\pmcreated{2013-03-22 18:39:35}
\pmmodified{2013-03-22 18:39:35}
\pmowner{asteroid}{17536}
\pmmodifier{asteroid}{17536}
\pmtitle{commutant of $B(H)$ is $\mathbb{C}I$}
\pmrecord{7}{41403}
\pmprivacy{1}
\pmauthor{asteroid}{17536}
\pmtype{Theorem}
\pmcomment{trigger rebuild}
\pmclassification{msc}{46L10}
\pmsynonym{center of $B(H)$}{CommutantOfBHIsmathbbCI}

% this is the default PlanetMath preamble.  as your knowledge
% of TeX increases, you will probably want to edit this, but
% it should be fine as is for beginners.

% almost certainly you want these
\usepackage{amssymb}
\usepackage{amsmath}
\usepackage{amsfonts}

% used for TeXing text within eps files
%\usepackage{psfrag}
% need this for including graphics (\includegraphics)
%\usepackage{graphicx}
% for neatly defining theorems and propositions
%\usepackage{amsthm}
% making logically defined graphics
%%%\usepackage{xypic}

% there are many more packages, add them here as you need them

% define commands here

\begin{document}
Let $H$ be a Hilbert space and $B(H)$ its algebra of bounded operators. We denote by $I$ the identity operator of $B(H)$ and by $\mathbb{C}I$ the set of all multiples of $I$, that is $\mathbb{C}I := \{ \lambda I : \lambda \in \mathbb{C}\}$. Let $B(H)'$ denote the commutant of $B(H)$, which is precisely the center of $B(H)$.

$\,$

{\bf Theorem -} We have that $B(H)' = \mathbb{C}I$.

As a particular case, we see that the center of the matrix algebra $Mat_{n \times n} (\mathbb{C})$ consists solely of the multiples of the identity matrix, i.e. a matrix in $Mat_{n \times n} (\mathbb{C})$ that commutes with all other matrices is necessarily a multiple of the identity matrix.

$\,$

{\bf \emph{\PMlinkescapetext{Proof}:}} For each $x, y \in H$ we denote by $T_{x,y}$ the operator given by
\begin{align*}
T_{x,y} z:= \langle z, x \rangle y\,, \qquad\qquad z \in H
\end{align*}

Let $S \in B(H)'$. We must have $ST_{x,y} = T_{x,y}S$ for all $x,y \in H$, hence
\begin{align}
\langle z, x \rangle Sy = \langle Sz,x \rangle y\,, \qquad\qquad \forall x,y,z \in H
\end{align}

Choosing a non-zero $x$ and taking $z = x$, we see that

\begin{align*}
Sy = \frac{\langle Sx, x \rangle}{\langle x, x \rangle} y\,, \qquad\qquad y \in H, x \in H \setminus \{0\}
\end{align*}

Hence, $\displaystyle \frac{\langle Sx, x \rangle}{\langle x, x \rangle}$ must be constant for all $x \in H \setminus \{0\}$. Denote by $\lambda \in \mathbb{C}$ this constant.

We have that $Sy = \lambda y$ for all $y \in H$, which simply means that $S = \lambda I$. Thus, $B(H)' \subseteq \mathbb{C} I$.

It is clear that the multiples of the identity operator commute with all operators, hence we also have $\mathbb{C} I \subseteq B(H)'$.

We conclude that $B(H)' = \mathbb{C} I$. $\square$
%%%%%
%%%%%
\end{document}
