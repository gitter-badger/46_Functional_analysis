\documentclass[12pt]{article}
\usepackage{pmmeta}
\pmcanonicalname{CalgebraHomomorphismsHaveClosedImages}
\pmcreated{2013-03-22 17:44:37}
\pmmodified{2013-03-22 17:44:37}
\pmowner{asteroid}{17536}
\pmmodifier{asteroid}{17536}
\pmtitle{$C^*$-algebra homomorphisms have closed images}
\pmrecord{9}{40193}
\pmprivacy{1}
\pmauthor{asteroid}{17536}
\pmtype{Theorem}
\pmcomment{trigger rebuild}
\pmclassification{msc}{46L05}
\pmsynonym{image of $C^*$-homomorphism is a $C^*$-algebra}{CalgebraHomomorphismsHaveClosedImages}

% this is the default PlanetMath preamble.  as your knowledge
% of TeX increases, you will probably want to edit this, but
% it should be fine as is for beginners.

% almost certainly you want these
\usepackage{amssymb}
\usepackage{amsmath}
\usepackage{amsfonts}

% used for TeXing text within eps files
%\usepackage{psfrag}
% need this for including graphics (\includegraphics)
%\usepackage{graphicx}
% for neatly defining theorems and propositions
%\usepackage{amsthm}
% making logically defined graphics
%%%\usepackage{xypic}

% there are many more packages, add them here as you need them

% define commands here

\begin{document}
\PMlinkescapeword{image}
\PMlinkescapeword{closed}

{\bf Theorem -} Let $f: \mathcal{A} \longrightarrow \mathcal{B}$ be a *-homomorphism between the \PMlinkname{$C^*$-algebras}{CAlgebra} $\mathcal{A}$ and $\mathcal{B}$. Then $f$ has \PMlinkname{closed}{ClosedSet} \PMlinkname{image}{Function}, i.e. $f(\mathcal{A})$ is closed in $\mathcal{B}$.

Thus, the image $f(\mathcal{A})$ is a $C^*$-subalgebra of $\mathcal{B}$.

$\,$

{\bf \emph{Proof:}} The kernel of $f$, $\mathrm{Ker} f$, is a closed two-sided ideal of $\mathcal{A}$, since $f$ is continuous (see \PMlinkname{this entry}{HomomorphismsOfCAlgebrasAreContinuous}). Factoring threw the quotient $C^*$-algebra $\mathcal{A}/\mathrm{Ker} f$ we obtain an injective *-homomorphism $\widetilde{f}:\mathcal{A}/\mathrm{Ker} f \longrightarrow \mathcal{B}$.

Injective *-homomorphisms between $C^*$-algebras are known to be isometric (see \PMlinkname{this entry}{InjectiveCAlgebraHomomorphismIsIsometric}), hence the image $\widetilde{f}(\mathcal{A}/\mathrm{Ker} f)$ is closed in $\mathcal{B}$.

Since the images $\widetilde{f}(\mathcal{A}/\mathrm{Ker} f)$ and $f(\mathcal{A})$ coincide we conclude that $f(\mathcal{A})$ is closed in $\mathcal{B}$. $\square$
%%%%%
%%%%%
\end{document}
