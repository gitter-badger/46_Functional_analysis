\documentclass[12pt]{article}
\usepackage{pmmeta}
\pmcanonicalname{ProofOfVonNeumannDoubleCommutantTheorem}
\pmcreated{2013-03-22 18:40:29}
\pmmodified{2013-03-22 18:40:29}
\pmowner{asteroid}{17536}
\pmmodifier{asteroid}{17536}
\pmtitle{proof of von Neumann double commutant theorem}
\pmrecord{7}{41422}
\pmprivacy{1}
\pmauthor{asteroid}{17536}
\pmtype{Proof}
\pmcomment{trigger rebuild}
\pmclassification{msc}{46L10}
\pmclassification{msc}{46H35}
\pmclassification{msc}{46K05}

% this is the default PlanetMath preamble.  as your knowledge
% of TeX increases, you will probably want to edit this, but
% it should be fine as is for beginners.

% almost certainly you want these
\usepackage{amssymb}
\usepackage{amsmath}
\usepackage{amsfonts}

% used for TeXing text within eps files
%\usepackage{psfrag}
% need this for including graphics (\includegraphics)
%\usepackage{graphicx}
% for neatly defining theorems and propositions
%\usepackage{amsthm}
% making logically defined graphics
%%%\usepackage{xypic}

% there are many more packages, add them here as you need them

% define commands here

\begin{document}
\PMlinkescapeword{Lemma}
\PMlinkescapeword{proposition}

{\bf Lemma -} Let $H$ be a Hilbert space and $B(H)$ its algebra of bounded operators. Let $\mathcal{N}$ be a *-subalgebra of $B(H)$ that contains the identity operator and is closed in the strong operator topology. If $T \in \mathcal{N}''$, the double commutant of $\mathcal{N}$, then for each $x \in H$ there is an operator $A \in \mathcal{N}$ such that $\|(A-T)x\| < 1$.

{\bf \emph{\PMlinkescapetext{Proof}:}} Let $\overline{\mathcal{N}x} \subseteq H$ be the closure of the subspace $\mathcal{N}x := \{Sx: S \in \mathcal{N}\}$. It is clear that $\mathcal{N}x$ is an invariant subspace for $\mathcal{N}$, hence so is its closure $\overline{\mathcal{N}x}$ (see \PMlinkname{this entry}{InvariantSubspacesForSelfAdjointAlgebrasOfOperators}, Proposition 5).

Let $P$ be the orthogonal projection onto $\overline{\mathcal{N}x}$. Since $\overline{\mathcal{N}x}$ is invariant for $\mathcal{N}$, we have that $P \in \mathcal{N}'$ (see \PMlinkname{this entry}{InvariantSubspacesForSelfAdjointAlgebrasOfOperators}, last theorem). Since $\mathcal{N}$ contains the identity operator, we know that $x$ belongs to $\overline{\mathcal{N}x}$. Hence,

\begin{align*}
Tx = TPx = PTx
\end{align*}
where the last equality comes from the fact that $T \in \mathcal{N}''$ and $P \in \mathcal{N}'$. Thus, we see that $Tx \in \overline{\mathcal{N}x}$, which implies that there exists an $A \in \mathcal{N}$ such that $\|Tx - Ax\| < 1$. $\square$

$\,$

{\bf \PMlinkescapetext{\emph{Proof of the von Neumann double commutant theorem}:}} $(1) \Longrightarrow (2)$ Since $\mathcal{M}''$ is the commutant of some set, namely it is the commutant of $\mathcal{M}'$, it follows that $\mathcal{M}''$ is closed in the weak operator topology (see \PMlinkname{this entry}{CommutantIsAWeakOperatorClosedSubalgebra}). But we are assuming that $\mathcal{M} = \mathcal{M}''$, hence $\mathcal{M}$ is closed in the weak operator topology.

$(2) \Longrightarrow (3)$ This part is obvious since the weak operator topology is weaker than the strong operator topology.

$(3) \Longrightarrow (1)$ Suppose $\mathcal{M}$ is closed in the strong operator topology.

A subset of $B(H)$ is always contained in its double commutant, thus $\mathcal{M} \subseteq \mathcal{M}''$. So it remains to prove the opposite inclusion.

Let $T \in \mathcal{M}''$. We are going to prove that $T$ belongs to the strong operator closure of $\mathcal{M}$, and since $\mathcal{M}$ is closed under this topology, it will follow that $T \in \mathcal{M}$.

Recall that the strong operator topology is the topology in $B(H)$ generated by the family of seminorms $\| \cdot \|_x, x \in H$ defined by $\|S\|_x:=\|Sx\|$. A local base around $T$, in this topology, consists of sets of the form

\begin{align*}
V(x_1, \dots, x_n;\, \epsilon):= \{S \in B(H) : \|(S - T)x_i\| \leq \epsilon, i=1, \dots, n\}\,, \qquad\qquad x_1, \dots, x_n \in H, \epsilon >0
\end{align*}

We can however consider $\epsilon$ to be $1$, since $V(x_1, \dots, x_n;\, \epsilon)= V(\epsilon^{-1}x_1, \dots, \epsilon^{-1} x_n; 1)$.

For every $x_1, \dots, x_n \in H$ we want to find $A \in \mathcal{M}$ such that $A \in V(x_1, \dots, x_n;\, 1)$, i.e. such that $\|(A - T)x_i\| < 1$, for each $i$.

Let $\widetilde{H}$ be the direct sum of Hilbert spaces $\widetilde{H}:= \oplus_{i = 1}^n H$. For every $A \in B(H)$ let $\widetilde{A} \in B(\widetilde{H})$ be the \PMlinkname{direct sum of bounded operators}{DirectSumOfBoundedOperatorsOnHilbertSpaces} $\widetilde{A}:= \oplus_{i=1}^n A$, i.e.

\begin{align*}
\widetilde{A} (y_1, \dots, y_n) = (Ay_1, \dots, Ay_n)\,, \qquad\qquad y_1, \dots, y_n \in H
\end{align*}

We have that $\mathcal{N}:= \{\widetilde{A}: A \in \mathcal{M}\}$ is a *-algebra of bounded operators in $\widetilde{H}$.

{\bf Claim 1 -} $\widetilde{T} \in \mathcal{N}''$.

The algebra $B(\widetilde{H})$ can be canonically identified with the algebra of $n \times n$ matrices with entries in $B(H)$, and $\mathcal{N}$ corresponds to the diagonal matrices with an element $A \in \mathcal{M}$ in the diagonal. Thus, it is easy to check that $\mathcal{N}'$ is precisely the set of matrices whose entries belong to $\mathcal{M}'$. 

Since the \PMlinkname{unit matrices}{UnitMatrix} belong to $\mathcal{N}'$, it follows that $\mathcal{N}''$ consists solely of diagonal matrices with one element on the diagonal (see \PMlinkname{this entry}{CentralizerOfMatrixUnits}). It is easy to check that $\mathcal{N}''$ is precisely the set of diagonal matrices with one element of $\mathcal{M}''$ in the diagonal. Hence, we conclude that $\widetilde{T} \in \mathcal{N}''$, and Claim 1 is proved.

Now, we observe that $\mathcal{N}$ is a *-subalgebra of $B(\widetilde{H})$ that contains the identity operator. Since $\mathcal{M}$ is closed in the strong operator topology, it follows easily that $\mathcal{N}$ is also closed in the strong operator topology. Since $\widetilde{T} \in \mathcal{N}''$, Lemma 1 \PMlinkescapetext{states} that for each $\widetilde{x} := (x_1, \dots, x_n) \in \widetilde{H}$ there exists an operator $\widetilde{A} \in \mathcal{N}$ such that $\|(\widetilde{A}- \widetilde{T})\widetilde{x}\| < 1$. But this is implies that $\|(A-T)x_i\| < 1$ for each $1 \leq i \leq n$.

Thus, $T \in V(x_1, \dots, x_n;\, 1)$. Hence we conclude that $T$ belongs to the \PMlinkescapetext{strong} operator closure of $\mathcal{M}$, but since $\mathcal{M}$ is closed under this topology, $T \in \mathcal{M}$.

We conclude that $\mathcal{M}'' = \mathcal{M}$. $\square$
%%%%%
%%%%%
\end{document}
