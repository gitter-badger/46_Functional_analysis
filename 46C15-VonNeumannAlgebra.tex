\documentclass[12pt]{article}
\usepackage{pmmeta}
\pmcanonicalname{VonNeumannAlgebra}
\pmcreated{2013-03-22 17:21:44}
\pmmodified{2013-03-22 17:21:44}
\pmowner{asteroid}{17536}
\pmmodifier{asteroid}{17536}
\pmtitle{von Neumann algebra}
\pmrecord{29}{39722}
\pmprivacy{1}
\pmauthor{asteroid}{17536}
\pmtype{Definition}
\pmcomment{trigger rebuild}
\pmclassification{msc}{46C15}
\pmclassification{msc}{46H35}
\pmclassification{msc}{46L10}
\pmsynonym{$W^*$-algebra}{VonNeumannAlgebra}
%\pmkeywords{operator algebras}
%\pmkeywords{C*-algebras}
%\pmkeywords{groupoid C*-dynamical systems}
%\pmkeywords{Hilbert spaces}
\pmrelated{CAlgebra}
\pmrelated{TopologicalAlgebra}
\pmrelated{Commutant}
\pmrelated{GroupoidCDynamicalSystem}
\pmrelated{Algebras2}
\pmrelated{CAlgebra3}
\pmrelated{WeakHopfCAlgebra2}
\pmrelated{HAlgebra}
\pmrelated{LocallyCompactQuantumGroup}
\pmrelated{QuantumGroupsAndVonNeumannAlgebras}

\endmetadata

% this is the default PlanetMath preamble.  as your knowledge
% of TeX increases, you will probably want to edit this, but
% it should be fine as is for beginners.

% almost certainly you want these
\usepackage{amssymb}
\usepackage{amsmath}
\usepackage{amsfonts}

% used for TeXing text within eps files
%\usepackage{psfrag}
% need this for including graphics (\includegraphics)
%\usepackage{graphicx}
% for neatly defining theorems and propositions
%\usepackage{amsthm}
% making logically defined graphics
%%%\usepackage{xypic}

% there are many more packages, add them here as you need them

% define commands here

\begin{document}
\PMlinkescapeword{theory}
\PMlinkescapeword{theories}

\section*{Definition}

Let $H$ be an Hilbert space, and let $B(H)$ be the *-algebra of bounded operators in $H$.

A {\bf von Neumann algebra} (or $W^*$-algebra) $\mathcal M$ is a *-subalgebra of $B(H)$ 
that contains the identity operator and satisfies one of the following equivalent conditions:
\begin{enumerate}
\item $\mathcal M$ is closed in the weak operator topology.
\item $\mathcal M$ is closed in the strong operator topology.
\item $\mathcal M = \mathcal M''$, i.e. $\mathcal M$ equals its double commutant.
\end{enumerate}

The equivalence between the above conditions is given by the von Neumann double commutant theorem.

Since the weak and strong operator topology are weaker than the norm topology, it follows that every von Neumann algebra is a norm closed *-subalgebra of $B(H)$. Thus, von Neumann algebras are a particular class of \PMlinkname{$C^*$-algebras}{CAlgebra} and the results and tools from the $C^*$ theory are also applicable in the setting of von Neumann algebras. Nevertheless, the philosophy behind von Neumann algebras is quite different from that of $C^*$-algebras and the tools and techniques for each theory turn out to be different as well.

\section*{Examples:}
\begin{enumerate}
\item $B(H)$ is itself a von Neumann algebra.
\item \PMlinkname{$L^{\infty}(\mathbb{R})$}{LinftyXDmu} as subalgebra of $B(L^2(\mathbb{R}))$ is a von Neumann algebra.
\end{enumerate}


%%%%%
%%%%%
\end{document}
