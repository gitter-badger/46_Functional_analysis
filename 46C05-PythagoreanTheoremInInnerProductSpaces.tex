\documentclass[12pt]{article}
\usepackage{pmmeta}
\pmcanonicalname{PythagoreanTheoremInInnerProductSpaces}
\pmcreated{2013-03-22 17:32:13}
\pmmodified{2013-03-22 17:32:13}
\pmowner{asteroid}{17536}
\pmmodifier{asteroid}{17536}
\pmtitle{Pythagorean theorem in inner product spaces}
\pmrecord{5}{39934}
\pmprivacy{1}
\pmauthor{asteroid}{17536}
\pmtype{Theorem}
\pmcomment{trigger rebuild}
\pmclassification{msc}{46C05}
\pmsynonym{Pythagoras theorem in inner product spaces}{PythagoreanTheoremInInnerProductSpaces}
\pmrelated{PythagorasTheorem}

% this is the default PlanetMath preamble.  as your knowledge
% of TeX increases, you will probably want to edit this, but
% it should be fine as is for beginners.

% almost certainly you want these
\usepackage{amssymb}
\usepackage{amsmath}
\usepackage{amsfonts}

% used for TeXing text within eps files
%\usepackage{psfrag}
% need this for including graphics (\includegraphics)
%\usepackage{graphicx}
% for neatly defining theorems and propositions
%\usepackage{amsthm}
% making logically defined graphics
%%%\usepackage{xypic}

% there are many more packages, add them here as you need them

% define commands here

\begin{document}
{\bf \PMlinkescapetext{Pythagorean theorem} -} Let $X$ be an inner product space (over $\mathbb{R}$ or $\mathbb{C}$) and $x, y \in X$ two orthogonal vectors. Then
\begin{displaymath}
\|x+y\|^2 = \|x\|^2 + \|y\|^2.
\end{displaymath}

{\bf Proof :} As $x \perp y$ one has $\langle x, y \rangle = 0$. Then
\begin{eqnarray*}
\|x+y\|^2 & = & \langle x+y, x+y \rangle \\
& = & \langle x, x \rangle + \langle x, y \rangle + \langle y, x \rangle + \langle y, y \rangle \\
& = & \|x\|^2 + \langle x, y \rangle + \overline{\langle x, y \rangle} +\|y\|^2 \\
& = & \|x\|^2 + \|y\|^2 \qquad\qquad\qquad\quad \square
\end{eqnarray*}

$Remark -$ This theorem is valid (with the same proof) for spaces with a semi-definite inner product.
%%%%%
%%%%%
\end{document}
