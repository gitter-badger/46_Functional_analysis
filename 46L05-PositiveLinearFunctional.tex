\documentclass[12pt]{article}
\usepackage{pmmeta}
\pmcanonicalname{PositiveLinearFunctional}
\pmcreated{2013-03-22 17:45:05}
\pmmodified{2013-03-22 17:45:05}
\pmowner{asteroid}{17536}
\pmmodifier{asteroid}{17536}
\pmtitle{positive linear functional}
\pmrecord{11}{40202}
\pmprivacy{1}
\pmauthor{asteroid}{17536}
\pmtype{Definition}
\pmcomment{trigger rebuild}
\pmclassification{msc}{46L05}

\endmetadata

% this is the default PlanetMath preamble.  as your knowledge
% of TeX increases, you will probably want to edit this, but
% it should be fine as is for beginners.

% almost certainly you want these
\usepackage{amssymb}
\usepackage{amsmath}
\usepackage{amsfonts}

% used for TeXing text within eps files
%\usepackage{psfrag}
% need this for including graphics (\includegraphics)
%\usepackage{graphicx}
% for neatly defining theorems and propositions
%\usepackage{amsthm}
% making logically defined graphics
%%%\usepackage{xypic}

% there are many more packages, add them here as you need them

% define commands here

\begin{document}
\subsubsection{Definition}
Let $\mathcal{A}$ be a \PMlinkname{$C^*$-algebra}{CAlgebra} and $\phi$ a linear functional on $\mathcal{A}$.

We say that $\phi$ is a {\bf positive linear functional} on $\mathcal{A}$ if $\phi$ is such that $\phi(x)\geq 0$ for every $x \geq 0$, i.e. for every positive element $x \in \mathcal{A}$.

\subsubsection{Properties}
Let $\phi$ be a positive linear functional on $\mathcal{A}$. Then

\begin{itemize}
\item $\phi(x^*) = \overline{\phi(x)}\;\;$ for every $x \in \mathcal{A}$.
\end{itemize}
\begin{itemize}
\item $|\phi(x^*y)|^2 \leq \phi(x^*x)\phi(y^*y)\;\;$ for every $x, y \in \mathcal{A}$. This is an analog of the Cauchy-Schwartz inequality
\end{itemize}

Let $\phi$ be a linear functional on a $C^*$-algebra $\mathcal{A}$ with identity element $e$. Then
\begin{itemize}
\item $\phi$ is positive if and only if $\phi$ is \PMlinkname{bounded}{ContinuousLinearMapping} and $\|\phi\|= \phi(e)$.
\end{itemize}

\subsubsection{Examples}
\begin{itemize}
\item Let $X$ be a locally compact Hausdorff space and $C_0(X)$ the $C^*$-algebra of continuous functions $X \longrightarrow \mathbb{C}$ that vanish at infinity. Let $\mu$ be a regular Radon measure on $X$. The linear functional $\phi$ defined by integration against $\mu$,
\begin{displaymath}
\phi(f) := \int_X f\;d\mu\;, \qquad\qquad f \in C_0(x)
\end{displaymath}
is a positive linear functional on $C_0(X)$. In fact, by the \PMlinkname{Riesz representation theorem}{RieszRepresentationTheoremOfLinearFunctionalsOnFunctionSpaces}, all positive linear functionals of $C_0(X)$ are of this form.
\end{itemize}
%%%%%
%%%%%
\end{document}
