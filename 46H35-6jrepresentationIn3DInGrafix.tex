\documentclass[12pt]{article}
\usepackage{pmmeta}
\pmcanonicalname{6jrepresentationIn3DInGrafix}
\pmcreated{2013-03-22 18:22:38}
\pmmodified{2013-03-22 18:22:38}
\pmowner{bci1}{20947}
\pmmodifier{bci1}{20947}
\pmtitle{6j-representation in 3D in grafix}
\pmrecord{38}{41019}
\pmprivacy{1}
\pmauthor{bci1}{20947}
\pmtype{Example}
\pmcomment{trigger rebuild}
\pmclassification{msc}{46H35}
\pmclassification{msc}{81T45}
\pmclassification{msc}{57R56}
\pmsynonym{6j-representation in 3D}{6jrepresentationIn3DInGrafix}
%\pmkeywords{6j-representation in 3D}
%\pmkeywords{tetrahedron}
%\pmkeywords{example of TQFT state}
\pmrelated{Tetrahedron}
\pmrelated{StateOnTheTetrahedron}
\pmrelated{RegularTetrahedron3}
\pmrelated{TriangleMidSegmentTheorem}
\pmrelated{Heptahedron}
\pmrelated{GergonneTriangle}
\pmrelated{RegularPolyhedron}
\pmrelated{NormalLine}
\pmdefines{tetrahedron graphics}

\endmetadata

% this is the default PlanetMath preamble.  
%\usepackage{psfrag}
% need this for including graphics (\includegraphics)
%\usepackage{graphicx}
% for neatly defining theorems and propositions
%\usepackage{amsthm}
% making logically defined graphics
%%%\usepackage{xypic}
\usepackage{amssymb}
\usepackage{amsmath}
\usepackage{amsfonts}

\usepackage{pstricks}


% there are many more packages, add them here as you need them


% define commands here

\begin{document}
\subsection{Graphical representation of the topological quantum field (TQFT) state on the tetrahedron}
\begin{center}
\begin{pspicture}(-4,-1)(4,4)
\psline[linecolor=blue,linewidth=0.04]{->}(-1.5,0)(0,-1.5)
\psline[linecolor=blue,linewidth=0.04]{->}(0,-1.5)(3,0)
\psline[linecolor=blue,linewidth=0.04]{->}(3,0)(0.5,3)
\psline[linecolor=blue,linewidth=0.04]{->}(0.5,3)(-1.5,0)
\rput(-1.7,0){2}
\rput(0,-1.75){3}
\rput(3.2,0){4}
\rput(0.5,3.25){1}
\psline[linecolor=blue,linewidth=0.04](0,-1.5)(0.5,3)
\psline[linestyle=dashed](-1.5,0)(3,0)
\end{pspicture}
\end{center}

{\em Notes:}
 The $pstricks$ code for the graphics is courtesy of pahio.
 
 Arrows indicate the orientations defined for the TQFT state on tetrahedron by the $6j$-symbols. 



%%%%%
%%%%%
\end{document}
