\documentclass[12pt]{article}
\usepackage{pmmeta}
\pmcanonicalname{MinkowskiFunctional}
\pmcreated{2013-03-22 14:50:44}
\pmmodified{2013-03-22 14:50:44}
\pmowner{Mathprof}{13753}
\pmmodifier{Mathprof}{13753}
\pmtitle{Minkowski  functional}
\pmrecord{18}{36515}
\pmprivacy{1}
\pmauthor{Mathprof}{13753}
\pmtype{Definition}
\pmcomment{trigger rebuild}
\pmclassification{msc}{46B20}
%\pmkeywords{Minkowski}

% this is the default PlanetMath preamble.  as your knowledge
% of TeX increases, you will probably want to edit this, but
% it should be fine as is for beginners.

% almost certainly you want these
\usepackage{amssymb}
\usepackage{amsmath}
\usepackage{amsfonts}

% used for TeXing text within eps files
%\usepackage{psfrag}
% need this for including graphics (\includegraphics)
%\usepackage{graphicx}
% for neatly defining theorems and propositions
%\usepackage{amsthm}
% making logically defined graphics
%%%\usepackage{xypic}

% there are many more packages, add them here as you need them

% define commands here
\begin{document}
Let $X$ be a normed space and let 
   $K$ an absorbing  convex subset of $X$ such that 
   $0$ is in the interior of $K$.
Then the 
  \emph{Minkowski functional} 
  $\rho \colon X \to \mathbb{R}$ is defined as 
$$
  \rho(x) = \inf \{ \lambda>0 \colon x \in \lambda K \}.
$$

We put $\rho(x) = 0$ whenever $x = 0$.  Clearly $\rho(x) \geq 0$ for all $x$.

It is important to note that in general $\rho(x) \neq \rho(-x)$.

{\bf Properties}\\
$\rho$ is positively $1$- homogeneous.  This means that $$\rho(s \cdot x) = s \cdot \rho(x)$$
for $s > 0$.
%%%%%
%%%%%
\end{document}
