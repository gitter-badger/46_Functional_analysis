\documentclass[12pt]{article}
\usepackage{pmmeta}
\pmcanonicalname{HyperplaneSeparation}
\pmcreated{2013-03-22 17:19:01}
\pmmodified{2013-03-22 17:19:01}
\pmowner{stevecheng}{10074}
\pmmodifier{stevecheng}{10074}
\pmtitle{hyperplane separation}
\pmrecord{4}{39667}
\pmprivacy{1}
\pmauthor{stevecheng}{10074}
\pmtype{Theorem}
\pmcomment{trigger rebuild}
\pmclassification{msc}{46A55}
\pmclassification{msc}{49J27}
\pmclassification{msc}{46A20}
\pmsynonym{separating hyperplane}{HyperplaneSeparation}
\pmrelated{HahnBanachgeometricFormTheorem}

% The standard font packages
\usepackage{amssymb}
\usepackage{amsmath}
\usepackage{amsfonts}

% For neatly defining theorems and definitions
\usepackage{amsthm}

% Including EPS/PDF graphics (\includegraphics)
%\usepackage{graphicx}

% Making matrix-based graphics
%%%\usepackage{xypic}

% Enumeration lists with different styles
%\usepackage{enumerate}

% Set up the theorem environments
\newtheorem{thm}{Theorem}
%\newtheorem*{thm*}{Theorem}

\providecommand{\defnterm}[1]{\emph{#1}}

% The standard number systems
\newcommand{\complex}{\mathbb{C}}
\newcommand{\real}{\mathbb{R}}
\newcommand{\rat}{\mathbb{Q}}
\newcommand{\nat}{\mathbb{N}}
\newcommand{\intset}{\mathbb{Z}}

% Absolute values and norms
% Normal, wide, and big versions of the delimeters
\providecommand{\abs}[1]{\lvert#1\rvert}
\providecommand{\absW}[1]{\left\lvert#1\right\rvert}
\providecommand{\absB}[1]{\Bigl\lvert#1\Bigr\rvert}
\providecommand{\norm}[1]{\lVert#1\rVert}
\providecommand{\normW}[1]{\left\lVert#1\right\rVert}
\providecommand{\normB}[1]{\Bigl\lVert#1\Bigr\rVert}

% Differentiation operators
\providecommand{\od}[2]{\frac{d #1}{d #2}}
\providecommand{\pd}[2]{\frac{\partial #1}{\partial #2}}
\providecommand{\pdd}[2]{\frac{\partial^2 #1}{\partial #2}}
\providecommand{\ipd}[2]{\partial #1 / \partial #2}

% Differentials on integrals
\newcommand{\dx}{\, dx}
\newcommand{\dt}{\, dt}
\newcommand{\dmu}{\, d\mu}

% Inner products
\providecommand{\ip}[2]{\langle {#1}, {#2} \rangle}

% Calligraphic letters
\newcommand{\sF}{\mathcal{F}}
\newcommand{\sD}{\mathcal{D}}

% Standard spaces
\newcommand{\Hilb}{\mathcal{H}}
\newcommand{\Le}{\mathbf{L}}

% Operators and functions occassionally used in my articles
\DeclareMathOperator{\D}{D}
\DeclareMathOperator{\linspan}{span}
\DeclareMathOperator{\rank}{rank}
\DeclareMathOperator{\lindim}{dim}
\DeclareMathOperator{\sinc}{sinc}

% Probability stuff
\newcommand{\PP}{\mathbb{P}}
\newcommand{\E}{\mathbb{E}}

\begin{document}
Let $X$ be a vector space, and $\Phi$ be any subspace
of linear functionals on $X$.
Impose on $X$ the weak topology generated by $\Phi$.

\begin{thm}[Hyperplane Separation Theorem I]
Given a weakly closed convex subset $S \subset X$,
and $a \in X \setminus S$.
there is $\phi \in \Phi$ such that
\[
\phi(a) < \inf_{x \in S} \phi(x)\,.
\]
\begin{proof}
The weak topology on $X$ can be generated by the semi-norms
$x \mapsto \abs{ p(x) }$ for $p \in \Phi$.
A subbasis for the weak topology
consists of neigborhoods of the form
$\{ x \in X \colon \abs{p(x-y)} < \epsilon \}$
for $y \in X$, $p \in \Phi$ and $\epsilon > 0$.
Since $X \setminus S$ is weakly open,
there exist $f_1, \dotsc, f_n \in \Phi$  and $\epsilon > 0$
such that
\[
\abs{f_i(x) - f_i(a)} = \abs{f_i(x-a)} < \epsilon \,, \text{ for all $i=1, \dotsc, n$ \quad implies }
x \in X \setminus S\,.
\]
In other words, if $x \in S$ then at least one of 
$\abs{f_i(x) - f_i(a)}$ is $\geq \epsilon$.

Define a map $F\colon X \to \real^n$
by $F(x) = ( f_1(x), \dotsc, f_n(x) )$.
The set $\overline{F(S)}$ is evidently
closed and convex in $\real^n$, a Hilbert space
under the standard inner product.
So there is a point $b \in \overline{F(S)}$
that minimizes the norm $\norm{b - F(a)}$.

It follows that $\ip{y-b}{b-F(a)} \geq 0$ for all $y \in \overline{F(S)}$;
for otherwise we can attain a smaller value of the norm
by moving from the point $b$ along a line towards $y$.
(Formally, we have
$0 \leq \left.\frac{d}{dt}\right|_{t=0} \norm{ty + (1-t)b - F(a)}^2 = 2\ip{y-b}{b-F(a)}$.)

Take $\phi = \sum_{i=1}^n \lambda_i f_i$
where $\lambda = b-F(a)$.
Then we find, for all $x \in S$,
\begin{align*}
\phi(x-a) &= \ip{b - F(a)}{F(x-a)} \\
&= \ip{b-F(a)}{b-F(a)} + \ip{b-F(a)}{y-b}\,, \quad y = F(x)  \in \overline{F(S)}\\
&\geq \norm{b-F(a)}^2 + 0 \geq \epsilon^2\,. \qedhere
\end{align*}
\end{proof}
\end{thm}

\begin{thm}[Hyperplane Separation Theorem II]
Let $S \subset X$ be a weakly closed convex subset,
and $K \subset X$ a compact convex subset,
that do not intersect each other.
Then there exists $\phi \in \Phi$ such that 
\[
\sup_{y \in K} \phi(y) < \inf_{x \in S} \phi(x)\,.
\]
\begin{proof}
We show that 
$S - K = \{ x - y \colon x \in S\,, y \in K\}$
 is weakly closed in $X$.
Let $\{ z_\alpha = x_\alpha - y_\alpha\} \subseteq A$ 
be a net convergent to $z$.
Since $K$ is compact, $\{ y_\alpha \}$ has a subnet $\{ y_{\alpha(\beta)} \}$
convergent to $y \in K$.
Then the subnet $x_{\alpha(\beta)} = z_{\alpha(\beta)} + y_{\alpha(\beta)}$
is convergent to $x = z + y$.
The point $x$ is in $S$ since $S$ is closed;
therefore $z = x-y$ is in $S - K$.

Also, $S-K$ is convex since $S$ and $K$ are.
Noting that $0 \notin S-K$ (otherwise $S$ and $K$ would have a common point),
we apply the previous theorem to
obtain a $\phi \in \Phi$ such
that 
\[
0 = \phi(0) < \inf_{z \in S-K} \phi(z) \leq \phi(x-y) \,, \text{ for all $x \in S$ and $y \in K$. }
\]
The desired conclusion follows at once.
\end{proof}
\end{thm}

%%%%%
%%%%%
\end{document}
