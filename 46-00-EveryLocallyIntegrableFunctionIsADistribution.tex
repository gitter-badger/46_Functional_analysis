\documentclass[12pt]{article}
\usepackage{pmmeta}
\pmcanonicalname{EveryLocallyIntegrableFunctionIsADistribution}
\pmcreated{2013-03-22 13:44:25}
\pmmodified{2013-03-22 13:44:25}
\pmowner{matte}{1858}
\pmmodifier{matte}{1858}
\pmtitle{every locally integrable function is a distribution}
\pmrecord{8}{34433}
\pmprivacy{1}
\pmauthor{matte}{1858}
\pmtype{Theorem}
\pmcomment{trigger rebuild}
\pmclassification{msc}{46-00}
\pmclassification{msc}{46F05}

\endmetadata

% this is the default PlanetMath preamble.  as your knowledge
% of TeX increases, you will probably want to edit this, but
% it should be fine as is for beginners.

% almost certainly you want these
\usepackage{amssymb}
\usepackage{amsmath}
\usepackage{amsfonts}

% used for TeXing text within eps files
%\usepackage{psfrag}
% need this for including graphics (\includegraphics)
%\usepackage{graphicx}
% for neatly defining theorems and propositions
%\usepackage{amsthm}
% making logically defined graphics
%%%\usepackage{xypic}

% there are many more packages, add them here as you need them

% define commands here

\newcommand{\sR}[0]{\mathbb{R}}
\newcommand{\sC}[0]{\mathbb{C}}
\newcommand{\sN}[0]{\mathbb{N}}
\newcommand{\sZ}[0]{\mathbb{Z}}
\begin{document}
\newcommand{\cD}[0]{\mathcal{D}}

Suppose $U$ is an open set in $\sR^n$ and $f$ is a locally
integrable function on $U$, i.e., $f\in L^1_{\scriptsize{\mbox{loc}}}(U)$.
Then the mapping
\begin{eqnarray*}
T_f: \cD(U) &\to& \sC \\
     u      &\mapsto& \int_U f(x) u(x) dx
\end{eqnarray*}
is a zeroth order distribution. (See parent entry for notation $\cD(U)$.)

\PMlinkname{(proof)}{T_fIsADistributionOfZerothOrder}

If $f$ and $g$ are both locally integrable functions on an open set $U$,
and $T_f=T_g$, then it follows (see 
\PMlinkname{this page}{TheoremForLocallyIntegrableFunctions}),
that $f=g$ almost everywhere. Thus, the mapping $f\mapsto T_f$
is a linear injection when $L^1_{\scriptsize{\mbox{loc}}}$ is equipped with
the usual equivalence relation for an $L^p$-space. For this reason,
one usually writes $f$ for the distribution $T_f$.
%%%%%
%%%%%
\end{document}
