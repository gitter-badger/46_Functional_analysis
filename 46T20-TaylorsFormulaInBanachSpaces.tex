\documentclass[12pt]{article}
\usepackage{pmmeta}
\pmcanonicalname{TaylorsFormulaInBanachSpaces}
\pmcreated{2013-03-22 15:28:27}
\pmmodified{2013-03-22 15:28:27}
\pmowner{stevecheng}{10074}
\pmmodifier{stevecheng}{10074}
\pmtitle{Taylor's formula in Banach spaces}
\pmrecord{10}{37328}
\pmprivacy{1}
\pmauthor{stevecheng}{10074}
\pmtype{Result}
\pmcomment{trigger rebuild}
\pmclassification{msc}{46T20}
\pmclassification{msc}{26B12}
\pmclassification{msc}{41A58}
%\pmkeywords{mean value theorem}
%\pmkeywords{power series}
%\pmkeywords{Taylor series}
%\pmkeywords{Taylor polynomial}

\usepackage{amssymb}
\usepackage{amsmath}
\usepackage{amsfonts}
%\usepackage{amsthm}
\usepackage{enumerate}

% used for TeXing text within eps files
%\usepackage{psfrag}
% need this for including graphics (\includegraphics)
%\usepackage{graphicx}
% making logically defined graphics
%%%\usepackage{xypic}

% define commands here
\newcommand{\complex}{\mathbb{C}}
\newcommand{\real}{\mathbb{R}}
\newcommand{\rat}{\mathbb{Q}}
\newcommand{\nat}{\mathbb{N}}

\providecommand{\abs}[1]{\lvert#1\rvert}
\providecommand{\absW}[1]{\left\lvert#1\right\rvert}
\providecommand{\absB}[1]{\Bigl\lvert#1\Bigr\rvert}
\providecommand{\norm}[1]{\lVert#1\rVert}
\providecommand{\normW}[1]{\left\lVert#1\right\rVert}
\providecommand{\normB}[1]{\Bigl\lVert#1\Bigr\rVert}
\providecommand{\defnterm}[1]{\emph{#1}}

\DeclareMathOperator{\D}{D}
\DeclareMathOperator{\linspan}{span}
\begin{document}
Let $U$ be an open subset of a real Banach space $X$.
If $f\colon U \to \real$ is differentiable $n+1$ times on  $U$,
it may be expanded by Taylor's formula:
\begin{equation}\label{taylor}
f(x) = f(a) + \D f(a) \cdot h + \frac{1}{2!} \D^2 f(a) \cdot h^2
+ \dotsb + \frac{1}{n!} \D^n f(a) \cdot h^n + R_n(x)\,,
\end{equation}
with the following expressions for the remainder term $R_n(x)$:
\begin{align*}
R_n(x) &= \frac{1}{n!} \, \D^{n+1}f(\eta) \cdot (x-\eta)^n h & \text{Cauchy form of remainder} \\
R_n(x) &= \frac{1}{(n+1)!} \, \D^{n+1}f(\xi) \cdot h^{n+1} & \text{Lagrange form of remainder} \\
R_n(x) &= \frac{1}{n!} \int_0^1 \D^{n+1} f(a+th) \cdot ((1-t)h)^n h \, dt & \text{integral form of remainder}
\end{align*}
Here $a$ and $x$ must be points of $U$ such that the line segment
between $a$ and $x$ lie inside $U$, $h$ is $x-a$,
and the points $\xi$ and $\eta$ lie on the same line segment,
strictly between $a$ and $x$.

The $k$th Fr\'echet derivative of $f$ at $a$ is being denoted by
$\D^k f(a)$, to be viewed as a multilinear map $X^k \to \real$.
The $\cdot h^k$ notation means to evaluate a multilinear map
at $(h, \dotsc, h)$.

\section{Remainders for vector-valued functions}
If $Y$ is a Banach space, we may also consider
Taylor expansions for $f\colon U \to Y$.
Formula \eqref{taylor} takes the same form,
but the Cauchy and Lagrange forms of the remainder
will not be exact;
they will only be bounds on $R_n(x)$.
That is, for $f\colon U \to Y$,
\begin{align*}
\norm{R_n(x)} &\leq \frac{1}{n!} \, \normW{\D^{n+1}f(\eta) \cdot (x-\eta)^n h} & \text{Cauchy form of remainder} \\
\norm{R_n(x)} &\leq \frac{1}{(n+1)!} \, \normW{\D^{n+1}f(\xi) \cdot h^{n+1}} & \text{Lagrange form of remainder}
\end{align*}
It is not hard to find counterexamples if we attempt to remove the norm signs or change
the inequality to equality in the above formulas.  

However, the integral form of the remainder continues to hold for $Y \neq \real$,
although strictly speaking it only applies if the integrand is \emph{integrable}.
The integral form is also applicable when $X$ and $Y$ are complex Banach spaces.

\section*{Mean Value Theorem}
The Mean Value Theorem can be obtained
as the special case $n=0$ with the Lagrange form of the remainder:
for $f\colon U \to Y$ differentiable,
\begin{equation}\label{mean-value}
\norm{f(x) - f(a)} \leq \norm{\D f(\xi) \cdot (x-a)}
\end{equation}
If $Y = \real$, then the norm signs may be removed from
\eqref{mean-value}, and the inequality replaced by equality.

Formula \eqref{mean-value} also holds under the much
weaker hypothesis
that $f$ only has a directional derivative along the line
segment from $a$ to $x$.

\section*{Weaker bounds for the remainder}
If $f \colon U \to Y$ is only differentiable $n$ times \emph{at} $a$,
then we cannot quantify the remainder by the $n+1$th derivative,
but it is still true
that
\begin{equation}
R_n(x) = o( \norm{x-a}^n ) \text{ as $x \to a$. }
\end{equation}

\section*{Finite-dimensional case}
If $X = \real^m$ and $Y = \real$,
$\D^k$ has the following expression in terms of coordinates:
\[
\D^k f(a) \cdot (\xi_1, \dotsc, \xi_k) = \sum_{i_1, \dotsc, i_k}
\frac{\partial^k f}{\partial x^{i_1} \dotsm \partial x^{i_k}} \,
\xi_1^{i_1} \dotsm \xi_k^{i_k}\,,
\]
where each $i_j$ runs from $1, \dotsc, m$ in the sum.

If we collect the equal mixed partials
(assuming that they are continuous)
then
\[
\frac{1}{k!} \, \D^k f(a) \cdot h^k
= \sum_{\abs{J} = k} \frac{1}{J!} \frac{\partial^{\abs{J}} f}{\partial x^J} h^J\,,
\]
where $J$ is a multi-index of $m$ components, and each component $J_i$ indicates
how many times the derivative with respect to the $i$th coordinate should be taken,
and the exponent that the $i$th coordinate of $h$ should be raised to
in the monomial $h^J$.
The multi-index $J$ runs through all combinations
such that $J_1 + \dotsb + J_m = \abs{J} = k$ in the sum.
The notation $J!$ means $J_1! \dotsm J_m!$.

All this is more easily assimilated if we remember that
$\D^k f(a) \cdot h^k$ is supposed to be a polynomial of degree $k$.
Also $\abs{J}!/J!$ is just the multinomial coefficient.

\section*{Taylor series}
If $\lim_{n \to \infty} R_n(x) = 0$,
then we may write
\begin{equation}\label{taylor}
f(x) = f(a) + \D f(a) \cdot h + \frac{1}{2!} \D^2 f(a) \cdot h^2 + \dotsb
\end{equation}
as a convergent infinite series. Elegant as such an expansion is,
it is not seen very often,
for the reason that higher order Fr\'echet derivatives, especially in infinite-dimensional spaces,
are often difficult to calculate.

But a notable exception occurs if a function $f$ is defined by a convergent ``power series''
\begin{equation}\label{power-series}
f(x) = \sum_{k=0}^\infty M_k \cdot (x-a)^k
\end{equation}
where $\{ M_k : k = 0, 1, \dotsc \}$ is a family of continuous symmetric multilinear functions $X^k \to Y$.
In this case, the series \eqref{power-series} is the Taylor series for $f$ at $a$.

\begin{thebibliography}{3}
\bibitem{Wouk}
Arthur Wouk. {\it A course of applied functional analysis}. Wiley-Interscience, 1979.
\bibitem{Zeidler}
Eberhard Zeidler. {\it Applied functional analysis: main principles and their applications}. Springer-Verlag, 1995.
\bibitem{Spivak}
Michael Spivak. {\it Calculus}, third edition. Publish or Perish, 1994.
\end{thebibliography}
%%%%%
%%%%%
\end{document}
