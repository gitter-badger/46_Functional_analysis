\documentclass[12pt]{article}
\usepackage{pmmeta}
\pmcanonicalname{ProofOfQuotientsInCalgebras}
\pmcreated{2013-03-22 17:41:56}
\pmmodified{2013-03-22 17:41:56}
\pmowner{asteroid}{17536}
\pmmodifier{asteroid}{17536}
\pmtitle{proof of quotients in $C^*$-algebras}
\pmrecord{7}{40141}
\pmprivacy{1}
\pmauthor{asteroid}{17536}
\pmtype{Proof}
\pmcomment{trigger rebuild}
\pmclassification{msc}{46L05}

\endmetadata

% this is the default PlanetMath preamble.  as your knowledge
% of TeX increases, you will probably want to edit this, but
% it should be fine as is for beginners.

% almost certainly you want these
\usepackage{amssymb}
\usepackage{amsmath}
\usepackage{amsfonts}

% used for TeXing text within eps files
%\usepackage{psfrag}
% need this for including graphics (\includegraphics)
%\usepackage{graphicx}
% for neatly defining theorems and propositions
%\usepackage{amsthm}
% making logically defined graphics
%%%\usepackage{xypic}

% there are many more packages, add them here as you need them

% define commands here

\begin{document}
\PMlinkescapeword{lies on}

{\bf Proof:} We have that $\mathcal{I}$ is \PMlinkname{self-adjoint}{InvolutaryRing}, since it is a closed ideal of a \PMlinkname{$C^*$-algebra}{CAlgebra} (see \PMlinkname{this entry}{ClosedIdealsInCAlgebrasAreSelfAdjoint}). Hence, the involution in $\mathcal{A}$ induces a well-defined involution in $\mathcal{A}/\mathcal{I}$ by $(x+\mathcal{I})^*:=x^*+\mathcal{I}$.

Recall that, since $\mathcal{I}$ is closed, the quotient norm is indeed a norm in $\mathcal{A}/\mathcal{I}$ that makes $\mathcal{A}/\mathcal{I}$ a Banach algebra (see \PMlinkname{this entry}{QuotientsOfBanachAlgebras}). Thus we only have to prove the $C^*$ \PMlinkescapetext{equality} to prove that $\mathcal{A}/\mathcal{I}$ is a $C^*$-algebra.

Recall that \PMlinkname{$C^*$-algebras have approximate identities}{CAlgebrasHaveApproximateIdentities}. Notice that $\mathcal{I}$ itself is a $C^*$-algebra and pick an approximate identity $(e_{\lambda})$ in $\mathcal{I}$ such that
\begin{itemize}
\item each $e_{\lambda}$ is positive.
\item $\|e_{\lambda}\|\leq 1$
\end{itemize}

We will only prove the case when $\mathcal{A}$ has an identity element $e$. For the non-unital case, one can consider $\mathcal{A}$ as a $C^*$-subalgebra of its minimal unitization and the same proof will still work.

Let $\|\cdot\|_{q}$ denote the quotient norm in $\mathcal{A}/\mathcal{I}$. We claim that for every $x \in \mathcal{A}$:
\begin{align}
\|x + \mathcal{I}\|_q = \lim_{\lambda}\|x(e-e_{\lambda})\|
\end{align}
We will prove the above equality as a lemma at the end of the entry. Assuming this result, it follows that for every $a \in \mathcal{A}$
\begin{align*}
\|x+\mathcal{I}\|_q^2 = \lim \|x(e-e_{\lambda})\|^2 = \lim \|(e-e_{\lambda})x^*x(e-e_{\lambda})\| \leq \lim \|(e-e_{\lambda})\|\|x^*x(e-e_{\lambda})\|
\end{align*}

Since each $e_{\lambda}$ is positive and $\|e_{\lambda}\|\leq 1$ we know that its spectrum lies on the interval $[0,1]$. Hence $e-e_{\lambda}$ is also positive and its spectrum also lies on the interval $[0,1]$. Thus, $\|e-e_{\lambda}\|\leq 1$. Therefore:
\begin{align*}
\|x+\mathcal{I}\|_q^2 \leq \lim \|(e-e_{\lambda})\|\|x^*x(e-e_{\lambda})\| \leq \lim \|x^*x(e-e_{\lambda})\| = \|x^*x+\mathcal{I}\|_q
\end{align*}

Since $\mathcal{A}/\mathcal{I}$ is a Banach algebra, we also have $\|x^*x+\mathcal{I}\|_q\leq \|x+\mathcal{I}\|_q^2$ and so
\begin{displaymath}
\|x+\mathcal{I}\|_q^2=\|x^*x+\mathcal{I}\|_q
\end{displaymath}
which proves that $\mathcal{A}/\mathcal{I}$ is a $C^*$-algebra. $\square$

$\;$

We now prove equality (1) as a lemma.

{\bf Lemma -} Suppose $\mathcal{A}$ is a $C^*$-algebra with identity element $e$. Let $\mathcal{I} \subset \mathcal{A}$ be a closed ideal and $(e_{\lambda})$ be an approximate identity in $\mathcal{I}$ such that each $e_{\lambda}$ is positive and $\|e_{\lambda}\|\leq 1$. Then
\begin{displaymath}
\|x + \mathcal{I}\|_q = \lim_{\lambda}\|x(e-e_{\lambda})\|
\end{displaymath}
for every $x$ in $\mathcal{A}$.

{\bf Proof:} Since $y(e-e_{\lambda})\longrightarrow 0$ for every $y \in \mathcal{I}$ it follows that
\begin{eqnarray*}
\limsup \|x(e-e_{\lambda})\|& = & \limsup\|x-xe_{\lambda}-y+ye_{\lambda}\|\\
& = & \limsup\|(x-y)(e-e_{\lambda})\|\\
& \leq & \|x-y\|
\end{eqnarray*}

Therefore, taking the infimum over all $y \in \mathcal{I}$ we obtain:
\begin{displaymath}
\limsup \|x(e-e_{\lambda})\| \leq \inf_{y \in \mathcal{I}} \|x-y\| = \|x+\mathcal{I}\|_q
\end{displaymath}

Also, since $xe_{\lambda} \in \mathcal{I}$,
\begin{displaymath}
\liminf\|x(e-e_{\lambda})\|\geq \inf_{y \in \mathcal{I}} \|x-y\| = \|x+\mathcal{I}\|_q
\end{displaymath}
and this proves the lemma. $\square$
%%%%%
%%%%%
\end{document}
