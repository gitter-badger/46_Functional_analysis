\documentclass[12pt]{article}
\usepackage{pmmeta}
\pmcanonicalname{DifferentiationUnderTheIntegralSign}
\pmcreated{2013-03-22 16:26:37}
\pmmodified{2013-03-22 16:26:37}
\pmowner{stevecheng}{10074}
\pmmodifier{stevecheng}{10074}
\pmtitle{differentiation under the integral sign}
\pmrecord{16}{38599}
\pmprivacy{1}
\pmauthor{stevecheng}{10074}
\pmtype{Topic}
\pmcomment{trigger rebuild}
\pmclassification{msc}{46F10}
\pmclassification{msc}{28A25}
\pmclassification{msc}{26B15}
\pmclassification{msc}{26A24}
\pmsynonym{Leibniz's rule}{DifferentiationUnderTheIntegralSign}
\pmrelated{DerivativeOfRiemannIntegral}
\pmrelated{IntegrationUnderIntegralSign}
\pmrelated{HolomorphicFunctionAssociatedWithContinuousFunction}

\endmetadata

\usepackage{amssymb}
\usepackage{amsmath}
\usepackage{amsfonts}
\usepackage{amsthm}
\usepackage{enumerate}
%\usepackage{graphicx}
%\usepackage{psfrag}
%%%\usepackage{xypic}

% define commands here
\newcommand{\complex}{\mathbb{C}}
\newcommand{\real}{\mathbb{R}}
\newcommand{\rat}{\mathbb{Q}}
\newcommand{\nat}{\mathbb{N}}

\providecommand{\abs}[1]{\lvert#1\rvert}
\providecommand{\absW}[1]{\left\lvert#1\right\rvert}
\providecommand{\absB}[1]{\Bigl\lvert#1\Bigr\rvert}
\providecommand{\norm}[1]{\lVert#1\rVert}
\providecommand{\normW}[1]{\left\lVert#1\right\rVert}
\providecommand{\normB}[1]{\Bigl\lVert#1\Bigr\rVert}
\providecommand{\defnterm}[1]{\emph{#1}}

\providecommand{\ddx}[1]{\frac{d #1}{dx}}
\providecommand{\pdx}[1]{\frac{\partial #1}{\partial x}}
\providecommand{\ipdx}[1]{\partial #1 / \partial x}
\providecommand{\pdxi}[1]{\frac{\partial #1}{\partial x_i}}
\providecommand{\ipdxi}[1]{\partial #1 / \partial x_i}

\newtheorem{thm}{Theorem}

\providecommand{\fnl}[2]{\langle {#1}, {#2} \rangle}
\newcommand{\sD}{\mathcal{D}}

\begin{document}
\PMlinkescapeword{sides}
\PMlinkescapeword{theory}
\PMlinkescapeword{satisfies}
\PMlinkescapeword{restriction}
\PMlinkescapeword{conjecture}
\PMlinkescapeword{place}
\PMlinkescapeword{even}
\PMlinkescapeword{variations}
\PMlinkescapeword{type}
\PMlinkescapeword{types}
\PMlinkescapeword{volumes}

The technique of \emph{differentiation under the integral sign}
concerns the interchange of the operation of differentiation
with respect to a parameter with the operation of integration
over some other variable:
\[
\pdx{} \int_\Omega f(x, \omega) \, d\omega = \int_\Omega 
\pdx{} f(x,\omega) \, d\omega\,.
\]

Intuitively, the rule ought to work because 
differentiation commutes with finite summation,
and one may conjecture that it can also 
commute with infinite summation (in the form of the integral),
at least in some cases.

The theorems below give some sufficient conditions,
in increasing generality and sophistication,
for which the swap of differentiation and
integration is legal.

\subsection*{Formal statements}

\begin{thm}[Elementary Calculus version]\label{thm:elem}
Let $f\colon [a,b] \times Y \to \real$ be a function,
with $[a,b]$ being a closed interval, and
$Y$ being a compact subset\footnote{Assumed to be Jordan-measurable if the Riemann integral is to be used.} of $\real^n$.
Suppose that both $f(x,y)$ and $\ipdx{f(x,y)}$ are continuous 
in the variables $x$ and $y$ jointly.
Then $\int_Y f(x.y) \, dy$ exists as a continuously differentiable function of
$x$ on $[a,b]$, with derivative
\[
\ddx{} \int_Y f(x, y) \, dy = \int_Y \pdx{} f(x,y) \, dy\,.
\]
\end{thm}

Theorem 1 is the formulation of integration
under the integral sign that usually appears in elementary Calculus texts.
Unfortunately, its restriction that $Y$ must
be compact can be quite severe for applications: e.g.
integrals over $(-\infty,+\infty)$ are not included.
Theorem 2 below addresses this problem and others:

\begin{thm}[Measure theory version]\label{thm:meas}
Let $X$ be an open subset of $\real$, and $\Omega$ be a measure space.
Suppose $f\colon X \times \Omega \to \real$ satisfies
the following conditions:
\begin{enumerate}
\item
$f(x, \omega)$ is a Lebesgue-integrable function of $\omega$ for each $x \in X$.
\item
For almost all 
$\omega \in \Omega$, the derivative $\ipdx{f(x,\omega)}$ exists for all $x \in X$.
\item
There is an integrable function $\Theta\colon \Omega \to \real$ such that 
$\absW{\ipdx{f(x,\omega)}} \leq \Theta(\omega)$
for all $x \in X$.
\end{enumerate}
Then for all $x \in X$,
\[
\ddx{} \int_\Omega f(x, \omega) \, d\omega = \int_\Omega 
\pdx{} f(x,\omega) \, d\omega\,.
\]
\end{thm}

Theorem 2 suffices for many applications,
but using the Fundamental Theorem of Calculus for Lebesgue integration,
we can weaken the hypotheses for differentiating under
the integral sign even further:

\begin{thm}
Let $X$ be an open subset of $\real$, and $\Omega$ be a measure space.
Suppose that a function $f\colon X \times \Omega \to \real$ satisfies
the following conditions:
\begin{enumerate}
\item
$f(x,\omega)$ is a measurable function of $x$ and $\omega$
jointly, and is integrable over $\omega$, for almost all $x \in X$ held fixed.
\item
For almost all 
$\omega \in \Omega$, $f(x,\omega)$ is an absolutely continuous
function of $x$.  (This guarantees that $\ipdx{f(x,\omega)}$ exists
almost everywhere.)
\item
$\ipdx{f}$ is ``locally integrable'' --- that is, for
all compact intervals $[a,b]$ contained in $X$:
\[
\int_{a}^b \int_\Omega \absW{ \pdx{} f(x,\omega)} \, d\omega \, dx < \infty\,.
\]
\end{enumerate}
Then $\int_\Omega f(x,\omega) \, d\omega$ is an absolutely continuous function of $x$, and for almost every $x \in X$, its derivative exists
and is given by
\[
\ddx{} \int_\Omega f(x, \omega) \, d\omega = \int_\Omega 
\pdx{} f(x,\omega) \, d\omega\,.
\]
\end{thm}

If the Kurzweil-Henstock integral --- which has 
a stronger \PMlinkname{Fundamental Theorem of Calculus}{FundamentalTheoremOfCalculusForKurzweilHenstockIntegral}
 ---
is used in place
of the Lebesgue integral,
Theorem 3 can be generalized to a formulation
that provides also the \emph{necessary} 
conditions for differentiation under the integral sign.
See \cite{Talvila} for the full details.

Yet this is not the end of the story.
There are some applications in which the integrand is too ``irregular'', 
or the integral of the differentiated integrand
becomes divergent, and neither Theorem 2 or Theorem 3 would apply.
However, if we use generalized functions 
(all of which can be differentiated at will),
then we can extend the technique of differentiation under 
the integral sign further,
and make sense of any ``irregular'' integrals that may result:

\begin{thm}[Distribution theory version]
Let $X$ be an open set in $\real^m$, and $\Omega$ be a measure space.
Given $f(x,\omega)$, for each $\omega \in \Omega$, a generalized function of $x \in X$ (in the sense of Schwartz's theory of distributions), 
define:
\[
\fnl{\int_\Omega f(\cdot,\omega) \, d\omega}{\phi} := 
\int_\Omega \fnl{f(\cdot, \omega)}{\phi} \, d\omega\,,
\quad \phi \in \sD(X)\,.
\]
Assume
the above integral is well-defined and gives a distribution.
Then
\[
\pdxi{}  \,
\int_\Omega f(x, \omega) \, d\omega
= \int_\Omega \pdxi{} f(x, \omega) \, d\omega\,.
\]
where $\ipdxi{}$ refers to the generalized derivative 
of generalized functions on both sides of the equation.
\end{thm}

For an absolutely continuous function,
the generalized derivative coincides with the ordinary derivative,
so Theorem 4 indeed generalizes Theorem 3.
On the other hand, there are cases where 
the integrand is not absolutely continuous --- and so
has a generalized derivative different from the ordinary derivative
--- yet its integral has a classical derivative that is
represented by the final equation of Theorem 4.  For instance,
the integrand may involve a step function,
and its derivative would thus involve a Dirac delta distribution,
that when integrated, yields an ordinary locally-integrable
function (of the parameter $x$).

Theorem 4 is not so well-publicized,
but appears, for example, in \cite{Jones}, and hinted
at in a comment in \cite{Schwartz}.

\subsection*{Other variations}

There are other frequently-used variations of the theorems above.

\textbf{Moving domains of integration}.
Not only can the integrand vary with the parameter,
we can consider domains of integration,
subsets of $\real^n$, that vary with the parameter.  

In the one-dimensional case,
for continuously differentiable functions
$\alpha\colon [a,b] \to \real$, $\beta\colon [a,b] \to \real$, and 
$f\colon [a,b] \times \real \to \real$, we have:
\[
\ddx{} \int_{\alpha(x)}^{\beta(x)} f(x,y) \, dy
= \int_{\alpha(x)}^{\beta(x)}
\pdx{f(x,y)} \, dy +
\frac{d\beta(x)}{dx} f(x, \beta(x)) - \frac{d\alpha(x)}{dx} f(x, \alpha(x))\,.
\] 
This result can be extrapolated from Theorem \ref{thm:elem},
with the help of the Fundamental Theorem of Calculus and
the \PMlinkname{multi-variate chain rule}{ChainRuleSeveralVariables}.

Generalizations to varying smooth surfaces or volumes --- or, more generally, 
$k$-dimensional differentiable manifolds in $\real^n$ --- can be obtained
by using integrals of differential forms on \PMlinkname{chains}{NChain},
and Stokes' Theorem.  Details can be found in \cite{Flanders}.

\textbf{Different types of integrals}.
The differentiation can also be taken under
integrals other than of the standard Riemann type,
such as the line integrals and surface integrals of vector calculus,
or complex contour integrals.
(Actually, these kinds of integrals can be re-formulated as
Lebesgue integrals, so Theorem \ref{thm:meas} applies to them.)

\textbf{Complex variables}. 
Other applications require differentiating holomorphic functions with respect to a complex variable, and Theorem \ref{thm:meas} generalizes directly to this situation, without requiring differentiation with respect to real variables as an intermediary.


\begin{thebibliography}{XXXXXX}
\bibitem[Flanders]{Flanders}
Harley Flanders. ``Differentiation under the Integral Sign''.
\emph{American Mathematical Monthly}, vol. 80 (June-July 1973), p. 615-627.

\bibitem[Folland]{Folland}
Gerald B. Folland. \emph{Real Analysis: Modern Techniques and Their Applications}, second ed. Wiley-Interscience, 1999.

\bibitem[Jones]{Jones}
D. S. Jones. \emph{The Theory of Generalized Functions}, second ed.
Cambridge University Press, 1982.

\bibitem[Munkres]{Munkres}
James R. Munkres. \emph{Analysis on Manifolds}.
Westview Press, 1991.

\bibitem[Schwartz]{Schwartz}
Laurent Schwartz. \emph{Th\'eorie des Distributions}, vol. I.
Hermann, 1957.

\bibitem[Talvila]{Talvila}
Erik Talvila. ``\PMlinkexternal{Necessary and Sufficient Conditions for Differentiating Under the Integral Sign}{http://www.math.ualberta.ca/~etalvila/papers/difffinal.pdf}''.
\emph{American Mathematical Monthly}, vol. 108 (June-July 2001), p. 544-548.

\end{thebibliography}

The author of this entry has also written an exposition,
``\PMlinkexternal{Differentiation under the Integral Sign using Weak Derivatives}{http://gold-saucer.afraid.org/math/diff-int/diff-int.pdf}'',
containing a proof of Theorem 4 along with detailed computational examples.

%%%%%
%%%%%
\end{document}
