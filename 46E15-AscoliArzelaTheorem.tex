\documentclass[12pt]{article}
\usepackage{pmmeta}
\pmcanonicalname{AscoliArzelaTheorem}
\pmcreated{2013-03-22 12:41:00}
\pmmodified{2013-03-22 12:41:00}
\pmowner{paolini}{1187}
\pmmodifier{paolini}{1187}
\pmtitle{Ascoli-Arzel\`a theorem}
\pmrecord{13}{32961}
\pmprivacy{1}
\pmauthor{paolini}{1187}
\pmtype{Theorem}
\pmcomment{trigger rebuild}
\pmclassification{msc}{46E15}
\pmsynonym{Arzel\`a-Ascoli theorem}{AscoliArzelaTheorem}
\pmrelated{MontelsTheorem}

\endmetadata

% this is the default PlanetMath preamble.  as your knowledge
% of TeX increases, you will probably want to edit this, but
% it should be fine as is for beginners.

% almost certainly you want these
\usepackage{amssymb}
\usepackage{amsmath}
\usepackage{amsfonts}

% used for TeXing text within eps files
%\usepackage{psfrag}
% need this for including graphics (\includegraphics)
%\usepackage{graphicx}
% for neatly defining theorems and propositions
\usepackage{amsthm}
% making logically defined graphics
%%%\usepackage{xypic}

% there are many more packages, add them here as you need them

% define commands here
\newcommand{\R}{\mathbb R}
%\newtheorem{theorem}
\begin{document}
%\begin{theorem}
Let $\Omega$ be a bounded subset of $\R^n$ and $(f_k)$ a sequence of functions $f_k\colon \Omega\to \R^m$. If $\{f_k\}$ is equibounded and uniformly equicontinuous then there exists a uniformly convergent subsequence $(f_{k_j})$.
%\end{theorem}

A more abstract (and more general) version is the following.

%\begin{theorem}
Let $X$ and $Y$ be totally bounded metric spaces and let $F\subset \mathcal C(X,Y)$ be an uniformly equicontinuous family of continuous mappings from $X$ to $Y$. 
Then $F$ is totally bounded (with respect to the uniform convergence metric induced by $\mathcal C (X,Y)$).
%\end{theorem}

Notice that the first version is a consequence of the second. 
Recall, in fact, that a subset of a complete metric space is totally bounded if and only if its closure is compact (or sequentially compact).
Hence $\Omega$ is totally bounded and all the functions $f_k$ have image in a totally bounded set. Being $F=\{f_k\}$ totally bounded means that $\overline F$ is sequentially compact and hence $(f_k)$ has a convergent subsequence.
%%%%%
%%%%%
\end{document}
