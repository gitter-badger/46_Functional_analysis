\documentclass[12pt]{article}
\usepackage{pmmeta}
\pmcanonicalname{Cone1}
\pmcreated{2013-03-22 15:32:58}
\pmmodified{2013-03-22 15:32:58}
\pmowner{matte}{1858}
\pmmodifier{matte}{1858}
\pmtitle{cone}
\pmrecord{16}{37447}
\pmprivacy{1}
\pmauthor{matte}{1858}
\pmtype{Definition}
\pmcomment{trigger rebuild}
\pmclassification{msc}{46-00}
\pmrelated{ProperCone}
\pmrelated{GeneralizedFarkasLemma}
\pmdefines{blunt cone}
\pmdefines{pointed cone}
\pmdefines{salient cone}
\pmdefines{cone with vertex}
\pmdefines{wedge}
\pmdefines{proper cone}
\pmdefines{generating}

\endmetadata

% this is the default PlanetMath preamble.  as your knowledge
% of TeX increases, you will probably want to edit this, but
% it should be fine as is for beginners.

% almost certainly you want these
\usepackage{amssymb}
\usepackage{amsmath}
\usepackage{amsfonts}
\usepackage{amsthm}

\usepackage{mathrsfs}

% used for TeXing text within eps files
%\usepackage{psfrag}
% need this for including graphics (\includegraphics)
%\usepackage{graphicx}
% for neatly defining theorems and propositions
%
% making logically defined graphics
%%%\usepackage{xypic}

% there are many more packages, add them here as you need them

% define commands here

\newcommand{\sR}[0]{\mathbb{R}}
\newcommand{\sC}[0]{\mathbb{C}}
\newcommand{\sN}[0]{\mathbb{N}}
\newcommand{\sZ}[0]{\mathbb{Z}}

 \usepackage{bbm}
 \newcommand{\Z}{\mathbbmss{Z}}
 \newcommand{\C}{\mathbbmss{C}}
 \newcommand{\F}{\mathbbmss{F}}
 \newcommand{\R}{\mathbbmss{R}}
 \newcommand{\Q}{\mathbbmss{Q}}



\newcommand*{\norm}[1]{\lVert #1 \rVert}
\newcommand*{\abs}[1]{| #1 |}



\newtheorem{thm}{Theorem}
\newtheorem{defn}{Definition}
\newtheorem{prop}{Proposition}
\newtheorem{lemma}{Lemma}
\newtheorem{cor}{Corollary}
\begin{document}
\PMlinkescapeword{vertex}

\begin{defn}
Suppose $V$ is a real (or complex) vector space with a subset $C$.
\begin{enumerate}
\item If $\lambda C \subset C$ for any real $\lambda >0$,
then $C$ is called a {\bf cone}.
\item If the origin belongs to a cone, then the cone is said to be {\bf pointed}.
Otherwise, the cone is {\bf blunt}.
\item A pointed cone is {\bf salient}, if it contains no
$1$-dimensional vector subspace of $V$.
\item If $C-x_0$ is a cone for some $x_0$ in $V$,
then $C$ is a {\bf cone with vertex} at $x_0$.
\item A convex pointed cone is called a {\bf wedge}.
\item A {\bf proper cone} is a convex cone $C$ with vertex at $0$, such that $C\cap (-C)=\lbrace 0\rbrace$.  A slightly more specific definition of a proper cone is this \PMlinkname{entry}{ProperCone}, but it requires the vector space to be topological.
\item A cone $C$ is said to be {\bf generating} if $V=C-C$.  In this case, $V$ is said to be {\bf generated by} $C$.
\end{enumerate}
\end{defn}

\subsubsection*{Examples}
\begin{enumerate}
\item In $\sR$, the set $x>0$ is a blunt cone.
\item In $\sR$, the set $x\ge 0$ is a pointed salient cone.
\item Suppose $x\in \sR^n$. Then for any $\varepsilon>0$, the set
$$
C=\bigcup \{\, \lambda B_x(\varepsilon) \mid \lambda >0 \,\}
$$
is an open cone. If $|x| < \varepsilon$, then $C=\sR^n$.
Here,
$B_x(\varepsilon)$ is the open ball at $x$ with radius $\varepsilon$.
\item In a normed vector space, a blunt cone $C$ is completely
determined by the intersection of $C$ with the unit sphere. 
\end{enumerate}

\subsubsection*{Properties}
\begin{enumerate}
\item The union and intersection of a collection of cones is a cone.  In other words, the set of cones forms a complete lattice.
\item The complement of a cone is a cone.  This means that the complete lattice of cones is also a complemented lattice. 
\item A cone $C$ is convex iff $C+C\subseteq C$.
\begin{proof}
If $C$ is convex and $a,b\in C$, then $\frac{1}{2}a,\frac{1}{2}b\in C$, so their sum, being the convex combination of $a,b$, is in $C$, and therefore $a+b=2(\frac{1}{2}a+\frac{1}{2}b)\in C$ also.  Conversely, suppose a cone $C$ satisfies $C+C\subseteq C$, and $a,b\in C$.  Then $\lambda a,(1-\lambda)b\in C$ for $\lambda> 0$ (the case when $\lambda=0$ is obvious).  Therefore their sum is also in $C$.
\end{proof}
\item A cone containing $0$ is a cone with vertex at $0$.  As a result, a wedge is a cone with vertex at $0$.
\item The only cones that are subspaces at the same time are wedges.
\end{enumerate}


\begin{thebibliography}{9}
\bibitem{reed} M. Reed, B. Simon,
\emph{Methods of Modern Mathematical Physics: Functional Analysis I},
Revised and enlarged edition, Academic Press, 1980.
\bibitem{horvath} J. Horv\'ath, \emph{Topological Vector Spaces and Distributions},
Addison-Wesley Publishing Company, 1966.
\bibitem{edwards} R.E. Edwards, \emph{Functional Analysis: Theory and Applications},
Dover Publications, 1995.
\bibitem{glazman} I.M. Glazman, Ju.I. Ljubic, \emph{Finite-Dimensional Linear Analysis, A systematic Presentation in Problem Form},
Dover Publications, 2006.
\end{thebibliography}
%%%%%
%%%%%
\end{document}
