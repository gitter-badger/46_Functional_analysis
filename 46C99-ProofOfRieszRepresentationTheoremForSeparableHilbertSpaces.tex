\documentclass[12pt]{article}
\usepackage{pmmeta}
\pmcanonicalname{ProofOfRieszRepresentationTheoremForSeparableHilbertSpaces}
\pmcreated{2013-03-22 14:34:20}
\pmmodified{2013-03-22 14:34:20}
\pmowner{asteroid}{17536}
\pmmodifier{asteroid}{17536}
\pmtitle{proof of Riesz representation theorem for separable Hilbert spaces}
\pmrecord{6}{36130}
\pmprivacy{1}
\pmauthor{asteroid}{17536}
\pmtype{Proof}
\pmcomment{trigger rebuild}
\pmclassification{msc}{46C99}
%\pmkeywords{Hilbert}
%\pmkeywords{Riesz}
%\pmkeywords{representation}

% this is the default PlanetMath preamble.  as your knowledge
% of TeX increases, you will probably want to edit this, but
% it should be fine as is for beginners.

% almost certainly you want these
\usepackage{amssymb}
\usepackage{amsmath}
\usepackage{amsfonts}

% used for TeXing text within eps files
%\usepackage{psfrag}
% need this for including graphics (\includegraphics)
%\usepackage{graphicx}
% for neatly defining theorems and propositions
%\usepackage{amsthm}
% making logically defined graphics
%%%\usepackage{xypic}

% there are many more packages, add them here as you need them

% define commands here
\begin{document}
Let $\lbrace {\bf e}_0, {\bf e}_1, {\bf e}_2, \ldots \rbrace$ be an orthonormal basis for the Hilbert space $\mathcal{H}$.  Define $$c_i = f({\bf e}_i)\qquad \mbox{ and } \qquad v = \sum_{k=0}^n {\bar c}_i {\bf e}_i.$$  The \PMlinkname{linear map}{ContinuousLinearMapping} $f$ is continuous if and only if it is bounded, i.e. there exists a constant $C$ such that $|f(v)| \le C \|v\|$.  Then $$f(v) = \sum_{k=0}^n {\bar c}_k f({\bf e}_k) = \sum_{k=0}^n |c_k|^2 \le C \sqrt{\sum_{k=0}^n |c_k|^2}.$$  Simplifying, $\sum_{k=0}^n |c_k|^2 \le C^2$.  Hence $\sum_{k=0}^\infty c_k {\bf e}_k$ converges to an element $u$ in $H$.

For every basis element, $f({\bf e}_i) = c_k = \langle u, {\bf e}_i \rangle$.  By linearity, it will also be true that $$f(v) = \langle u, v \rangle\mbox{ if $v$ is a finite superposition of basis vectors.}$$  Any vector in the Hilbert space can be written as the limit of a sequence of finite superpositions of basis vectors hence, by continuity, $$f(v) = \langle u, v \rangle\mbox{ for all }v \in \mathcal{H}$$

It is easy to see that $u$ is unique.  Suppose there existed two vectors $u_1$ and $u_2$ such that $f(v) = \langle u_1, v \rangle = \langle u_2, v \rangle$.  Then $\langle u_1 - u_2, v \rangle = 0$ for all vectors $v \in \mathcal{H}$.  But then, $\langle u_1 - u_2, u_1 - u_2 \rangle = 0$ which is only possible if $u_1 - u_2 = 0$, i.e. if $u_1 = u_2$.
%%%%%
%%%%%
\end{document}
