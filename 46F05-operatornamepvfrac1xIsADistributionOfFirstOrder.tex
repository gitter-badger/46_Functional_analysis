\documentclass[12pt]{article}
\usepackage{pmmeta}
\pmcanonicalname{operatornamepvfrac1xIsADistributionOfFirstOrder}
\pmcreated{2013-03-22 13:46:07}
\pmmodified{2013-03-22 13:46:07}
\pmowner{Koro}{127}
\pmmodifier{Koro}{127}
\pmtitle{$\operatorname{p.\!v.}(\frac{1}{x})$ is a distribution of first order}
\pmrecord{7}{34473}
\pmprivacy{1}
\pmauthor{Koro}{127}
\pmtype{Proof}
\pmcomment{trigger rebuild}
\pmclassification{msc}{46F05}
\pmclassification{msc}{46-00}

% this is the default PlanetMath preamble.  as your knowledge
% of TeX increases, you will probably want to edit this, but
% it should be fine as is for beginners.

% almost certainly you want these
\usepackage{amssymb}
\usepackage{amsmath}
\usepackage{amsfonts}

% used for TeXing text within eps files
%\usepackage{psfrag}
% need this for including graphics (\includegraphics)
%\usepackage{graphicx}
% for neatly defining theorems and propositions
%\usepackage{amsthm}
% making logically defined graphics
%%%\usepackage{xypic}

% there are many more packages, add them here as you need them

% define commands here

\newcommand{\sR}[0]{\mathbb{R}}
\newcommand{\sC}[0]{\mathbb{C}}
\newcommand{\sN}[0]{\mathbb{N}}
\newcommand{\sZ}[0]{\mathbb{Z}}

% The below lines should work as the command
% \renewcommand{\bibname}{References}
% without creating havoc when rendering an entry in 
% the page-image mode.
\makeatletter
\@ifundefined{bibname}{}{\renewcommand{\bibname}{References}}
\makeatother

\newcommand*{\norm}[1]{\lVert #1 \rVert}
\newcommand*{\abs}[1]{| #1 |}
\begin{document}
\PMlinkescapeword{finite}
\newcommand{\cpv}[0]{\operatorname{p.\!v.}(\frac{1}{x})}
\newcommand{\cD}[0]{\mathcal{D}}
\newcommand{\supp}[0]{\operatorname{supp}}
(Following \cite{reed, igari}.)
%{\bf Proof.} 
Let $u\in \cD(U)$. Then $\supp u \subset [-k,k]$ for
some $k>0$. For any $\varepsilon>0$,  $u(x)/x$ is Lebesgue integrable 
in $|x|\in[\varepsilon,k]$. Thus, by a change of variable, we have
\begin{eqnarray*}
\cpv(u) &=& \lim_{\varepsilon\to0+} \int_{[\varepsilon,k]} \frac{u(x)-u(-x)}{x} dx.
\end{eqnarray*}
Now it is clear that the integrand is continuous for all $x\in \sR\setminus\{0\}$.
What is more, the integrand approaches $2u'(0)$ for $x\to 0$, so the
integrand has a removable discontinuity at $x=0$. That is, by assigning the value 
$2u'(0)$ to the integrand at $x=0$, the integrand becomes continuous in $[0,k]$.
This means that the integrand is Lebesgue measurable on $[0,k]$. 
Then, by defining
$f_n(x) = \chi_{[1/n,k]} \big(u(x)-u(-x)\big)/x$
(where $\chi$ is the characteristic function), and
applying the Lebesgue dominated convergence theorem, we have
\begin{eqnarray*}
\cpv(u) &=& \int_{[0,k]} \frac{u(x)-u(-x)}{x} dx.
\end{eqnarray*}
It follows that $\cpv(u)$ is finite, i.e., $\cpv$ takes values in $\sC$. 
Since $\cD(U)$ is a vector space, if follows easily from the above expression 
that $\cpv$ is linear. 

To prove that $\cpv$ is continuous, we shall use condition (3) on 
\PMlinkname{this page}{Distribution4}.      
For this, suppose $K$ is a compact subset of $\sR$ and 
$u\in \cD_K$. Again, we can assume that $K\subset [-k,k]$ for some $k>0$.
For $x>0$, we have 
\begin{eqnarray*}
|\frac{u(x)-u(-x)}{x}| &=& |\frac{1}{x}\int_{(-x,x)} u'(t) dt| \\
&\le &  2 ||u'||_\infty,
\end{eqnarray*}
where $||\cdot||_\infty$ is the supremum norm. In the first equality we have used the
fundamental theorem of calculus for the Lebesgue integral (valid since 
$u$ is absolutely continuous on $[-k,k]$). Thus
$$ |\cpv(u)| \le 2k ||u'||_\infty$$
and $\cpv$ is a \PMlinkname{distribution of first order}{Distribution4} as claimed. $\Box$

\begin{thebibliography}{9}
\bibitem{reed} M. Reed, B. Simon,
  \emph{Methods of Modern Mathematical Physics: Functional Analysis I},
Revised and enlarged edition,  Academic Press, 1980. 
\bibitem{igari} S. Igari, \emph{Real analysis - With an introduction to Wavelet Theory}, American Mathematical Society, 1998.
 \end{thebibliography}
%%%%%
%%%%%
\end{document}
