\documentclass[12pt]{article}
\usepackage{pmmeta}
\pmcanonicalname{SpectralInvarianceTheoremforCalgebras}
\pmcreated{2013-03-22 17:29:53}
\pmmodified{2013-03-22 17:29:53}
\pmowner{asteroid}{17536}
\pmmodifier{asteroid}{17536}
\pmtitle{spectral invariance theorem (for $C^*$-algebras)}
\pmrecord{7}{39886}
\pmprivacy{1}
\pmauthor{asteroid}{17536}
\pmtype{Theorem}
\pmcomment{trigger rebuild}
\pmclassification{msc}{46H10}
\pmclassification{msc}{46L05}
\pmsynonym{spectral invariance theorem}{SpectralInvarianceTheoremforCalgebras}
\pmsynonym{invariance of the spectrum of $C^*$-subalgebras}{SpectralInvarianceTheoremforCalgebras}
\pmdefines{invertibility in $C^*$-subalgebras}

% this is the default PlanetMath preamble.  as your knowledge
% of TeX increases, you will probably want to edit this, but
% it should be fine as is for beginners.

% almost certainly you want these
\usepackage{amssymb}
\usepackage{amsmath}
\usepackage{amsfonts}

% used for TeXing text within eps files
%\usepackage{psfrag}
% need this for including graphics (\includegraphics)
%\usepackage{graphicx}
% for neatly defining theorems and propositions
%\usepackage{amsthm}
% making logically defined graphics
%%%\usepackage{xypic}

% there are many more packages, add them here as you need them

% define commands here

\begin{document}
The spectral permanence theorem (\PMlinkescapetext{parent} entry) relates the spectrums $\sigma_{\mathcal{B}}(x)$ and $\sigma_{\mathcal{A}}(x)$ of an element $x \in \mathcal{B} \subseteq \mathcal{A}$ relatively to the Banach algebras $\mathcal{B}$ and $\mathcal{A}$.

For \PMlinkname{$C^*$-algebras}{CAlgebra} the situation is quite \PMlinkescapetext{simple}.

{\bf Spectral invariance theorem -} Suppose $\mathcal{A}$ is a unital $C^*$-algebra and $\mathcal{B} \subseteq \mathcal{A}$ a $C^*$-subalgebra that contains the identity of $\mathcal{A}$. Then for every $x \in \mathcal{B}$ one has
\begin{displaymath}
\sigma_{\mathcal{B}}(x)=\sigma_{\mathcal{A}}(x).
\end{displaymath}

The spectral invariance theorem is a straightforward corollary of the next more general theorem about invertible elements in $C^*$-subalgebras.

{\bf Theorem -} Let $x \in \mathcal{B} \subset \mathcal{A}$ be as above. Then $x$ is invertible in $\mathcal{B}$ if and only if $x$ invertible in $\mathcal{A}$.

{\bf Proof :}
\begin{itemize} 
\item $(\Longrightarrow)$

If $x$ is invertible in $\mathcal{B}$ then it is clearly invertible in $\mathcal{A}$.

\item $(\Longleftarrow)$

If $x$ is invertible in $\mathcal{A}$, then so is $y=x^*x$. Thus, $0 \notin \sigma_{\mathcal{A}}(y)$.

Since $y$ is \PMlinkname{self-adjoint}{InvolutaryRing}, $\sigma_{\mathcal{A}}(y) \subseteq \mathbb{R}$ (see this \PMlinkname{entry}{SpecialElementsInACAlgebraAndTheirSpectralProperties}), and so $\mathbb{C} - \sigma_{\mathcal{A}}(y)$ has no \PMlinkname{bounded}{Bounded} connected components.

By the \PMlinkname{spectral permanence theorem}{SpectralPermanenceTheorem} we must have $\sigma_{\mathcal{B}}(y)=\sigma_{\mathcal{A}}(y)$. Hence, $0 \notin \sigma_{\mathcal{B}}(y)$, i.e. $y$ is invertible in $\mathcal{B}$.

It follows that $x^{-1}=(x^*x)^{-1}x^*=y^{-1}x^* \in \mathcal{B}$, i.e. $x$ is invertible in $\mathcal{B}$. $\square$
\end{itemize}

%%%%%
%%%%%
\end{document}
