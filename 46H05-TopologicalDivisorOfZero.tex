\documentclass[12pt]{article}
\usepackage{pmmeta}
\pmcanonicalname{TopologicalDivisorOfZero}
\pmcreated{2013-03-22 16:12:15}
\pmmodified{2013-03-22 16:12:15}
\pmowner{CWoo}{3771}
\pmmodifier{CWoo}{3771}
\pmtitle{topological divisor of zero}
\pmrecord{7}{38299}
\pmprivacy{1}
\pmauthor{CWoo}{3771}
\pmtype{Definition}
\pmcomment{trigger rebuild}
\pmclassification{msc}{46H05}
\pmsynonym{generalized divisor of zero}{TopologicalDivisorOfZero}

\endmetadata

\usepackage{amssymb,amscd}
\usepackage{amsmath}
\usepackage{amsfonts}

% used for TeXing text within eps files
%\usepackage{psfrag}
% need this for including graphics (\includegraphics)
%\usepackage{graphicx}
% for neatly defining theorems and propositions
%\usepackage{amsthm}
% making logically defined graphics
%%\usepackage{xypic}
\usepackage{pst-plot}
\usepackage{psfrag}

% define commands here

\begin{document}
Let $A$ be a normed ring.\, An element $a\in A$ is said to be a \emph{left topological divisor of zero} if there is a sequence $a_n$ with\, $\|a_n\|=1$\, for all $n$ such that\, $$\lim_{n\to\infty}\|aa_n\| = 0.$$\, Analogously, $a$ is a \PMlinkescapetext{\emph{right topological divisor of zero}} if\, $$\lim_{n\to\infty}\|b_na\| = 0,$$ for some sequence $b_n$ with\, $\|b_n\|=1$.\, The element $a$ is a \emph{topological divisor of zero} if it is both a left and a \PMlinkescapetext{right} topological divisor of zero.

\textbf{Remarks}.  
\begin{itemize}
\item Any zero divisor is a topological divisor of zero.
\item If $a$ is a (left) topological divisor of zero, then $ba$ is a (left) topological divisor of zero.  As a result, $a$ is never a unit, for if $b$ is its inverse, then $1=ba$ would be a topological divisor of zero, which is impossible.
\item In a commutative Banach algebra $A$, an element is a topological divisor of zero if it lies on the boundary of $U(A)$, the group of units of $A$.
\end{itemize}
%%%%%
%%%%%
\end{document}
