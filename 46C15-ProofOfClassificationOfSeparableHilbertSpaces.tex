\documentclass[12pt]{article}
\usepackage{pmmeta}
\pmcanonicalname{ProofOfClassificationOfSeparableHilbertSpaces}
\pmcreated{2013-03-22 14:34:11}
\pmmodified{2013-03-22 14:34:11}
\pmowner{rspuzio}{6075}
\pmmodifier{rspuzio}{6075}
\pmtitle{proof of classification of separable Hilbert spaces}
\pmrecord{5}{36127}
\pmprivacy{1}
\pmauthor{rspuzio}{6075}
\pmtype{Proof}
\pmcomment{trigger rebuild}
\pmclassification{msc}{46C15}
%\pmkeywords{Hilbert space}
%\pmkeywords{separable}
\pmrelated{VonNeumannAlgebra}

\endmetadata

% this is the default PlanetMath preamble.  as your knowledge
% of TeX increases, you will probably want to edit this, but
% it should be fine as is for beginners.

% almost certainly you want these
\usepackage{amssymb}
\usepackage{amsmath}
\usepackage{amsfonts}

% used for TeXing text within eps files
%\usepackage{psfrag}
% need this for including graphics (\includegraphics)
%\usepackage{graphicx}
% for neatly defining theorems and propositions
%\usepackage{amsthm}
% making logically defined graphics
%%%\usepackage{xypic}

% there are many more packages, add them here as you need them

% define commands here
\begin{document}
The strategy will be to show that any separable, infinite
dimensional Hilbert space $H$ is equivalent to $\ell^2$, where
$\ell^2$ is the space of all square summable sequences.  Then it
will follow that any two separable, infinite dimensional Hilbert
spaces, being equivalent to the same space, are equivalent to each
other.

Since $H$ is separable, there exists a countable dense subset $S$ of
$H$.  Choose an enumeration of the elements of $S$ as $s_0, s_1,
s_2, \ldots$.  By the Gram-Schmidt orthonormalization procedure, one
can exhibit an orthonormal set $e_0, e_1, e_2, \ldots$ such that
each $e_i$ is a finite linear combination of the $s_i$'s.

Next, we will demonstrate that Hilbert space spanned by the $e_i$'s
is in fact the whole space $H$.  Let $v$ be any element of $H$.
Since $S$ is dense in $H$, for every integer $n$, there exists an
integer $m_n$ such that
 $$\| v - s_{m_n} \| \le 2^{-n}$$
The sequence $(s_{m_0}, s_{m_1}, s_{m_2}, \ldots)$ is a Cauchy
sequence because
 $$\| s_{m_i} - s_{m_j} \| \le \| s_{m_i} - v\| + \|
v - s_{m_j} \| \le 2^{-i} + 2^{-j}$$
 Hence the limit of this sequence must lie in the Hilbert space spanned by
$\{s_0, s_1,  s_2, \ldots\}$, which is the same as the Hilbert space
spanned by $\{e_0, e_1, e_2, \ldots\}$.  Thus, $\{e_0, e_1, e_2,
\ldots\}$ is an orthonormal basis for $H$.

To any $v \in H$ associate the sequence $U(v) = ( \langle v, s_0
\rangle, \langle v, s_1 \rangle, \langle v, s_2 \rangle, \ldots )$.
That this sequence lies in $\ell^2$ follows from the generalized Parseval equality
 $$\|v\|^2 = \sum_{k=0}^\infty \langle v, s_k \rangle$$
which also shows that $\|U(v)\|_{\ell^2} = \|v\|_H$.  On the other
hand, let $(w_0, w_1, w_2, \ldots)$ be an element of $\ell^2$. Then,
by definition, the sequence of partial sums $(w_0^2, w_0^2 + w_1^2,
w_0^2 + w_1^2 + w_2^2, \ldots)$ is a Cauchy sequence.  Since
 $$\| \sum_{i=0}^m w_i e_i - \sum_{i=0}^n w_i e_i \|^2 = \sum_{i=0}^m w_i^2 -
 \sum_{i=0}^n w_i^2$$
 if $m>n$, the sequence of partial sums of $\sum_{k=0}^\infty w_i e_i$
is also a Cauchy sequence, so $\sum_{k=0}^\infty w_i e_i$ converges
and its limit lies in $H$.  Hence the operator $U$ is invertible and
is an isometry between $H$ and $\ell^2$.
%%%%%
%%%%%
\end{document}
