\documentclass[12pt]{article}
\usepackage{pmmeta}
\pmcanonicalname{SobolevInequality}
\pmcreated{2013-03-22 15:05:14}
\pmmodified{2013-03-22 15:05:14}
\pmowner{paolini}{1187}
\pmmodifier{paolini}{1187}
\pmtitle{Sobolev inequality}
\pmrecord{10}{36812}
\pmprivacy{1}
\pmauthor{paolini}{1187}
\pmtype{Theorem}
\pmcomment{trigger rebuild}
\pmclassification{msc}{46E35}
\pmsynonym{Sobolev embedding}{SobolevInequality}
\pmsynonym{sobolev immersion}{SobolevInequality}
\pmsynonym{Gagliardo Nirenberg inequality}{SobolevInequality}
\pmrelated{LpSpace}
\pmdefines{Sobolev conjugate}
\pmdefines{Sobolev exponent}

% this is the default PlanetMath preamble.  as your knowledge
% of TeX increases, you will probably want to edit this, but
% it should be fine as is for beginners.

% almost certainly you want these
\usepackage{amssymb}
\usepackage{amsmath}
\usepackage{amsfonts}

% used for TeXing text within eps files
%\usepackage{psfrag}
% need this for including graphics (\includegraphics)
%\usepackage{graphicx}
% for neatly defining theorems and propositions
%\usepackage{amsthm}
% making logically defined graphics
%%%\usepackage{xypic}

% there are many more packages, add them here as you need them

% define commands here
\newtheorem{theorem}{Theorem}
\newcommand{\R}{\mathbb R}
\begin{document}
For $1\le p < n$, define the \emph{Sobolev conjugate \PMlinkescapetext{exponent}} of $p$ as
\[
  p^* := \frac {np}{n-p}.
\]
Note that $-n/p^* = 1-n/p$.

In the following statement $\nabla$ represent the weak derivative and $W^{1,p}(\Omega)$ 
is the Sobolev space of functions $u\in L^p(\Omega)$ whose weak derivative $\nabla u$ is itself in $L^p(\Omega)$.

\begin{theorem}
Assume that $p\in [1,n)$ and let $\Omega$ be a bounded, open subset of $\R^n$
with Lipschitz boundary. 
Then there is a constant $C>0$ such that, for all $u\in W^{1,p}(\Omega)$ one has
\[
 \Vert u \Vert_{L^{p^*}(\Omega)} \le C \Vert \nabla u \Vert_{L^p(\Omega)}.
\]
\end{theorem}

We can restate the previous Theorem by saying that the Sobolev space $W^{1,p}(\Omega)$ is a subspace of the Lebesgue space $L^{p^*}(\Omega)$ and that the inclusion map $i\colon W^{1,p}(\Omega)\to L^{q^*}(\Omega)$ is continuous.
%%%%%
%%%%%
\end{document}
