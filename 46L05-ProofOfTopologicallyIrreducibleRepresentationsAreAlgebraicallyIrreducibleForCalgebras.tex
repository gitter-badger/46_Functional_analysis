\documentclass[12pt]{article}
\usepackage{pmmeta}
\pmcanonicalname{ProofOfTopologicallyIrreducibleRepresentationsAreAlgebraicallyIrreducibleForCalgebras}
\pmcreated{2013-03-22 19:04:12}
\pmmodified{2013-03-22 19:04:12}
\pmowner{karstenb}{16623}
\pmmodifier{karstenb}{16623}
\pmtitle{proof of topologically irreducible representations are algebraically irreducible for $C^*$-algebras}
\pmrecord{8}{41954}
\pmprivacy{1}
\pmauthor{karstenb}{16623}
\pmtype{Proof}
\pmcomment{trigger rebuild}
\pmclassification{msc}{46L05}

\endmetadata

\usepackage{amssymb}
\usepackage{amsmath}
\usepackage{amsfonts}
\usepackage{amsthm}
\usepackage{mathrsfs}
\usepackage{latexsym}
\usepackage[sort&compress]{natbib}

% commands
\newcommand{\scal}[2]{\langle #1, #2 \rangle}

%theorems
\theoremstyle{definition}
\newtheorem{Def}{Definition}

\theoremstyle{plain}
\newtheorem*{Lem}{Theorem}
\newtheorem*{Lem2}{Lemma}
\newtheorem{Cor}{Corollary}
\newtheorem{Rem}{Remark}

\begin{document}
Denote by $\mathcal{H}$ an arbitrary Hilbert space.
To fix notation let $\mathcal{U} \subset \mathcal{L}(\mathcal{H})$ be a $C^{\ast}$ subalgebra of $\mathcal{L}(\mathcal{H})$. We then define the \emph{commutator} of $\mathcal{U}$ by
\begin{align*}
\mathcal{U}' &:= \{T \in \mathcal{L}(\mathcal{H}) : TU = UT \ \forall U \in \mathcal{U}\}
\end{align*}

Note that $\mathcal{U}'$ is closed with regard to the weak topology (see \PMlinkname{this entry}{CommutantIsAWeakOperatorClosedSubalgebra}). So $\mathcal{U}'$ is always a von Neumann algebra.

As an immediate consequence of Schur's Lemma for group representations on a Hilbert space we obtain the following result.

\textbf{Lemma.}
Let $\mathcal{U}$ be a $\ast$-algebra and let $\pi$ be a $\ast$-representation of $\mathcal{U}$ on the Hilbert space $\mathcal{H}$. Then $\pi$ is topologically irreducible iff $\pi(\mathcal{U})' = \mathbb{C} I$. 


We can now prove the result.

\textbf{Theorem.}
Let $\mathcal{U}$ be a $C^{\ast}$ algebra. Assume the $\ast$-representation $\pi$ of $\mathcal{U}$ on the Hilbert space $\mathcal{H}$ is topologically irreducible. Then $\pi$ is algebraically irreducible.   


\begin{proof}
By the Lemma it follows that $\pi(\mathcal{U})' = \mathbb{C} I$. Hence $\pi(\mathcal{U})'' = \mathcal{L}(\mathcal{H})$. By the \PMlinkname{double commutant theorem}{VonNeumannDoubleCommutantTheorem} every operator in $\mathcal{L}(\mathcal{H})_1$ (the unit ball in the set of bounded operators $\mathcal{L}(\mathcal{H})$) belongs to the strong operator closure of $\pi(\mathcal{U})_1$ (the unit ball in $\pi(\mathcal{U})$). 

To show the algebraical irreducibility of $\pi(\mathcal{U})$ it is enough to find for two given vectors $x, y \in \mathcal{H}, x \not= 0$ an element $T \in \mathcal{U}$ such that $\pi(T) x = y$ holds. Indeed, it is enough to consider the case $\|x\| = \|y\| = 1$. 

Now construct the rank one approximation $\tilde{T}_1 := y \otimes x$ ($\Leftrightarrow \tilde{T}_1 z = \scal{x}{z} y, z \in \mathcal{H} \Rightarrow \tilde{T}_1 x = \|x\| y = y$) with a corresponding $T_1 \in \mathcal{U}, \pi(T_1) \in \pi(\mathcal{U})_1$, so that $\|y - \pi(T_1) x\| = \|\tilde{T}_1 x - \pi(T_1) x\| \leq \frac{1}{2}$. 

Approximate further $\tilde{T}_2 := (y - \pi(T_1) x) \otimes x \in \frac{1}{2} \mathcal{L}(\mathcal{H})_1$ and choose $\pi(T_2) \in \frac{1}{2} \pi(\mathcal{U})_1$ with $\|y - \pi(T_1) x - \pi(T_2) x\| = \|\tilde{T}_2 x- \pi(T_2) x\| \leq \frac{1}{2^2}$.

Proceed by induction with $\tilde{T}_n := ( y- \sum_{j=1}^{n-1} \pi(T_j) x) \otimes x \in 2^{-j} \mathcal{L}(\mathcal{H})_1$. Choose $\pi(T_n) \in 2^{-n} \pi(\mathcal{U})_1$ with $\|y - \sum_{j=1}^{n} \pi(T_j) x\| = \|\tilde{T}_n x - \pi(T_n) x \| \leq 2^{-n}$. Then we have $\pi(T) := \sum_{j=1}^n \pi(T_n)$ in $\mathcal{U}$ and $\pi(T) x = y$ which completes the proof. 
\end{proof}

%%%%%
%%%%%
\end{document}
