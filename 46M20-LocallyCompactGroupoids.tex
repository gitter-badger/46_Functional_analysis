\documentclass[12pt]{article}
\usepackage{pmmeta}
\pmcanonicalname{LocallyCompactGroupoids}
\pmcreated{2013-03-22 18:17:51}
\pmmodified{2013-03-22 18:17:51}
\pmowner{bci1}{20947}
\pmmodifier{bci1}{20947}
\pmtitle{locally compact groupoids}
\pmrecord{45}{40915}
\pmprivacy{1}
\pmauthor{bci1}{20947}
\pmtype{Topic}
\pmcomment{trigger rebuild}
\pmclassification{msc}{46M20}
\pmclassification{msc}{81R15}
\pmclassification{msc}{46L05}
\pmclassification{msc}{81R50}
\pmclassification{msc}{18B40}
\pmclassification{msc}{55U40}
\pmclassification{msc}{22A22}
\pmsynonym{locally compact topological groupoids}{LocallyCompactGroupoids}
%\pmkeywords{locally compact groupoids}
%\pmkeywords{second countable topological space}
%\pmkeywords{locally compact Hausdorff space}
%\pmkeywords{topological groupoids}
%\pmkeywords{continuous inversion maps}
\pmrelated{Groupoids}
\pmrelated{TopologicalGroupoid}
\pmrelated{LocallyCompact}
\pmrelated{LocallyCompactHausdorffSpace}
\pmrelated{T2Space}
\pmrelated{SecondCountable}
\pmrelated{QuantumGroupoids2}
\pmrelated{GroupoidAndGroupRepresentationsRelatedToQuantumSymmetries}
\pmrelated{GroupoidRepresentationsInducedByMeasure}
\pmrelated{LocallyCompactHausdorffSpace}
\pmrelated{Cat}
\pmdefines{locally compact groupoid}
\pmdefines{groupoid as a small category}
\pmdefines{topological groupoid}
\pmdefines{analytic groupoid}
\pmdefines{locally compact topological group}
\pmdefines{second countable locally compact groupoid}

% this is the default PlanetMath preamble.  as your 

% almost certainly you want these
\usepackage{amssymb}
\usepackage{amsmath}
\usepackage{amsfonts}

% used for TeXing text within eps files
%\usepackage{psfrag}
% need this for including graphics (\includegraphics)
%\usepackage{graphicx}
% for neatly defining theorems and propositions
%\usepackage{amsthm}
% making logically defined graphics
%%%\usepackage{xypic}

% there are many more packages, add them here as you need them

% define commands here
\usepackage{amsmath, amssymb, amsfonts, amsthm, amscd, latexsym, enumerate}
\usepackage{xypic, xspace}
\usepackage[mathscr]{eucal}
\usepackage[dvips]{graphicx}
\usepackage[curve]{xy}

\setlength{\textwidth}{6.5in}
%\setlength{\textwidth}{16cm}
\setlength{\textheight}{9.0in}
%\setlength{\textheight}{24cm}

\hoffset=-.75in     %%ps format
%\hoffset=-1.0in     %%hp format
\voffset=-.4in


\theoremstyle{plain}
\newtheorem{lemma}{Lemma}[section]
\newtheorem{proposition}{Proposition}[section]
\newtheorem{theorem}{Theorem}[section]
\newtheorem{corollary}{Corollary}[section]

\theoremstyle{definition}
\newtheorem{definition}{Definition}[section]
\newtheorem{example}{Example}[section]
%\theoremstyle{remark}
\newtheorem{remark}{Remark}[section]
\newtheorem*{notation}{Notation}
\newtheorem*{claim}{Claim}

\renewcommand{\thefootnote}{\ensuremath{\fnsymbol{footnote}}}
\numberwithin{equation}{section}

\newcommand{\Ad}{{\rm Ad}}
\newcommand{\Aut}{{\rm Aut}}
\newcommand{\Cl}{{\rm Cl}}
\newcommand{\Co}{{\rm Co}}
\newcommand{\DES}{{\rm DES}}
\newcommand{\Diff}{{\rm Diff}}
\newcommand{\Dom}{{\rm Dom}}
\newcommand{\Hol}{{\rm Hol}}
\newcommand{\Mon}{{\rm Mon}}
\newcommand{\Hom}{{\rm Hom}}
\newcommand{\Ker}{{\rm Ker}}
\newcommand{\Ind}{{\rm Ind}}
\newcommand{\IM}{{\rm Im}}
\newcommand{\Is}{{\rm Is}}
\newcommand{\ID}{{\rm id}}
\newcommand{\grpL}{{\rm GL}}
\newcommand{\Iso}{{\rm Iso}}
\newcommand{\rO}{{\rm O}}
\newcommand{\Sem}{{\rm Sem}}
\newcommand{\SL}{{\rm Sl}}
\newcommand{\St}{{\rm St}}
\newcommand{\Sym}{{\rm Sym}}
\newcommand{\Symb}{{\rm Symb}}
\newcommand{\SU}{{\rm SU}}
\newcommand{\Tor}{{\rm Tor}}
\newcommand{\U}{{\rm U}}

\newcommand{\A}{\mathcal A}
\newcommand{\Ce}{\mathcal C}
\newcommand{\D}{\mathcal D}
\newcommand{\E}{\mathcal E}
\newcommand{\F}{\mathcal F}
%\newcommand{\grp}{\mathcal G}
\renewcommand{\H}{\mathcal H}
\renewcommand{\cL}{\mathcal L}
\newcommand{\Q}{\mathcal Q}
\newcommand{\R}{\mathcal R}
\newcommand{\cS}{\mathcal S}
\newcommand{\cU}{\mathcal U}
\newcommand{\W}{\mathcal W}

\newcommand{\bA}{\mathbb{A}}
\newcommand{\bB}{\mathbb{B}}
\newcommand{\bC}{\mathbb{C}}
\newcommand{\bD}{\mathbb{D}}
\newcommand{\bE}{\mathbb{E}}
\newcommand{\bF}{\mathbb{F}}
\newcommand{\bG}{\mathbb{G}}
\newcommand{\bK}{\mathbb{K}}
\newcommand{\bM}{\mathbb{M}}
\newcommand{\bN}{\mathbb{N}}
\newcommand{\bO}{\mathbb{O}}
\newcommand{\bP}{\mathbb{P}}
\newcommand{\bR}{\mathbb{R}}
\newcommand{\bV}{\mathbb{V}}
\newcommand{\bZ}{\mathbb{Z}}

\newcommand{\bfE}{\mathbf{E}}
\newcommand{\bfX}{\mathbf{X}}
\newcommand{\bfY}{\mathbf{Y}}
\newcommand{\bfZ}{\mathbf{Z}}

\renewcommand{\O}{\Omega}
\renewcommand{\o}{\omega}
\newcommand{\vp}{\varphi}
\newcommand{\vep}{\varepsilon}

\newcommand{\diag}{{\rm diag}}
\newcommand{\grp}{{\mathsf{G}}}
\newcommand{\dgrp}{{\mathsf{D}}}
\newcommand{\desp}{{\mathsf{D}^{\rm{es}}}}
\newcommand{\grpeod}{{\rm Geod}}
%\newcommand{\grpeod}{{\rm geod}}
\newcommand{\hgr}{{\mathsf{H}}}
\newcommand{\mgr}{{\mathsf{M}}}
\newcommand{\ob}{{\rm Ob}}
\newcommand{\obg}{{\rm Ob(\mathsf{G)}}}
\newcommand{\obgp}{{\rm Ob(\mathsf{G}')}}
\newcommand{\obh}{{\rm Ob(\mathsf{H})}}
\newcommand{\Osmooth}{{\Omega^{\infty}(X,*)}}
\newcommand{\grphomotop}{{\rho_2^{\square}}}
\newcommand{\grpcalp}{{\mathsf{G}(\mathcal P)}}

\newcommand{\rf}{{R_{\mathcal F}}}
\newcommand{\grplob}{{\rm glob}}
\newcommand{\loc}{{\rm loc}}
\newcommand{\TOP}{{\rm TOP}}

\newcommand{\wti}{\widetilde}
\newcommand{\what}{\widehat}

\renewcommand{\a}{\alpha}
\newcommand{\be}{\beta}
\newcommand{\grpa}{\grpamma}
%\newcommand{\grpa}{\grpamma}
\newcommand{\de}{\delta}
\newcommand{\del}{\partial}
\newcommand{\ka}{\kappa}
\newcommand{\si}{\sigma}
\newcommand{\ta}{\tau}

\newcommand{\med}{\medbreak}
\newcommand{\medn}{\medbreak \noindent}
\newcommand{\bign}{\bigbreak \noindent}

\newcommand{\lra}{{\longrightarrow}}
\newcommand{\ra}{{\rightarrow}}
\newcommand{\rat}{{\rightarrowtail}}
\newcommand{\ovset}[1]{\overset {#1}{\ra}}
\newcommand{\ovsetl}[1]{\overset {#1}{\lra}}
\newcommand{\hr}{{\hookrightarrow}}

\newcommand{\<}{{\langle}}

%\newcommand{\>}{{\rangle}}

%\usepackage{geometry, amsmath,amssymb,latexsym,enumerate}
%%%\usepackage{xypic}

\def\baselinestretch{1.1}


\hyphenation{prod-ucts}

%\grpeometry{textwidth= 16 cm, textheight=21 cm}

\newcommand{\sqdiagram}[9]{$$ \diagram  #1  \rto^{#2} \dto_{#4}&
#3  \dto^{#5} \\ #6    \rto_{#7}  &  #8   \enddiagram
\eqno{\mbox{#9}}$$ }

\def\C{C^{\ast}}

\newcommand{\labto}[1]{\stackrel{#1}{\longrightarrow}}

%\newenvironment{proof}{\noindent {\bf Proof} }{ \hfill $\Box$
%{\mbox{}}

\newcommand{\quadr}[4]
{\begin{pmatrix} & #1& \\[-1.1ex] #2 & & #3\\[-1.1ex]& #4&
 \end{pmatrix}}
\def\D{\mathsf{D}}

\begin{document}
\section{Locally compact groupoids}
 This is a specific topic entry defining the basics of locally compact groupoids and related concepts. 

 Let us first recall the related concepts of groupoid and \emph{topological groupoid}, 
together with the appropriate notations needed to define a \emph{locally compact groupoid}.
 
\subsubsection{Groupoids and topological groupoids: categorical definitions}

 Recall that a groupoid $\grp$ is a small category with inverses 
over its set of objects $X = Ob(\grp)$~. One writes $\grp^y_x$ for 
the set of morphisms in $\grp$ from $x$ to $y$~. 

 \emph{A topological groupoid} consists of a space $\grp$, a distinguished subspace 
$\grp^{(0)} = \obg \subset \grp$, called {\it the space of objects} of $\grp$, 
together with maps
\begin{equation}
r,s~:~ \xymatrix{ \grp \ar@<1ex>[r]^r \ar[r]_s & \grp^{(0)} }
\end{equation}

called the {\it range} and {\it source maps} respectively,
together with a law of composition
\begin{equation}
\circ~:~ \grp^{(2)}: = \grp \times_{\grp^{(0)}} \grp = \{
~(\gamma_1, \gamma_2) \in \grp \times \grp ~:~ s(\gamma_1) =
r(\gamma_2)~ \}~ \lra ~\grp~,
\end{equation}

such that the following hold~:~
\begin{enumerate}
\item[(1)]
$s(\gamma_1 \circ \gamma_2) = r(\gamma_2)~,~ r(\gamma_1 \circ
\gamma_2) = r(\gamma_1)$~, for all $(\gamma_1, \gamma_2) \in
\grp^{(2)}$~.

\item[(2)]
$s(x) = r(x) = x$~, for all $x \in \grp^{(0)}$~.

\item[(3)]
$\gamma \circ s(\gamma) = \gamma~,~ r(\gamma) \circ \gamma =
\gamma$~, for all $\gamma \in \grp$~.

\item[(4)]
$(\gamma_1 \circ \gamma_2) \circ \gamma_3 = \gamma_1 \circ
(\gamma_2 \circ \gamma_3)$~.

\item[(5)]
Each $\gamma$ has a two--sided inverse $\gamma^{-1}$ with $\gamma
\gamma^{-1} = r(\gamma)~,~ \gamma^{-1} \gamma = s (\gamma)$~.

Furthermore, only for topological groupoids the inverse map needs be continuous.
It is usual to call $\grp^{(0)} = Ob(\grp)$ {\it the set of objects}
of $\grp$~. For $u \in Ob(\grp)$, the set of arrows $u \lra u$ forms a
group $\grp_u$, called the \emph{isotropy group of $\grp$ at $u$}.
\end{enumerate}

Thus, as is well kown, a topological groupoid is just a groupoid internal to the 
\PMlinkname{category of topological spaces and continuous maps}{CategoryOfPointedTopologicalSpaces}. The notion of internal groupoid has proved significant in a number of fields, since groupoids generalize bundles of groups, group actions, and equivalence relations. For a further study of groupoids we refer the reader to ref. \cite{Brown2006}.


\subsection{Locally compact and analytic groupoids}

\begin{definition}
A \emph{locally compact groupoid} $\grp_{lc}$ is defined as a groupoid that has also the topological structure of a second countable, \PMlinkname{locally compact Hausdorff space}{LocallyCompactHausdorffSpace}, and if the product and also inversion maps are continuous. Moreover, each $\grp_{lc}^u$ as well as the unit space $\grp_{lc}^0$ is closed in $\grp_{lc}$. 
\end{definition}

\begin{remark}
 The locally compact Hausdorff second countable spaces are {\em analytic}.

 One can therefore say also that $\grp_{lc}$ is analytic.

 When the groupoid $\grp_{lc}$ has only one object in its object space, that is, when it becomes a group, the above
definition is restricted to that of a \emph{locally compact topological group}; it is then a special case of a one-object category with all of its morphisms being invertible, that is also endowed with a locally compact, topological structure.

\end{remark}
 
\begin{thebibliography}{9}

\bibitem{Brown2006}
R. Brown. (2006). \emph{Topology and Groupoids}. BookSurgeLLC
\end{thebibliography}

%%%%%
%%%%%
\end{document}
