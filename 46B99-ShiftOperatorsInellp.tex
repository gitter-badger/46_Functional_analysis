\documentclass[12pt]{article}
\usepackage{pmmeta}
\pmcanonicalname{ShiftOperatorsInellp}
\pmcreated{2013-03-22 15:17:40}
\pmmodified{2013-03-22 15:17:40}
\pmowner{matte}{1858}
\pmmodifier{matte}{1858}
\pmtitle{shift operators in $\ell^p$}
\pmrecord{4}{37092}
\pmprivacy{1}
\pmauthor{matte}{1858}
\pmtype{Definition}
\pmcomment{trigger rebuild}
\pmclassification{msc}{46B99}
\pmclassification{msc}{54E50}

% this is the default PlanetMath preamble.  as your knowledge
% of TeX increases, you will probably want to edit this, but
% it should be fine as is for beginners.

% almost certainly you want these
\usepackage{amssymb}
\usepackage{amsmath}
\usepackage{amsfonts}
\usepackage{amsthm}

\usepackage{mathrsfs}

% used for TeXing text within eps files
%\usepackage{psfrag}
% need this for including graphics (\includegraphics)
%\usepackage{graphicx}
% for neatly defining theorems and propositions
%
% making logically defined graphics
%%%\usepackage{xypic}

% there are many more packages, add them here as you need them

% define commands here

\newcommand{\sR}[0]{\mathbb{R}}
\newcommand{\sC}[0]{\mathbb{C}}
\newcommand{\sN}[0]{\mathbb{N}}
\newcommand{\sZ}[0]{\mathbb{Z}}

 \usepackage{bbm}
 \newcommand{\Z}{\mathbbmss{Z}}
 \newcommand{\C}{\mathbbmss{C}}
 \newcommand{\F}{\mathbbmss{F}}
 \newcommand{\R}{\mathbbmss{R}}
 \newcommand{\Q}{\mathbbmss{Q}}



\newcommand*{\norm}[1]{\lVert #1 \rVert}
\newcommand*{\abs}[1]{| #1 |}



\newtheorem{thm}{Theorem}
\newtheorem{defn}{Definition}
\newtheorem{prop}{Proposition}
\newtheorem{lemma}{Lemma}
\newtheorem{cor}{Corollary}
\begin{document}
Let $\F$ be $\R$ or $\C$, and let $1\le p\le \infty$,
let $\ell^p(\F), \Vert\cdot \Vert_p$ be as in the parent entry.

The right and left \emph{shift operators}
   $S_r, S_l\colon \ell^p(\F)\to \ell^p(\F)$
as defined as follows. For $a=(a_1,a_2, \ldots)\in \ell^p(\F)$, 
$$
  S_r(a)=(0,a_1, a_2, \ldots)
$$
and 
$$
  S_l(a)=(a_2, a_3, \ldots).
$$

\subsubsection*{Properties}
\begin{enumerate}
\item $S_l \circ S_r$ is the identity, but $S_r\circ S_l$ is not. 
\item $S_r$ is an isometry; $\Vert S_r(a)\Vert = \Vert a \Vert$,
    and $\Vert S_l(a)\Vert_p \le \Vert a\Vert$. Both shift operators
    are therefore bounded (and continuous). 
\end{enumerate}
%%%%%
%%%%%
\end{document}
