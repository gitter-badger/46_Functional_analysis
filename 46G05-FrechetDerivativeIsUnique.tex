\documentclass[12pt]{article}
\usepackage{pmmeta}
\pmcanonicalname{FrechetDerivativeIsUnique}
\pmcreated{2013-03-22 16:08:35}
\pmmodified{2013-03-22 16:08:35}
\pmowner{Mathprof}{13753}
\pmmodifier{Mathprof}{13753}
\pmtitle{Fr\'echet derivative is unique}
\pmrecord{12}{38221}
\pmprivacy{1}
\pmauthor{Mathprof}{13753}
\pmtype{Theorem}
\pmcomment{trigger rebuild}
\pmclassification{msc}{46G05}
\pmrelated{derivative}

\endmetadata

% this is the default PlanetMath preamble.  as your knowledge
% of TeX increases, you will probably want to edit this, but
% it should be fine as is for beginners.

% almost certainly you want these
\usepackage{amssymb}
\usepackage{amsmath}
\usepackage{amsfonts}

% used for TeXing text within eps files
%\usepackage{psfrag}
% need this for including graphics (\includegraphics)
%\usepackage{graphicx}
% for neatly defining theorems and propositions
\usepackage{amsthm}
% making logically defined graphics
%%%\usepackage{xypic}

% there are many more packages, add them here as you need them

% define commands here

\begin{document}
{\textbf{Theorem}
The Fr\'echet derivative is unique.\\
{\textbf{Proof.}
Assume that both $A$ and $B$ in $L(\mathsf{V,W})$ satisfy the condition for the \PMlinkname{Fr\'echet derivative}{derivative2} at the point $\mathbf{x}$. To prove that they are equal we will show that for all $\varepsilon >0$ the operator norm $\|A-B\|$ is not greater than $\varepsilon$. By the definition of limit there exists a positive $\delta$ such that for all $\|\mathbf{h}\|\leq\delta$
\[\|f(\mathbf{x}+\mathbf{h})-f(\mathbf{x})-A\mathbf{h}\|\leq\frac{\varepsilon}{2}\cdot\|\mathbf{h}\|
\mbox{ and }
\|f(\mathbf{x}+\mathbf{h})-f(\mathbf{x})-B\mathbf{h}\|\leq\frac{\varepsilon}{2}\cdot\|\mathbf{h}\|\]
holds. This gives
\begin{align*}
\|(A-B)\mathbf{h}\|&=\|(f(\mathbf{x}+\mathbf{h})-f(\mathbf{x})-A\mathbf{h})-(f(\mathbf{x}+\mathbf{h})-f(\mathbf{x})-B\mathbf{h})\|\\
&\leq\|f(\mathbf{x}+\mathbf{h})-f(\mathbf{x})-A\mathbf{h}\|+\|f(\mathbf{x}+\mathbf{h})-f(\mathbf{x})-B\mathbf{h}\|\\
&<\varepsilon\cdot\|\mathbf{h}\|.
\end{align*}
Now we have
\[\delta\cdot\|A-B\|=\delta\cdot\sup_{\|\mathbf{g}\|\leq 1}\|(A-B)\mathbf{g}\|=\sup_{\|\mathbf{g}\|\leq\delta}\|(A-B)\mathbf{g}\|\leq\sup_{\|\mathbf{g}\|\leq\delta}\varepsilon\cdot\|\mathbf{g}\|\leq\varepsilon\cdot\delta,
\]
thus $\|A-B\|\leq\varepsilon$ as we wanted to show.
%%%%%
%%%%%
\end{document}
