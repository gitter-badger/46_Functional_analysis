\documentclass[12pt]{article}
\usepackage{pmmeta}
\pmcanonicalname{Calgebra}
\pmcreated{2013-03-22 12:57:55}
\pmmodified{2013-03-22 12:57:55}
\pmowner{asteroid}{17536}
\pmmodifier{asteroid}{17536}
\pmtitle{$C^*$-algebra}
\pmrecord{34}{33334}
\pmprivacy{1}
\pmauthor{asteroid}{17536}
\pmtype{Definition}
\pmcomment{trigger rebuild}
\pmclassification{msc}{46L05}
\pmclassification{msc}{46L87}
\pmsynonym{C*-algebra}{Calgebra}
\pmsynonym{C* algebra}{Calgebra}
%\pmkeywords{C*-algebra properties}
%\pmkeywords{relation to groupoids}
%\pmkeywords{norm and spectral radius}
%\pmkeywords{noncommutative C*-algebras}
%\pmkeywords{C*-axiom}
\pmrelated{GroupCAlgebra}
\pmrelated{VonNeumannAlgebra}
\pmrelated{NoncommutativeGeometry}
\pmrelated{GroupoidCConvolutionAlgebra}
\pmrelated{GroupoidCDynamicalSystem}
\pmrelated{CAlgebra3}
\pmrelated{NuclearCAlgebra}
\pmrelated{HomomorphismsOfCAlgebrasAreContinuous}
\pmrelated{ContinuousLinearMapping}
\pmrelated{OperatorNorm}
\pmrelated{C_cG}
\pmrelated{UniformContinuityOverLocallyCompa}
\pmdefines{$C^*$ axiom}

\endmetadata

\usepackage{amssymb}
\usepackage{amsmath}
\usepackage{amsfonts}
\usepackage{amsthm}

% used for TeXing text within eps files
%\usepackage{psfrag}
% need this for including graphics (\includegraphics)
%\usepackage{graphicx}
% making logically defined graphics
%%%\usepackage{xypic}

% my maths package

\newcommand*{\Nset}{\mathbb{N}}
\newcommand*{\Zset}{\mathbb{Z}}
\newcommand*{\Qset}{\mathbb{Q}}
\newcommand*{\Rset}{\mathbb{R}}
\newcommand*{\Cset}{\mathbb{C}}
\newcommand*{\Hset}{\mathbb{H}}
\newcommand*{\Oset}{\mathbb{O}}
\newcommand*{\Bset}{\mathbb{B}}
\newcommand*{\Kset}{\mathbb{K}}
\newcommand*{\Sset}{\mathbb{S}}
\newcommand*{\Tset}{\mathbb{T}}
\newcommand*{\GLgrp}{\mathrm{GL}}
\newcommand*{\SLgrp}{\mathrm{SL}}
\newcommand*{\Ogrp}{\mathrm{O}}
\newcommand*{\SOgrp}{\mathrm{SO}}
\newcommand*{\Ugrp}{\mathrm{U}}
\newcommand*{\SUgrp}{\mathrm{SU}}
\newcommand*{\e}{\mathop{\mathrm{e}}\nolimits}
\newcommand*{\im}{\mathord{\mathrm{i}}}
\newcommand*{\identity}{\mathord{\mathrm{1\!\!\!\:I}}}
\newcommand*{\tr}{\mathop{\mathrm{tr}}}
\newcommand*{\Tr}{\mathop{\mathrm{Tr}}}
\newcommand*{\norm}[1]{\Vert #1\Vert}
\renewcommand*{\d}{\mathrm{d}}
\newcommand*{\deriv}[2]{\frac{\d #1}{\d #2}}
\newcommand*{\pderiv}[2]{\frac{\partial #1}{\partial #2}}
\newcommand*{\fderiv}[2]{\frac{\delta #1}{\delta #2}}

% my noncommutative geometry package

\newcommand*{\algebra}[1][A]{\mathord{\mathcal{#1}}}
\newcommand*{\hilbert}[1][H]{\mathord{\mathcal{#1}}}
\newcommand*{\hilbmod}[1][E]{\mathord{\mathcal{#1}}}
\newcommand*{\Matrix}[2]{\mathord{\mathrm{M}_{#1}(#2)}}
\newcommand*{\dixmier}{\mathop{\mathrm{Tr}_\omega}}
\newcommand*{\Res}{\mathop{\mathrm{Res}}}
\newcommand*{\Wres}{\mathop{\mathrm{Wres}}}
\newcommand*{\Aut}{\mathop{\mathrm{Aut}}\nolimits}
\newcommand*{\Inn}{\mathop{\mathrm{Inn}}\nolimits}
\newcommand*{\Out}{\mathop{\mathrm{Out}}\nolimits}
\newcommand*{\Diff}{\mathop{\mathrm{Diff}}\nolimits}
\newcommand*{\Ker}{\mathop{\mathrm{Ker}}\nolimits}
\newcommand*{\Coker}{\mathop{\mathrm{Coker}}\nolimits}
\newcommand*{\Img}{\mathop{\mathrm{Im}}\nolimits}
\newcommand*{\End}{\mathop{\mathrm{End}}\nolimits}
\newcommand*{\spin}{\mathop{\mathrm{spin}}\nolimits}
\newcommand*{\Ind}{\mathop{\mathrm{Ind}}\nolimits}
\newcommand*{\KK}{\mathit{KK}}
\newcommand*{\HH}{\mathit{HH}}
\newcommand*{\HC}{\mathit{HC}}
\newcommand*{\ch}{\mathop{\mathrm{ch}}\nolimits}

% my category theory package

\newcommand*{\mathcat}[1]{\mathord{\mathbf{#1}}}
\newcommand*{\id}{\mathrm{id}}
\newcommand*{\op}{\mathrm{op}}
\newcommand*{\boxprod}{\mathbin{\square}}

%
\newcommand{\diag}{{\rm diag}}
\newcommand{\grp}{{\mathbb G}}
\newcommand{\dgrp}{{\mathbb D}}
\newcommand{\desp}{{\mathbb D^{\rm{es}}}}
\newcommand{\Geod}{{\rm Geod}}
\newcommand{\geod}{{\rm geod}}
\newcommand{\hgr}{{\mathbb H}}
\newcommand{\mgr}{{\mathbb M}}
\newcommand{\ob}{{\rm Ob}}
\newcommand{\obg}{{\rm Ob(\mathbb G)}}
\newcommand{\obgp}{{\rm Ob(\mathbb G')}}
\newcommand{\obh}{{\rm Ob(\mathbb H)}}
\newcommand{\Osmooth}{{\Omega^{\infty}(X,*)}}
\newcommand{\ghomotop}{{\rho_2^{\square}}}
\newcommand{\gcalp}{{\mathbb G(\mathcal P)}}
\renewcommand{\a}{\alpha}
\newcommand{\be}{\beta}
\newcommand{\ga}{\gamma}
\newcommand{\Ga}{\Gamma}
\newcommand{\de}{\delta}
\newcommand{\del}{\partial}
\newcommand{\ka}{\kappa}
\newcommand{\si}{\sigma}
\newcommand{\ta}{\tau}

\newcommand{\med}{\medbreak}
\newcommand{\U}{{\rm U}}

\newcommand{\A}{\mathcal A}
\newcommand{\Ce}{\mathcal C}
\newcommand{\D}{\mathcal D}
\newcommand{\E}{\mathcal E}
\newcommand{\F}{\mathcal F}
\newcommand{\G}{\mathcal G}
\newcommand{\Q}{\mathcal Q}
\newcommand{\R}{\mathcal R}
\newcommand{\cS}{\mathcal S}
\newcommand{\cU}{\mathcal U}
\newcommand{\W}{\mathcal W}

\newcommand{\lra}{{\longrightarrow}}
\newcommand{\ra}{{\rightarrow}}



% my environments

\newtheoremstyle{inlinedefn}{}{0pt}{}{}{\bfseries}{.}{0.5em}{}
\theoremstyle{inlinedefn}
\newtheorem{definition}{Definition}

\newtheoremstyle{break}{\baselineskip}{\baselineskip}{\itshape}{}{\bfseries}{}{\newline}{}
\theoremstyle{break}
\newtheorem{example}{Example}

% misc commands

\newcommand*{\defn}[1]{\textbf{#1}}
\begin{document}
\PMlinkescapeword{type}
\PMlinkescapeword{algebraic}
\PMlinkescapeword{structure}
\PMlinkescapeword{axiom}
\PMlinkescapeword{algebraic structure}
\PMlinkescapeword{property}
\PMlinkescapeword{self-adjoint}
\PMlinkescapeword{self-adjoint elements}
\PMlinkescapeword{unitary}
\PMlinkescapeword{unitary elements}
\PMlinkescapeword{positive}
\PMlinkescapeword{positive elements}
\PMlinkescapeword{interpretation}
\PMlinkescapeword{decomposition}
\PMlinkescapeword{similar}


\section{Definition}

$C^*$-algebras are a type of involutive Banach algebras which arise
in the study of operators on Hilbert spaces, Lie group 
representations, locally compact topological spaces, knots, noncommutative \PMlinkescapetext{geometry}, among other
topics in mathematics and theoretical physics . Their study was initiated in the 1930's with the purpose of axiomatizing  quantum mechanics, and still today, $C^*$-algebras play a decisive role in formulations of quantum statistical mechanics and quantum \PMlinkescapetext{field theory}.

 The defining property of these algebras is
that the norm and the involution are related in a very special way.

{\bf Definition  1} - \emph{A \textbf{$C^*$-algebra} $\mathcal{A}$ is a Banach *-algebra such that
$\norm{a^*a} = \norm{a}^2$ for all $a \in \mathcal{A}$}.

The equality in Definition 1 is sometimes called the {\bf $C^*$ axiom}. It turns out that one can weaken this condition and still specify
the same \PMlinkescapetext{class} of algebras.

{\bf Definition  2} - \emph{A \textbf{$C^*$-algebra} $\mathcal{A}$ is a Banach algebra with an antilinear 
involution $*$ such that $\norm{a}^2 \leq \norm{a^* a}$ for all $a \in \mathcal{A}$.}

{\bf Definition  3} - \emph{A \textbf{$C^*$-algebra} $\mathcal{A}$ is a Banach algebra with an antilinear 
involution $*$ such that $\norm{a^* a} = \norm{a^*}\norm{a}$}


\section{C* Norm}

$C^*$-algebras are a very peculiar type of topological algebras. The $C^*$ axiom, deceptively \PMlinkescapetext{simple}, imposes severe \PMlinkescapetext{restrictions} on the the algebraic and
 topological structure of a $C^*$-algebra.

A most striking consequence of the $C^*$ axiom is that the norm is solely determined by the algebraic structure of the algebra. More specifically,
\begin{displaymath}
\|a\| = \sqrt{R_{\sigma}(a^*a)}
\end{displaymath}
where $R_{\sigma}(x)$ denotes the spectral radius of the element $x \in \mathcal{A}$. For $C^*$ algebras with an identity element $e$ we can specify even further: the norm of an element $a \in \mathcal{A}$ is determined by
\begin{displaymath}
\|a\|^2 = \sup \{ |\lambda|: \lambda \in \mathbb{C} \;\text{and}\; a^*a- \lambda e \;\text{is not invertible}\}
\end{displaymath}

This also implies that the norm in a $C^*$-algebra is unique, in the sense that there is no other norm in the algebra that satisfies that $C^*$ axiom, i.e. that turns the algebra into a $C^*$-algebra. This is a stark contrast to the case 
of general normed algebras, where one may find many norms which are
\PMlinkescapetext{compatible} with the algebraic structure.

Moreover, the $C^*$ norm occupies a unique \PMlinkescapetext{place} amongst the possible
norms for an involutive algebra.  Suppose that $\mathcal{A}$ is a $C^*$ algebra
with norm $\norm{\cdot}_{C^*}$.  If $\norm{\cdot}_{B}$ is
any other norm for which $\mathcal{A}$ is a Banach *-algebra, then we must have
\begin{displaymath}
\|a\|_{C^*} \leq \|a\|_B \,, \qquad \forall a \in \mathcal{A}
\end{displaymath}
Hence we see that the $C^*$ norm enjoys an extremal property --- it
is the smallest possible norm for which $\mathcal{A}$ is a Banach *-algebra.

There are many other surprising consequences of the $C^*$ axiom, like: *-homomorphisms between $C^*$-algebras are automatically continuous and every $C^*$-algebra is semi-simple, which again are not true for general involutive algebras.

\section{Elements of a C*-algebra}

Like in involutory rings, there are some special elements in $C^*$-algebras that deserve some attention. We recall some definitions here:

Let $\mathcal{A}$ be a $C^*$-algebra with identity element $e$. An element $a \in \mathcal{A}$ is said to be
\begin{itemize}
\item {\bf self-adjoint} if $a^* = a$
\item {\bf unitary} if $a^*a = aa^* = e$
\item {\bf positive} if $a = b^*b$ for some element $b \in \mathcal{A}$
\end{itemize}

It is many times useful to have some interpretation for this elements. One of this interpretations comes from complex analysis: we regard the elements of a $C^*$-algebra as functions with values in $\mathbb{C}$ and the involution as complex conjugation. 

In this frame, self-adjoint elements correspond to real functions, unitary elements correspond to functions whose values lie in the unit circle in $\mathbb{C}$ and positive elements correspond to positive functions (functions with values in $\mathbb{R^+_0}$).

It is easily seen that self-adjoint elements are closed under addition, multiplication and multiplication by real numbers. It can be proven the same for positive elements (with multiplication by positive numbers).

There are some decompositions of elements in a $C^*$-algebra analogous to some decompositions in complex analysis. For instance, every element $a$ in a $C^*$-algebra has a unique decomposition of the form
\begin{displaymath}
a = x + i y
\end{displaymath}
where $x, y$ are self-adjoint. This is similar to the decomposition of a complex valued function in its real and imaginary parts.

Moreover, every self-adjoint element $a$ is of the form
\begin{displaymath}
a = x - y
\end{displaymath}
where $x, y$ are positive elements. This is similar to the decomposition of real valued functions in its positive and negative parts.

There are many other aspects of the theory of $C^*$-algebras for which this kind of interpretation proves to be very insightful.
 
For example, $C^*$-algebras happen to have a natural partial ordering. One can define an
ordering by declaring that $x > y$ when $x - y$ is positive. Given this ordering, one can then
speak of such things as monotonic functions, monotonic sequences, 
and positive linear functionals on the algebra.  These notions, in
turn, prove to be extremely useful in the study of $C^*$-algebras.  

\section{Examples}

Having discussed these algebras in general terms, it is high time
that we illustrate the definition with some examples.

{\bf Example 1}

As our first class of examples, we consider algebras of functions.
Let $X$ be a compact Hausdorff topological space and let 
$C(X)$ be the algebra of continuous functions from $X$ to 
$\mathbb{C}$.  For the involution operation, 
we take pointwise complex conjugation and for the norm we take the 
norm of uniform convergence:
\[
 \norm{f} = \sup_{x \in X} |f(x)|
\]
It is a routine matter to check that the norm and involution 
satisfy the appropriate algebraic requirements.  Completeness 
under this norm follows from the fact that the uniform limit 
of continuous functions on a locally compact Hausdorff topological 
space is continuous.

More generally, instead of a compact space, we can take a locally compact Hausdorff space $X$ and consider the algebra $C_0(X)$ of continuous functions $X \to \mathbb{C}$ that vanish at infinity, endowed with the same norm and involution. These are important examples of $C^*$-algebras.

{\bf Example 2}

As our second class of examples, we consider operator algebras.
Let $H$ be a complex Hilbert space with inner product $\langle \cdot , \cdot \rangle$ and 
let $B(H)$ be the algebra of bounded operators on $H$. For the
involution, we take the adjoint operation and as a norm we 
take the usual operator norm:
\[
 \norm{T} = \sup_{\|\xi\| = 1} \|T\xi\|
\]
Again, it is straightforward to verify that the norm and
involution satisfy the appropriate algebraic requirements,
as is done in an attachment to this entry.  Completeness
under the norm follows from a well-known theorem of 
functional analysis.

\section{Commutative and noncommutative C*-algebras}

The algebras $C_0(X)$ in Example 1 above are more than just an example. In fact, all commutative $C^*$-algebras are *-isomorphic to $C_0(X)$ for some locally compact Hausdorff space $X$. Moreover, $X$ is compact if and only if the $C^*$-algebra has an identity element. This is the content of the Gelfand-Naimark theorem.

Furthermore, there is a correspondence between properties of the topological space and properties of the $C^*$-algebra. For example: a compactification of the space corresponds to a unitization of the $C^*$-algebra; the space is connected if and only if the $C^*$-algebra has no non-trivial projections, among many other interesting correspondences.

For this reason, the theory of (noncommutative) $C^*$-algebras is many times called noncommutative topology (click on the link for more information).

$\,$

The second example is also more than just an example of $C^*$-algebras. In fact, by the Gelfand-Naimark representation theorem, all $C^*$-algebras are *-isomorphic to a norm closed *-subalgebra of $B(H)$, for some Hilbert space $H$.

Note, however, that this does not provide a ``classification'' of $C^*$-algebras since we do not know in general what are the closed *-subalgebras of $B(H)$. This is merely a (very-important) structural theorem. The classification problem for $C^*$-algebras is still open.

\section{Additional Examples}

{\bf Example 3}

Compact operators in a Hilbert space $H$ form a closed ideal of $B(H)$. Moreover, this ideal is also closed for the involution of operators. Hence, the algebra of compact operators, $K(H)$, is a $C^*$-algebra.

{\bf Example 4}

Let $(X, \mathfrak{B}, \mu)$ be a measure space. The space \PMlinkname{$L^{\infty}(X)$}{LpSpace} is an algebra under pointwise operations. We can define an involution again by complex conjugation and we consider the essential supremum norm $\| \cdot \|_{\infty}$. It can be readily verified that, under these operations and norm, $L^{\infty}(X)$ is a $C^*$-algebra.

The algebras $L^{\infty}(X)$ are also particularly important since they are examples of von Neumann algebras, which are a specific kind of $C^*$-algebras.





%%%%%
%%%%%
\end{document}
